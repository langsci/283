\documentclass[output=paper]{langsci/langscibook}
\ChapterDOI{10.5281/zenodo.4049681}

\author{Olga Borik\affiliation{Universidad Nacional de Educaci\'on a Distancia (UNED), Madrid}\and Joan Borràs-Comes\affiliation{Universitat Aut\`onoma de Barcelona \& Universitat Pompeu Fabra, Barcelona}\lastand Daria Seres\affiliation{Universitat Aut\`onoma de Barcelona}
}
\title{Preverbal (in)definites in Russian: An experimental study}  
\abstract{This paper presents an experimental investigation aimed at determining the exact nature of the relationship between type of interpretation (definite or indefinite) and linear position (pre- or postverbal) of bare nominal\is{bare nominal} subjects of intransitive predicates in Russian.\il{Russian|(} The results of our experiment confirm that preverbal position correlates with a definite interpretation, and postverbal position with an indefinite interpretation. However, we also discovered that the acceptance rate of preverbal indefinites is reasonably high. We suggest an explanation for the appearance of indefinites in preverbal subject position in terms of lexical accessibility,\is{accessibility} which is couched in general terms of D-linking.\is{discourse linking (D-linking)}}


\begin{document}
\maketitle

\section{Introduction}\label{2sec:1}
This paper is devoted to the study of bare singular nominals in Russian in pre- and postverbal subject position and a possible correlation between their (in)definite-ness and their linear position in a sentence. Russian, as is well known, is a language without articles, i.\,e., a language that does not express definiteness as a grammatical category in a strict sense. This means that to establish the referential status\is{referential status} of a bare nominal\is{bare nominal} as a definite or an indefinite expression (a contrast\is{contrast} that seems to be perceivable for native speakers of Russian), the communication participants have to rely on a combination of clues and use various indicators provided both at a sentential and at a discourse level. In this paper, we are interested in establishing the role of the linear position of a nominal in this combination of factors that Russian uses to signal (in)definiteness.

{
To tackle this problem we conducted an experimental study, the empirical coverage of which is limited to subjects of stage-level intransitive verbs.\is{stage-level predicates} In this study, native speakers of Russian were asked to judge the acceptability of sentences containing pre- and postverbal bare nominals\is{bare nominal} in two types of contexts: definiteness- and indefiniteness-suggesting contexts. In definiteness-suggesting contexts we used anaphoric bare nominals,\is{bare nominal} i.\,e., those that are linked to a referent in the previous context. This practical decision suggests a familiarity\is{familiarity} theory of definiteness \citep{christophersen:39,heim:82} as a theoretical basis for the paper. The familiarity\is{familiarity} approach to definiteness is based on the idea that the referent of the definite description is known/familiar to both the speaker and the addressee. Definites are assumed to pick out an existing referent from the discourse, whereas indefinites introduce new referents \citep[see specifically][]{heim:82,kamp:81}.
}

A different and very influential theory of definiteness is based on the uniqueness\is{uniqueness} property of definite nominals \citep{russell:05}, which is usually taken to be part of the presupposition associated with a definite NP \citep{frege:1879,strawson:50}. The two approaches are not mutually exclusive, and familiarity\is{familiarity} is sometimes claimed to be subsumed by uniqueness\is{uniqueness} \citep[see, for instance,][]{farkas:02,beaver:coppock:15}. The basic idea behind the uniqueness\is{uniqueness} approach is that a definite description is felicitous if, within a certain pragmatically determined domain, there is exactly one entity, in the case of singulars, or unique maximal set, in the case of plurals, satisfying the description.\footnote{Some other relevant theoretical notions related to definiteness in the current literature are determinacy\is{determinacy} \citep{coppock:beaver:15} and salience\is{discourse salience} \citep{vonheus:97}. We will not discuss these here, since a deep theoretical discussion of what it means to be definite is outside the scope of this paper. We limit our attention to one particular type of definite expression in this paper, although it is very well known that there are various types of definites (cf., for instance, \citealt{lyons:99}, for an overview).}

In this paper, we follow \cite{farkas:02}, who introduced the notion of determined reference\is{determined reference} to ``capture what is common to anaphoric and unique reference'' \citep[][221]{farkas:02}. Determined reference\is{determined reference} simply means that the value assigned to the variable introduced by a definite DP is fixed: there is no choice of entities that satisfy the descriptive content of a definite nominal. Although definite descriptions\is{definite descriptions} always have a determined reference,\is{determined reference} it can be achieved in different ways: definite DPs have determined reference\is{determined reference} if the descriptive content of a nominal (i.e., {\emph{cat}} in {\emph{the cat}}) denotes a singleton set relative to a context or if they are used anaphorically. In our experimental study, the nominal that appears in the definiteness-suggesting contexts is always anaphoric, by a link to a previous antecedent or by bridging.\is{anaphora!bridging anaphora} 

The main conclusion drawn from our experiment is that linear word order in Russian cannot be considered the primary means for expressing definiteness and indefiniteness of bare nominals.\is{bare nominal} Apart from the fact that we have confirmed a strong and clear correlation between linear position and interpretation of bare nominals,\is{bare nominal} in the sense that preverbal bare subjects are mostly interpreted definitely and postverbal subjects indefinitely, we also report on another important result: some (not all) indefinite preverbal subjects are judged as acceptable by native speakers. It is this result that we are focusing on: in this paper, our aim is to ascertain what makes it possible or impossible to use a bare nominal\is{bare nominal} subject in preverbal position in an indefiniteness-suggesting context. Thus, the main theoretical contribution of this paper consists in identifying requirements that facilitate the acceptability of what is considered an outcast, i.\,e., preverbal subjects with an indefinite interpretation. We propose that the general mechanism employed in licensing preverbal indefinite subjects is D-linking\is{discourse linking (D-linking)} and identify some conditions for indefinites in Russian to be D-linked,\is{discourse linking (D-linking)} justifying our proposal through an item-by-item analysis of all our preverbal contexts.

{
The rest of the paper is organised as follows. In Section \ref{2sec:2}, we discuss the category of (in)definiteness and various means of expressing it, especially in those languages that lack articles and hence, do not have a straightforward way of signalling when a nominal is (in)definite. Our discussion is limited to Russian, a well-known representative of languages without articles. Section \ref{2sec:3} is devoted to the experimental study that we have conducted with pre- and postverbal subjects of intransitive verbs. In this section, we outline the design and the methodology used in the experimental study, describe the results and present our interpretation of the results. In Section \ref{2sec:4}, we discuss some theoretical issues that can be raised on the basis of the results of our experiment, and Section \ref{2sec:5} concludes the paper.
}

\section{The category of (in)definiteness and its realizations}\label{2sec:2}

The category of definiteness (with two values, definite and indefinite) is mostly discussed in the literature in relation to articles, although it is often assumed that this category is, in fact, semantically universal and also present in those languages that do not possess formal means to express definiteness. The intuition is, indeed, that one of the differences in the interpretation of the nominal subject in (\ref{2ex:1a}) vs. (\ref{2ex:1b}) in Russian corresponds to the contrast between (\ref{2ex:2a}) and (\ref{2ex:2b}) in English,\il{English} where the (in)definiteness of the subject is overtly expressed. In Russian, even though the nominal subject appears in the same morphological form and without any additional markers in both sentences, the interpretation that the speakers are likely to attribute to the subject {\emph{ko\v{s}ka}} in (\ref{2ex:1a}) by default seems to be indefinite, and thus comparable to the interpretation of {\emph{a cat}} in (\ref{2ex:2a}). However, the same nominal in (\ref{2ex:1b}) is most likely to be interpreted as definite, and hence is comparable to the definite subject in the English example (\ref{2ex:2b}).

\begin{exe}
\ex\label{2ex:1}
	\begin{xlista}
	\ex\label{2ex:1a}
	\gll V uglu spit ko\v{s}ka. \\
	in corner.{\sc{loc}} sleeps cat.{\sc{nom}} \\
	\glt 
	\ex\label{2ex:1b}
	\gll Ko\v{s}ka spit v uglu. \\
	cat.{\sc{nom}} sleeps in corner.{\sc{loc}} \\
	\glt 
	\end{xlista}
\ex\label{2ex:2}
	\begin{xlista}
	\ex\label{2ex:2a}
	A cat is sleeping in the corner. / There is a cat sleeping in the corner.
	\ex\label{2ex:2b}
	The cat is sleeping in the corner. 
	\end{xlista}
\end{exe}

Thus, at least at first impression, definiteness forms part of the inventory of semantic contrasts/categories\is{contrast} that can be expressed in Russian since the contrast between definite and indefinite readings of nominals can be easily perceived and understood by speakers. This observation is supported by the literature, where common wisdom seems to be that languages without articles can express definiteness contrasts despite the absence of an article system. In fact, all the literature on definiteness in Russian simply assumes that it is entirely legitimate to talk about definite and indefinite readings. The only question that is discussed and debated is {\emph{how}} (in)definiteness is expressed \citep[see, for instance,][]{galkina:fedoruk:63,pospelov:70,krylov:84,nesset:99}.

{
From a formal/compositional semantic perspective, a sentence like (\ref{2ex:3}) needs some functional semantic operations to make sure that the result of combining a nominal phrase and a predicate in a simple intransitive sentence is well formed.
}

\begin{exe}
\ex\label{2ex:3} 
\gll Ko\v{s}ka spit.\\
cat.{\sc{nom}} sleep\\
\glt `A/the cat is sleeping.'
\end{exe}

\newpage
In formal semantics, common nouns like {\emph{cat}} in English\il{English} or {\emph{ko\v{s}ka}} in Russian are expressions of the type $\langle$e,t$\rangle$, i.\,e., they denote a set of entities that can be characterized as cats.\footnote{{See \cite{chierchia:98} for the claim that common nouns can be lexically of different logical types in different languages, although in his system English\il{English} and Russian belong to the same group of languages, where the denotation of a common noun is taken to be of a predicate (i.e., $\langle$e,t$\rangle$) type.}}
Intransitive verbs are standardly given the same type $\langle$e,t$\rangle$, as they denote a set of entities that sleep. Technically speaking, the two elements in (\ref{2ex:3}) could be combined even though they are of the same type without the need to introduce any other semantic operations, for instance, by intersection, in which case we would end up with a set of entities that are sleeping cats. A way to combine the two elements without resorting to any type-shifting\is{type-shifting} operations could be by pseudo-incorporation\is{pseudo-incorporation} \citep[e.\,g.,][]{mithun:84}. However, in these cases the meaning predicted for the whole expression is far from the actual meaning of (\ref{2ex:3}): (\ref{2ex:3}) does not denote either a set of sleeping cats or a sleeping action as typically performed by cats. Moreover, the nominal itself does not exhibit any of the properties of pseudo-incorporated\is{pseudo-incorporation} nominals (cf. \citealt{borik:gehrke:15} for an overview of such properties). The sentence in (\ref{2ex:3}) is a typical predication, where something is said (asserted) about a cat entity. As a proposition, (\ref{2ex:3}) can also be given a truth value. In order to properly derive the truth conditions of (\ref{2ex:3}), we need to resort to type-shifting\is{type-shifting} operations \citep{chierchia:84,partee:87} which turn an argument (in this case, {\emph{ko\v{s}ka}} `cat') into an entity $\langle$e$\rangle$ or a quantifier $\langle$$\langle$e,t$\rangle,$t$\rangle$. 

In languages with articles, one of the functions that is attributed to an article or, in more general terms, a determiner, is shifting the noun denotation from a predicate type to an argument type. In particular, a type-shifting\is{type-shifting} operation that the definite article `performs' is called an {\emph{iota}} shift\is{iota shift} and is formally defined as follows \citep[see][998]{heim:11}:

\begin{exe}
\ex\label{2ex:4}
$\lint${\emph{the}}$\rint$ = $\lambda P$:$\exists x\forall y[P(y) \leftrightarrow x=y].\iota x.P(x)$\\
where $\iota x.$ abbreviates ``the unique x such that''
\end{exe}

{
It is reasonable to hypothesize that in Russian the same type-shifting\is{type-shifting} rules can be applied as in English,\il{English} although in the case of Russian the type-shifter itself is not lexically expressed. In fact, it has been proposed by \cite{chierchia:98} that exactly the same set of type-shifting\is{type-shifting} operators that are used to formally derive argument types in languages like English\il{English} can be employed in languages without articles to reflect various types of readings (entity type, predicate type or quantifier type) of nominal phrases. The proposal is quite attractive since it postulates a universal set of semantic operations that are used to model various denotations of nominal constituents. The only difference is that in some languages these operators are lexicalized (languages with articles), whereas in others they are not (languages without articles), as suggested, for instance, by \cite{dayal:04}.
}

Thus, from a theoretical viewpoint, it is rather attractive to assume that a universal set of formal operators, called type-shifting\is{type-shifting} operators, is postulated to derive various readings of nominal phrases in different (possibly all) languages. Both definite and indefinite readings can then be derived by using appropriate type-shifting\is{type-shifting} operations, regardless of language. It seems that we have ample empirical evidence from languages without articles like Russian that these readings do, indeed, exist, so that type-shifting\is{type-shifting} operators are not vacuous, but give rise to various interpretations of nominal arguments, as illustrated in (\ref{2ex:1}) and (\ref{2ex:3}) above. However, in the absence of any obligatory lexical items that would reflect (in)definiteness of the corresponding nominal phrase, the question that arises is how we know when a nominal phrase is interpreted as a definite or as an indefinite one in a language like Russian.

\subsection{Expressing (in)definiteness in Russian: lexical and grammatical means}\label{2sec:21}


Languages without articles possess various means to indicate the referential status\is{referential status} of a nominal argument. In this section, we will review various means that can be employed in Russian to facilitate different (definite or indefinite) readings of a nominal. 


First of all, there are lexical elements, including determiners, quantifiers and demonstrative\is{demonstratives} pronouns,\footnote{Here we refer to a class of canonical, not pragmatic, demonstratives\is{demonstratives} \citep[cf.][]{elbourne:08}. Canonical demonstratives\is{demonstratives} are strongly associated with definiteness in the literature \citep[see, for instance,][]{lyons:99,wolter:04,elbourne:08}.} that can be used to indicate whether the nominal they modify or combine with has a definite or an indefinite reading. Some examples of such lexical items are given in (\ref{2ex:5}) below:

\begin{exe}
\ex\label{2ex:5}
	\begin{xlista}
	\ex\label{2ex:5a}
	\gll	Vasja 	znaet 	{\emph{ėtogo}} 		{\emph{studenta}}. \\
		Vasja 	knows 	this.{\sc{acc.m}} 	student.{\sc{acc.m}} \\
	\glt	`Vasja knows this student.'
	\ex\label{2ex:5b}
	\gll	{\emph{Odna}} 		{\emph{znakomaja}}			prixodila 	v\v{c}era 	v 	gosti. \\
		one.{\sc{nom.f}}	acquaintance.{\sc{nom.f}} 	came 	yesterday 	to 	guests \\
	\glt	`A (particular) friend came to visit yesterday.'
	\ex\label{2ex:5c}
	\gll	Vasju 			iskala 		{\emph{kakaja-to}} 		{\emph{studentka}}. \\
		Vasja.{\sc{acc}} 	looked.for 		some.{\sc{nom.f}} 	student.{\sc{nom.f}} \\
	\glt	`Some student was looking for Vasja.'
	\ex\label{2ex:5d}
	\gll 	Vasja opjat' 	kupil 		{\emph{kakuju-nibud'}}	{\emph{erundu}}. \\
		Vasja again 	bought 	some.{\sc{acc.f}} 		nonsense.{\sc{acc.f}} \\
	\glt	`Vasja bought some useless thing again.'
	\end{xlista}
\end{exe}

In (\ref{2ex:5a}), the direct object {\emph{student}} is preceded by a demonstrative,\is{demonstratives} which gives the whole nominal phrase a definite interpretation: it is a particular, contextually unique and identifiable (possibly deictically) student that the nominal phrase refers to. In (\ref{2ex:5b}), we are dealing with a specificity\is{specificity} marker {\emph{odin}} (lit. `one', see \citealt{ionin:13}) and hence the whole noun phrase {\emph{one friend}} is a specific indefinite. Similarly, the (postverbal) subject in (\ref{2ex:5c}) is also a specific indefinite, although the marker here is different from the previous example. The last example, (\ref{2ex:5d}), features a marker for non-specific low scope indefinites, so the object argument in this example is a weak indefinite.\footnote{Various indefiniteness markers in Russian are discussed in detail in the literature, especially in relation to specificity.\is{specificity} See, for instance, \cite{haspelmath:97}; \cite{pereltsvaig:00}; \cite{yanovich:05}; \cite{geist:08}; \cite{ionin:13}, etc.} In all these examples, there is a lexical determiner that indicates the definiteness status of a nominal argument, although these elements are really not like articles in the sense that it is never (or almost never) obligatory to use them.

Apart from lexical means, there are some grammatical tools in Russian that can affect the definiteness status of a nominal phrase. The two most well-known ones are case and aspect\is{aspect}: both grammatical categories primarily affect the definiteness status of nominal arguments in direct object position. The influence of case-marking on referential properties can be demonstrated by the genitive/accusative case alternation on the direct object. For instance, mass nominal arguments of perfective verbs marked by the genitive case receive a partitive (indefinite) interpretation,\is{partitive interpretation} whereas the same object in the accusative case can be interpreted as definite: 

\begin{exe}
\ex\label{2ex:6}
	\begin{xlista}
	\ex\label{2ex:6a}
 	\gll 	Vasja	kupil 		moloka. \\
		Vasja 	bought 	milk.{\sc{gen}} 	\\
	\glt	`Vasja bought (some) milk.'
	\ex\label{2ex:6b}
	\gll	Vasja 	kupil 		moloko. \\
		Vasja 	bought 	milk.{\sc{acc}} \\
	\glt	`Vasja bought (the) milk.'
	\end{xlista}
\end{exe}

Note, however, that the accusative case in (\ref{2ex:6b}) allows for, but does not guarantee, a definite reading of the direct object {\emph{moloko}} (milk.{\sc{acc}}), so that the observed effect is not strong enough to postulate a direct link between definiteness and case-marking.\footnote{{Speaking more generally, there is no correlation between case-marking and definiteness in Russian. There are languages that seem to exhibit such a correlation, especially with respect to direct object marking, such as Turkish,\il{Turkish} Persian\il{Persian} \citep{comrie:81} and Sakha\il{Sakha} \citep{baker:15}.}}

As for aspect,\is{aspect} the question of whether/how perfectivity\is{perfectivity} influences the interpretation of a direct object in \ili{Slavic languages} has been widely discussed in the literature (see \citealt{wierzbicka:67}; \citealt{krifka:92}; \citealt{schoorlemmer:95}; \citealt{verkuyl:98}; \citealt{filip:99}, etc.). It is often claimed (ibid.) that plural and mass objects of perfective verbs\is{aspect!perfective aspect} receive a definite interpretation,\footnote{In the case of \cite{verkuyl:98}, the terminology that is used is `quantized', not `definite'.} whereas imperfective aspect\is{aspect!imperfective aspect} does not impose any restrictions on the interpretation of a direct object. The claim is illustrated in (\ref{2ex:7}) below:

\begin{exe}
\ex\label{2ex:7}
	\begin{xlista}
	\ex\label{2ex:7a}
	\gll	Vasja 	risoval 			pejza\v{z}i. \\
		Vasja 	painted.{\sc{ipfv}} 	landscapes.{\sc{acc}} \\
	\glt	`Vasja painted landscapes.'
	\ex\label{2ex:7b}
	\gll	Vasja 	narisoval 			pejza\v{z}i. \\
		Vasja 	painted.{\sc{pfv}} 	landscapes.{\sc{acc}} \\
	\glt	`Vasja painted the landscapes.'
	\end{xlista}
\end{exe}

The effect of aspect\is{aspect} on the interpretation of direct objects can be demonstrated very clearly in Bulgarian,\il{Bulgarian} another Slavic language, which, in contrast to Russian, does have a definite article. The example in (\ref{2ex:8}) below, taken from \cite[][944]{dimitrova:vulchanova:12}, illustrates that the definite article cannot be omitted if the verb is perfective: 

\begin{exe}
\ex\label{2ex:8}
	\begin{xlista}
	\ex\label{2ex:8a}
	\gll	Ivan 	pi 			vino. \\
		Ivan 	drank.{\sc{ipfv}} 	wine.{\sc{acc}} 	\\
	\glt	`Ivan drank/was drinking wine.'
	\ex\label{2ex:8b}
	\gll 	Ivan 	izpi 			vino*(-to).\\
		Ivan 	drank.{\sc{pfv}} 	wine.{\sc{acc}}-the\\
	\glt	`Ivan drank the wine.'
	\end{xlista}
\end{exe}

Thus, the correlation between the aspectual marking of a verb and the interpretation of its direct object seems, indeed, to be very strong. However, as illustrated above, perfective aspect\is{aspect!perfective aspect} is also compatible with an indefinite partitive interpretation\is{partitive interpretation} if the object appears in the genitive case. Thus, in example (\ref{2ex:6a}), the object is clearly indefinite and best translated as `some (indefinite quantity of) milk' and not `some of the milk'. Future, or non-past \citep[][]{borik:06} tense on a verb is another factor that can neutralize the effect of perfectivity:\is{perfectivity}  if the verb in (\ref{2ex:7b}) is used in a non-past tense, the inferred definiteness of the direct object is considerably weakened or even invalidated. This means that the effect of aspect\is{aspect} on definiteness of an internal argument is really just a tendency and might be overruled by other factors. But even in the strongest cases comparable to (\ref{2ex:7b}), the correlation between perfectivity\is{perfectivity}  and definiteness in Russian is not absolute. \cite[][92]{borik:06} provides an example where the internal argument of a perfective verb can have a non-maximal/existential interpretation:\footnote{{In fact, \cite{borik:06}  claims that the interpretation in this case is `generic'.\is{generic sentence} However, since the sentence itself is not interpreted generically but rather refers to an episodic event, `existential interpretation' is a more accurate term. We thank one of the reviewers for pointing this out to us.}}

\begin{exe}
\ex\label{2ex:9}
\gll	Petja 	razdelil 			ljudej 			na 	dobryx 		i 	zlyx. \\
	Petja 	divided.{\sc{pfv}} 	people.{\sc{acc}} 	in 	kind.{\sc{acc}} 	and 	mean.{\sc{acc}} \\
\glt	`Petja divided people into kind ones and mean ones.'
\end{exe}

{
To summarize, we have seen that there are some grammatical factors, such as case or aspect,\is{aspect} that can favour or facilitate a certain (definite or indefinite) interpretation of a nominal argument, but there are no strict correlations between definiteness and other grammatical categories. The lexical means that Russian possesses to signal (in)definiteness are only optional and cannot be semantically compared to articles. The interim conclusion is, then, that there is nothing so far in the grammatical system of Russian that would allow us to predict whether a nominal argument will necessarily be interpreted as a definite or an indefinite one. 
}

Another factor often mentioned in the discussion of (in)definiteness in Russian is the effect of word order on the interpretation of nominal arguments, the phenomenon which underlies the experimental part of the paper. In the next subsection, we will briefly discuss word order in Russian and its (potential) relation to definiteness, and provide motivation for the experiment that will be reported in Section \ref{2sec:3} of the paper. 

\subsection{The effects of word order on the interpretation of nominal arguments}\label{2sec:22}

Russian is a classic example of a so-called `free word order' language, i.\,e., a language where the linear order of the elements in a sentence is determined not so much by grammatical functions like subject and object, or grammatical properties like case assignment, but by the requirements imposed by discourse and information structure\is{information structure} (see \citealt{mathesius:64}; \citealt{sgall:72}; \citealt{hajicova:74}; \citealt{isacenko:76}; \citealt{yokoyama:86}; \citealt{comrie:89}; among others). However, more cautious typological sources always point out that the `free' word order is to a large extent an illusion, since various permutations of sentence constituents are usually not entirely free but guided by some pragmatic or information structure\is{information structure} principles \citep[see, for instance,][]{dryer:13}. For these languages, the connection is often made between the linear position of a nominal argument and its definiteness status. In particular, it is often stated in the literature that preverbal (subject) position is strongly associated with a definite interpretation (\citealt{pospelov:70}; \citealt{fursenko:70}; \citealt{kramsky:72}; \citealt{chvany:73}; \citealt{szwedek:74}; \citealt{topolinjska:09}; etc.), whereas nominals in postverbal position are likely to be interpreted as indefinites. This descriptive generalization is primarily assumed to hold for subjects, as the canonical word order in Russian is SVO, and objects, unless they are topicalized, follow the verb rather than precede it. 

The relationship between definiteness and preverbal subjects is often mediated by topicality,\is{topicality} a notion that plays a crucial role in the interpretation of arguments in languages with a (relatively) free word order. Both preverbal subjects and objects are considered topics\is{topic} when they appear in sentence-initial position \citep[][]{jasinskaja:14}. As illustrated in the examples below, both subject (see (\ref{2ex:10})) and object (see (\ref{2ex:11})) in the leftmost position can also be left-dislocated (creating, arguably, a bi-clausal structure), a construction that we consider to be a reasonable, although not clear-cut diagnostic for topichood \citep[][]{reinhart:81}.

\begin{exe}
\ex\label{2ex:10}
	\begin{xlista}
	\ex\label{2ex:10a}
	\gll	Tolya v\v{c}era 		razgovarival 	s 	Anej. \\
		Tolya yesterday 	talked.{\sc{ipfv}} 	with	Anya 	\\
	\glt	`Tolya yesterday talked to Anya.'
	\ex\label{2ex:10b}
	\gll 	\v{C}to	kasaetsja	Toli, 		to 	on	v\v{c}era 	razgovarival 	s 	Anej. \\
		what 	concerns 	Tolya 	that	he	yesterday	talked.{\sc{ipfv}} 	with	Anya \\
	\glt	`As for Tolya, he talked to Anya yesterday.'
	\end{xlista}
\ex\label{2ex:11}
	\begin{xlista}
	\ex\label{2ex:11a}
	\gll	Varenje	ja	v\v{c}era 	s'el. \\
		jam 		I 	yesterday 	ate.{\sc{pfv}} \\
	\glt	`I ate the jam yesterday.'
	\ex\label{2ex:11b}
	\gll	\v{C}to 	kasaetsja	varenja, 	to 	ja	ego 	v\v{c}era 	s'el. \\
		what 	concerns 	jam 		that 	I 	it 	yesterday 	ate.{\sc{pfv}} \\
	\glt	`As for the jam, I ate it yesterday.'
	\end{xlista}
\end{exe}

The type of topic\is{topic} illustrated in (\ref{2ex:10a}) and (\ref{2ex:11a}) is called aboutness topic\is{topic!aboutness topic} \citep{reinhart:81} or, what we believe to be essentially the same phenomenon, internal topics\is{topic!internal topic} \citep{king:95}. The connection between definiteness and topicality\is{topicality} is based on a descriptive generalization that is accepted in a lot of semantic literature on topics\is{topic} in general, namely, that elements that appear in topic position\is{topic position} can only be referential, i.\,e., definite or specific indefinite (see \citealt{reinhart:81}; \citealt{erteschik:shir:97}; \citealt{portner:yabushita:01}; \citealt{endriss:09}; etc.). An appealing intuitive idea behind this generalization is that if there is no entity that the nominal topic\is{topic} refers to, this expression cannot be aboutness topic\is{topic!aboutness topic} because then there is no entity to be talked about.

Nevertheless, a number of examples from various Romance languages\il{Romance languages} have been brought up in the literature to show that a topicalized left-dislocated element can, in fact, be interpreted non-specifically. The following examples from \cite{leonetti:10} illustrate the phenomenon in Spanish\il{Spanish} (\ref{2ex:12a}) and Italian\il{Italian} (\ref{2ex:12b}):

\begin{exe}
\ex\label{2ex:12}
	\begin{xlista}
	\ex\label{2ex:12a}
	\gll	Buenos	vinos, 	(los) 				hay 				en	Castilla. \\
		good 	wines	({\sc{cl.m.pl}}) 	have.{\sc{prs.3sg}}	in 	Castile \\
	\glt 	`There are good wines in Castile.'
	\ex\label{2ex:12b}
	\gll	Libri 		in	inglese,	{(li/ne)} 						può 				trovare	al 	 	secondo	piano. \\
		books 	in 	English 	{({\sc{cl.m.pl}}/{\sc{cl.part}})}	can.{\sc{prs.3sg}}	find		on.the	second 	floor \\
	\glt	`English books can be found on the second floor.'
	\end{xlista}
\end{exe}

As suggested by \cite{leonetti:10}, non-specific or weak indefinites are highly restricted in topic position.\is{topic position} He identifies two conditions that must be met to allow for non-specific indefinites to appear as topics.\is{topic} First, they can be licensed by certain kinds of contrast\is{contrast} that cannot lead to a specific reading. This condition has to do with intonation\is{intonation} and stress,\is{stress} factors that fall outside the scope of the discussion in this paper. Second, they can be licensed in the sentential context with which the topic\is{topic} is linked. In other words, this second condition means that what matters for licensing non-specific indefinite topics\is{topic} is the presence of supporting context. In general, the examples in (\ref{2ex:12}) illustrate that the correlation between topic\is{topic} and definiteness and/or topic and referentiality\is{referentiality} is not a strict dependency but rather a strong tendency. 

For Russian, as well as for other languages with free word order, it is important to dissociate the effects that can be attributed to topicality\is{topicality} from those that can (potentially) arise merely from word order. In particular, the question that we address in this paper is whether the linear position of a subject, regardless of topichood, correlates with its definiteness or not. Therefore, our experimental items include preverbal subjects which are not topics,\is{topic} i.\,e., which do not appear in a sentence-initial position. It has been claimed in the literature that preverbal subjects that are not topics,\is{topic} for instance, preverbal subjects of thetic sentences,\is{thetic statements} can be both definite and indefinite (cf. \citealt{geist:10}, among others), but the results of our experiment suggest that there is nonetheless a strong dependency between linear position and interpretation. In particular, we will show that indefinites have a relatively low acceptance rate when they appear preverbally in non-topical contexts. Just like (weak) indefinite topics,\is{topic} preverbal non-topical indefinites seem to still need contextual support, so the conditions for licensing indefinites in preverbal position appear to be quite rigorous. Thus, the generalization seems to be that preverbal indefinites need special contextual conditions to facilitate their use, regardless of whether they are topics.\is{topic}


\section{The experimental study}\label{2sec:3}

The relationship between the syntactic position of a bare nominal\is{bare nominal} and its interpretation has been found in other languages (e.\,g., \citealt{cheng:sybesma:14}, for Mandarin Chinese);\il{Chinese!Mandarin} it has even been claimed that the pattern where the preverbal nominal is interpreted definitely and the postverbal nominal is interpreted indefinitely is universal \citep{leiss:07}. However, there have not been many experimental studies based on articleless languages to ascertain how speakers interpret bare nominal\is{bare nominal} subjects in preverbal and postverbal position. Some of the most relevant experimental studies that have been conducted for Slavic languages\il{Slavic languages} are discussed in the next subsection. The scarcity of experimental work concerning the interpretation of bare nominals\is{bare nominal} in Slavic languages\il{Slavic languages} in general and in Russian in particular motivated our study of Russian bare plural subjects.


\subsection{Previous experiments}\label{2sec:31}

The recent experimental studies on Slavic languages\il{Slavic languages} that we are aware of are the study of bare singular NPs in Czech\il{Czech} by \cite{simik:14}, a statistical analysis based on Polish\il{Polish} and English\il{English} texts by \cite{czardybon:hellwig:petersen:14} and \cite{simik:burianova:18}, who conducted a corpus study of bare nominals\is{bare nominal} found in pre- and postverbal position in Czech.\il{Czech} All the studies, even though methodologically different, show that there is a quite strong correlation between word order and the interpretation of nominal arguments.

\v{S}im\'ik's (\citeyear{simik:14}) experiment tested the preference for a definite or an indefinite reading of an NP in initial or final position in a sentence. The study demonstrated that the initial position (topicality)\is{topicality} of the subject increased the probability of a definite interpretation; however, it was not a sufficient force to ensure this type of reading. Even though the indefinite interpretations were selected less for NPs in initial position than in final position, they were still not excluded. Moreover, indefinite interpretations were overall preferred over definite ones.

{
A comparative study of Polish\il{Polish} translations of English original texts by \cite{czardybon:hellwig:petersen:14} aimed to provide a quantitative assessment of the interaction between word order and (in)definiteness in Polish.\il{Polish} The results of this quantitative evaluation support previous theories about the correlation between the verb-relative position and the interpretation of bare nominals:\is{bare nominal} preverbal position is strongly associated with definiteness and postverbal position is connected to the indefinite reading of an NP. Nevertheless, the study revealed quite a high number of preverbal indefinite NPs, which the authors were not expecting \citep[][147-148]{czardybon:hellwig:petersen:14}. However, as pointed out by \cite{simik:burianova:18}, \cite{czardybon:hellwig:petersen:14} did not distinguish between preverbal and sentence-initial position, which complicates the interpretation of their results considerably.
}

Some important and relevant findings concerning the relationship between definiteness of a nominal argument and its linear position in a sentence are reported in \cite{simik:burianova:18}, who conducted a corpus study and discovered that in Czech,\il{Czech} clause-initial position shows very high intolerance towards indefinite nominal phrases. \cite{simik:burianova:18} argue that definiteness of bare nominals\is{bare nominal} in Slavic\il{Slavic languages} is affected by an absolute (i.\,e., clause-initial vs. clause-final) but not a relative (i.\,e., preverbal vs. postverbal) position of this nominal in a clause. Our experimental findings, which will be described in the next section, seem to contradict this conclusion. In particular, we find that preverbal indefinites in non-initial position have much lower acceptability than postverbal ones. We therefore argue that preverbal indefinites need additional anchoring mechanisms to be activated, which would ensure their successful use in a given context. We will propose that this anchoring mechanism is D-linking,\is{discourse linking (D-linking)} a general discourse coherence principle that can be defined by a set of specific conditions.

All the studies reviewed in this section demonstrate that, at least to some extent, NPs with an indefinite interpretation do appear preverbally, where they are not generally expected. Our experiment will also confirm this result.

\newpage
\subsection{Overall characteristics of the experiment}\label{2sec:32}

{
This section provides a general description of the experimental study we conducted. As mentioned above, the aim of our study was to investigate the relationship between the interpretation of bare nominals\is{bare nominal} in Russian and their position in the sentence (preverbal or postverbal), which relies on the long-standing assumption that word order in articleless Slavic languages\il{Slavic languages} is one of the means of expressing (in)definiteness. The main goal of the experimental study was to see whether the claim that preverbal bare subjects are interpreted definitely, while postverbal bare subjects are interpreted indefinitely correlates with native speaker judgements.
}

Given that we limited our study to anaphoric definiteness,\is{definiteness!anaphoric} our initial hypothesis can be formulated as follows:

\begin{exe}
\ex\label{2ex:13}
{\emph{The preverbal position of the bare subject expresses definiteness (familiarity)\is{familiarity} and the postverbal position expresses indefiniteness (novelty).\is{novelty}}}
\end{exe}

In order to verify this initial hypothesis, a survey was created. The interpretation of bare subject NPs was examined using an Acceptability Judgement Test (AJT)\is{Acceptability Judgement Test} with a scale from 1 (not acceptable) to 4 (fully acceptable). The survey was administered online using the SurveyMonkey software. The items were presented to participants visually and acoustically, so as to avoid a possible change in the interpretation due to intonation,\is{intonation} as it has been claimed in the literature that if a preverbal noun receives a nuclear accent,\is{accent} it may be interpreted indefinitely (\citealt{pospelov:70}; \citealt{jasinskaja:14}; among others). Potentially, the effect of intonation\is{intonation} may be stronger than the word order restriction described above (see the initial hypothesis). Thus, we considered it important to exclude variation in judgement caused by intonation\is{intonation} and all the experimental items were recorded with the usual, most neutral intonation\is{intonation} contour used for statements in Russian (intonation\is{intonation} contour 1, cf. \citealt{bryzgunova:77}). This intonation\is{intonation} contour is characterized by a flat, level pitch before the stressed syllable of the intonational nucleus, i.\,e., the stressed syllable of the most informative word in a sentence. In our examples, the stress\is{stress} was always on the last word of the sentence.

A total of 120 anonymous participants took part in the survey. Demographic information about the participants was collected in a pre-survey sociological questionnaire. All participants claimed to be native Russian speakers; the gender distribution was 102 women, 17 men, one non-binary; the mean age (in years) was 36.59 (SD = 8.55); 91 participants claimed to have received a university degree in language-related fields.

{
The experimental items contained a bare subject nominal in a preverbal or postverbal position. All predicates were stage-level,\is{stage-level predicates} according to Carlson's (\citeyear{carlson:77}) classification, expressed by an intransitive verb. All subject nominals were plural for the sake of uniformity; however, we expected the effects found in the course of the experiment to be manifested in the case of singular nominals as well. The experimental sentences were presented in a brief situational context, which suggested either novelty\is{novelty} (associated with indefiniteness) or familiarity\is{familiarity} (associated with definiteness) of the subject. A total of 48 items were presented to participants: 16 definiteness-suggesting (8 preverbal and 8 postverbal) plus 16 indefiniteness-suggesting (8 preverbal and 8 postverbal) experimental scenarios, and 16 fillers. The average answer time was 22 minutes.
}

Below we provide some examples of our experimental items. 

\begin{exe}
\ex\label{2ex:14}
{\emph{Preverbal subject in an indefiniteness-suggesting context:}}
\exi{}
{
Ėto \v{z}e pustynja, ėto samaja nastoja\v{s}\v{c}aja pustynja. 
V ėtoj mestnosti do six por ne bylo ni odnogo \v{z}ivogo su\v{s}\v{c}estva. 
No na pro\v{s}loj nedele {\bf{pticy}} {\emph{prileteli}}}.\footnote{In order to make the examples easier to understand, the bare NP subject is in bold, while the verb is in italics. This marking does not reflect any stress\is{stress} pattern.}\\
`It's a desert, it's a real desert. In this area there has never been a living creature. But last week birds came (lit. {\bf{birds}} {\emph{came.flying}}).'

\ex\label{2ex:15}
{\emph{Postverbal subject in an indefiniteness-suggesting context:}}
\exi{}
\v{C}to-to strannoe stalo proisxodit' v na\v{s}ej kvartire. 
V kuxne vsegda bylo o\v{c}en' \v{c}isto, nikogda ne bylo ni odnogo nasekomogo. 
No nedelju nazad {\emph{obnaru\v{z}ilis'}} {\bf{tarakany}}. \\
`Something strange started happening in our flat. It has always been very clean in the kitchen, there has never been a single insect. But a week ago cockroaches were found (lit. {\emph{found.themselves}} {\bf{cockroaches}}).'

\ex\label{2ex:16}
{\emph{Preverbal subject in a definiteness-suggesting context:}}
\exi{}
Kogda Katja i Boris vernulis' iz otpuska, oni obnaru\v{z}ili, \v{c}to ix dom ograblen. 
Pervym delom Katja brosilas' v spal'nju i proverila seif. 
Ona uspokoilas'. 
{\bf{Dragocennosti}} {\emph{le\v{z}ali}} na meste. \\
`When Katja and Boris came back from holiday, they discovered that their house had been burgled. First of all, Katja rushed into the bedroom and checked the safe. She calmed down. The jewellery was still there (lit. {\bf{jewelleries}} {\emph{lay}} in place).'

\filbreak
\ex\label{2ex:17}
{\emph{Postverbal subject in a definiteness-suggesting context:}}
\exi{}
O\v{z}ivlenije spalo, publika potixon'ku potjanulas' domoj. 
``Po\v{c}emu vse uxodjat?'' -- sprosil Mi\v{s}a. 
``Gonki zakon\v{c}ilis'. V gara\v{z}i {\emph{vernulis'}} {\bf{ma\v{s}iny}}.'' \\
`The agitation decmidruled, the public slowly started going home. ``Why is everybody leaving?'' Misha asked. ``The race has finished. The cars have returned to their garages.'' ' (lit. to garages {\emph{returned}} {\bf{cars}}).
\end{exe}

In the following section we discuss the results of the experiment.

\subsection{General results}\label{2sec:33}

A total of 3,840 data points were collected (120 participants $\times$ 2 definiteness conditions [indefinite, definite] $\times$ 2 positions in which the NP appeared in the sentence with respect to the verb [preverbal, postverbal] $\times$ 8 scenarios). These responses were analyzed using a Linear Mixed Model\is{Linear Mixed Model} using the GLMM interface from IBM SPSS Statistics 24.

The Linear Mixed Model\is{Linear Mixed Model} was applied to the data. The model was defined with Participant as the subject structure and Situation $\times$ Position as the repeated measures structure (Covariance Type: Diagonal). The participants' perceived acceptability of the sentences was set as the dependent variable. The fixed factors were Definiteness, Position, and their interaction. Regarding the random factors, a random intercept was set for Participant, with a random slope over Position (Covariance Structure: Variance Components).

The two main effects were found to be significant: Definiteness, $\text{F}(1, 3829)$ = $44.700$, $\text{p} < .001$, such that indefinite sentences obtained significantly more acceptability than definite sentences (diff = .164, SE = .024, p $<$ .001), and Position, F(1, 3829) = 14.236, p $<$ .001, indicating that preverbal NPs obtained more acceptability than postverbal NPs (diff = .113, SE = .030, p $<$ .001).

{
The interaction Definiteness $\times$ Position was found to be significant, $\text{F}(1, 3829)$ = $4958.853$, $\text{p} < .001$, which could be interpreted in the following two ways. On the one hand, in preverbal position definites were more adequate than indefinites (diff = \minus 1.561, SE = .035, p $<$ .001), and in postverbal position indefinites were more adequate than definites (diff = 1.888, SE = .034, p $<$ .001). On the other hand, indefinites were found to be more adequate in postverbal rather than in preverbal position (diff = \minus 1.612, SE = .037, p $<$ .001), while definites were found to be more adequate in preverbal rather than in postverbal position (diff = 1.837, SE = .040, $\text{p} < .001$). Figure \ref{2fig:1} shows the mean perceived acceptability that the participants ascribed to the experimental items on the 4-point Likert scale,\is{Likert scale} from 1 (not acceptable) to 4 (fully acceptable).
}

\begin{figure}[H]
\centering
\includegraphics[width=10.5cm]{chapters/borikFig1.png}
\caption{Average perceived acceptability that our participants attributed to the experimental sentences. Error bars depict the 95\% confidence interval.}\label{2fig:1}
\end{figure}

{
The most perceptible result seen from the graph is that the participants favoured two out of the four possible combinations of Definiteness and Position, i.\,e., postverbal subjects in indefiniteness-suggesting contexts (M = 3.399, SD = .791) and preverbal subjects in definiteness-suggesting contexts (M = 3.289, $\text{SD} = .874$), giving substantially lower ratings to preverbal subjects in indefiniteness-suggesting contexts (M = 1.831, SD = .885) and postverbal subjects in definiteness-suggesting contexts (M = 1.657, SD = .932).
}

Besides the optimal combinations (preverbal NP + definiteness-suggesting context and postverbal NP + indefiniteness-suggesting context), additional statistically significant results were obtained. Firstly, an overall superior acceptability for NPs in indefiniteness-suggesting contexts (regardless of the syntactic position of the NP) as compared to definiteness-suggesting ones was observed. Secondly, the acceptability of bare nominals\is{bare nominal} in preverbal position was higher compared to the postverbal position, regardless of type of preceding context.

%{
The results, in our view, can be interpreted in the following way. First of all, there is quite a strong preference for interpreting preverbal NPs definitely and postverbal NPs indefinitely. However, there is no clear one-to-one correspondence, which suggests that the linear position of a subject nominal in Russian cannot be considered a means of expressing its definiteness/indefiniteness. So, our initial hypothesis has to be modified. Instead of saying that the word order encodes the referential status\is{referential status} of a nominal (i.\,e., its definiteness or indefiniteness), we think the results only show that preverbal nominal subjects are {\emph{much more likely}} to be interpreted as definites. Indefinites are not rare in this position either and their acceptability is fairly high, so our next question is what the factors are that influence speakers' judgements in the case of preverbal indefinites. 
%}

In an attempt to answer this question we looked at our preverbal definiteness \hspace*{-0.33em}- and indefiniteness-suggesting contexts one by one and tried to analyze every preverbal context that we had used in our experimental study. The results that we obtained are reported in the next section.

\subsection{Item-by-item analysis of preverbal subjects}\label{2sec:34}

One of the main theoretical questions that we try to answer in this paper is what makes it possible for a particular nominal to appear in a preverbal subject position. In search for a possible answer, we looked at the information status\is{information status} of the subject NPs in the experimental sentences. \cite{baumann:riester:12} claim that, for an adequate analysis of the information status\is{information status} of a nominal expression occurring in natural discourse, it is important to investigate two levels of givenness:\is{givenness} referential\is{givenness!referential givenness} and lexical.\is{givenness!lexical givenness} The authors propose a two-level annotation scheme for the analysis of an NP's information status:\is{information status} {\emph{the RefLex}} scheme.\is{RefLex scheme} In this article we adopt this scheme in order to investigate the correlation between acceptability of an item in preverbal position and its information status.\is{information status}

\subsubsection{Definiteness-suggesting contexts}\label{2sec:341}

In definiteness-suggesting contexts, the subject NPs can be labelled, according to Baumann \& Riester's RefLex scheme\is{RefLex scheme} (\citeyear{baumann:riester:12}: 14), as {\emph{r-given}} or {\emph{r-bridging}} at a referential level. The {\emph{r-given}} label is used when the anaphor co-refers with the antecedent in the previous discourse. {\emph{R-bridging}} is assigned when the anaphor does not co-refer with an antecedent but rather depends on the previously introduced scenario. At a lexical level, the items can be classified \citep[][18-19]{baumann:riester:12} as {\emph{l-given-syn}} (the nouns are at the same hierarchical level, i.\,e., synonyms), {\emph{l-given-super}} (the noun is lexically superordinate to the nominal antecedent), {\emph{l-accessible-sub}} (the noun is lexically subordinate to the nominal antecedent) or {\emph{l-accessible-other}} (two related nouns, whose hierarchical lexical relation cannot be clearly determined).

Table \ref{2table:1} represents the experimental scenarios with definiteness-suggesting contexts. It provides the anchor nominal from the previous context, the target nominal (the preverbal subject NP from the experimental sentence), the RefLex\is{RefLex scheme} labels of the target nominal, the mean acceptability given (M; in a 0 to 1 scale)\footnote{The original acceptability variable was changed from 1-4 to 0-1 for reasons of clarity. This change was the result of the following formula: (acceptability -- 1)/3. We consider it to be easier to interpret what an acceptability score of .4 on a 0-1 scale represents than the equivalent score of 2.2 on a 1-4 scale, which might be misconceived as if it was a 0-4 scale, thus indicating more than a half of accepted readings.} and the standard deviation (SD) acceptability figures for each item.

\begin{table}[H]

{
{\small{
\begin{tabular}{m{1pt}llm{112pt}m{17pt}m{18pt}}
\lsptoprule
 & {\bf{Previous context}} & {\bf{Target nominal}} & {\bf{RefLex annotation}} & {\bf{M}} & {\bf{SD}} \\
 \midrule
1 & boy and girl\footnote{We are not using articles here as, naturally, they are absent in the Russian examples.}  & children & r-given, l-given-syn & .8333 & .2520 \\ 
2 & family of tigers & animals & r-given, l-given-super & .7750 & .2806 \\ %\hline
3 & safe & jewellery & r-bridging, l-accessible other & .8833 & .2102 \\ %\hline
4 & canary and parrot & birds & r-given, l-given-super & .8418 & .2166 \\ %\hline
5 & crucians & fishes & r-given, l-given-super & .6863 & .3016 \\ %\hline
6 & family silverware & cutlery & r-given, l-given-syn & .7583 & .2930 \\ %\hline
7 & Plato and Aristotle & philosophers & r-given, l-given-super & .5972 & .3372 \\ %\hline
8 & races & cars & r-bridging, l-accessible-sub & .7306 & .3155 \\ %\hline
\lspbottomrule
\end{tabular}
}}
}
\caption{{Annotation of target nominals in definiteness-suggesting contexts}}\label{2table:1}
\end{table}
\vspace*{-3mm}

As can be seen from Table \ref{2table:1}, the acceptability of preverbal nominals in definiteness-suggesting contexts is quite high and fairly uniform. This is an expected result as preverbal position is strongly related with familiarity/identifiability\is{familiarity}\is{identifiability} of the referent and the degree of givenness,\is{givenness} which is high in all cases (as can be seen from the labels). So, it is natural for NPs to appear preverbally in definiteness-suggesting contexts, when they are anaphorically or situationally related to an antecedent in a previous context.

The item with the lowest (although still high, in absolute terms) acceptability is 7, given in (\ref{2ex:18}):

\begin{samepage}
\begin{exe}
\ex\label{2ex:18}
Sredi mnogo\v{c}islennyx anti\v{c}nyx prosvetitelej, otmetiv\v{s}ixsja v istorii, mo\v{z}no vydelit' neskol'ko naibolee va\v{z}nyx. 
Platon i Aristotel' izvestny vo vs\"em mire. {\bf{Filosofy}} {\emph{\v{z}ili}} v Drevnej Grecii. \\
`Among numerous classical thinkers that left their trace in the history it is possible to distinguish a few most important ones. Plato and Aristotle are known all over the world. The philosophers lived in Ancient Greece (lit. {\bf{philosophers}} {\emph{lived}} in Ancient Greece).'
\end{exe}
\end{samepage}

In terms of its information status,\is{information status} the bare nominal\is{bare nominal} subject {\emph{philosophers}} is not really different from the subjects of other items: it is {\emph{r-given}}. A lower acceptability rate must then be due to other factors, e.\,g., the use of proper names or attributing a generic type of interpretation\is{generic sentence} to the last sentence (i.\,e., `In general, philosophers lived...'), which would cancel the anaphoric connection.\footnote{It is interesting to note that a similar effect has been observed for an item with a postverbal subject in a definiteness-suggesting context. While other items with definite postverbal subjects were given low acceptability as expected, the acceptability of this particular item was quite high (M = .4667, SD = .3441). The English translation of the item is given in (i):
\vspace*{-1mm}
\begin{exe}
\exi{(i)}
I love birds and I advise all my friends to have at least one feathered pet. They are generally undemanding, although sometimes they make noises and give you extra trouble. At home I have {\emph{lit.}} {\bf{canary and parrot}}. Yesterday I forgot to close the cage's door, and all day long {\emph{lit.}} around room {\emph{flew}} {\bf{birds.}}
\end{exe}
\vspace*{-1mm}
\noindent
Our hypothesis is that the informants processed the antecedent and the anaphor NPs in this example as non-co-referential, therefore interpreting the target NP as referentially new, which made it possible for the subject to be accepted in postverbal position.
}

Thus, apart from one item discussed above (item 7), all the definiteness-suggesting contexts show the same result: a high acceptability rate for the preverbal bare nominal\is{bare nominal} subject. 

\subsubsection{Indefiniteness-suggesting contexts}\label{2sec:342}

In all indefiniteness-suggesting contexts, the existence of referents was negated; thus, the novelty\is{novelty} of the target nominal was presupposed. Using Baumann \& Riester's (\citeyear{baumann:riester:12}: 14) annotation scheme, at a referential level all the target NPs are classified as {\emph{r-new}}, i.\,e., specific or existential indefinites introducing a new referent. At a lexical level, they are either {\emph{l-accessible-sub}} or {\emph{l-accessible-other}}. Table \ref{2table:2} presents the annotation results for bare nominals\is{bare nominal} in preverbal position in indefiniteness-suggesting contexts.

\begin{table}[H]
{\small{
\begin{tabular}{m{1pt}lllm{20pt}m{20pt}}
\lsptoprule
 & Previous context & Target nominal & RefLex annotation & M & SD \\ 
\midrule
1 & no rodents & mice & r-new, l-accessible-sub & .3833 & .3195 \\ 
2 & no insects & cockroaches & r-new, l-accessible-sub & .2167 & .2755 \\ 
3 & empty street & people & r-new, l-accessible-other & .1750 & .2766\\ 
4 & no fruit & bananas & r-new, l-accessible-sub & .2778 & .2778\\ 
5 & no living creatures & birds & r-new, l-accessible-sub & .2861 & .2940\\ 
6 & no mail & postcards & r-new, l-accessible-sub & .3056	 & .3163\\ 
7 & no domestic animals & cats & r-new, l-accessible-sub & .3194 & .2847\\ 
8 & no wild animals & wild boar & r-new, l-accessible-sub & .2521 & .2709\\
\lspbottomrule
\end{tabular}
}}
\caption{Annotation of target nominals in indefiniteness-suggesting contexts}\label{2table:2}
\end{table}


As can be seen from Table \ref{2table:2}, the acceptability of preverbal NPs in indefinite-ness-suggesting contexts is uniformly low, but high enough to be statistically significant (see Section \ref{2sec:32}). All these NPs are referentially new. However, it should be pointed out that at a lexical level, the target nominals in indefiniteness-suggesting contexts are accessible, being a subset of a superset mentioned in the previous context.

The item that received the lowest ranking in Table \ref{2table:2} is item 3 (M = .1750, SD= .2766), which has a slightly different information status\is{information status} label at a lexical level. It has an {\emph{l-accessible-other}} label, which means that, unlike other items with a clear lexical relation of hyponymy,\is{hyponymy} the hierarchical relation between the context and the target NP cannot be clearly established in the given scenario. Item 3 is provided in (\ref{2ex:19}):

\begin{exe}
\ex\label{2ex:19}
Bystro stemnelo, nastupil ve\v{c}er. Na ulice bylo tixo i pustynno. Vdrug iz-za ugla {\bf{ljudi}} vy\v{s}li. \\
`It got darker, the night came very quickly. 
{\emph{Lit.}} In the street it was silent and empty. Suddenly from around the corner {\emph{lit.}} {\bf{people}} came out.'
\end{exe}

As opposed to other experimental scenarios, in this context there is no NP to which the target nominal {\emph{ljudi}} `people' could be anchored. Even though it can be linked to the whole context, given our common knowledge that people usually walk in the streets, this vague type of contextual support does not seem to be enough to `license'\footnote{We use the term `license' here in a loose sense, without appealing to anything as strict as `licensing conditions', the way they are understood in syntax.} the bare nominal\is{bare nominal} {\emph{ljudi}} `people' to appear in preverbal position. Even though it is just one example, we believe that the lower acceptability rate of this example might not be accidental. In the next section we will discuss the factors that could make this sentence different from the other experimental items in the same group.


\section{Some theoretical considerations}\label{2sec:4}

We begin this final section of the paper by suggesting a tentative answer to the question posed in the previous sections: what are the conditions that bare indefinites have to meet to be accepted in preverbal position? Having analyzed the data in all preverbal contexts, we can propose a plausible hypothesis as an answer to this question, although the validity of this hypothesis should be further confirmed in future empirical and experimental studies.

{
An item-by-item analysis of our experimental scenarios suggests that if an item is {\emph{r-given}}, it has a tendency to appear preverbally, and this combination (i.\,e., {\emph{r}}-givenness\is{givenness!referential givenness} and preverbal position) is judged highly acceptable by native speakers of Russian. This is illustrated by the item-by-item analysis of our definiteness-suggesting contexts. If, however, a nominal is {\emph{r-new}}, it is judged much less acceptable in preverbal position, even though it is still tolerable: the acceptability rate for these items was about 1.8 on a 4-point scale, as we saw in Section \ref{2sec:33}, where the general results of the experiments were discussed. What our data seems to indicate is that it is not only referential givenness\is{givenness!referential givenness} but also accessibility\is{accessibility} at a lexical level that plays a significant role in licensing bare nominals\is{bare nominal} in preverbal position. Thus, if a bare nominal\is{bare nominal} is {\emph{r-new}}, it can still appear preverbally in those cases where it establishes a clear lexical connection with a nominal phrase in the previous context. However, in the example where the connection between the previous context and the target item is looser and the item can only be classified as {\emph{l-accessible-other}} (i.\,e., a target nominal can only be pragmatically related to the whole context), the acceptability rate drops and the item is judged close to unacceptable.
}

It might be too early to draw any far-reaching theoretical conclusions on the basis of just one experiment with 16 test items. However, we believe that the results we obtained in our experimental study for preverbal bare nominals\is{bare nominal} in Russian at least allow us to identify some conditions that seem to facilitate the use of bare nominal\is{bare nominal} phrases in indefiniteness contexts in preverbal position in Russian: {\emph{r}}-givenness\is{givenness!referential givenness} and {\emph{l}}-accessibility. We would like to suggest that these conditions could be connected with a much broader phenomenon, which might serve as a general explanation for a reduced and restricted, but still accepted, appearance of indefinite nominal phrases in preverbal position. The phenomenon that we refer to is called D-linking.\is{discourse linking (D-linking)}

\cite{pesetsky:87} described discourse linking (or D-linking)\is{discourse linking (D-linking)} as a phenomenon where one constituent is anchored to another one in the preceding discourse or extralinguistic context. \cite[][73]{dyakonova:09}, building on this idea, gives the following definition of D-linking:\is{discourse linking (D-linking)}

\begin{exe}
\ex\label{2ex:20}
{
A constituent is D-linked\is{discourse linking (D-linking)} if it has been explicitly mentioned in the previous discourse, is situationally given by being physically present at the moment of communication, or can be easily inferred from the context by being in the set relation with some other entity or event figuring in the preceding discourse.
}
\end{exe}

{
As can be seen from this definition, D-linking\is{discourse linking (D-linking)} is a rather broad phenomenon that allows for various connections to be established between a constituent X and the preceding discourse or a situational context. We suggest that this general phenomenon could be split into a set of specific conditions that would allow us to achieve a more precise characterisation of D-linking.\is{discourse linking (D-linking)}
}

As was pointed out in Section \ref{2sec:22}, discourse support seems to play a role in licensing non-specific (weak) indefinite nominals in topic position\is{topic position} in Romance languages.\il{Romance languages} For instance, \cite{leonetti:10} identifies two conditions under which non-specific indefinites appear as topics\is{topic} in Romance languages:\il{Romance languages} contrast\is{contrast} and contextual support. We suspect that the latter could fit into what we describe as D-linking\is{discourse linking (D-linking)} although the precise characterization of what it means to be contextually supported is yet to be established.

Our experiment has shown that native speakers of Russian, a language which does not encode (in)definiteness by any grammatical means, can accept an indefinite interpretation of a bare nominal\is{bare nominal} in preverbal position, even though a general acceptability rate for preverbal indefinites is much lower than for preverbal definites. What we have suggested in this paper is that for indefiniteness contexts, not only referential, but also lexical linking to a previous nominal element can play a role. Those nominals that were strongly supported by the previous contexts by lexical relations such as hyponymy/hyperonymy\is{hyperonymy}\is{hyponymy} are judged more acceptable than those which do not have this type of support. This, of course, is almost directly captured by the definition of D-linking\is{discourse linking (D-linking)} given in (\ref{2ex:20}): in all our test sentences, the preverbal nominals in indefiniteness-suggesting contexts were lexically accessible, as they were in the set relation with an entity from the preceding context, except for item 3 (which obtained the lowest acceptability). This fact indicates that it would be plausible to explore the role of D-linking\is{discourse linking (D-linking)} principle(s) for a general account of the distribution of bare nominals\is{bare nominal} with indefinite readings in Russian.

\filbreak
To conclude this section, we would like to raise another theoretical issue that has come to our attention, both while conducting the experiment that we have reported on here and in the course of the interpretation of the results. It concerns the notion of definiteness and the status of our test items with respect to this category.

As we pointed out in the introduction, the debate on what definiteness actually means in semantic terms continues, although here we follow \cite{farkas:02} in assuming that the familiarity\is{familiarity} and the uniqueness\is{uniqueness} approaches to definiteness could converge. In the experiment reported on here, in the definiteness-suggesting contexts, we test nominal phrases that are anaphorically linked to a referent in the preceding discourse, and we consider our experimental items to be fully legitimate candidates for definite nominals in the definiteness contexts because they are familiar to the speaker. However, all our anaphoric nominals in definiteness-suggesting contexts are also given, so the question that presents itself is whether the results of the experiment are influenced by the givenness\is{givenness} status of the tested items.\footnote{We thank a reviewer for bringing up this question and for suggesting that we look into the role of givenness\is{givenness} in the distribution of nominal arguments.}

Givenness\is{givenness} is a category related, although not equivalent, to definiteness. An element is given if there is an antecedent for it in the preceding discourse, so givenness\is{givenness} is an information-structural category that is also closely related to anaphoricity.\is{anaphoricity} Any constituent of a sentence can have the status of `given', including, of course, nominal arguments.

The relationship between definiteness and givenness\is{givenness} is not straightforward: in principle, both definite and indefinite arguments can be either given or new. For instance, any contextually unique definite mentioned for the first time is not given but new (e.\,g., {\emph{The UV is very high today, The head of the department just called me}}), whereas any anaphoric definite is given. Crucially, however, the given/new status seem to correlate with stress:\is{stress} deaccentuation and word order are common ways to indicate givenness\is{givenness} of a certain constituent \citep[cf.][]{krifka:08}.\footnote{Although see \cite{rochemont:16} for the claim that only salient-based givenness\is{givenness} is associated with deaccenting.}  As for Slavic languages,\il{Slavic languages} \cite{simik:wierzba:15} present a thorough study of the interaction between givenness,\is{givenness} word order and stress\is{stress} in Czech.\il{Czech}

{
As we have already mentioned, in our experiment we tried to eliminate the stress\is{stress} factor, by recording all our example sentences with a neutral intonation,\is{intonation} flat pitch, and a phrasal stress\is{stress} at the end of a sentence. Hence, all preverbal items (in definiteness and indefiniteness contexts) were unstressed and postverbal items (in definiteness and indefiniteness contexts) were stressed only when they also appeared in a sentence-final position. It might be that givenness\is{givenness} is the factor that influenced the acceptability judgements in our experiment, especially in the case of nominals in definiteness contexts, because the speakers might have been less willing to accept a postverbal stressed given nominal, since the nominal appeared in sentence-final position. However, to properly answer this question we need to design an experiment with postverbal definites that appear in sentence-final and non-final position, and also manipulate stress.\is{stress} Stress\is{stress} might be particularly important for indefinite nominals, as it has been noted in the literature that stressed indefinites become more acceptable in preverbal position. All in all, we think that studying the role of givenness\is{givenness} versus definiteness in the distribution of bare nominal\is{bare nominal} arguments is an exciting task for a (near) future project.
}

\section{Conclusions}\label{2sec:5}
In this paper, we have discussed the relationship between the definiteness status of a bare nominal\is{bare nominal} and its linear position in a sentence in Russian. We have confirmed that, according to the results of the experiment that we conducted with native speakers of Russian,\is{Russian|)} the general tendency is, indeed, to associate preverbal position with a definite interpretation and postverbal with an indefinite one, although it cannot be stated that this connection is a strict correspondence. Consequently, we cannot say that linear position `encodes' definiteness or indefiniteness:  the observed correlations are tendencies rather than strict rules. 

One of the other results of our experiment is the reasonably high ranking that is assigned to bare nominals\is{bare nominal} with an indefinite interpretation that appear in preverbal position. We carried out an item-by-item analysis of all the preverbal nominals with the aim of identifying a specific condition or a set of specific conditions that would make indefinite nouns acceptable in this position. Our conclusion was that the condition has to do with the level of accessibility\is{accessibility} of a target noun at a lexical level: if a (subset) lexical relation can be established between a target noun and its antecedent, the acceptability rate of the target noun in preverbal position increases. We link this condition to a more general principle of D-linking,\is{discourse linking (D-linking)} which, by hypothesis, is the same principle that can be used to explain various exceptional occurrences of weak indefinites in topic position.\is{topic position} Thus, we suggest that D-linking\is{discourse linking (D-linking)} principles facilitate the use of indefinite nominals in preverbal position, whether they are topics\is{topic} or not.


\section*{Acknowledgments}
{
This study has been supported by three grants: a grant from the Spanish MINECO FFI2017-82547-P (all authors), an ICREA Academia fellowship awarded to M. Teresa Espinal and a grant from the Generalitat de Catalunya (2017SGR634) awarded to the Centre for Theoretical Linguistics of UAB (the second and the third author).
}


{\sloppy\printbibliography[heading=subbibliography,notkeyword=this]}
\end{document}
