\documentclass[output=paper]{langsci/langscibook}
\ChapterDOI{10.5281/zenodo.4049679}

\author{Walter Bisang\affiliation{Johannes Gutenberg University, Mainz \& Zhejiang University}\lastand Kim Ngoc Quang\affiliation{Johannes Gutenberg University, Mainz \& University of Social Sciences and Humanities, Ho Chi Minh City} }

\title{(In)definiteness and Vietnamese classifiers}  

\abstract{Vietnamese \il{Vietnamese|(} numeral classifiers\is{classifiers!numeral classifiers} (CL) in the bare classifier\is{classifiers|(} construction [CL+N] can be interpreted as definite and as indefinite. Based on a corpus of written and oral texts with a broad range of different contexts for the potential use of classifiers, this paper aims at a better understanding of the factors and linguistic contexts which determine the use of the classifier in [CL+N] and its specific functions. The following results will be presented: (a) Even though classifiers tend to be interpreted as definite, they are also used as indefinites, irrespective of word order (subject/preverbal or object/postverbal). (b) There is a strong tendency to use the [CL+N] construction with definite animate nouns in the subject position, while bare nouns [N] preferably occur with indefinite inanimate nouns in the object position. (c) The vast majority of nouns occurring with a classifier are sortal nouns\is{noun types!sortal noun} with the features [\minus unique, \minus relational]. (d) Discourse and information structure\is{information structure} are the most prominent factors which determine the grammar of Vietnamese classifiers. The influence of discourse is reflected in the pragmatic definiteness\is{definiteness!pragmatic} expressed by the classifier. Moreover, information structure\is{information structure} enhances the use of a classifier in contexts of contrastive topic,\is{topic!contrastive topic} contrastive focus\is{focus!contrastive focus} and focus particles.\is{focus!focus particles} Finally, thetic statements\is{thetic statements} and some special constructions (existential clauses,\is{existential sentences} verbs and situations of appearance) provide the environment for the indefinite interpretation of classifiers.}

\begin{document}
\maketitle


\section{Introduction}\label{1sec:1}

Numeral classifiers\is{classifiers!numeral classifiers} are an areal characteristic of East \il{East Asian languages} and mainland Southeast Asian languages \il{Southeast Asian languages} in the context of counting. This fact is well known and has been frequently discussed in the literature since the 1970s \citep{greenberg:72}. What is less well known and has been discussed only in more recent times is the use of the same classifiers in the contexts of definiteness and indefiniteness, when they occur in the [CL+N] construction 
\citep[bare classifier construction; cf.][]{bisang:99,cheng:sybesma:99,simpson:05,wu:bodomo:09,li:bisang:12,jiang:15,simpson:17,bisang:wu:17}. 
In Vietnamese, classifiers in [CL+N] are clearly associated with reference. What is controversial in the literature is the question of whether they are used only in the context of definiteness or in contexts of definiteness and indefiniteness. \citet{tran:11} claims that classifiers only have a definite interpretation, while \citet{nguyen:04} argues for both interpretations \citep[see also][]{trinh:11}. A look at an example from \citet{nguyen:04} in (\ref{1ex:nguyen}) shows that both interpretations are possible. In this respect, it differs significantly from many \ili{Sinitic languages} with [CL+N] constructions. While the definiteness/indefiniteness interpretation of classifiers depends on the preverbal or postverbal position of the [CL+N] construction in most of these languages,\footnote{In Wang's (\citeyear{wang:15}) survey of Sinitic \il{Sinitic languages} classifiers as markers of reference, the definiteness/indefiniteness distinction is independent of word order relative to the verb in only 10 out of his 120 sample languages (cf. Type I classifiers in his terminology).} Vietnamese classifiers can have both interpretations in both positions. In (\ref{1ex:nguyen.a}), {\emph{con b\`o}} [CL cow] is in the subject position and is open to both interpretations (`the cow'/`a cow'). Similarly, {\emph{cu{\dao}n s\'ach}}  [CL book] in the object position of (\ref{1ex:nguyen.b}) can be definite as well as indefinite (`the book'/`a book'):

\begin{exe}
\ex\label{1ex:nguyen}
\citet{nguyen:04}:
	\begin{xlista}
	\ex\label{1ex:nguyen.a}
	\gll Con b\`o \u{a}n l\'ua k\`ia!\\
  	{\textsc{cl}} cow eat paddy {\textsc{sfp}}\\
   	\glt `Look! A/the cow is eating your paddy!'
	\ex\label{1ex:nguyen.b}
	\gll Mang cu{\dao}n s\'ach ra \textcrd\^ay!\\
	bring {\textsc{cl}} book out here\\
   	\glt `Get a/the book!'
	\end{xlista}
\end{exe}

As can be seen from the following example, nouns without a classifier (bare nouns) can also be interpreted in both ways in both positions. In Nguyen's (\citeyear{nguyen:04}) analysis, the only difference between the bare noun construction and the [CL+N] construction is that the former can be interpreted as singular or plural, while the latter can only have a singular reading:

\begin{exe}
\ex\label{1ex:nguyen2}
\citet{nguyen:04}:
	\begin{xlista}
	\ex
	\gll B\`o \u{a}n l\'ua k\`ia!\\
   	cow eat paddy {\textsc{sfp}}\\
   	\glt `Look! A/the cow(s) is/are eating your paddy!'
	\ex
	\gll Mang s\'ach ra \textcrd\^ay!\\
	bring book out here\\
   	\glt `Get a/the book(s), will you?'
	\end{xlista}
\end{exe}

Even though these examples show that classifier use is not obligatory and that classifiers can be interpreted as definite as well as indefinite, neither the conditions under which classifiers have these functions nor their specific referential meaning are well understood. Analyzing Vietnamese classifiers in the [CL+N] construction as variables whose interpretation depends on semantic, syntactic and discourse-pragmatic contexts, it is the aim of this paper to define the contexts which determine their use in terms of obligatoriness and their interpretation as definite and indefinite. Since the use of the classifier in the [CL+N] construction and its interpretation in terms of (in)definiteness in Vietnamese strongly depends on discourse and information structure,\is{information structure} as in many other East \il{East Asian languages} and mainland Southeast Asian languages,\il{Southeast Asian languages} looking at individual examples in isolation is not sufficient for modeling the function and the use of the [CL+N] construction. What is needed are texts, both written and oral. For that reason, we decided to set up our own corpus of Vietnamese, which is based on written and oral reports, by native speakers of Vietnamese, on the content of two silent movies (for details, cf. Section \ref{1sec:2}).

{
The analysis of the data from our Vietnamese corpus confirms the general observation that classifiers in [CL+N] can be interpreted as definite as well as indefinite, irrespective of word order. It also shows that the interpretation of numeral classifiers\is{classifiers!numeral classifiers} in terms of definiteness and indefiniteness in [CL+N] depends on semantic and syntactic (preverbal/postverbal or subject/object) factors, as well as on discourse and information structure.\is{information structure} In addition to that, it turns out that the definite function is much more frequent than the indefinite function. Instances of indefinite [CL+N] constructions mainly occur in special contexts and constructions, such as thetic statements,\is{thetic statements} existential clauses,\is{existential sentences} and constructions with verbs which introduce previously unidentified referents into discourse (i.\,e., verbs of appearance). Given the relative rareness and the functional specifics of indefinite classifiers, it may not come as a surprise that the indefinite interpretations of classifiers remained unnoticed in a number of studies of Vietnamese classifiers.
}

To find out more about the function of classifiers and the factors that determine their use, the following criteria will be studied in more detail:

\begin{itemize}
\setlength\itemsep{-0.1em}
\item Definiteness and indefiniteness of the nominal expression
\item The semantic feature of [\pm animate] of the noun
\item The semantic features of [\pm unique] and [\pm relational] in terms of L\"obner's (\citeyear{lobner:85,lobner:11}) four basic types of nouns
\item The syntactic criterion of word order (position of the noun in the subject/ preverbal or object/postverbal position)
\item The role of discourse and the relevance of identifiability\is{identifiability}
\item Information structure\is{information structure} and the use of a classifier (contrastive topics,\is{topic!contrastive topic} certain types of focus)\is{focus} as well as its function (i.\,e., theticity and indefiniteness)
\item The combination with specific verbs (i.\,e., existential verbs and verbs of appearance).
\end{itemize}

{
The structure of the paper is as follows: after the discussion of methodological issues in Section \ref{1sec:2}, Section \ref{1sec:3} will describe classifiers in their definite functions and the criteria that determine their use. Section 4 will do the same with classifiers in their indefinite function. The conclusion in Section 5 will briefly summarize our findings and situate them with regard to other languages with numeral classifiers\is{classifiers!numeral classifiers} that are used in contexts of definiteness as well as indefiniteness.
}

\section{Methodology}\label{1sec:2}

Our analysis of the function and the use of classifiers in the [CL+N] construction is based on a Vietnamese corpus of 30 written texts and 30 oral texts produced by native speakers of Vietnamese who were asked to report on the content of two films which were previously presented to them on the screen of a personal computer. One of the films was used to create a written corpus, the other an oral corpus. The total number of informants involved was 46 (25 female and 21 male informants). Fourteen informants (five female and nine male) from among these 46 informants participated in both experiments and thus produced a written and an oral text.\footnote{Since it was more difficult to find male informants, we had to ask more males to take part in both experiments. Six of the remaining 12 male informants only produced a written text, while the other six only were involved in the oral experiment.} In total, there were 15 male and 15 female informants, as well as 15 graduate and 15 undergraduate informants for each experiment. The reason for this arrangement was to check for potential effects originating from differences in gender or modality (written vs. oral). Since we did not find any significant differences, we will not address this issue in the present paper.

The experiments were carried out by Kim Ngoc Quang in Ho Chi Minh city (Southern Vietnam) with the support of assistants who played the role of addressees (readers/listeners). This arrangement was necessary to avoid speaker assumptions about information shared with the addressee. Thus, the informants reported their stories in a situation in which it was clear that the addressee did not know the story.

For the purpose of our study, we needed two films with multiple protagonists, frequently changing scenes with different perspectives and a large number of animate and inanimate objects involved in a variety of actions expressed by transitive and intransitive verbs. The first film, with the title `Cook, Papa, Cook', is a silent movie of nine minutes and 38 seconds in length.\footnote{The film can be seen on YouTube at {\url{https:://www.youtube.com/watch?v=OITJxh51z3Q}}.} This very lively film, which was used to create the written corpus, has three protagonists: a husband, a wife and their son. The story is characterized by intense quarrels between the husband and his wife. Because of this, the wife decides that she is no longer prepared to make breakfast for her husband. His attempts to make it himself are met by a number of obstacles and end up turning the kitchen into a total mess. When he finally manages to make his own kind of breakfast, his wife refuses to eat it.

The second film, which was used to set up the oral corpus, is from the `Pear Stories' \citep{chafe:80}.\footnote{The film can be seen on YouTube at {\url{https:://www.youtube.com/watch?v=bRNSTxTpG7U}}.} It is five minutes and 54 seconds long. It has two protagonists: a farmer and a young boy, who steals the farmer's pears from some baskets, while the farmer is up a tree picking the rest of the pears. When cycling away from the farmer, he inadvertently rides over a stone because he is distracted by a girl cycling in the opposite direction. As a consequence, the pears roll out of the basket and scatter all over the road. Three other boys arrive and help the boy to pick up the pears. As a reward for their help, the boy offers them each a pear. Later on, the three boys walk past the farmer while eating their pears. The film ends with the farmer trying to understand what has happened.

The length of the 30 written texts varies between 491 and 1,944 words. The written corpus as a whole consists of 31,663 words. The total length of the oral corpus is 17,777 words, after transcription. The length of the 30 oral texts varies between 321 and 1,061 words.

\largerpage
In this paper, the two corpora are employed as sources of examples of a broad range of different classifier functions and different conditions responsible for their occurrence. Moreover, the data from these corpora are used for some generalizations about frequency, as far as that is possible on the basis of calculating simple percentages.


\section{Classifiers and definiteness}\label{1sec:3}

This section examines the correlation between classifiers with a definite interpretation in the [CL+N] construction\footnote{Notice that we do not discuss instances of [NUM CL N] with numerals > 1 because we do not have enough data in our corpus.} from various perspectives. \sectref{1sec:31} discusses the semantic feature of animacy\is{animacy} and its interaction with definiteness. An examination of the semantic features of uniqueness\is{uniqueness} and relationality\is{relationality} in \sectref{1sec:32} shows that the vast majority of nouns occurring with a classifier are sortal nouns,\is{noun types!sortal noun} defined by their features of [\minus unique]/[\minus relational]. The interaction of word order (preverbal/subject and postverbal/object) with animacy\is{animacy} and definiteness is explored in \sectref{1sec:33}. Finally, the roles of discourse (identifiability)\is{identifiability} and information structure\is{information structure} (contrastive topics,\is{topic!contrastive topic} focus particles\is{focus!focus particles} and contrastive focus)\is{focus!contrastive focus} are discussed in \sectref{1sec:34}.


\subsection{Animacy and definiteness}\label{1sec:31}

Animacy\is{animacy} plays an important role in grammar. This can be clearly seen from the animacy\is{animacy} hierarchy as introduced by \citet{silverstein:76} and \citet{dixon:79}, which is involved in such divergent domains of grammar as alignment, differential object marking,\is{differential object marking} direct/inverse marking and number marking on nouns (to name just a few). An examination of this hierarchy in its full form, as it is presented in \citet[][130]{croft:03}, shows that it is not only concerned with animacy\is{animacy} but also with person and referentiality.\is{referentiality}

\begin{exe}
\ex Animacy hierarchy\is{animacy!animacy hierarchy} \citep[][130]{croft:03}:\\
        first/second person pronouns > third person pronoun > proper names > human common
        noun > non-human animate common noun > inanimate common noun.
\end{exe}

The role of animacy\is{animacy} in a strict sense is limited to the animacy scale\is{animacy!animacy hierarchy}, which goes from human to animate to inanimate. Animacy\is{animacy} generally contributes to prominence\is{prominence} \citep[for a good survey, cf.][]{bornkessel:schlesewsky:09}. Another important scale that contributes to prominence\is{prominence} is the definiteness scale\is{definiteness scale} that runs from personal pronoun to proper name, to definite NP, to indefinite specific NP, to non-specific NP (cf. \citealt{aissen:03}, on the relevance of these two scales for differential object marking).\is{differential object marking} As will be shown in this subsection, based on the Vietnamese data from our experiments, both scales have their impact in the use of classifiers inasmuch as there is a strong tendency for classifiers to be used with definite animate nouns.

As for animacy,\is{animacy} \tabref{1table:1} below shows a clear correlation between the feature of [\pm animate] and classifier use. Out of 1,698 instances with animate nouns, 1,571 instances\footnote{1,571 is the result of all [+animate] nouns with a classifier in the written corpus (978 + 19 + 176 + 3) plus all [+animate] nouns with a classifier in the oral corpus (262 + 0 + 133 + 0) in \tabref{1table:6}.} (92.5\%) take a classifier, while only 127 instances\footnote{127 is the result of all [+animate] nouns with no classifier in the written corpus (8 + 1 + 114 + 0) plus all [+animate] nouns with no classifier in the oral corpus (0 + 1 + 3 + 0) in \tabref{1table:6}.} (7.5\%) occur without a classifier. In contrast, only 742 instances\footnote{742 is the result of all [\minus animate] nouns with a classifier in the written corpus (34 + 9 + 256 + 94) plus all [\minus animate] nouns with a classifier in the oral corpus (55 + 2 + 230 + 62) in \tabref{1table:6}.} of  [\minus animate] nouns (27.6\%) occur with a classifier, while 1,948
instances\footnote{1,948 is the result of all [\minus animate] nouns with no classifier in the written corpus (78 + 31 + 1,092 + 365) plus all [\minus animate] nouns with no classifier in the oral corpus (12 + 0 + 324 + 46) in \tabref{1table:6}.} (72.4\%) are bare nouns.\footnote{The frequencies of classifier use in the tables in this paper are for those occurrences in [CL+N] constructions; hence sequences such as {\emph{hai cu{\dao}n s\'ach}} [two CL book] `two books' would not be counted in these tables.}

\begin{table}
\begin{tabularx}{\textwidth}{Xll}
\lsptoprule
{\textbf{Nouns in narratives}} & {\textbf{[+animate]}} & {\textbf{[\minus animate]}} \\
\midrule
{\textbf{[CL+N]}} &1,571 instances (92.5\%) & 742 instances (27.6\%) \\
%\midrule
{\textbf{[N]}} & 127 instances (7.5\%) & 1,948 instances (72.4\%) \\
\lspbottomrule
\end{tabularx}
\caption{Token frequency of classifier use with [\pm animate] nouns in written texts and oral texts (in our Vietnamese corpus)}\label{1table:1}
\end{table}

Our Vietnamese data also show that classifiers can be interpreted as definite as well as indefinite but that there is a strong tendency towards definite interpretation in our written and in our oral corpus. This can be seen from \tabref{1table:2}, in which 1,444 instances of [CL+N] in the written corpus are definite (92.0\%; 1,154 + 290), while only 125 instances are indefinite (8.0\%; 22 + 103). Similarly, the oral corpus shows 680 instances of classifiers in their definite function (91.4\%; 395 + 285), which contrast with only 64 classifiers with an indefinite reading (8.6\%; 0 + 64). The same table additionally shows that definiteness clusters with animacy.\is{animacy} In the written corpus, 1,154 animate definite nouns with a classifier (90.4\%) {\mbox{contrast}} with only 122 animate definite nouns with no classifier (9.6\%). In the case of oral texts, animate definite nouns reach an even higher percentage: 100\% of these nouns take a classifier. As for inanimate definite nouns, only 19.9\% of the written corpus (290 out of a total of 1,460) and 45.9\% of the oral corpus (285 out of 621) take a classifier.

\begin{table}
\begin{tabular}{m{35pt}rrrrrrrr}
\lsptoprule
 & \multicolumn{4}{c}{\textbf{Written texts}} & \multicolumn{4}{c}{\textbf{Oral texts}} \\
\cmidrule(lr){2-4}\cmidrule(lr){5-9}
 & \multicolumn{2}{c}{\textbf{[+animate]}} & \multicolumn{2}{c}{\textbf{[\minus animate]}} & \multicolumn{2}{c}{\textbf{[+animate]}} & \multicolumn{2}{c}{\textbf{[\minus animate]}} \\
\cmidrule(lr){2-3}\cmidrule(lr){4-5}\cmidrule(lr){6-7}\cmidrule(lr){8-9}
 & {\textbf{[CL+N]}} & {\textbf{[N]}} & {\textbf{[CL+N]}} & {\textbf{[N]}} & {\textbf{[CL+N]}} & {\textbf{[N]}} & {\textbf{[CL+N]}} & {\textbf{[N]}} \\
\midrule 
{\textbf{Definite}} & {\textbf{1,154}} & {\textbf{122}} & {\textbf{290}} & {\textbf{1,170}} & {\textbf{395}} & {\textbf{3}} & {\textbf{285}} & {\textbf{336}} \\
%\hline
{\textbf{Indefinite}} & 22 & 1 & 103 & 396 & 0 & 1 & 64 & 46 \\
%\hline
{\textbf{Total}} & 1,176 & 123 & 393 & 1,566 & 395 & 4 & 349 & 382 \\
\lspbottomrule
\end{tabular}
\caption{Token frequency of [\pm animate] nouns and their interpretation as definite and indefinite in written texts and oral texts (in our Vietnamese corpus)}\label{1table:2}
\end{table}


The following two examples illustrate the use of animate nouns with a classifier. In (\ref{1ex:4}), the classifier occurs with one of the human protagonists of the story, who is clearly identifiable and definite at the point at which he is mentioned in that example. In example (\ref{1ex:5}), the classifier is interpreted as indefinite. The animate noun {\emph{d\^e}} `goat' is introduced into the story.\footnote{Notice, however, that in the continuation of this text, the goat is further specified as a {\emph{d\^e n\'ui}} [goat mountain] `wild goat' and does not take a classifier. With this type of compound, classifiers are often omitted.} As will be seen later in \sectref{1sec:42}, the co-occurrence with the copula verb {\emph{l\`a}} `to be' is one of the typical contexts in which [CL+N] is interpreted as indefinite (cf. example \ref{1ex:31}):

\begin{exe}
\ex\label{1ex:4}
[+animate, +CL, +DEF] (Oral text 19, sentence 12)
\exi{}
\gll {\textbf{Cậu}} {\textbf{bé}}   th{\daa}y  th{\dae}  tặng        mỗi   người    một  trái  lê … \\
{\textsc{cl}}  boy  see   that  present  each  person   one  {\textsc{cl}}   pear \\
\glt `The boy saw that he gave each of them one pear.'
\ex\label{1ex:5}
[+animate, +CL, \minus DEF] (Oral text 16, sentence 7) 
\exi{}
\gll Có 	một 	người   dẫn  con,  con  đó      {chắc là}   {\textbf{con}}  {\textbf{dê}},    đi    ngang   qua. \\
have  one   person  lead  {\textsc{cl}}   {\textsc{cl}}  {\textsc{dem}}  maybe    {\textsc{cl}}  goat   go   pass      over \\
\glt `There was a man who led a, a, it may be a goat, passing by.' \\
\end{exe}

\newpage
The following example shows how inanimate nouns tend to be realized as bare nouns, even if they are definite. The referents expressed by {\emph{thang}} `ladder' and {\emph{cây}} `tree' have already been mentioned but do not have classifier marking:\footnote{One of our reviewers asks if {\emph{thang}} `ladder' and {\emph{cây}} `tree' may be analyzed as instances of incorporation into the verb plus preposition. Given that both referents represented by these nouns can be clearly identified from their previous mention as individuated countable concepts in the text, such an analysis does not seem to be very likely.}

\begin{exe}
\ex\label{1ex:6}
[\minus animate, \minus CL, +DEF] (Oral text 27, sentence 3)
\exi{}
\gll Sau    đó,  {ông {\daa}y}  lại     leo     lên       {\textbf{thang}}   và       leo     lên       \\
after that  3.{\textsc{sg}}     again climb {\textsc{prep}} ladder   {\textsc{conj}} climb {\textsc{prep}}  \\
\glt
\exi{}
\gll {\textbf{cây}}  hái     ti{\dae}p. \\
tree  pluck continue \\
\glt `After that, he [the farmer] climbed up the ladder and climbed onto the tree again to continue picking [pears].'
\end{exe}

The comparatively less frequent combination of inanimate nouns with classifiers is illustrated by the following two examples:

\begin{exe}
\ex\label{1ex:7}
[\minus animate, +CL, +DEF] (Written text 2, sentence 8)
\exi{}
\gll {Cậu ta}  đã     đặt   {\textbf{cái}}  {\textbf{xô}}      ngay  giữa   b{\dao}    và   mẹ.  \\
3.{\textsc{sg}}   {\textsc{perf}}  place  {\textsc{cl}}  bucket  right  between father and  mother  \\
\glt `He put the bucket right between his father and mother.'
%%
\ex\label{1ex:8}
[\minus animate, +CL, \minus DEF] (Written text 24, sentence 14)
\exi{}
\gll Lúc  này,   người {đàn ông} thức  dậy, l{\daa}y  {\textbf{cái}}  {\textbf{bình}}  rót     nước  vào      ly, \\
time {\textsc{dem}} {\textsc{cl}}     man       wake up   take {\textsc{cl}} bottle pour  water {\textsc{prep}}  glass \\
\glt `At this time, the man woke up, he took a bottle and poured water into a glass,'
\end{exe}

In (\ref{1ex:7}), the inanimate noun {\emph{xô}} `bucket' was previously introduced into the scene by one of the protagonists (the boy). Given that the bucket is activated in the hearer's mind, the classifier marks definiteness in this example. In (\ref{1ex:8}), the noun {\emph{bình}} `bottle' refers to a newly introduced concept. Thus, the classifier {\emph{cái}} marks indefiniteness in this context.

The relationship between animacy/definiteness\is{animacy} and word order (the position of the [CL+N] construction relative to the preverbal and postverbal positions) will be discussed in \sectref{1sec:33}.

\subsection{The semantic features of uniqueness\is{uniqueness} and relationality\is{relationality}}\label{1sec:32}

{
The distinction between {\textpm}relational\footnote{Relational
    nouns\is{noun types!relational noun} have not only a referential argument, but also an additional relational argument (cf. the relational noun\is{noun types!relational noun} {\emph{daughter [of someone]}} in contrast to the absolute noun {\emph{girl}}).
    }
 and {\textpm}unique\footnote{Unique
    nouns denote concepts which are uniquely determined in a given situation (e.\,g., {\emph{the sun, the pope}}). Notice that the default use of uniqueness\is{uniqueness} is singular definite. Plural, indefinite and quantificational uses require special marking.} %\todo{got rid of brackets and \textpm to fit the line}
nouns as discussed by \cite{lobner:85,lobner:11} is of crucial importance for describing the use of classifiers in Vietnamese. The combination of these features with their two values yields the following four basic types of nouns, which correspond to four types of concepts\is{concept types} or four logical types: sortal nouns\is{noun types!sortal noun} ([\minus relational]/[\minus unique]; $\langle$e,t$\rangle$), individual nouns\is{noun types!individual noun} ([\minus relational]/[+unique]; $\langle$e$\rangle$), relational nouns\is{noun types!relational noun} ([+relational]/[\minus unique]; $\langle$e,$\langle$e,t$\rangle\rangle$) and functional nouns\is{noun types!functional noun} ([+relational]/[+unique]; $\langle$e,e$\rangle$).
}

{
\tabref{1table:3} presents our data on the presence or absence of classifiers in the context of L\"obner's (\citeyear{lobner:85,lobner:11}) basic types of nouns. As can be seen, the vast majority of nouns occurring with a classifier are sortal nouns\is{noun types!sortal noun} ([\minus unique]/[\minus relational]): out} of a total of 2,313 nouns with a classifier, 2,309 (99.8\%) belong to this type. Moreover, only three [+unique] nouns (marked in bold) out of 108 (2+83+1+22) take a classifier (2.8\%), while 105 of them are realized as bare nouns (97.2\%). In a similar way, relational nouns\is{noun types!relational noun} ([\minus unique]/[+relational]) have a strong tendency to occur without a classifier. Only one out of a total of 57 instances of this type (1.8\%) takes a classifier.


\begin{table}

\fittable{\small{
\begin{tabular}{lrrrrrrrr}	%{|l|c|c|c|c|c|c|c|c|}
\lsptoprule
{\textbf{[\pm relational]}},  & \multicolumn{4}{c}{\textbf{[+relational]}} & \multicolumn{4}{c}{{\textbf{[\minus relational]}}} \\
\cmidrule(lr){2-5}\cmidrule(lr){6-9}
{\textbf{[\pm unique]}} & \multicolumn{2}{c}{\textbf{[+unique]}} & \multicolumn{2}{c}{\textbf{[\minus unique]}} & \multicolumn{2}{c}{\textbf{[+unique]}} & \multicolumn{2}{c}{\textbf{[\minus unique]}} \\
{\textbf{nouns}} & \multicolumn{2}{c}{\textbf{functional}} & \multicolumn{2}{c}{\textbf{relational}} & \multicolumn{2}{c}{\textbf{individual}} & \multicolumn{2}{c}{\textbf{sortal}} \\
\cmidrule(lr){2-3}\cmidrule(lr){4-5}\cmidrule(lr){6-7}\cmidrule(lr){8-9}
 & {\textbf{[CL+N]}} & {\textbf{[N]}} & {\textbf{[CL+N]}} & {\textbf{[N]}} & {\textbf{[CL+N]}} & {\textbf{[N]}} & {\textbf{[CL+N]}} & {\textbf{[N]}} \\
\midrule
{\textbf{Written texts}} & 2 & 76 & 0 & 48 & 0 & 2 & 1,567 & 1,563 \\
%\hline
{\textbf{Oral texts}} & 0 & 7 & 1 & 8 & 1 & 20 & 742 & 351 \\
%\hline
{\textbf{Total}} & {\textbf{2}} & 83 & {\textbf{1}} & 56 & {\textbf{1}} & 22 & {\textbf{2,309}} & {\textbf{1,914}} \\
\lspbottomrule
\end{tabular}
}}
\caption{Token frequency of classifier with [\pm relational], [\pm unique] nouns in written texts and oral texts (in our Vietnamese corpus)}\label{1table:3}
\end{table}

\largerpage
From the four non-sortal nouns with a classifier, two are used in anaphoric situations. In example (\ref{1ex:9}), the [+unique/+relational] noun {\emph{mông}} `buttocks' is first introduced into the story by a bare noun. The second time it is mentioned, the same noun occurs with the general classifier {\emph{cái}}, its interpretation being definite because the object it denotes is now activated in the hearer's mind:

\begin{exe}
\ex\label{1ex:9}
(Written text 1, sentence 45)
\exi{}
\gll Bị         nóng   {\textbf{mông}},    {anh ta}  mở     vòi-nước    xịt      mát   cho   {\textbf{cái}}    {\textbf{mông}}, \\
{\textsc{pass}}   hot      buttock  {\textsc{3.sg}}    open   water-tap   spray cool  for     {\textsc{cl}}   buttock  \\
\glt 
\exi{}
\gll thì        lúc     đó,      bạn     {anh ta}   ch{\daob}m     từ      ngoài    {cửa sổ}    vào   h{\dao}i \\
{\textsc{conj}}  time   {\textsc{dem}}  friend  {\textsc{3.sg}}     prance   from outside  window  in     urge  \\
\glt 
\vspace*{-1mm}
\exi{} 
\gll {anh ta}  nhanh-lên  kẻo           {trễ giờ}.  \\
3.{\textsc{sg}}    hurry-up    otherwise  late \\
\glt `[His] buttocks were burnt, he turned on the tap and sprayed cool water onto the buttocks, at that time, his friend gesticulated from outside of the window to urge him to hurry up as otherwise he would be late.'
%}}
\end{exe}

A similar pattern is found in example (\ref{1ex:10}) with the [\minus unique/+relational] noun {\emph{chân}} `leg', which is expressed by a bare noun when it is first mentioned. Later on, it is taken up together with the general classifier {\emph{cái}} expressing definiteness in this context:

\begin{exe}
\ex\label{1ex:10}
(Oral text 4, sentence 21) 
\exi{}
\gll Lê    đổ    ra    {tung toé},      {hình như}  nó     bị         đau   {\textbf{chân}}  nữa,   th{\daa}y  nó sờ       sờ       {\textbf{cái}}   {\textbf{chân}}. \\
pear pour out  everywhere seems       3.{\textsc{sg}}   {\textsc{pass}} hurt  leg     more  see   3.{\textsc{sg}} touch  touch  {\textsc{cl}}   leg \\
\glt `The pears rolled out everywhere, it seemed that his leg was hurt, (because I saw) he touched [his] leg.'
\end{exe}

{
In the other two instances of the [CL+N] construction with a non-sortal noun, the use of the classifier is due to information structure\is{information structure} (focus).\is{focus} For that reason, the relevant examples will be discussed in \sectref{1sec:343} (cf. (\ref{1ex:23}) and (\ref{1ex:25})).
}

\subsection{Word order, definiteness and animacy\is{animacy}}\label{1sec:33}
In many Sinitic \il{Sinitic languages} numeral classifier systems,\is{classifiers!numeral classifiers} the referential status\is{referential status} associated with the classifier in [CL+N] constructions depends on word order relative to the verb (see \citealt{wang:15} for a survey). The following examples in (\ref{1ex:11}) and (\ref{1ex:12}) from \cite{li:bisang:12} show how the preverbal subject position and the postverbal object position are associated with definiteness and indefiniteness in Mandarin,\il{Chinese!Mandarin} in the Wu\il{Chinese!Wu} dialect of Fuyang and in Cantonese.\il{Chinese!Cantonese}

While the [CL+N] construction in the subject position is ungrammatical in Mandarin Chinese\il{Chinese!Mandarin} (\ref{1ex:11a}), it is interpreted in terms of definiteness in the Wu\il{Chinese!Wu} dialect of Fuyang (\ref{1ex:11b}) and in Cantonese\il{Chinese!Cantonese} (\ref{1ex:11c}).

\begin{exe}
\ex\label{1ex:11}
[CL+N] in the subject position \citep[][338]{li:bisang:12} \\
Context: Where is the book?
	\begin{xlista}
	\ex\label{1ex:11a}
	Mandarin:\il{Chinese!Mandarin}
	\exi{}
	\gll nà    běn   shū,     {\textbf{(*ge)}} {\textbf{xuéshēng}}  mǎi-zǒu     le. \\
	     that  {\textsc{cl}}   book    {\textsc{cl}}    student       buy-away  {\textsc{pf}} \\
	\glt `The book, the student(s) has/have bought it.'
	\ex\label{1ex:11b}
	Wu Chinese:\il{Chinese!Wu}
	\exi{}
	\gll pen   cy      ke    ia\textglotstop sn        ma    le         tçhi   die. \\
	      {\textsc{cl}}    book, {\textsc{cl}}  student    buy   {\textsc{pfv}}     go    {\textsc{sfp}} \\
	\glt `The book, the student bought (it).'
	\ex\label{1ex:11c}
	Cantonese:\il{Chinese!Cantonese}
	\exi{}
	\gll bun  syu,    go   hoksaang    maai-jo     la. \\
	      {\textsc{cl}}   book  {\textsc{cl}}   student       buy-{\textsc{pfv}}   {\textsc{sfp}} \\
 	\glt `The book, the student bought (it).'
	\end{xlista}
\end{exe}

In the object position, the classifier in [CL+N] is associated with indefiniteness in Mandarin\il{Chinese!Mandarin} (\ref{1ex:12a}) and the Wu\il{Chinese!Wu} dialect of Fuyang (\ref{1ex:12b}). In Cantonese,\il{Chinese!Cantonese} it goes with definiteness and indefiniteness (\ref{1ex:12c}):

\begin{exe}
\ex\label{1ex:12}
[CL+N] in the object position \citep[][338-339]{li:bisang:12}
\begin{xlista}
	\ex\label{1ex:12a}
	Mandarin:\il{Chinese!Mandarin}
	\exi{}
	\gll wǒ  mǎi-le       {\textbf{liàng}}  {\textbf{chē}}. \\
	     I      buy-{\textsc{pfv}}  {\textsc{cl}}     car \\
	\glt `I bought a car.'
	\ex\label{1ex:12b}
	Wu Chinese:\il{Chinese!Wu}
	\exi{}
	\gll Nge   ma   le      {\textbf{bu}}    {\textbf{tsʰotsʰi}}. \\
	     I        buy  {\textsc{pfv}}  {\textsc{cl}}   car \\
	\glt `I bought a car.'
	\ex\label{1ex:12c}
	Cantonese:\il{Chinese!Cantonese}
	\exi{}
	\gll Keuih   maai-zo    {\textbf{gaa}}  {\textbf{ce}}. \\
	     he         sell-{\textsc{pfv}}   {\textsc{cl}}   car \\
	\glt `I sold a car/the car.'
	\end{xlista}
\end{exe}

\largerpage
As can be seen from \tabref{1table:4}, the situation is different in Vietnamese. The [CL+N] construction occurs preverbally and postverbally and the classifier can be associated with definiteness as well as indefiniteness in both positions. A closer look reveals that the definite interpretation of the classifier is generally preferred. The overall percentage of definite [CL+N] constructions is 91.8\% in contrast to only 8.2\% of classifiers with an indefinite function.\footnote{The total number of definite [CL+N] constructions is 2,124 (1,012 + 432 + 317 + 363); the total number of indefinite [CL+N] constructions is 189 (28 + 97 + 2 + 62).} The dominance of the definite interpretation is even stronger in the subject position (cf. the figures printed in bold). If the written and oral texts are combined, 1,329 out of 1,359 [CL+N] constructions, or 97.8\%, are definite.\footnote{The total number of definite [CL+N] constructions in subject position is 1,329 (1,012 + 317); the total number of indefinite [CL+N] constructions is 30 (28 + 2).} In the object position, the asymmetry between the definite and the indefinite interpretation is not as strong as in the subject position. In spite of this, the definite interpretation still clearly dominates, with 795 (432 + 363) instances
(83.3\%), compared with only 159 (97 + 62) instances (16.7\%) with an indefinite interpretation.\footnote{Recall that bare nouns in Vietnamese can also occur in both subject positions and object positions and be interpreted as either definite or indefinite.}

\begin{table}
{\small{
\begin{tabularx}{\textwidth}{Xrrrrrrrr}%{|l|c|c|c|c|c|c|c|c|}
\lsptoprule
{\textbf{Nouns in}}  & \multicolumn{4}{c}{\textbf{Written texts}} & \multicolumn{4}{c}{{\textbf{Oral texts}}} \\
\cmidrule{2-9}
{\textbf{narrative}} & \multicolumn{2}{c}{\textbf{Subject}} & \multicolumn{2}{c}{\textbf{Object}} & \multicolumn{2}{c}{\textbf{Subject}} & \multicolumn{2}{c}{\textbf{Object}} \\
\cmidrule(lr){2-5}\cmidrule(lr){6-9}
 & {\textbf{[CL+N]}} & {\textbf{[N]}} & {\textbf{[CL+N]}} & {\textbf{[N]}} & {\textbf{[CL+N]}} & {\textbf{[N]}} & {\textbf{[CL+N]}} & {\textbf{[N]}} \\
\cmidrule(lr){2-3}\cmidrule(lr){4-5}\cmidrule(lr){6-7}\cmidrule(lr){8-9}
{\textbf{Definite}} & {\textbf{1012}} & 86 & 432 & 1206 & {\textbf{317}} & 12 & 363 & 327 \\
%\hline
{\textbf{Indefinite}} & {\textbf{28}} & 32 & 97 & 365 & {\textbf{2}} & 1 & 62 & 46 \\
%\hline
{\textbf{Total}} & 1040 & 118 & 529 & 1571 & 319 & 13 & 425 & 373 \\
\lspbottomrule
\end{tabularx}
}}
\caption{Token frequency of the presence/absence of a classifier in subject and object positions in relation to definite vs. indefinite function (in our Vietnamese corpus)}\label{1table:4}
\end{table}

\largerpage[2]
The two examples in (\ref{1ex:13}) and (\ref{1ex:14}) illustrate the definite function of the classifier in [CL+N]. In (\ref{1ex:13}), {\emph{con lừa}} [CL donkey] `the donkey' is in the subject position. Because it is mentioned in the previous context, the classifier {\emph{con}} has a definite reading. In (\ref{1ex:14}), {\emph{cô vợ}} [CL wife] `the wife' is in the object position. Since it is mentioned in the preceding text, it is also interpreted as definite:

\begin{exe}
\ex\label{1ex:13}
Definite subject (Oral text 26, sentence 9) 
\exi{}
\gll {\textbf{Con}} {\textbf{lừa}}        cứ        nhìn vào    các  \\
{\textsc{cl}}  donkey  always look inside {\textsc{pl}}  \\
\glt
\exi{}
\gll c{\daab}n  {xé lê}   như  mu{\dao}n {đứng lại}  và        ăn   lê.   \\
{\textsc{cl}}  pear   like   want  stop         {\textsc{conj}}  eat  pear \\
\glt `The donkey kept on looking into the baskets as if it wanted to stand by and eat them.'
%%
\ex\label{1ex:14}
Definite object (Written text 9, sentence 14) 
\exi{}
\gll {Bực mình},  anh ch{\daob}ng      {đóng s{\daab}m}   cửa   khi{\dae}n  {\textbf{cô}}  {\textbf{vợ}}      {giật mình}, \\
angry         {\textsc{cl}}  husband   slam          door  cause  {\textsc{cl}} wife  startled \\
\glt 
\exi{}
\gll r{\daob}i        bỏ     vào     {nhà tắm}. \\
{\textsc{conj}}  leave enter   bathroom \\
\glt `Annoyed, the husband slammed the door. This upset [his] wife, then he went to the bathroom.'
\end{exe}

The following two examples focus on the object position and indefiniteness (for indefinite [CL+N] constructions in the subject position, cf. \sectref{1sec:41}). At the same time, they also illustrate how classifiers in the same syntactic position can be interpreted as indefinite or definite, depending on context. In example (\ref{1ex:15}) from our data on written texts, we find the same expression ({\emph{chi{\dae}c xe}} [CL car] `a/the car') in both functions.

\begin{exe}
\ex\label{1ex:15}
[Indefinite object, \pm DEF] (Written text 1, sentence 26)
\exi{}
\gll {Anh ta} bước vào     nhà     thì         lại          bị        đứa  con  chơi  {\textbf{chi{\dae}c}} {\textbf{xe$_1$}}  \\
3.{\textsc{sg}}    step  {\textsc{prep}} house  {\textsc{conj}}   {\textsc{emph}}   {\textsc{pass}}  {\textsc{cl}}   son  play   {\textsc{cl}}  car \\
\glt
\gll đẩy    trúng  vào    chân  khi{\dae}n  {anh ta}  {ngã ngửa}  vào      {\textbf{chi{\dae}c}} {\textbf{xe$_2$}}. \\
push  {\textsc{res}}   {\textsc{prep}} leg    cause  3.{\textsc{sg}}    fall.back    {\textsc{prep}}  {\textsc{cl}} car \\
\glt `When he entered the house, he ran into his son who was playing and he got hit by a car [a toy car] into [one of his legs]. [This] made him fall down onto the car.'
\end{exe}

In the first line, the noun {\emph{xe}} `car' in {\emph{chi{\dae}c xe}} is not activated by previous context. Thus, the classifier must be interpreted as indefinite. In the second line, the same car is taken up again with the same classifier ({\emph{chi{\dae}c}}), which now has a definite interpretation. The next example is from our oral corpus:

\begin{samepage}
\begin{exe}
\ex\label{1ex:16}
[Indefinite object, \pm DEF] (Oral text 26, sentence 1)
\exi{}
\gll 	Có    một	người	{đàn ông}	đang			ở          		trên  {\textbf{cái}}  {\textbf{thang}}   bắc        lên       {\textbf{cây}}  {\textbf{lê}}  và         đang    hái       {\textbf{trái}}  {\textbf{lê}}. \\
	exist one	{\textsc{cl}}     man		{\textsc{prog}}		{\textsc{prep}}   	top   {\textsc{cl}}  ladder  connect  {\textsc{prep}}  {\textsc{cl}}  pear {\textsc{conj}}  {\textsc{prog}}   pluck  {\textsc{cl}}   pear\\
\glt `There was a man on [a] ladder which was propped up against [a] pear tree. He was picking [its] pears.'
\end{exe}
\end{samepage}

{
In this example, we find three [CL+N] constructions, i.\,e., {\emph{cái thang}} [CL\textsubscript{general} ladder], {\emph{cây lê}} [CL\textsubscript{tree} pear] and {\emph{trái lê}} [CL\textsubscript{fruit} pear]. Since the first two nominal concepts are newly introduced, the corresponding [CL+N] constructions are interpreted as indefinite (`a ladder' and `a pear tree'). The third [CL+N] construction is associated with the previously mentioned pear tree. For that reason, the classifier {\emph{trái}} for fruits can be interpreted as definite through bridging\is{anaphora!bridging anaphora} (`its pears [i.\,e., the pears of the previously mentioned tree]').
}

If the data on classifier use in the subject and in the object position is combined with the semantic feature of animacy\is{animacy} as in \tabref{1table:5}, it can be seen that there is a clear preference for animate nouns in the subject position. There are 1,269 instances (85.2\%) of [+animate] nouns in the subject position, which contrast with only 221 instances (14.8\%) of [\minus animate] nouns. Similarly, the object position is characterized by its clear preference for [\minus animate] nouns. There are 2,469 [\minus animate] object nouns (85.2\%) and only 429 [+animate] object nouns (14.8\%). Thus, the data in \tabref{1table:5} reflect the well-known preference of animate subjects and inanimate objects (cf. \citealt{givon:79}, \citealt{dubois:87} and many later publications).

\begin{table}

\begin{tabularx}{\textwidth}{Xrrrr}%{|l|c|c|c|c|}
\lsptoprule
 & \multicolumn{2}{c}{\textbf{Subject}} & \multicolumn{2}{c}{\textbf{Object}} \\
\cmidrule(lr){2-3}\cmidrule(lr){4-5}
 & {\textbf{[+animate]}} & {\textbf{[\minus animate]}} & {\textbf{[+animate]}} & {\textbf{[\minus animate]}} \\
\midrule
Written texts & 1,006	 & 152 & 293 & 1,807 \\
%\hline
Oral texts & 263 & 69 & 136 & 662 \\
%\hline
Total & 1,269 (85.2\%) & 221 (14.8\%) & 429 (14.8\%) & 2,469 (85.2\%)\\
\lspbottomrule
\end{tabularx}

\caption{Distribution of instances of [\pm animate] nouns in the positions of subject and object (in our Vietnamese corpus)}\label{1table:5}
\end{table}

{
Finally, the combination of the three parameters of word order (subject vs. object), reference (definite vs. indefinite) and animacy\is{animacy} (animate vs. inanimate) yields the following results for the presence/absence of the classifier ([CL+N] vs. [N]):
}

\begin{table}

\begin{tabularx}{\textwidth}{Q rr p{2mm} rr p{4mm} rr p{2mm} rr}
\lsptoprule
{\textbf{[+def]}} {\textbf{vs.}} {\textbf{[\minus def]}}  & \multicolumn{10}{c}{\textbf{Written texts}} \\
\cmidrule{2-12}
  & \multicolumn{5}{c}{\textbf{Subject}} && \multicolumn{5}{c}{\textbf{Object}} \\
\cmidrule(lr){2-6}\cmidrule(lr){8-12}
 & \multicolumn{2}{c}{\textbf{[CL+N]}} && \multicolumn{2}{c}{\textbf{[N]}} && \multicolumn{2}{c}{\textbf{[CL+N]}} && \multicolumn{2}{c}{\textbf{[N]}} \\
 & \multicolumn{2}{c}{\textbf{[\pm ani]}} && \multicolumn{2}{c}{\textbf{[\pm ani]}} && \multicolumn{2}{c}{\textbf{[\pm ani]}} && \multicolumn{2}{c}{\textbf{[\pm ani]}} \\
\cmidrule(lr){2-3}\cmidrule(lr){5-6}\cmidrule(lr){8-9}\cmidrule(lr){11-12}
& {\textbf{+}} & {\textbf{\minus}} && {\textbf{+}} & {\textbf{\minus}} && {\textbf{+}} & {\textbf{\minus}} && {\textbf{+}} & {\textbf{\minus}}\\
\midrule
{\textbf{+def}} & {\textbf{978}} & {\textbf{34}} && 8 & 78 && 176 & 256 && 114 & 1,092  \\
%\hline
{\textbf{\minus def}} & {\textbf{19}} & {\textbf{9}} && 1 & 31 && 3 & 94 && 0 & 365  \\
%\hline
\midrule
{\textbf{Total}} & {\textbf{997}} & {\textbf{43}} && 9 & 109 && 179 & 350 && 114 & 1,457  \\
\lspbottomrule
\end{tabularx}

\noindent
\begin{tabularx}{\textwidth}{Q rr p{2mm} rr p{4mm} rr p{2mm} rr}
\lsptoprule
{\textbf{[+def]}} {\textbf{vs.}} {\textbf{[\minus def]}}   & \multicolumn{10}{c}{\textbf{Oral texts}} \\
\cmidrule{2-12}
  & \multicolumn{5}{c}{\textbf{Subject}} && \multicolumn{5}{c}{\textbf{Object}} \\
\cmidrule(lr){2-6}\cmidrule(lr){8-12}
 & \multicolumn{2}{c}{\textbf{[CL+N]}} && \multicolumn{2}{c}{\textbf{[N]}} && \multicolumn{2}{c}{\textbf{[CL+N]}} && \multicolumn{2}{c}{\textbf{[N]}} \\
 & \multicolumn{2}{c}{\textbf{[\pm ani]}} && \multicolumn{2}{c}{\textbf{[\pm ani]}} && \multicolumn{2}{c}{\textbf{[\pm ani]}} && \multicolumn{2}{c}{\textbf{[\pm ani]}} \\
\cmidrule(lr){2-3}\cmidrule(lr){5-6}\cmidrule(lr){8-9}\cmidrule(lr){11-12}
& {\textbf{+}} & {\textbf{\minus}} && {\textbf{+}} & {\textbf{\minus}} && {\textbf{+}} & {\textbf{\minus}} && {\textbf{+}} & {\textbf{\minus}}\\
\midrule
{\textbf{+def}} &  {\textbf{262}} & {\textbf{55}} && 0 & 12 && 133 & 230 && 3 & 324 \\
%\hline
{\textbf{\minus def}} & {\textbf{0}} & {\textbf{2}} && 1 & 0 && 0 & 62 && 0 & 46 \\
%\hline
\midrule
{\textbf{Total}} & {\textbf{262}} & {\textbf{57}} && 1 & 12 && 133 & 292 && 3 & 370 \\
\lspbottomrule
\end{tabularx}

\caption{Presence/absence of classifiers depending on the features of [\pm animate], subject vs. object and definite vs. indefinite (in our Vietnamese corpus)}\label{1table:6}
\end{table}

\tabref{1table:6} reveals that, of the 1,012 definite [CL+N] constructions in the subject position of the written text corpus, 978 (96.6\%) are [+animate] nouns. Only 34 definite [CL+N] constructions in the subject position (3.4\%) are [\minus animate]. Similarly in oral texts, 262 animate definite subject [CL+N] constructions (82.6\%) contrast with only 55 inanimate definite subject [CL+N] constructions (17.4\%). In the object position, the percentage of animate nouns with definite subject [CL+N] constructions is much lower: 40.7\% (176 vs. 256) in the corpus of written texts and 36.6\% (133 vs. 230) in the corpus of oral texts. The results from \tabref{1table:6} combined with the results from \tabref{1table:4} (general preference of definite classifier interpretation, particularly with [CL+N] constructions in the subject position) plus \tabref{1table:5} (preference of animate subjects) show that the classifier prototypically occurs with definite animate nouns in the subject position.

These observations can be visualized more clearly by means of the bar chart in \figref{1fig:1}. The blue columns represent definiteness, while the green ones stand for indefiniteness:

\begin{figure}
% \includegraphics[width=\textwidth]{chapters/bisang-fig1.png}
\pgfplotstableread{
1 1240      19
2 89        11
3 8         2
4 90        31
5 309       3
6 486       156
7 117       0
8 1416      411
}\dataset
\begin{tikzpicture}
\begin{axis}[ybar,
        width=\textwidth,
        height=.3\textheight,
        ymin=0,
        % ymax=100,
        % ylabel={},
        % xlabel={},
        xlabel style = {yshift=-8mm},
        xtick=data,
        bar width=5mm,
        nodes near coords,
        xticklabels = {
            {[Subj, +CL +ani]},
            {[Subj, +CL -ani]},
            {[Subj, -CL +ani]},
            {[Subj, -CL -ani]},
            {[Obj, +CL +ani]},
            {[Obj, +CL -ani]},
            {[Obj, -CL +ani]},
            {[Obj, -CL -ani]}
        },
      axis y line*=right,
      axis x line*=bottom,
      legend style={at={(.15,1)},anchor=north west},
      x tick label style={align=center,text width=1cm},
      ticklabel style = {font=\scriptsize},
      ]
\addplot[draw=lsMidDarkBlue!80!black,fill=lsMidDarkBlue] table[x index=0,y index=1] \dataset;
\addplot[draw=lsLightBlue!80!black,fill=lsLightBlue] table[x index=0,y index=2] \dataset;
\legend{Definite,Indefinite}
\end{axis}
\end{tikzpicture}

\caption{Token frequency of [\pm animate] nouns in subject and object function, marking definiteness or indefiniteness with or without a classifier (in our Vietnamese corpus)}\label{1fig:1}
\end{figure}


In accordance with the data in \tabref{1table:6}, the blue columns representing definiteness are generally higher than the green columns, reflecting again the overall dominance of the definite function of Vietnamese classifiers. Moreover, the blue column in \figref{1fig:1} clearly dominates over the green column at the leftmost pole representing animate subjects with classifiers [Subj, +CL, +ani]. The preference for classifier use with animate subjects is further corroborated if the total number of tokens with the features [Subj, +CL, +ani] in the written and the oral corpus is compared with the total number of tokens with the features [Subj, \minus CL, +ani]. The figure for [Subj, +CL, +ani] is 1,259 (978 + 19 + 262 + 0), while the figure for [Subj, \minus CL, +ani] is just 10 (8 + 1 + 0 + 1). Thus, the use of the classifier with animate subjects overwhelmingly dominates over its absence with 99.2\%. In addition to these results, the rightmost pole in \figref{1fig:1} with the features [Obj, \minus CL, \minus ani] demonstrates that inanimate object nouns tend to occur without a classifier. The overall number of tokens with the features [Obj, \minus CL, \minus ani] from the written and the oral texts is 1,827 (1,092 + 365 + 324 + 46), while the overall number of tokens with the features [Obj, +CL, \minus ani] is only 642 (256 + 94 + 230 + 62). Thus, the percentage of inanimate object nouns without a classifier is 74.0\% against 26.0\% with a classifier. Taken together, there is a clear preference for animate subjects to occur with a classifier and for inanimate objects to occur as bare nouns.

\largerpage
To conclude, the data presented in this subsection show that the (in)definite\-ness interpretation of the classifier is not rigidly determined by the position of the [CL+N] construction relative to the verb (subject vs. object position). In fact, there is an overall preference for interpreting classifiers in [CL+N] as definite even though indefinite [CL+N] constructions are found in both positions. In spite of this, there are other factors which operate against this general tendency as well as against the use of classifiers in definite contexts. The semantic factors were presented above in \sectref{1sec:32}. \sectref{1sec:34} will discuss aspects of discourse and information structure.\is{information structure}

\subsection{Discourse and information structure\is{information structure}}\label{1sec:34}
Discourse and information structure\is{information structure} affect the meaning of Vietnamese classifiers as well as their presence or absence in a given context. As discussed in \sectref{1sec:341} on meaning, the definiteness expressed by the classifier is discourse-based. The same subsection also shows how discourse enhances the use of classifiers with [+unique] nouns which otherwise show a strong preference for occurring as bare nouns in our data (cf. \sectref{1sec:32}). \sectref{1sec:342} and \sectref{1sec:343} illustrate how information structure\is{information structure} determines the presence of a classifier. It will be shown that contrastive topics\is{topic!contrastive topic} generally take a classifier (cf. \sectref{1sec:342}). Similarly, focus,\is{focus} as it manifests itself in contrastive focus\is{focus!contrastive focus} and focus particles,\is{focus!focus particles} can support the use of a classifier, even with non-sortal nouns (\sectref{1sec:343}).


\subsubsection{Definiteness, identifiability\is{identifiability} and information structure\is{information structure}}\label{1sec:341}
Classifiers in [CL+N] constructions  very rarely  occur  with [+unique] nouns (cf. \sectref{1sec:32} on the strong preference for sortal nouns\is{noun types!sortal noun} ([\minus unique]/[\minus relation\-al])). Moreover, the majority of definite classifiers are used in anaphoric contexts, in which a previously introduced concept is taken up with a classifier in order to highlight the speaker's assumption that it can be identified by the hearer (cf. examples (\ref{1ex:4}), (\ref{1ex:7}), (\ref{1ex:13}), (\ref{1ex:14}) and (\ref{1ex:15})). Even two of the four non-sortal nouns with a classifier acquire their classifier in an anaphoric context (cf. (\ref{1ex:9}) and (\ref{1ex:10}); for the other two, cf. \sectref{1sec:343} on focus).\is{focus} Taken together, these facts are strong indicators that the definiteness expressed by the classifier marks pragmatic definiteness\is{definiteness!pragmatic} rather than semantic definiteness\is{definiteness!semantic} in terms of \cite{lobner:85}. In Schwarz's (\citeyear{schwarz:09,schwarz:13}) framework, Vietnamese definite classifiers express anaphoric or ``strong'' definiteness\is{definiteness!anaphoric}\is{definiteness!strong} rather than unique or ``weak'' definiteness.\is{definiteness!weak}

With these properties, the definiteness associated with the classifier corresponds to the findings of \cite[][17]{li:bisang:12} on identifiability.\is{identifiability} As they show in example (\ref{1ex:17}) from the Wu\il{Chinese!Wu} dialect of Fuyang, uniqueness\is{uniqueness} is not a necessary condition for the definite interpretation of the [CL+N] construction. Unique concepts can be expressed either by bare nouns or by the [CL+N] construction. A [+unique] [\minus relational] noun like {\emph{thin}} `sky' in (\ref{1ex:17}) occurs in its bare form if the sky is understood generically as the one and only one sky. Thus (\ref{1ex:17a}) is a generic sentence\is{generic sentence} expressing the fact that the sky is blue in general. In contrast, the classifier in {\emph{ban thin}} [CL sky] (\ref{1ex:17b}) indicates that the speaker means the sky as it is relevant for a given speech situation with its temporal or spatial index, and that s/he thinks that the hearer can identify it \citep[][17]{li:bisang:12}:

\newpage
\begin{exe}
\ex\label{1ex:17}
Wu dialect of Fuyang \citep[][17]{li:bisang:12}:\il{Chinese!Wu}
\begin{xlista}
\ex\label{1ex:17a}
Generic use:\is{generic sentence}
\exi{}
\gll {\textbf{Thin}}  zi   lan     ko. \\
    sky    be  blue   {\textsc{sfp}} \\
\glt `The sky is blue (in general).'
%%


\ex\label{1ex:17b}
Episodic use:
\exi{}
\gll {\textbf{Ban}}         {\textbf{thin}}    gints{\textbf{\textopeno}}   man   lan.  \\
    {\textsc{cl}}\textsubscript{piece}  sky     today    very   blue   \\
\glt `The sky is blue today.'
\end{xlista}
\end{exe}

In Vietnamese, the situation seems to be similar. Since a much larger corpus than the two corpora used here would be needed to find examples like (\ref{1ex:17}), we present another example from a Vietnamese dictionary in (\ref{1ex:18}) \citep[][116 and 1686]{nguyen:05}. In (\ref{1ex:18a}), we find {\emph{trời}} `sky' as a bare noun. In this form, the sky is understood generically as the endless outer space seen from the earth with its general property of being full of stars. In contrast, {\emph{b{\daab}u trời}} [CL sky] `the sky' in (\ref{1ex:18b}) with a classifier denotes the inner space seen from the earth as it is currently relevant to the speech situation. The speaker employs the classifier to inform the hearer that s/he is referring to the sky as it currently matters and as it can be identified by the speaker and the hearer in a shared temporal or spatial environment.

\begin{exe}
\ex\label{1ex:18}
	\begin{xlista}
	\ex\label{1ex:18a}
	\gll Trời  đ{\daab}y  sao. \\
		sky   full  star \\
	\glt `The sky is full of stars.'
	\ex\label{1ex:18b}
	\gll {\textbf{B{\daab}u}}  trời  {đêm nay}  đ{\daab}y  sao. \\
		{\textsc{cl}}   sky   tonight    full  star \\
	\glt `THE sky tonight is full of stars.'
	\end{xlista}
\end{exe}

{
Further evidence for the discourse-dependency of classifier use with [+unique] nouns comes
from the fact that the noun {\emph{trời}} `sky' can take several different classifiers, e.\,g., {\emph{bầu trời}} [CL\textsubscript{round} sky], {\emph{khung trời}} [CL\textsubscript{frame} sky] or {\emph{vùng trời}} [CL\textsubscript{area} sky], etc. The selection of a specific classifier out of a set of possible classifiers depends on the particular property of the sky the speaker wants to highlight to facilitate its identifiability\is{identifiability} to the hearer. In such a situation, selecting a particular classifier is even compulsory:
}

\begin{exe}
\ex\label{1ex:19}
\gll *({\textbf{Khung/b{\daab}u/vùng}}) trời  {mơ ước}  của     hai   {chúng ta}  đây   r{\daob}i! \\
\hspace*{2mm}{\textsc{cl}}  sky  dream    {\textsc{poss}}  two   2.{\textsc{pl}}       here  {\textsc{sfp}}\\
\glt `Our dream sky/world is here!'
\end{exe}

In the above example, the speaker creates a specific notion of the sky as it is relevant for her/him and the hearer. This `dream sky' is then anchored in space and time as relevant to the speech situation by a classifier.

{
In another of our four examples of non-sortal nouns with a classifier in (\ref{1ex:23}), the [+unique, \minus relational] noun {\emph{đ{\daa}t}} `earth, ground' is marked by the classifier {\emph{mặt}} `face/surface' in a situation of contrastive focus.\is{focus!contrastive focus} As in the case of the sky in (\ref{1ex:19}), this noun is also compatible with other classifiers, among them {\emph{mảnh/mi{\dae}ng}} `piece' and {\emph{vùng}} `area'. The selection of a specific classifier depends again on the properties of the concept expressed by the noun as they are relevant to the speech situation.
}

\subsubsection{Contrastive topics\is{topic!contrastive topic}}\label{1sec:342}

There is an impressive body of literature on contrastive topics.\is{topic!contrastive topic} For the purpose of this paper, Lambrecht's (\citeyear[][183, 291, 195]{lambrecht:94}) discourse-based definition in terms of two activated topic\is{topic} referents which are contrasted will be sufficient. This type of topic\is{topic} is quite frequent in our Vietnamese corpus. A look at the statistics shows that classifier use is very strongly associated with contrastiveness. In fact, there is a classifier in each of the 84 instances of contrastive focus\is{focus!contrastive focus} (66 in the written corpus and 18 in the oral corpus). Moreover, all nouns occurring in this function are [+animate].

In most examples, the action/state of one protagonist is contrasted with the action/state of another protagonist. As shown in (\ref{1ex:20}), the actions of the son in the kitchen are contrasted with the actions of his mother in the bedroom (described as `the wife' from the perspective of the husband). The son takes the classifier {\emph{đứa}} for young boys, while the mother takes the classifier {\emph{bà}} for women. The contrast\is{contrast} between these two protagonists is supported by the adverbial subordinator {\emph{còn }} `while/whereas':

\largerpage[2]
\begin{exe}
\ex\label{1ex:20}
(Written text 26, sentence 23)
\exi{}
\gll {\textbf{Đứa}} {\textbf{con trai}}  thì     đứng lên  kệ-b{\dae}p           và        vẽ     bậy           lên  tường, \\
{\textsc{cl}}   son         {\textsc{top}}  stand up  kitchen-bar  {\textsc{conj}} draw disorderly on   wall \\
\glt 
\exi{}
\gll còn     {\textbf{bà}}  {\textbf{vợ}}      thì     nằm  ăn  {đ{\daob} ăn nhanh}  với    {vẻ mặt}         {khoái chí}. \\
while  {\textsc{cl}} wife  {\textsc{top}}  lie    eat  fast.food        {\textsc{pre}}  expression  delightful \\
\glt `[His] son stood on the kitchen base (cabinet) and scribbled [something] onto the wall, while [his] wife was lying in bed, eating fast food with a facial expression of delight.'
\end{exe}

In (\ref{1ex:21}), the husband is contrasted with his wife. The husband's anger and his intention to make his wife eat some food is mirrored against his wife's reaction of refusing to give in. Both nouns take a classifier. The husband occurs with the classifier {\emph{ông}} for men and the wife again takes the classifier {\emph{bà}} for women. The contrast\is{contrast} is explicitly expressed by the disjunctive conjunction {\emph{nhưng}} `but':

\begin{exe}
\ex\label{1ex:21}
(Written text 26, sentence 36)
\exi{}
\gll Th{\daa}y  {thái độ}   của       vợ-mình,  {\textbf{ông}} {\textbf{ch{\daob}ng}}     điên-máu-lên  và \\
see     attitude  {\textsc{poss}}   wife-self   {\textsc{cl}}  husband  get.crazy        and \\
\glt 
\exi{}
\gll {bắt ép}   ăn,  {\textbf{nhưng}}  {\textbf{bà}}   {\textbf{vợ}}     vẫn  không  ăn.  \\
force    eat   {\textsc{conj}}   {\textsc{cl}}  wife  still  {\textsc{neg}}   eat \\
\glt `Seeing the behaviour of his wife, the husband went crazy and [tried to] force her to eat, but [his] wife still did not eat.'
\end{exe}

In the final example of this subsection, there is a contrast\is{contrast} between a protagonist and a non-protagonist. The noun {\emph{bé}} `boy', as one of the two protagonists in the Pear Story, is contrasted with the children ({\emph{trẻ}} `child'). What is contrasted is the boy's action of leaving on a bike and the children's action of walking away. Again, both nouns occur with a classifier ({\emph{thằng}} for the boy and {\emph{bọn}} for the children) and there is a contrastive conjunction ({\emph{còn}} `while, whereas'):

\begin{samepage}
\begin{exe}
\ex\label{1ex:22}
(Oral text 6, sentence 31)
\exi{}
\gll {\textbf{Thằng}} {\textbf{bé}}    {tập tễnh}  dắt   xe      đi   vài  bước, còn      {\textbf{bọn}} {\textbf{trẻ}}     thì \\
{\textsc{cl}}      boy  limping  lead  bike  go  few  step   {\textsc{conj}}  {\textsc{cl}}  kid   {\textsc{top}} \\
\glt 
\exi{}
\gll đi   theo       hướng       {ngược lại}. \\
go  toward   direction   opposite \\
\glt `The boy led the bike limpingly, while the children walked in the opposite direction of the boy.'
\end{exe}
\end{samepage}


\subsubsection{Focus\is{focus}}\label{1sec:343}
Classifiers are also selected in various types of focus.\is{focus} This will be shown by the discussion of the two remaining non-sortal nouns with a classifier (cf. \sectref{1sec:32}) plus two additional examples. The first example is on the [+unique, \minus relational] noun {\emph{đ{\daa}t}} `earth/ground'. In (\ref{1ex:23}), this noun is interpreted as definite by the classifier {\emph{mặt}}\footnote{{\emph{Mặt}} has the meaning of `face'. In this context, it is a classifier for objects with a flat surface. As a full noun, it can be interpreted as a [+relational] noun as in {\emph{mặt bàn}} [surface table] `the surface of the table'.} for flat surfaces because it has the function of contrastive focus.\is{focus!contrastive focus} The author of this text starts her story from the perspective of the protagonist, a farmer, who is up `on a tree' ({\emph{trên một cái cây}}). Having described a series of the farmer's actions up there, her attention suddenly moves to the position of the baskets `down on the ground' ({\emph{dưới mặt đ{\daa}t}}), which is contrasted to the position up on the tree.\footnote{One of our reviewers suggests that {\emph{dưới mặt đất}} `down on the ground' is a frame-setter\is{frame-setter} \citep[e.\,g.,][]{krifka:08}. This interpretation cannot be fully excluded. However, we would like to point out that the contrast\is{contrast} between the position `up in the tree' and the position `down on the ground' is clearly given in the way the scenes are presented in the film.}

\begin{exe}
\ex\label{1ex:23}
(Oral text 13, sentence 8)
\exi{}
\gll Ổng    leo      lên       một  cái  thang  để         ổng    leo    lên       một  cái cây  để \\
3.{\textsc{sg}}   climb {\textsc{prep}}   one  {\textsc{cl}} ladder  {so that}  3.{\textsc{sg}} climb {\textsc{prep}}  one  {\textsc{cl}} tree  to \\
\glt
\exi{}
\gll ổng    hái.    Ổng    hái    xong,  thì        ổng    leo      xu{\dao}ng  cái    thang   đó, \\
3.{\textsc{sg}} pluck  3.{\textsc{sg}}  pluck {\textsc{res}}   {\textsc{conj}}  3.{\textsc{sg}}  climb  down   {\textsc{cl}}   ladder   {\textsc{dem}}\\
\glt 
\exi{}
\gll xu{\dao}ng  đó.     R{\daob}i      ổng,    {\textbf{dưới}}   {\textbf{mặt}}  {\textbf{đ{\daa}t}}         sẽ     có       ba    cái   giỏ ...\\
down  {\textsc{dem}} {\textsc{conj}}  3.{\textsc{sg}}   down  {\textsc{cl}}   ground  {\textsc{fut}}  have  three {\textsc{cl}}   basket \\
\glt `He climbed a ladder to get on [a] tree to pick [the fruits]. Having picked [them], he went down [the] ladder. Then, he, down on the ground, there were three baskets...'
\end{exe}

In contrast to (\ref{1ex:23}), {\emph{đ{\daa}t}} `earth, ground' does not have a classifier in the non-contrastive situation of the following example:

\begin{samepage}
\begin{exe}
\ex\label{1ex:24}
(Oral text 28, sentence 20)
\exi{}
\gll Thì      có     ba     sọt   {trái cây} dưới  {\textbf{đ{\daa}t}},         \\
{\textsc{conj}} have three {\textsc{cl}}  fruit       under ground  \\
\glt
\exi{}
\gll không có     ai     {trông nom}  h{\dae}t. \\
{\textsc{neg}}   have who take-care    at.all \\
\glt `There were three baskets of fruit on the ground, but nobody was taking care of them.'
\end{exe}
\end{samepage}

Another context that induces classifier use is the context of focus particles,\is{focus!focus particles} which typically mark the inclusion or exclusion of alternatives \citep{konig:91}. The other two examples to be discussed here both belong to this type of focus.\is{focus} The first example (\ref{1ex:25}) is on the [+unique/+relational] noun {\emph{mặt}} `face', which occurs with the two focus particles\is{focus!focus particles} {\emph{chỉ còn}} `only' and {\emph{mỗi}} `only'. The noun {\emph{mặt}} `face' takes the position between these two particles to emphasize the fact that the foam covers almost the whole of the husband's body, leaving only his face unaffected. Thus, the two particles exhaustively single out one part of the body, which is excluded from the disturbing presence of foam:

\begin{exe}
\ex\label{1ex:25}
(Written text 29, sentence 31)
\exi{}
\gll Lúc  {b{\daa}y giờ}, người  ch{\daob}ng     nghe th{\daa}y   bèn      tr{\daob}i  lên khỏi     mặt       nước, toàn    thân  ông    là      {bọt xà phòng}  {chỉ còn}  th{\daa}y  mỗi   {\textbf{khuôn}} {\textbf{mặt}}. \\
time that        {\textsc{cl}}      husband hear   {\textsc{res}}  {\textsc{conj}}  rise  out {out of}  surface  water whole body 3.{\textsc{sg}} {\textsc{cop}}  foam                only      see   only  {\textsc{cl}}       face \\
\glt `At that time, the husband heard (the bell), then he moved out of the water. His whole body was full of soap foam, except {\textbf{[the] face}} [lit.: one can just only see {\textbf{[his] face}}].'
\end{exe}

{
Our next two examples are not included in the statistics in \tabref{1table:3} because they contain a possessive construction, and thus go beyond the distinction of bare noun vs. [CL+N]. In spite of this, they are relevant because classifiers very rarely occur with non-sortal head nouns of possessor constructions. In (\ref{1ex:26}), the [+unique, +relational] possessee head noun {\emph{ch{\daob}ng}} `husband' in {\emph{ch{\daob}ng của mình}} [husband CL self] `husband of her' takes the classifier {\emph{ông}}. Since non-sortal nouns of this type do not have a classifier in our data (e.\,g., {\emph{ch{\daob}ng (của) mình}} [husband (possessive marker) self-\textsubscript{reflexive pronoun}] `[her] husband', {\emph{vợ (của) mình}} [wife (possessive marker) self-\textsubscript{reflexive pronoun}] `[his] wife', {\emph{con trai họ}} [son (possessive marker) selves-\textsubscript{reflexive pronoun}] `[their] son', etc.), it is reasonable to assume that the presence of the classifier is due to the focus particle\is{focus!focus particles} {\emph{ngoài}} `except':
}

\begin{exe}
\ex\label{1ex:26}
(Written text 30, sentence 8)
\exi{}
\gll Khi    {giật mình}  vì            bị        tạt      nước,  bà  vợ    li{\daeb}n                {thức gi{\daa}c}  nhìn {xung quanh} xem  ai     làm  và        chả  có      ai    {\textbf{ngoài}}   {\textbf{ông}} ch{\daob}ng     của     mình. \\
when startle        because {\textsc{pass}}  throw water, {\textsc{cl}} wife immediately  awake       look  around         see   who do    {\textsc{conj}} {\textsc{neg}} have who except  {\textsc{cl}}  husband {\textsc{poss}}  self \\
\glt `Being startled by the water, the wife awoke immediately, looked around to see who did it. But there was nobody, except [her] husband.'
\end{exe}

Finally, the classifier even occurs with non-sortal head nouns of possessive constructions, if the relevant focus\is{focus} situation can only be derived from context without the explicit presence of a focus marker.\is{focus!focus marker} This is illustrated by (\ref{1ex:27}), in which we find the two non-sortal nouns {\emph{chân}} `foot' and {\emph{mông}} `buttocks', the former without a classifier, the latter with a classifier. The interpretation of this sentence crucially depends on the function of the adverbial subordinator {\emph{nên}} `therefore, to the extent that' which creates a context in which the situation becomes worse and worse until it culminates in a rather unexpected situation, in which the husband even burns his buttocks. This situation can be compared to the situation created by a focus particle\is{focus!focus particles} like {\emph{even}}:

\begin{exe}
\ex\label{1ex:27}
(Written text 13, sentence 35)
\exi{}
\gll {Ông {\daa}y}  bị        bỏng  và         đau  quá   nên       ôm    {\textbf{chân}}  lên    và   không  giữ  \\
3.{\textsc{sg}}     {\textsc{pass}}  burn   {\textsc{conj}}  hurt  very  {\textsc{conj}}   hold  foot   {\textsc{res}}  and {\textsc{neg}}   keep \\
\glt
\exi{}
\gll được	 {thăng bằng}   nên      té   vào   chi{\dae}c chảo đang    cháy  đó,       {\textbf{cái}}   {\textbf{mông}} \\
{\textsc{res}}  balance          {\textsc{conj}}  fall into   {\textsc{cl}}    pan   {\textsc{prog}} burn   {\textsc{dem}}  {\textsc{cl}}  buttock \\
\exi{}
\gll {\textbf{của}}     {\textbf{ông {\daa}y}}  đã        bị        phỏng. \\
{\textsc{poss}} {\textsc{3.sg}}     {\textsc{perf}}  {\textsc{pass}}  burn.\\
\glt `He got burnt and he got hurt, therefore, he lifted [his] leg up to hold it, then he was no longer able to keep his balance to the extent that he fell down on [the] burning pan and [as a consequence] even [his] buttocks got burnt.'
\end{exe}


\section{Classifiers and indefiniteness}\label{sec:4}

Classifiers with indefinite interpretation are limited to particular contexts: the indefinite function of classifiers in the subject position of thetic statements\is{thetic statements} is presented in \sectref{1sec:41}.  \sectref{1sec:42} discusses the [CL+N] construction in existential clauses,\is{existential sentences} while \sectref{1sec:43} describes [CL+N] constructions in combination with verbs of appearance.

\subsection{Thetic statements\is{thetic statements}}\label{1sec:41}

As can be seen from \tabref{1table:4}, indefinite subjects are rather rare: 97.8\% of the preverbal [CL+N] constructions of the written and the spoken corpus together are definite (cf. \sectref{1sec:33}). The vast majority of the remaining 2.2\% of indefinite preverbal [CL+N] constructions are subjects of thetic constructions\is{thetic statements} \citep{kuroda:72,sasse:87,sasse:95}. Thetic utterances\is{thetic statements} are seen in contrast to categorical utterances.\is{categorical statement} \cite{sasse:95} defines both types as follows:\\

\begin{quote}
Categorical utterances\is{categorical statement} are said to be bipartite predications, involving a {\textbf{predication base}}, the entity about which the predication is made, and a {\textbf{predicate}}, which says something about the predication base. In other words, one of the arguments of the predicate is picked out as a ``topic''\is{topic} in the literal sense, namely, an object about which something is asserted. Thetic utterances,\is{thetic statements} on the other hand, are {\textbf{monomial}} predications (called ``simple assertions'' in \citealt{sasse:87}); no argument is picked out as a predication base; the entire situation, including all of its participants, is asserted as a unitary whole. \citep[][4-5]{sasse:95}
\end{quote}

In utterances of this type, the entire clause is an `all-new' utterance that is seen as inactivated information (often backgrounded) that is assumed by the speaker not to be present in the hearer’s mind. Thus, nominal participants of thetic utterances\is{thetic statements} are generally indefinite. The following two examples constitute the beginning of the story as told by two different informants. They provide a description of the initial scene as it was presented in the film. In the first sentence of both examples, the subject {\emph{đ{\daob}ng h{\daob} báo thức}} `alarm clock' is marked by the classifier {\emph{chi{\dae}c}}. Similarly, the subject {\emph{đàn ông}} `man' has the default classifier for humans, {\emph{người}}, in the second sentence of both examples:

\begin{exe}
\ex\label{1ex:28}
Indefinite Subject (Written text 12, sentence 1, 2)
\exi{}
\gll {\textbf{Chi{\dae}c}} {\textbf{đ{\daob}ng h{\daob}}} {\textbf{báo thức}}  reo   lên  lúc   8         giờ        đúng. \\
{\textsc{cl}}     clock       alarm       ring  up   at     eight  o'clock  exactly \\
\glt 
\exi{}
\gll {\textbf{Người}} {\textbf{đàn ông}}  đang    ngủ    thì        bị       nước   {văng tung tóe}   vào     mặt. \\
{\textsc{cl}}      man         {\textsc{prog}} sleep  {\textsc{conj}}  {\textsc{pass}}  water  splatter            {\textsc{prep}} face \\
\glt `The alarm clock rang at exactly eight o'clock. There was a man, who was sleeping and then [his] face was splattered with water.'
%%%
\ex\label{1ex:29}
Indefinite Subject (Written text 1, sentence 1, 2)
\exi{}
\gll {\textbf{Chi{\dae}c}} {\textbf{đ{\daob}ng h{\daob}}} {\textbf{báo thức}}   reo   lên     {báo hiệu}    đã 	       tám 	giờ 	     sáng. \\
{\textsc{cl}}     clock      alarm         ring  {\textsc{res}}  signaling   {\textsc{perf}}  eight	o'clock  morning \\
\glt 
\exi{}
\gll {\textbf{Người}} {\textbf{đàn ông}} 	mở    mắt 	{li{\dae}c nhìn} 	sang 	     vợ-mình. \\
{\textsc{cl}}      man          open eye 	glance        toward    wife-self \\
\glt `[The] alarm clock rang to signal that it was already 8 o'clock in the morning. [A] man opened his eye and glanced at [his] wife.'
\end{exe}


\subsection{Existential expressions}\label{1sec:42}
Existential sentences\is{existential sentences} of the type `there is an X' are typically used to introduce previously unidentified referents. Thus, [CL+N] constructions occurring in this type of construction are typically indefinite. Since they are positioned after the verb, they form a considerable part of the indefinite object classifiers in our data (but cf. inanimate nouns below). A good example is (\ref{1ex:30}) from our oral corpus, in which the [CL+N] construction is preceded by the verb {\emph{có}} `have, there is':

\begin{exe}
\ex\label{1ex:30}
(Written text 20, sentence 28)
\exi{}
\gll Lúc  này,     có    {\textbf{viên}} {\textbf{cảnh sát}}     vào    hỏi   xem  {tình hình}  vì             hai \\
time {\textsc{dem}}  have {\textsc{cl}}   policeman  enter  ask   see   situation  because   two \\
\glt
\exi{}
\gll vợ      ch{\daob}ng      cãi     nhau. \\
wife   husband  argue {\textsc{recip}} \\
\glt `This time, [a] policeman entered and asked why this couple was arguing with each other.'
\end{exe}

Another verb that implies indefiniteness is the copula verb {\emph{là}} `be', which is used in identificational contexts (`this is an X') as well as in locative contexts (`Y is [placed] in/at/on an X').  The following example starts out with a locative expression\is{locatives} in the topic position\is{topic position} ({\emph{bên cạnh đó}} `at the side of it, beside'). The three subsequent objects following the copula {\emph{là}} are introduced as previously unmentioned elements into the scene by being situated within that locative\is{locatives} topic:\is{topic}


\begin{exe}
\ex\label{1ex:31}
(Written text 14, sentence 2)
\exi{}
\gll {Bên cạnh}  đó       là      {\textbf{cái}}  {\textbf{kệ}}      {\textbf{nhỏ}},    {\textbf{cái}}  {\textbf{bình}}     và    {\textbf{ly}}   {\textbf{nước}} được    đặt     lên           trên  nó.\\
beside      {\textsc{dem}}  {\textsc{cop}}  {\textsc{cl}}  shelf  small  {\textsc{cl}}  bottle   and  {\textsc{cl}} water {\textsc{pass}}  place  move.up  top    3.{\textsc{sg}}\\
\glt {`Beside [him] was a small shelf with a bottle and a glass of water placed on it.'} \\[1mm]
{\emph{Previous Context:}} The man who wore glasses awoke, opened his eyes for a moment, had a look around himself, ignored the alarm clock and went on sleeping.
\end{exe}

In contrast to the thetic utterances\is{thetic statements} of the preceding subsection, existential constructions\is{existential sentences} can also be combined with constructions other than [CL+N]. For that reason, their impact on postverbal indefinite classifiers in our data is less strict than the impact of thetic utterances\is{thetic statements} on indefinite classifiers in the subject position. As is shown by the following example, existential expressions can also occur with the [{\emph{một}} `one'+CL+N] construction:

\begin{exe}
\ex\label{1ex:32}
(Oral text 20, sentence 1, 2)
\exi{}
\gll Câu  chuyện  được     {bắt đ{\daab}u}  vào  một   {buổi sang}   tại  một  cánh đ{\daob}ng, \\
{\textsc{cl}}    story     {\textsc{pass}}   begin      in    one   morning     at    one  {\textsc{cl}}  field \\
\glt 
\exi{} 
\gll có     {\textbf{một}} {\textbf{người}} {\textbf{nông dân}} leo      lên trên một cái thang,   đang     hái   \\
have one  {\textsc{cl}}      farmer      climb up  on    one {\textsc{cl}} ladder  {\textsc{prog}}  pluck \\
\glt
\exi{}
\gll một  loại  {trái cây}  nào đó            gi{\dao}ng  trái  lê. \\
one  kind  fruit      some certain   like     {\textsc{cl}}   pear \\
\glt `The story began in a morning in a field. There was a farmer, who was climbing up a ladder to pick a kind of fruit like a pear.'
\end{exe}

Finally, there are also some instances of inanimate nouns which occur without a classifier in existential constructions.\is{existential sentences} This is illustrated by the following example with the noun {\emph{xe cứu hoả}} `fire truck' in its bare form:

\begin{exe}
\ex\label{1ex:33}
(Written text 20, sentence 25)
\exi{}
\gll {G{\daab}n đó},  có   {\textbf{xe}}  {\textbf{cứu}}  {\textbf{hỏa}}   và        {lập tức}           đ{\dae}n      xịt     nước  vào  chữa cháy nhưng  {làm cho}  {mọi thứ}       {hỏng h{\dae}t}.\\
nearby  have car save fire    {\textsc{conj}}  immediately  arrive  spray water into extinguish fire {\textsc{conj}}  cause      everything  ruin \\
\glt `Nearby, there was [a] fire truck, it arrived immediately to extinguish the fire. However, it also ruined everything.'
\end{exe}

The extent to which the use of the classifier ultimately depends on the animacy\is{animacy} of the noun cannot be determined from our data because we do not have enough examples.\footnote{In an alternative analysis, readers may be tempted to argue that the absence of the classifier is related to the complexity of the head noun (compounds vs. simple nouns) or to its status as a lexical item borrowed from \ili{Chinese}. Since \cite{emeneau:51}, it has often been claimed that nouns of this type take no classifiers. In spite of this, the noun {\emph{cảnh sát}} `policeman', which is borrowed from \ili{Chinese}
% \includegraphics[height=3mm]{chapters/jingcha.png}
{\cn 警察}
{\emph{jǐngchá}} `police(man)', does occur with the classifier {\emph{viên}} in (\ref{1ex:30}). Thus, we can at least exclude borrowing from \ili{Chinese} as a strong factor for determining classifier use in existential constructions.\is{existential sentences} In (\ref{1ex:31}) it seems that animacy\is{animacy} is more important. Ultimately, more data would be needed to enable more precise conclusions to be reached.}

\subsection{Verbs and situations of appearance}\label{1sec:43}
Vietnamese has quite a few verbs with the meaning of `appear, come up', `turn out to be' or `reveal', whose subsequent nouns introduce previously unidentified elements into the discourse. In such cases, the postverbal noun is indefinite. In the following example with the verb {\emph{lòi ra}} `come to light, appear', the noun {\emph{tẩu thu{\dao}c}} `smoking pipe' takes the general classifier {\emph{cái}}. Since the pipe was hidden in the husband's pocket, it is unknown to the audience/reader of the text and is interpreted as indefinite.

\begin{exe}
\ex\label{1ex:34}
(Written text 15, sentence 61)
\exi{}
\gll Nhưng sau  đó,   cái {túi áo}  của     ông ch{\daob}ng    bị       lủng,  lòi      ra  {\textbf{cái}} {\textbf{tẩu thu{\dao}c}},      chứ            {không phải}  {vật gì}          {có thể}  gây     {nguy hiểm}.\\
{\textsc{conj}}   after that  {\textsc{cl}} pocket  {\textsc{poss}}  {\textsc{cl}} husband   {\textsc{pass}}  burst,  show   out  {\textsc{cl}} smoking-pipe {\textsc{conj}}\textsubscript{emph}  {\textsc{neg}}            something  can      cause  danger \\
\glt `However, after that, his pocket burst and what came to light was a smoking pipe, definitely nothing that may cause any danger.'
\end{exe}

Sometimes, the meanings of verbs implying the emergence of unidentifiable concepts are highly specific. This can be shown by the verb {\emph{v{\daa}p}} `trip, walk into, stumble over', which creates a situation in which the object is unpredictable and has the status of being unidentifiable as in the following example:

\begin{samepage}
\begin{exe}
\ex\label{1ex:35}
(Oral text 25, sentence 13)
\exi{}
\gll {Mải mê}         nhìn gái nên      nó     v{\daa}p   phải     \\
passionately look girl {\textsc{conj}} 3.{\textsc{sg}}  trip  {\textsc{pass}}  \\
\glt
\exi{}
\gll {\textbf{cục}} {\textbf{đá}}      và        té   xu{\dao}ng  đường. \\
{\textsc{cl}}  stone  {\textsc{conj}} fall  down  road \\
\glt `[He] looked at the girl passionately and thus stumbled over [a] stone and fell down on the road.'
\end{exe}
\end{samepage}

{
Thus, the object {\emph{đá}} `stone' is marked by the classifier {\emph{cục}} in (\ref{1ex:35}). The boy, who is one of the two protagonists in the story, as well as the audience, cannot know what will happen when the boy is looking at the girl rather than at the road while riding his bike. The stone is clearly not activated and is interpreted as indefinite. 
}

\section{Conclusion}\label{1sec:5}
The aim of this study was to reach a better understanding of the referential functions of Vietnamese classifiers based on the systematic analysis of data from a corpus of written and oral texts which was designed to generate a broad variety of contexts which may trigger classifier use. The main results on the use and the functions of the Vietnamese classifier in [CL+N] can be summarized as follows:

\begin{itemize}
%\setlength\itemsep{-0.3em}
\item[(i)] 
{
Classifiers can be interpreted as definite as well as indefinite but there is a clear preference for using the classifier in definite contexts (cf. \sectref{1sec:31}).
}
%
\item[(ii)] There is a clear clustering of animacy\is{animacy} and definiteness: definite animate nouns occur much more frequently with a classifier than definite inanimate nouns (\sectref{1sec:31}).
%
\item[(iii)] There is a clear clustering of [CL+N] with [+definite, +animate, subject] and of bare nouns [N] with [\minus definite, \minus animate, object] (\sectref{1sec:33}).
%
\item[(iv)] The overwhelming majority of nouns occurring in the [CL+N] construction are sortal nouns\is{noun types!sortal noun} [\minus unique, \minus relational] (\sectref{1sec:32}).
%


\item[(v)] Discourse and information structure\is{information structure} play an important role in the function as well as in the presence/absence of a classifier:
%
	\begin{itemize}
	\item[a.] The definiteness with which classifiers are associated in [CL+N] is based on identifiability\is{identifiability} in discourse (\sectref{1sec:341});
	\item[b.] Information structure\is{information structure} is an important factor for determining the use of a classifier in [CL+N] (\sectref{1sec:342} and \sectref{1sec:343}) and its interpretation in terms of definiteness vs. indefiniteness (particularly cf. \sectref{1sec:41} on indefiniteness and theticity).
	\end{itemize}
%	
\item[(vi)]	There are certain semantic environments which support the indefinite interpretation of the classifier (existential clauses\is{existential sentences} and verbs of appearance; \sectref{1sec:42} and \sectref{1sec:43}).
\end{itemize}


The results in (i) to (iii) on animacy,\is{animacy} definiteness and subject/preverbal position tie in with general findings on prominence\is{prominence} at the level of the morphosyntax-semantics interface as they manifest themselves in hierarchies like the animacy hierarchy\is{animacy!animacy hierarchy} \citep{silverstein:76,dixon:79} or the accessibility hierarchy\is{accessibility hierarchy} \citep{keenan:comrie:77} \citep[for a survey, cf.][]{bornkessel:schlesewsky:09}.\footnote{Based on the relevance of (in)definiteness and animacy,\is{animacy} one may think of analyzing the use of the classifier in [CL+N] in the light of Differential Object Marking (DOM)\is{differential object marking} as suggested by one of our reviewers. In our view, such an account would be problematic for at least the following reasons: (i) The use of the classifier in the [CL+N] construction is strongly associated with sortal nouns\is{noun types!sortal noun} ([\minus relational]/[\minus unique]), while DOM marking\is{differential object marking} is not limited to this type of nouns. (ii) As pointed out by \cite[][439]{aissen:03}, ``it is those direct objects which most resemble typical subjects that get overtly case-marked''. If one takes the use of the classifier as a DOM marker,\is{differential object marking} one would expect the highest frequency of classifier use with [+definite] and [+animate] objects. This is clearly not borne out in the case of definiteness. As can be seen from \tabref{1table:4}, the ratio of definite subjects with CL is much higher than the ratio of definite objects with CL. There are 1,329 [= 1012 + 317] definite subjects with CL vs. 98 [= 86 + 12] definite subjects with no CL, i.\,e., 93.3\% of the definite subjects in our two corpora have a classifier. In contrast, only 34.1\% of the definite objects have a classifier (795 [= 432 + 363] definite objects with CL contrast with 1,533 [= 1206 + 327] without CL). In the case of animacy,\is{animacy} the difference between the two ratios is smaller but it is still higher with animate subjects. As can be seen from \tabref{1table:6}, there are 1,259 [= 997 + 262] animate subjects with CL and only 10 [= 9 + 1] animate subjects with no CL, i.\,e., 99.2\% of the animate subjects have a classifier. In the case of animate objects, the ratio is 72.7\% (312 [= 179 + 133] animate objects with CL contrast with 117 [= 114 + 3] animate objects with no CL). (iii) The results discussed in (ii) are remarkable from the perspective of split vs. fluid DOM languages\is{differential object marking} in terms of \cite{dehoop:malchukov:07}. In split languages, DOM marking\is{differential object marking} is obligatory for a particular feature, while it is optional in fluid systems. In most DOM languages,\is{differential object marking} DOM\is{differential object marking} is split for at least one category. As can be seen in (ii), this is not the case with the use of the classifier. Vietnamese classifiers are not obligatory with definite objects nor are they obligatory with animate objects.} The clustering observed in (ii) and (iii) additionally reflects a universal tendency to associate animate subjects in clause-initial positions of SVO languages with definiteness (\citealt{keenan:comrie:77}; \citealt{givon:79}; \citealt{dubois:87}; and many others). This tendency is also well known for word order in Sinitic languages \il{Sinitic languages} \citep{li:thompson:76,sun:givon:85,lapolla:95}. \cite[][1166]{chen:04} talks about definiteness-inclined preverbal positions and indefiniteness-inclined postverbal positions in Mandarin Chinese.\il{Chinese!Mandarin} As can be seen from (i), word order does not determine the (in)definiteness interpretation of the classifier in Vietnamese as rigidly as it does in Cantonese\il{Chinese!Cantonese} or in the Wu\il{Chinese!Wu} dialect of Fuyang (cf. the discussion of (\ref{1ex:11}) and (\ref{1ex:12}); for the discourse-based reasons for this, cf. below).\footnote{In the case of Sinitic,\il{Sinitic languages} \cite{li:bisang:12} argue that the definiteness interpretation of subjects is due to a process of grammaticalization\is{grammaticalization} in which the definiteness properties of the topic position\is{topic position} were passed on to the subject position (cf. the classical grammaticalization\is{grammaticalization} pathway from information structure\is{information structure} to syntax in \citealt{givon:79}). In a similar way, the observation that postverbal [CL+N] constructions are preferably indefinite but do not exclude definiteness in Sinitic \il{Sinitic languages} can be derived from the association of informational focus\is{focus} with the postverbal position \citep{xu:04}. As \cite[][262]{lambrecht:94} points out, focus\is{focus} differs from topic\is{topic} inasmuch as it is not necessarily identifiable or pragmatically salient in discourse. For that reason, it is open to indefinite and definite interpretation even though the default interpretation is indefinite. If this analysis is true, one may argue that in Sinitic \il{Sinitic languages} the classifier in [CL+N] is like a variable that takes on the [\pm definite] function that corresponds to its syntactic position if it is not overwritten by stronger factors. In Vietnamese, such a syntactic scenario turns out to be problematic because the classifier generally favours definite interpretation (cf. point (i)).}

{
The observation in (iv) that the vast majority of nouns occurring in the [CL+N] construction are sortal nouns\is{noun types!sortal noun} in the terms of \cite{lobner:85} confirms and further specifies the findings of \cite[][324]{simpson:17} on the Wu\il{Chinese!Wu} variety of Jinyun, that nouns denoting ``specifically unique individuals/elements'' predominantly appear as bare nouns [N] (cf. the three instances of [+unique] nouns taking a classifier in \tabref{1table:3}). These results show the potential relevance of L\"obner's (\citeyear{lobner:85,lobner:11}) four basic types of nouns for understanding definiteness/indefiniteness as associated with the [CL+N] construction in East \il{East Asian languages} and mainland Southeast Asian languages.\il{Southeast Asian languages}
}

Even though the factors of semantics (animacy,\is{animacy} uniqueness,\is{uniqueness} relationality)\is{relationality} and syntax (subject, object) clearly have an impact on the presence or absence of the classifier in contexts of definiteness and indefiniteness, we have evidence that discourse and information structure\is{information structure} are stronger than these factors. The dominance of discourse is reflected in the very function of the classifier itself. As discussed in \sectref{1sec:341}, classifiers mark identifiability\is{identifiability} rather than uniqueness\is{uniqueness} (cf. point (v.a), also cf. \citealt{li:bisang:12} on Sinitic).\il{Sinitic languages} Thus, they express pragmatic definiteness\is{definiteness!pragmatic} rather than semantic definiteness\is{definiteness!semantic} in terms of \cite{lobner:85,lobner:11} or anaphoric (``strong'') definiteness\is{definiteness!anaphoric}\is{definiteness!strong} rather than unique (``weak'') definiteness\is{definiteness!weak} in terms of \cite{schwarz:09,schwarz:13}. In addition to the discourse-based definiteness expressed by the classifier, contrastive topics\is{topic!contrastive topic} (\sectref{1sec:342}), as well as contrastive focus\is{focus!contrastive focus} and focus particles\is{focus!focus particles} (\sectref{1sec:343}), enhance the use of the [CL+N] construction. Thetic statements,\is{thetic statements} as another instantiation of information structure,\is{information structure} play an important role in the indefinite interpretation of [CL+N] in the subject position (\sectref{1sec:41}; also cf. (v.b)). Moreover, there are more specific discourse-based environments as mentioned in point (vi) which support the use of a classifier in contexts of indefinite interpretation (\sectref{1sec:42} and \sectref{1sec:43}). Finally, evidence of the dominance of discourse comes from data outside of our corpus. In order to disentangle the semantic effects of animacy\is{animacy} vs. discourse effects associated with protagonists, we looked for narrative texts with inanimate protagonists. In the three texts we found, the inanimate protagonists generally occur in the [CL+N] construction \citep{quang:forth}. One of the stories is about a flying carpet, which is already mentioned in the title, {\emph{T{\daa}m thảm bay}} [CL carpet fly] `The Flying Carpet'.\footnote{The story was published by Viet Nam Education Publisher in 2003.} After the protagonist is introduced by an indefinite construction of the type [one CL N], the noun {\emph{thảm}} `carpet' consistently occurs with a classifier. It is important to add in this context that the carpet has no anthropomorphic properties in the story, i.\,e., it does not act in any way. It is just the element that keeps the story going through many different events and episodes. Needless to say, such examples are hard to find in a corpus, no matter how large it is, because they are rare overall. The fact that even inanimate protagonists generally can take a classifier together with the findings summarized in (v) are good evidence for the dominance of discourse and information structure\is{information structure} over semantics and syntax.

Taking these findings together, the classifier in [CL+N] is used as a variable whose use and interpretation depend on prominence in discourse\is{discourse prominence} and interact with factors from the morphosyntax-semantics interface. The details of that interaction will undoubtedly need more research. What is remarkable and makes the data on Vietnamese and other East \il{East Asian languages} and mainland Southeast Asian languages \il{Southeast Asian languages} particularly relevant from a typological perspective is the observation that the different factors associated with (in)definiteness are well known, while cross-linguistic variation in how they interact is still under-researched. In Vietnamese, factors of discourse are particularly prominent. In order to further corroborate these observations and compare them with the situation in other mainland \ili{Southeast Asian languages}, it is necessary to look at how classifiers are used in actual discourse in text corpora. We understand the corpus discussed here as a starting point for Vietnamese.\il{Vietnamese|)}\is{classifiers|)}

\section*{Acknowledgements}
We would like to thank the editors for their time and support in the editorial processes. Comments from two anonymous reviewers have greatly improved the content. We owe special thanks to our 46 informants and friends in Ho Chi Minh city, Vietnam, for their help in participating in our experiments.

{\sloppy\printbibliography[heading=subbibliography,notkeyword=this]}
\end{document}

