\documentclass[output=paper]{langsci/langscibook}
\ChapterDOI{10.5281/zenodo.4049683}

\author{Eve Danziger\affiliation{University of Virginia}\lastand Ellen Contini-Morava\affiliation{University of Virginia}}
\title{Referential anchoring without a definite article: The case of Mopan (Mayan)}  

\abstract{Mopan Maya\il{Mopan (Mayan)|(} is a language in which pragmatic factors play a significant role in referential anchoring.\is{referential anchoring} Its article occurs in both definite and indefinite contexts, and so do bare nominals.\is{bare nominal} We discuss several forms that assist with referential anchoring,\is{referential anchoring} using Dryer's (\citeyear{dryer:14}) reference hierarchy\is{reference hierarchy} as an organizing framework, but none of these forms is obligatory for any of the functions in the hierarchy. Rather than explicitly encoding, e.\,g., definiteness or specificity,\is{specificity} their employment is sensitive to factors such as discourse salience.\is{discourse salience}}

\begin{document}
\maketitle

\section{Introduction}\label{3sec:1}
{
It is now well documented (e.\,g., \citealt{sasse:88}; \citealt{matthewson:98}; \citealt{gillon:09,gillon:13}; \citealt[][e201-e207]{davis:gillon:matthewson:14}; \citealt{lyon:15}) that languages exist in which determiners do not signal semantic gradations of relative `definiteness' \citep[degrees of identifiability\is{identifiability} and uniqueness,\is{uniqueness} see][]{hawkins:78,lobner:85,lobner:11,lyons:99,dryer:14}, as they do in most European languages (for example, by the contrast between English\il{English} {\emph{the}} and {\emph{a/an}}).  The question therefore arises whether degrees of definiteness are explicitly signaled in such languages, or if not, how related messages can be conveyed.  In the following, we discuss the case of Mopan Mayan (Yukatekan), a language in which the form that fills distributional criteria to be an article does not encode the semantic concept of definiteness or the related concepts of specificity\is{specificity} and uniqueness\is{uniqueness} \citep{contini:morava:danziger:fc}. We address the various means which are used in Mopan to indicate relative identifiability\is{identifiability} and uniqueness,\is{uniqueness} using a scale developed by \cite{dryer:14} specifically to handle the notions `definite/indefinite' in typological comparison.\footnote{Other scales and metalanguages, such as Gundel et al.'s (\citeyear{gundel:hedberg:zacharski:93}) givenness\is{givenness} hierarchy or L\"obner's (\citeyear{lobner:11}) uniqueness\is{uniqueness} scale, would have been reasonable alternatives. For present purposes, Dryer's hierarchy has the advantage in that it pursues degrees of `indefiniteness' as well as of `definiteness', and does not deal with contrasts other than those of `definiteness'. (For example, `relationality'\is{relationality} is not a dimension on Dryer's scale.)} We describe a number of forms which can be used to indicate relatively high or low degrees of identifiability,\is{identifiability} and we also document the fact that ART can occur at every position on Dryer's hierarchy, thus confirming that ART does not usefully convey information about identifiability\is{identifiability} or uniqueness.\is{uniqueness}
}

{
\hspace*{-1mm}We also note however that in very many Mopan discourse cases, explicit means of indicating the status of a referent vis-\`a-vis identifiability\is{identifiability} and uniqueness\is{uniqueness} are not in fact employed. We show that in all positions on Dryer's hierarchy, referents can be expressed by unmarked, or `bare', nominals,\footnote{The distinction between `noun' and `verb' as separate lexical classes is problematic in Mopan (\citealt{danziger:08}; see further below), but for ease of reference we will use the terms `noun' and `nominal' to mean `word understood as serving in a given utterance as an argument in the predication, as possessor in a possessive phrase, or as object of a preposition', and `verb' to mean `word understood as serving as predicator in a given utterance'.} and therefore no explicit information is provided about degrees of identifiability\is{identifiability} or uniqueness.\is{uniqueness} We conclude that the identification of referents in terms of degrees of previous mention, uniqueness,\is{uniqueness} specificity,\is{specificity} or familiarity\is{familiarity} to speech participants is not always explicitly formulated in Mopan. When this is the case, calculation of these properties of referents must be accomplished, if it is accomplished at all, by pragmatic means.
}

\subsection{Resources for referential anchoring\is{referential anchoring} in Mopan}\label{3sec:11}
In Mopan, information about the referential status\is{referential status} of argument expressions can be provided in a variety of ways. These include:

\begin{itemize}
\setlength\itemsep{-0.1em}
\item[(a)] use of the article (ART), which explicitly signals that the associated constituent is to be construed as an entity, and hence an argument; 
\item[(b)] use of a demonstrative\is{demonstratives} expression involving one of four stative deictic predicates\is{stative predicates} that specify proximity, visibility, and states of prior knowledge to various speech-act participants and discourse referents \citep{danziger:94}, with or without an accompanying NP; 
\item[(c)] use of the emphatic 3rd person pronoun {\emph{le'ek}}; %\\
\item[(d)] use of a numeral classifier phrase;\is{classifiers!numeral classifiers}
\item[(e)] a bare nominal:\is{bare nominal} absence of any explicit specification of referential status.\is{referential status} 
\end{itemize}

The chapter is organized as follows.  In \sectref{3sec:2}, we provide distributional and semantic characterization of the above forms.  In \sectref{3sec:3}, we introduce Dryer's (\citeyear{dryer:14}) reference hierarchy,\is{reference hierarchy} which we use as a framework for fuller description of the `definiteness/indefiniteness' functions of the forms listed above.  We show how some of the listed means of expression are restricted to certain portions of the scale, meaning that they can be characterized as conveying degrees of definiteness or indefiniteness. But we also show that both ART and `bare nominal'\is{bare nominal} can occur in any position on the hierarchy. This means on the one hand that ART does not usefully convey degrees of `definiteness', and on the other that  none of the definiteness-conveying means which we also document is actually required, even when its preferred segment of the scale is at issue in a given utterance. That is, it is often the case that degrees of, e.\,g., anaphora, specificity,\is{specificity} etc. are pragmatically inferred rather than semantically conveyed by reliance on dedicated grammatical forms. \sectref{3sec:4} provides a summary and conclusion. A table summarizing the data appears in this final section.

Our data are drawn primarily from Mopan narratives, including 75 narratives from eight speakers collected by Pierre Ventur in Guatemala in the 1970s \citep{ventur:76}, 14 texts of varied kinds from ten speakers collected by Matthew and Rosemary Ulrich \citep{ulrich:ulrich:82}, and narratives collected in Belize more recently by Eve Danziger (p.c.) and by Lieve Verbeeck \citep{verbeeck:99}. We also draw on conversational data elicited by Eve Danziger from Mopan speakers in Belize \citep{danziger:94}.


\section{Descriptive preliminaries}\label{3sec:2}

{
Mopan is a Mayan language spoken by several thousand people living in communities that span the Belize-Guatemala border in Eastern Central America.  It is a predominantly head-marking, predicate-initial language. Pluralization is optional, numeral classifiers\is{classifiers!numeral classifiers} are required for enumeration, and there is no copular verb.
}

\subsection{The article (ART) {\emph{a}} }\label{3sec:21}
The Mopan lexicon is characterized by the neutrality of many lexical items in relation to the traditional distinction between noun and verb \citep{danziger:08}.  Many lexical items which would translate into English\il{English} as nouns may play the role of a clause predicate without derivation. Such items fall into the category of `statives'\is{stative predicates} in Mopan (see \citealt{danziger:96}; for similar observations in other Yukatekan languages\il{Yucatecan languages} see \citealt{bricker:81}; \citealt{lucy:94}; \citealt{lois:vapnarsky:06}).  This can be seen in example (\ref{3ex:1a}), where the lexeme {\emph{winik}}, inflected with the pronominal suffix from the series known to Mayanists as Set B, is interpreted as a stative predicate\is{stative predicates} (`be a man').  In other contexts, such as (\ref{3ex:1b}), the same lexeme is construed as an argument (`the man' or `a man').  Note the presence of ART {\emph{a}} before {\emph{winik}} in example (\ref{3ex:1b}).

\begin{exe}
\ex\label{3ex:1}
Noun-verb neutrality in Mopan.
	\begin{xlista}
	\ex\label{3ex:1a}
	Stative lexeme with 2nd person Set B inflection.\footnote{Orthography is as preferred by the Academ\'ia de las Lenguas Mayas de Guatemala (ALMG, \citealt{england:elliott:90}). Interlinear glosses follow the Leipzig glossing rules (http://www.eva.mpg.de/lingua/resources/glossing-rules.php), with some additions; see Abbreviations at the end of the chapter.}\newline 
%the following additions: {\textsc{a}} = set A case-role marker (Actor or Possessor),  {\textsc{anim}} = animate, {\textsc{b}} = set B case-role marker (Undergoer), {\textsc{d}} = deictic, {\textsc{disc}} = discourse particle, {\textsc{emph}} = emphatic pronoun, {\textsc{ev}} = prosodic echo vowel, {\textsc{gm}} = gender marker, {\textsc{hsy}} = hearsay particle, {\textsc{inan}} = inanimate, {\textsc{inch}} = inchoative, {\textsc{int}} = intensifier, {\textsc{nv}} = non visible, {\textsc{poss}} = Possessive, {\textsc{prep}} = general preposition, {\textsc{scope}} = scope clitic, {\textsc{t}} = transitivizer, {\textsc{v}} = visible} \\
	[Author's data, {\emph{Ix Che'il etel B\"ak'}} `Wild Woman', J.\,S.]
	\exi{}
	\gll	inchech=e 		tan-{∅}				inw				il-ik-{\textbf{ech}}. \\
		2.{\textsc{emph=ev}}	be\_continuing-{\textsc{3b}}	1{\textsc{a}}(prevocalic)	see-{\textsc{tr.ipfv-{\textbf{2b}}}} \\
	\glt
	\exi{}
	\gll	winik-{\textbf{ech}}. \\
		man-{\textbf{{\textsc{2b}}}} \\
	\glt `As for you, I am looking at {\textbf{you}}. {\textbf{You're (a) man}}.'
	%%
	\ex\label{3ex:1b}
	Same lexeme with ART. \\
	$\lbrack$Source as in (\ref{3ex:1a})$\rbrack$
	\exi{}
	\gll ``...''	kut'an		{\textbf{a}}	{\textbf{winik}}		t-uy						\"atan=a. \\
		{}	3.{\textsc{quot}}	{\textsc{art}}		man			{\textsc{prep}}-3{\textsc{a}}(prevocalic)	wife={\textsc{ev}} \\
	\glt ` ``...'' said {\textbf{the [{\textsc{art}}] man}} to his wife.'
	\end{xlista}
\end{exe}

ART designates an entity that instantiates the content of the accompanying constituent (see \citealt{contini:morava:danziger:fc} for details). As such, it helps to distinguish arguments unambiguously from predicates.

{
We show below that ART does not usefully convey semantic contrasts on the definiteness dimension. If this is the case, are we justified in calling it an `article'?  Although some have argued that semantic criteria such as definiteness or specificity\is{specificity} are necessary for defining the category `article' \citep[e.\,g.,][833-4]{himmelmann:08}, others foreground distributional criteria.  For example, \cite[][158]{dryer:07} states that the term `article' can be applied to ``a set of words which occur with high frequency in noun phrases and which vary for certain grammatical features of the noun phrase''.\footnote{In his WALS study Dryer defines `articles' more narrowly as ``words or morphemes that occur in noun phrases…[that] must code something in the general semantic domain of definiteness or indefiniteness'' (\citeyear{dryer:14}: e234), but this was for the purpose of surveys specifically of definite and indefinite articles.} Mopan ART occurs in a fixed position preceding expressions that are to be construed as nominals. It is also in complementary distribution with forms that function as possessive pronouns (the pronoun series known to Mayanists as Set A), again suggesting determiner status.\footnote{A possessive construction involves two referents, each of which may require its own referential anchoring.\is{referential anchoring}  As such, they do not fit easily into Dryer's (\citeyear{dryer:14}) reference hierarchy,\is{reference hierarchy} used below as an organizing framework for our discussion. Dryer suggests that a possessor is inherently an indication of an NP's definiteness (fn 4, p. e234), and he does not include possessive constructions in his discussion. Others however \citep[e.\,g.,][]{alexiadou:05} argue that possessives are not always definite. Given the complexity of integrating possessive constructions with Dryer's hierarchy, we will not discuss them further here.}  ART is glossed as `the' in several previous works on Mopan \citep{shaw:71,ulrich:ulrich:82,ulrich:ulrich:peck:86}, even though it can be used in contexts that cannot be construed as definite (see \sectref{3sec:3} below); \cite{hofling:06} glosses it as DET[erminer]. We use the term `article' to distinguish {\emph{a}} from the Set A possessives with which it is in complementary distribution and which might also be considered to be {\textsc{det}}[erminer].
}

Aside from occurring before single lexical items as in example (\ref{3ex:1b}), ART also occurs  before `property concepts' and other expressions, if they are to function as arguments, as in (\ref{3ex:2}).

\begin{exe}
\ex\label{3ex:2}
ART preceding lexemes usually construed as adjectival modifiers.\newline
[\cite{ventur:76} 3:16, {\emph{Aj K\"an\"an Kax}} `The Chicken Keeper', E.\,S.]\footnote{{Ventur's collection of narratives, transcribed and translated into Spanish\il{Spanish} by Ventur and his Mopan consultants, was donated by Ventur to the Smithsonian. We provide our own interlinear glosses and translations into English.\il{English} Ventur's manuscript includes the names of the original narrators, but since we have no way to obtain permission to publish their names, we use only initials to refer to them. For examples from published sources we include the full names of the speakers.}}
\exi{}
\gll 	jok'-ij				{\textbf{a}}	{\textbf{nooch=o}}.	Tal-ij					a		{\textbf{nene'=e}}. \\
	exit-{\textsc{3b.intr.pfv}}	{\textsc{art}}	big={\textsc{ev}}	come-{\textsc{3b.intr.pfv}}	{\textsc{art}}	small={\textsc{ev}} \\
\glt	`{{\textbf{The [{\textsc{art}}] big (one)}} left off (lit. went out).  {\textbf{The/a [{\textsc{art}}] little (one)}} came.}'
\end{exe}


In some cases, ART's ability to allow forms that do not normally denote entities to function as clause arguments yields an English\il{English} translation as a relative clause. In example (\ref{3ex:3}), the article precedes something that would otherwise be interpreted as a predicate `he went under the bed').\footnote{A reviewer asks why we do not just use the gloss `nominalizer' for ART. One reason is its complementary distribution with the possessive pronouns, mentioned above as a criterion for determiner status. Another is that lexemes can function as `nominals' (clause arguments) in Mopan with or without ART (see example (\ref{3ex:8}) below).}

\begin{exe}
\ex\label{3ex:3}
ART preceding predicative expression. \newline
[\cite{ventur:76} 1:03, {\emph{Aj Okol ich Witz}}, `He who Enters the Mountain', R.\,K'.]
\exi{}
\gll	``...''	kut'an		b'in		{\textbf{a}}	{\textbf{b'in-ij}} 			{\textbf{yalan}}	{\textbf{kamaj=a}}.\\
	{}	3.{\textsc{quot}}	{\textsc{hsy}}	{\textsc{art}}	go.{\textsc{pfv-3b.intr.pfv}} 	under		bed={\textsc{ev}}\\
\glt	` ``...'' said {\textbf{the [{\textsc{art}}] (one who) had gone under the bed}}.'
\end{exe}


\subsubsection{ART with relativized deictic\is{relativized deictics} predicates}\label{3sec:211}

One frequent example of the relativizing function of ART that will be relevant to our discussion of referential anchoring\is{referential anchoring} is its use with a set of four dedicated stative deictic predicates\is{stative predicates} that provide information about referents with respect to their proximity, visibility, and states of prior knowledge to various speech-act participants.  These are {\emph{la'$\sim$d'a'}} `deictic stative 1st person', {\emph{kan(a')}} `deictic stative 2nd person', {\emph{lo'$\sim$d'o'}} `deictic stative 3rd person known through visual means', and {\emph{b'e'}} `deictic stative 3rd person known through other than visual means' \citep{danziger:94}.  When a predicate of this series is relativized using ART {\emph{a}}, the result is a form most simply rendered in English\il{English} as a deictic demonstrative\is{demonstratives} (`this one/that one').  A more literal translation recognizes the predicate content, and might read `one who/which is near me', `one who/which is near you', etc. (\citealt[][891-894]{danziger:94}, see also \citealt[][489-490]{jelinek:95} for similar analysis of Determiner Phrases in Straits Salish).\il{Straits Salish}  We therefore refer to these demonstrative expressions\is{demonstratives} as `relativized deictics'.\is{relativized deictics}  We do not include the deictic predicates {\emph{do'$\sim$lo'}} `deictic stative 3rd person visible', {\emph{da'$\sim$la'}} `deictic stative 1st person', and {\emph{kan(a')}} `deictic stative 2nd person' in the discussion which follows, because these forms are used primarily in face-to-face conversation, and the categories of Dryer's hierarchy are better suited for application to narrative contexts.

A relativized deictic\is{relativized deictics} can occur alone or together with lexical specification of the referent. (\ref{3ex:4}) is an example of the latter.

\begin{exe}
\ex\label{3ex:4}
Lexeme with relativized deictic\is{relativized deictics} expression. \newline
[\cite{ventur:76} 5:07, {\emph{Aj Ma' Na'oo'}} `The Orphans', J.\,I.]
\exi{}
\gll	pero 		{\textbf{a}} 	{\textbf{winik}} 	{\textbf{a}} 	{\textbf{b'e=e}}, 				u 		ka'		k\"ax-t-aj-∅  \\
	{\textsc{disc}}	{\textsc{art}}	man			{\textsc{art}}	{\textsc{d}}.3.{\textsc{nv=ev}}	3{\textsc{a}}	again	seek-{\textsc{t-tr.pfv-3b}} \\
\glt
\exi{}
\gll	u 		laak' 		uy				\"atan. \\
	3{\textsc{a}}	other		3{\textsc{a}}(prevocalic)	wife \\
\glt	`So {\textbf{that [{\textsc{art}}] man, known by other than visual means}}, he looked again for another wife.'
\end{exe}

As we will show below, a relativized deictic\is{relativized deictics} may be employed to indicate identifiability\is{identifiability} of a discourse referent.

\subsubsection{Emphatic pronoun\is{emphatic pronoun}}\label{3sec:212}

Mopan is a polysynthetic language in which verb arguments are frequently encoded only in obligatory person affixes of the verb.  (See for instance the 2nd person Set B affix in example (\ref{3ex:1a}), `you are a man'.) This includes arguments which denote referents previously mentioned in the discourse.

\begin{exe}
\ex\label{3ex:5}
3rd person undergoer affix for anaphoric reference.\footnote{The 3rd person Set B undergoer affix is a zero morpheme.} \newline
[\cite{ventur:76} 3:15, {\emph{Siete Kolor}} `Seven Colors', E.\,S.]
\exi{}
\gll	sas-aj-ij 					samal-il=i, \\
	lighten-{\textsc{inch-3b.intr.pfv}}	next.day-{\textsc{poss=ev}} \\
\glt
\exi{}
\gll	ka' 		b'in-oo'			tukadye' 		u 		k\"ax-t-aj-{\textbf{∅}}-oo' 		b'in.	\\
	again	go.{\textsc{pfv-3.pl}}	another.time	{\textsc{3a}}	seek-{\textsc{t-tr.pfv-3b-3.pl}}	{\textsc{hsy}} \\
\glt	`(When) it dawned the next day, they went and looked for {\textbf{it}} again.'
\end{exe}

In this example `it' (expressed by the zero Set B suffix) refers to previously mentioned coffee and cacao for the king's horse to eat, after the protagonists have been unsuccessful in finding this food the day before. The unusual food has already been named and discussed at length.

If emphasis on a particular argument is desired, it is possible to add a person-indicating independent pronoun. The third person in this series has the form {\emph{le'ek}} and is relevant to our discussion of referential anchoring.\is{referential anchoring} {\emph{Le'ek}} occurs twice in the example below, which comes from a story in which a young woman's father has shot a small hummingbird which he found in her bedroom, and now comes to understand that this hummingbird was actually a magical disguise for his daughter's lover, the Holy Sun. The first use of {\emph{le'ek}} (`that hummingbird I shot') occurs in combination with a nominal ({\emph{tz'unu'un}}, `hummingbird') and helps to specify which hummingbird we are talking about. The second use (`that was the Holy Sun') occurs alone as one side of an equational predication.

\begin{exe}
\ex\label{3ex:6}
{\emph{Le'ek}}, emphatic pronoun.\is{emphatic pronoun} \newline
[\cite{ventur:76} 1:05, {\emph{U kwentojil Santo K'in y Santo Uj}} `The Story of the Holy Sun and Holy Moon', R.\,K'.]
\exi{}
\gll	{\textbf{le'ek}} 	{\textbf{a}} 	{\textbf{tz'unu'un}}	 in 		tz'on-aj-∅=a, \\
	{\textsc{3.emph}}	{\textsc{art}}	hummingbird	1{\textsc{a}}	shoot-{\textsc{tr.pfv-3b=ev}} \\
\glt
\exi{}
\gll	{\textbf{le'ek}} 	a 	santo	k'in=i.  \\
	3.{\textsc{emph}}	{\textsc{art}}	holy		sun={\textsc{ev}} \\
\glt	`{\textbf{That hummingbird}} I shot, {\textbf{that}} was the Holy Sun!' [Lit. That which is a hummingbird I shot, is that which is the Holy Sun!]
\end{exe}


\subsubsection{Numeral + classifier construction\is{classifiers!numeral classifiers}}\label{3sec:213}

In Mopan, enumeration of nominals requires use of a numeral classifier.\is{classifiers!numeral classifiers}  A numeral classifier\is{classifiers!numeral classifiers} phrase consists of numeral + classifier (+ optional ART) + nominal.\is{classifiers!numeral classifiers} %Overwhelmingly frequently in this function, the numeral {\emph{jun}} `one' is found, 
It is overwhelmingly the numeral {\emph{jun}} `one' that is found in this function,
although other numerals can also introduce referents where appropriate. This construction is often used to introduce new discourse referents.\footnote{Use of the numeral `one' for discourse-new referents is common cross-linguistically and is often the source for indefinite articles \citep[see, e.\,g.,][]{lyons:99}.} An example is (\ref{3ex:7}), the first sentence in a story; see also \sectref{3sec:334} below.

\largerpage[2]
\begin{exe}
\ex\label{3ex:7}
Numeral classifier\is{classifiers!numeral classifiers} construction. \newline
[\cite{ventur:76} 1:08, {\emph{Aj Jook'}} `The Fisherman', R.\,K'.]
\exi{}
\gll	{\textbf{jun}} 	{\textbf{tuul}} 			b'in 		{\textbf{a}} 	{\textbf{winik=i}}, \\
	one		{\textsc{clf.anim}}	{\textsc{hsy}}		{\textsc{art}}	man{\textsc{=ev}} \\
\glt
\exi{}
\gll	top	ki'-{∅}			b'in		t-u			wich	a		jook'=o. \\
	very	be.good-3{\textsc{b}}	{\textsc{hsy}}	{\textsc{prep-3a}}	eye	{\textsc{art}}	fishing{\textsc{=ev}} \\
\glt 	`{\textbf{A man}}, fishing was very good in his eye(s) (he liked fishing very much).'
\end{exe}

 
\subsubsection{Bare nominal\is{bare nominal}}\label{3sec:214} 

Despite the abovementioned noun-verb lexical fluidity that is characteristic of Mopan, it is possible for a bare lexical item to be construed as an argument if its lexical meaning readily supports this.  An example is (\ref{3ex:8}).

\begin{exe}
\ex\label{3ex:8}
Bare lexical item interpreted as argument. \newline
[Author's data, {\emph{Ix Che'il etel B\"ak'}} `Wild Woman', J.\,S.]
\exi{}
\gll	o,	inen=e			waye'		watak-en \\
	oh	1.{\textsc{emph=ev}}	{\textsc{d.loc}}.1	be.imminent-1{\textsc{b}} \\
\glt
\exi{}
\gll	waye'		yan-Ø		in		kaal,			kut'an		{\textbf{winik=i}}. \\
	{\textsc{d.loc.1}}	exist-3{\textsc{b}}	{\textsc{1a}}	hometown		3.{\textsc{quot}}	man{\textsc{=ev}} \\
\glt 
` ``Oh, myself, I come from here. Here [this] is my home village,''\\said {\textbf{(the) man}}.'
\end{exe}

Here the word {\emph{winik}} `(be a) man' follows a direct quotation, along with the quotative {\emph{kut'an}}, so it is readily interpreted as the one doing the saying, i.\,e., as an argument. We will show that bare nominals\is{bare nominal} may be ascribed a wide range of definiteness interpretations in Mopan.


\section{Dryer's (\citeyear{dryer:14}) reference hierarchy\is{reference hierarchy}}\label{3sec:3}

As an organizing framework for discussing anchoring, we will use the reference hierarchy\is{reference hierarchy} described by \cite[][e235]{dryer:14}, the basis for his chapter on definite articles in the World Atlas of Language Structures (https://wals.info/chapter/37).  Dryer proposes that a hierarchical organization facilitates cross-linguistic comparison, and asserts that any article which accomplishes the leftmost functions in the hierarchy, to the exclusion of at least some functions on the right, should be declared a definite one \citep[][e241]{dryer:14}.\footnote{\cite[][e237-238]{dryer:14} treats preferential occurrence of an article on a contiguous span of his reference hierarchy\is{reference hierarchy} as the basis for classifying the article as `definite' or `indefinite', depending on whether its span is located toward the left or right of the hierarchy.  He classifies the Basque\il{Basque} article as `definite' even though it occurs in all positions of the hierarchy \citep[][e239]{dryer:14}, because it cannot occur in a subset of indefinite contexts (semantically nonspecific indefinites within the scope of negation).  This may be an acceptable heuristic for typological purposes (or it may not, see \citealt{contini:morava:danziger:fc}), but it does not solve the potential semantic ambiguity of actual occurrences of Mopan ART as regards identifiability\is{identifiability} or uniqueness,\is{uniqueness} when this form occurs in actual discourse. In fact ART can occur within the scope of negation, as in the following example, uttered by an unsuccessful hunter:

\begin{exe}
\exi{}
\gll 	ma' 		yan-{∅}	 	a 	b'\"ak=a. \\
	{\textsc{neg}}	exist-{\textsc{3b}}	{\textsc{art}}	game{\textsc{=ev}} \\
\glt 	`There isn't any game.'\newline
	[Author's data, {\emph{Ix Che'il etel B\"ak'}} `Wild Woman and Meat', J.\,S.]
\end{exe}
} 
Dryer's hierarchy was intended for typological comparison specifically of articles, but we include a broader set of anchoring devices in order to provide a fuller picture of referential anchoring\is{referential anchoring} in Mopan. The typological aspects of Dryer's proposal are of less interest to us here than the usefulness of his framework for descriptive organization in a single language.  His hierarchy is as follows:

\begin{exe}
\exi{}
Dryer's reference hierarchy\is{reference hierarchy} \citep[][e235]{dryer:14}\footnote{{\cite[][e235]{dryer:14} states that his hierarchy is based on Giv\'on's (\citeyear{givon:78}) `wheel of reference', but Dryer uses some different terminology and omits generics\is{generics} and predicate nominals from his hierarchy.}} \\
anaphoric definites > nonanaphoric definites > pragmatically specific indefinites > pragmatically nonspecific (but semantically specific) indefinites > semantically nonspecific indefinites
\end{exe}

A brief explanation of terms that may not be familiar to the reader \citep[see][e236-e237]{dryer:14}:  An anaphoric definite NP refers back in the discourse, i.\,e., is ``licensed by a linguistic antecedent'' \citep[][e236]{dryer:14}, whereas a non-anaphoric definite relies instead on shared knowledge between speaker and addressee; an example of the latter would be {\emph{the sun}} (in a context where there are not multiple suns). These notions of definiteness have much in common with prior understandings \citep[e.\,g.,][]{hawkins:78,lyons:99}, that definiteness is a matter of encoding `identifiability'\is{identifiability} and/or `inclusivity'\is{inclusivity} (more on these ideas below). It is useful for our purposes, however, that Dryer's hierarchy also extends to characterization of the semantics of indefinites. 

{
For Dryer, semantically specific indefinites are those where there is an entailment of existence (e.\,g., {\emph{I went to a movie last night}}).  Within this type, Dryer distinguishes between pragmatically specific indefinites which indicate a discourse participant that ``normally … is referred to again in the subsequent discourse'' \citep[][e236]{dryer:14}, and pragmatically nonspecific indefinites (an NP whose referent is not mentioned again, even though there is an entailment of existence).
}

Finally, a semantically nonspecific indefinite NP (which necessarily is also pragmatically nonspecific) does not entail existence of the referent, e.\,g., {\emph{John is looking for a new house}}.\footnote{\cite[][e237]{dryer:14} acknowledges that a semantically nonspecific referent can be mentioned again (i.\,e., could fit his definition of `pragmatically specific'), as in {\emph{John is looking for a new house.  It must be in the city...}} He also states, however, that ``articles that code pragmatic specificity\is{specificity!pragmatic} appear never to occur with semantically nonspecific noun phrases'' (ibid.).  He does not include the category of semantically nonspecific but pragmatically specific in his hierarchy.}

In the following, we document the distribution of the Mopan forms described above across each of the positions of Dryer's hierarchy.  One of our principal findings is the fact that the `bare nominal'\is{bare nominal} option is allowable across all positions in the hierarchy. This means that, even if other forms can be said (based on their distribution across the hierarchy) to encode definite or indefinite semantics, these forms are never obligatory in the relevant semantic contexts. In many cases, therefore, it seems that distinctions of referential anchoring\is{referential anchoring} in Mopan are made pragmatically, based on context. 

We also make special note of the fact that, while it is never obligatory, Mopan ART is allowable in all positions on the hierarchy.  ART, therefore, cannot be said to encode any sort of distinction between the semantic positions in the hierarchy (that is, it does not encode any semantics of definiteness).  

We now consider each position on Dryer's (\citeyear{dryer:14}) hierarchy in turn, describing the central Mopan possibilities in each case.


\subsection{Anaphoric definites}\label{3sec:31}

\subsubsection{ART}\label{3sec:311}
In Mopan, anaphoric definites are frequently preceded by ART. In (\ref{3ex:9}), the referent has been mentioned in the immediately preceding context and is known to both the storyteller and the addressee.

\begin{exe}
\ex\label{3ex:9}
ART in contexts consistent with anaphoric definiteness.\is{definiteness!anaphoric} \newline
[Author's data, {\emph{Ix Che'il etel B\"ak'}} `Wild Woman', J.\,S.]
\exi{}
\gll	``...'' 	kut'an		{\textbf{a}}	{\textbf{winik}}		t-uy					\"atan=a. \\
	{}	3.{\textsc{quot}}	{\textsc{art}}	man			{\textsc{prep-3a}}(prevocalic)	wife{\textsc{=ev}} \\
\glt	` ``...'' said {\textbf{the [{\textsc{art}}] man}} to his wife.'
\end{exe}

We will show, however, that ART does not explicitly encode anaphoric definiteness,\is{definiteness!anaphoric} since it can also be found in nonanaphoric and non-definite contexts (see Sections \sectref{3sec:32}-\sectref{3sec:35} below).

\subsubsection{ART + deictic predicate}\label{3sec:312}

More explicit indication of anaphoric definiteness\is{definiteness!anaphoric} may also be accomplished through the use of ART to create a relative clause from the deictic predicate {\emph{b'e'}} `near neither speaker nor hearer and known through non-visual means'. The non-visual means in question are commonly understood to include prior mention in discourse \citep{danziger:94}. This construction therefore yields an expression that is equivalent to an anaphoric deictic demonstrative.\is{demonstratives} This was shown in example (\ref{3ex:4}), repeated as (\ref{3ex:10}) for convenience.

\begin{exe}
\ex\label{3ex:10}
Deictic expression for anaphoric definite. \newline
[\cite{ventur:76} 5:07, {\emph{Aj Ma' Na'oo'}} `The Orphans', J.\,I.]
\exi{}
\gll	pero 		{\textbf{a}} 	{\textbf{winik}}		{\textbf{a}} 	{\textbf{b'e=e}}, 			u 		ka'		k\"ax-t-aj-∅ \\
	{\textsc{disc}}	ART	man	ART	{\textsc{d.3.nv=ev}}		{\textsc{3a}}	again	seek-{\textsc{t-tr.pfv-3b}} \\
\glt
\exi{}
\gll	u 		laak' 		uy					\"atan. \\
	3{\textsc{a}}	other		3{\textsc{a}}(prevocalic)		wife \\
\glt	`So {\textbf{that [{\textsc{art}}] man, known through non-visual means}}, he looked again for another wife.'
\end{exe}

The predicate {\emph{b'e'}} can itself occur alone with ART, yielding a referential expression translatable as `one which is near neither speaker nor hearer and which is known through non-visual means', as in (\ref{3ex:11}).

\begin{exe}
\ex\label{3ex:11}
Anaphoric definite with relativized deictic\is{relativized deictics} predicate {\emph{a b'e'}} `deictic stative 3rd person non-visible' used alone. \newline
[\cite{ventur:76} 3:11, {\emph{Uj y k'in}}  `Moon and Sun', E.\,S.]
\exi{}
\gll	top 		kich'pan-{∅} 			ti			in 		wich, \\
	very		be.beautiful-3{\textsc{b}}		{\textsc{prep}}		1{\textsc{a}}	eye \\
\glt
\exi{}
\gll	kut'an		b'in	 	{\textbf{a}} 	{\textbf{b'e'=e}}. \\
	3.{\textsc{quot}}	{\textsc{hsy}}	{\textsc{art}}	{\textsc{d}}.3.{\textsc{nv=ev}} \\
\glt	` ``I like it very much,'' said {\textbf{that one known through non-visual means}}.'
\end{exe}

Example (\ref{3ex:11}) comes in the middle of a story in which a young woman (the Moon) has been speaking to her father.  In the preceding context her quotations are interspersed with the expression {\emph{k'u t'an b'in}} `apparently [that is] what [s/he] said', which is very common for quotations in Mopan narrative, and completely lacks overt identification of the speaker.  This example comes at the end of her conversational turn, just before her father's reply.  Although it has been clear all along who the speaker is, here the narrator makes the anaphoric reference more explicit by means of the deictic, perhaps to mark the transition to a new speaker.  In any case, no lexical specification is needed, and the deictic is used alone. 


\subsubsubsection{Optionality of relativized deictic\is{relativized deictics} for explicit marking of anaphoric definiteness\is{definiteness!anaphoric}}\label{3sec:3121}

\largerpage[-1]
Recall that in the context immediately preceding example (\ref{3ex:11}) above there are several non-explicit allusions to the woman being quoted, in contrast with the deictic expression that appears in the cited example. This example thus illustrates another characteristic of referential anchoring\is{referential anchoring} of anaphoric definites in Mopan: even though this information can be conveyed by a deictic expression, a deictic is not obligatory with anaphoric definites. This is shown in (\ref{3ex:12}), in which there are two anaphoric NPs, but only the second one is marked by a deictic.

\begin{exe}
\ex\label{3ex:12}
Anaphoric definites with and without relativized deictic\is{relativized deictics} predicate. \newline
[Author's data, {\emph{Ix Che'il etel B\"ak'}} `Wild Woman and Meat', J.\,S.]
\exi{}
\gll	ma'			patal-{∅}		u		ch'uy-t-e'			{\textbf{a}}	{\textbf{b'\"ak'=\"a}} \\
	{\textsc{neg}}	be.able-{\textsc{3b}}	{\textsc{3a}}	hang-{\textsc{t-tr.irr.3b}}	{\textsc{art}}	meat{\textsc{=ev}} \\
\glt
\exi{} 
\gll	{\textbf{a}}	{\textbf{winik}}		{\textbf{a}}	{\textbf{b'e'}}. \\
	{\textsc{art}}	man			{\textsc{art}}	{\textsc{d.3.nv}} \\
\glt	`{\textbf{That [{\textsc{art}}] man, known through non-visual means}}, couldn't hoist up {\textbf{the [{\textsc{art}}] meat}}.'
\end{exe}

In example (\ref{3ex:12}), the protagonist has been mentioned several times, and has encountered a supernatural forest woman, who has brought him a large quantity of game.  The game is so heavy that the man can't lift it to take it home.  Here there are two anaphoric NPs:  {\emph{a b'\"ak'}} `the meat' and {\emph{a winik a b'e'}} `that man'.  The first is marked only by ART and the second by both ART and a relativized deictic.\is{relativized deictics}

One could ask why the deictic is used in (\ref{3ex:12}) at all, since this is the only man mentioned in the story so far.  Why not just use ART + nominal, as is done with the reference to the meat (also previously mentioned in the story)?  In this case the deictic appears to add emphasis:  in contrast with the woman, who had no trouble carrying the meat, and in contrast to other possible men who might also be able to carry it, that particular man was unable to lift it.\footnote{This interpretation is also consistent with the use of the deictic in example (\ref{3ex:11}), where the deictic marks a transition between speakers.} 

When referring to anaphoric definites, then, a relativized deictic\is{relativized deictics} can be used, but is not obligatory. It is also allowable, and far from unusual, for ART alone to occur in such contexts.  There may be a tendency for relativized deictics\is{relativized deictics} to be associated with contrast\is{contrast} or extra emphasis, but further research would be needed to confirm this.


\subsubsection{The emphatic pronoun\is{emphatic pronoun} {\emph{le'ek}} }\label{3sec:313}
{\emph{Le'ek}} `be it/be the one' is appropriately used for anaphoric mention, as in (\ref{3ex:13}).

\newpage
\begin{exe}
\ex\label{3ex:13}
{\emph{Le'ek}} for anaphoric mention. \newline
[\cite{ventur:76} 3:15, {\emph{Siete Kolor}} `Seven Colors', E.\,S.]
\exi{}
\gll 	k\"ak\"aj 	i	kafe, 	 \\
	cacao	and	coffee	 \\
\glt
\exi{}
\gll	{\textbf{le'ek}}	 	a 	walak-{∅}	 		u 		jan-t-ik-{∅}			in 		kabayoj=o. \\
	3.{\textsc{emph}}	{\textsc{art}}	be.habitual-{\textsc{3b}} 	3{\textsc{a}}	eat-{\textsc{t-tr.ipfv-3b}}	1{\textsc{a}}	horse{\textsc{=EV}} \\
\glt 	`Cacao and coffee, {\textbf{it is that}} which my horse eats.'
\end{exe}

{
Relativized deictics,\is{relativized deictics} including {\emph{b'e'}} `associated with neither speaker nor hearer and known through non-visual means' can occur with {\emph{le'ek}}, as shown in (\ref{3ex:14}).
}

\begin{exe}
\ex\label{3ex:14}
{\emph{Le'ek}} with relativized deictic\is{relativized deictics} {\emph{a b'e'}}. \newline
[\cite{ventur:76} 3:15, {\emph{Siete Kolor}} `Seven Colors', E.\,S.]
\exi{}
\gll	{\textbf{le'ek}} 	{\textbf{a}} 	{\textbf{b'e'}}		u 		p'o'-aj-∅=a. \\
	3.{\textsc{emph}}	{\textsc{art}}	{\textsc{d.3.nv}}	3{\textsc{a}}	do.laundry-{\textsc{tr.pfv-3b=ev}} \\
\glt	`{\textbf{It is he, known through non-visual means}}, who washed the clothes.'
\end{exe}

In this story, the hero has been secretly out winning the competition to marry the princess, but now returns home to the humble identity of a hard-working younger brother, assigned to menial domestic tasks.

Finally, {\emph{le'ek}} can also co-occur in anaphoric use with a nominal phrase with ART plus a relativized deictic,\is{relativized deictics} as shown in (\ref{3ex:15}).\footnote{This construction, applied to each of the deictic predicates in turn, is cognate with the current Yukatek\il{Yucatecan languages} demonstrative series\is{demonstratives} \citep{hanks:90}.}


\begin{exe}
\ex\label{3ex:15}
{\emph{Le'ek}} + ART + nominal + relativized deictic.\is{relativized deictics}  \newline
[\cite{ventur:76} 5:06, {\emph{Kompadre etel a Komadre}} `The Compadre and the Comadre', J.\,I.]
\exi{}
\gll	tz'a'-b'-ij	 				u 		meyaj	 \\
	give-{\textsc{pass-3b.intr.pfv}}		{\textsc{3a}}	work	\\
\glt
\exi{}
\gll	ichil 		jum	p'eel 		jardin. ... \\
	inside	one	{\textsc{clf.inan}}	garden \\
\glt
\exi{}
\gll	bueno. 	{\textbf{le'ek}} 	{\textbf{a}} 	{\textbf{meyaj}} 	{\textbf{a}}	{\textbf{b'e'}} 		u 		b'et-aj-∅=a  \\
	well		{\textsc{3.emph}}	{\textsc{art}}	work			{\textsc{art}}	{\textsc{d.3.nv}}	{\textsc{3a}}	do-{\textsc{tr.pfv-3b=ev}} \\
\glt	`He was given work in a garden. ... well, that work is what he did.'\newline 
[Lit. Well, {\textbf{it is that which is work which is known through non-visual means}} (that) he did]
\end{exe}

In addition to serving as a stative predicate,\is{stative predicates} the emphatic pronoun,\is{emphatic pronoun} then, may also appear with nominals, and it is an important resource in Mopan for indicating reference to a previously mentioned referent. As we have shown, however, {\emph{le'ek}} is not obligatory for anaphoric reference.


\subsubsection{Bare nominal\is{bare nominal}}\label{3sec:314}

In Mopan, it is possible for a bare nominal\is{bare nominal} to be used for anaphoric definite reference, as shown in (\ref{3ex:8}), repeated for convenience as (\ref{3ex:16}).

\begin{exe}
\ex\label{3ex:16}
Bare nominal\is{bare nominal} for anaphoric definite referent. \newline
[Author's data, {\emph{Ix Che'il etel B\"ak'}} `Wild Woman', J.\,S.] 
\exi{}
\gll	o,    	inen=e 			waye’           	watak-en \\
	oh     {\textsc{1.emph=ev}}	{\textsc{d.loc.1}}	be.imminent-{\textsc{1b}} \\
\glt
\exi{}
\gll	waye'           	yan-{∅}          	in     		kaal,			kut'an		{\textbf{winik=i}}. \\
	{\textsc{d.loc.1}}	exist-{\textsc{3b}}	{\textsc{1a}}	hometown		{\textsc{3.quot}}	man={\textsc{ev}} \\
\glt	` ``Oh, myself, I come from here. Here [this] is my home village,''\\said {\textbf{(the) man}}.'
\end{exe}

Here the man is the main protagonist in the story, and has been mentioned several times before.  As mentioned earlier, use of a bare referring expression occurs only when its lexical semantics support argument construal (see \citealt{contini:morava:danziger:fc} for details).


\subsection{Nonanaphoric definites}\label{3sec:32}

\subsubsection{ART for nonanaphoric definites}\label{3sec:321}

\cite[][e236]{dryer:14} defines nonanaphoric definites as definite noun phrases whose use ``is based only on shared knowledge of the speaker and hearer'', unlike anaphoric definites whose use is ``licensed by linguistic antecedents'' (ibid.).  With regard to prior mention, Dryer further states that ``in English,\il{English} one would not normally refer to the sun with the noun phrase {\emph{the aforementioned sun}}, even if there were a previous reference to it'' (ibid.), presumably because {\emph{the sun}} has a unique referent, so does not require re-identification.  Mopan ART occurs readily in such contexts, as shown in example (\ref{3ex:17}).

\begin{exe}
\ex\label{3ex:17}
ART for unique individuals. \newline
[\cite{ulrich:ulrich:82}, `Mopan Maya Concept of Earth and Heaven', Jos\'e Mar\'ia Cowoj, interviewed by Matthew Ulrich, line 41]\footnote{For all examples from this source, we regularize the orthography to that recommended by the ALMG (see supra note 3) and provide our own glossing.} 
\exi{}
\gll	{\textbf{a}}	{\textbf{uj=u}}			tan-{∅}				ilik 		u   		b'eel  \\
	{\textsc{art}}	moon={\textsc{ev}}		be.continuing-{\textsc{3b}}	{\textsc{int}}	{\textsc{3a}}	go{\textsc{.ipfv}} \\
\glt
\exi{}
\gll	jab'ix	ti 			tan-{∅}					u		b'eel 		{\textbf{a}}	{\textbf{k'in}}. 	 \\
	like	{\textsc{prep}}		be.continuing-{\textsc{3b}}		{\textsc{3a}}	go.{\textsc{ipfv}}	{\textsc{art}}	sun \\
\glt	`{\textbf{The [{\textsc{art}}] moon}} goes along just like {\textbf{the [{\textsc{art}}] sun}} goes.'
\end{exe}

In (\ref{3ex:17}), the moon and the sun could be construed as definite because each has a unique referent.\footnote{\cite[][282]{lobner:11} treats e.\,g., {\emph{moon}} as an `individual noun',\is{noun types!individual noun} marked by the feature [+Unique], i.\,e., as `semantically definite' and inherently unique.  By contrast, a noun like {\emph{man}} is a `sortal noun',\is{noun types!sortal noun} i.\,e., [\minus Unique], but it can be coerced into an individual reading by contextual information that identifies a particular individual, which can make it `pragmatically definite' in a given context (pp.\,307-308). We will see below (examples (\ref{3ex:30}) and (\ref{3ex:32})) that Mopan ART does not coerce an individual reading for the associated nominal. }


\subsubsection{Relativized deictic\is{relativized deictics} for nonanaphoric reference}\label{3sec:322}

Although the relativized deictic\is{relativized deictics} {\emph{a b'e'}} is most often used for anaphoric reference, it can also occur with unique referents, as in example (\ref{3ex:18}).


\begin{exe}
\ex\label{3ex:18}
Relativized deictic\is{relativized deictics} for nonanaphoric definite. \newline
[\cite{ventur:76} 3:07, {\emph{U Kweentojil aj Peedro}} `The Story of Pedro', E.\,S.]
\exi{}
\gll	ok-ij 						b'in 			ichil 		{\textbf{a}} 	{\textbf{ka'an}} 	{\textbf{a}} 	{\textbf{b'e'=e}}. \\
	enter-{\textsc{3b.intr.ipfv}}		{\textsc{hsy}}		inside	{\textsc{art}}	sky			{\textsc{art}}	{\textsc{d.3.nv=ev}} \\
\glt 	`He went inside {\textbf{that sky}}.'
\end{exe}

Example (\ref{3ex:18}) is from a story in which a man wants to enter the sky in order to see God, and he is finally allowed in after a series of negotiations with Saint Peter.  Like the sun and moon in (\ref{3ex:17}), the sky is unique, so even though the sky has been mentioned before in this story, according to Dryer's definition, example (\ref{3ex:18}) would not constitute anaphoric reference.  The relativized deictic\is{relativized deictics} is not being used in order to remind the hearer that we are talking about the same sky that has been mentioned before.  In this example it seems merely to add emphasis (cf. example (\ref{3ex:12}), discussed earlier).


\subsubsection{Emphatic pronoun\is{emphatic pronoun} {\emph{le'ek}} `be a 3rd person' for nonanaphoric reference}\label{3sec:323}

The independent pronoun {\emph{le'ek}} `be a third person' may also be used for inherently unique referents. In example (\ref{3ex:19}), from a legend in which Jesus is hunted down by evil pursuers, first mention of this very familiar and unique protagonist is made using {\emph{le'ek}}.\footnote{{Jesus is also introduced with ART, rather than with the masculine gender marker, which would normally be expected with the name of an ordinary human man \citep{contini:morava:danziger:18}.}} This usage helps to specify that we are talking about a unique referent rather than just one man among others who bears this name.

\begin{exe}
\ex\label{3ex:19}
{\emph{Le'ek}} for nonanaphoric definite. \newline 
[\cite{ventur:76} 5:09, {\emph{U Alkab'eeb' Jesus}} `The Chasing of Jesus', J.\,I.]
\exi{}
\gll	bueno, 	{\textbf{le'ek}} 	{\textbf{a}}	{\textbf{jesus=u}},  \\
	well		{\textsc{3.emph}}	{\textsc{art}}	Jesus={\textsc{ev}} \\
\glt	
\exi{}
\gll	ti 			kaj-ij 					alka'-b'-\"al \\
	{\textsc{prep}}		begin-{\textsc{3b.intr.ipfv}}	run-{\textsc{pass-intr.ipfv}} \\
\glt	`Well, {\textbf{he who is Jesus}}, when he was beginning to be chased, ...'
\end{exe}

\subsubsection{Bare nominal\is{bare nominal} for nonanaphoric reference}\label{3sec:324}

It is also possible for a bare nominal\is{bare nominal} to be used for nonanaphoric definite reference, as in (\ref{3ex:20}).

\begin{exe}
\ex\label{3ex:20}
Bare nominal\is{bare nominal} for nonanaphoric definite reference. \newline
[\cite{ulrich:ulrich:82}, `Little Brother', Genoveva Bol]
\exi{}
\gll	u		tz'-aj-{∅}			b'in 			ich 	{\textbf{k'aak'}}. \\
	{\textsc{3a}}	put-{\textsc{tr.pfv-3b}}	{\textsc{hsy}}		in	fire \\
\glt	`She put it on {\textbf{(the) fire}}.'
\end{exe}
	
Although the term `fire' does not inherently identify a unique individual, in the context of a Mopan house where cooking has been mentioned, only one fire can be intended \citep[see, e.\,g.,][285]{lobner:11}.


\subsection{Pragmatically (and also semantically) specific indefinites}\label{3sec:33}

Recall that in Dryer's hierarchy, semantically specific indefinites presuppose the existence of a referent ({\emph{I went to a movie last night}}), as opposed to semantically nonspecific indefinites, which do not make this presupposition ({\emph{John is looking for a new house}}). Semantically specific indefinites come in two pragmatic types. Pragmatically specific indefinites are those which will remain topical, i.\,e., are mentioned again in the discourse after they are introduced. Pragmatically non-specific indefinites are not mentioned again in the subsequent discourse.\footnote{We note that a category that is based on subsequent mention in the discourse is weighted toward connected discourse such as narrative.} By Dryer's definition (\citeyear[][e237]{dryer:14}), semantically nonspecific reference cannot be pragmatically specific.

\subsubsection{ART alone for pragmatically specific indefinites}\label{3sec:331}

Although new referents that will remain topical are typically introduced with the {\emph{jun}} + classifier construction\is{classifiers!numeral classifiers} (\sectref{3sec:334}), it is also possible for such a referent to be marked only with ART.  This is illustrated in (\ref{3ex:21}).

\begin{exe}
\ex\label{3ex:21}
Pragmatically specific new referent introduced with ART alone. \newline
[Author's data, `The Ring and the Fish', P.\,C.]
\exi{}
\gll	pues	a 	winik	a 	b’e=e 			u 		chaan-t-aj-∅  \\
	so	{\textsc{art}}	man	{\textsc{art}}	{\textsc{d.3.nv=ev}}		{\textsc{3a}}	gaze-{\textsc{t-tr.pfv-3b}} \\
\glt
\exi{}
\gll	t-u  			tzeel=e 		uy(prevocalic) 	il-aj-∅=a  yan-{∅} 		{\textbf{a}} 	{\textbf{b'ak=a}}.\\
	{\textsc{prep-3a}}	side={\textsc{ev}} 	{\textsc{3a}} 		see-{\textsc{tr.pfv-3b=ev}}  exist-{\textsc{3b}} 	{\textsc{art}}		bone={\textsc{ev}} \\
\glt	`So the mentioned man looked next to him, he saw there were {\textbf{[{\textsc{art}}] bone[s]}}.'
\end{exe}

Example (\ref{3ex:21}) is from a story in which an ogre disguised as a woman has lured a man's brothers into the forest and killed them.  The bones, mentioned here for the first time, are evidence that the brothers have been killed. As such they are extremely important to the storyline, and they are mentioned again as the story continues.  Here the {\emph{jun}} `one' + classifier construction\is{classifiers!numeral classifiers} would be less felicitous, since more than one bone is involved (pluralization is optional in Mopan), but other numbers would be over-specific in this context.  


\subsubsection{Relativized deictic\is{relativized deictics} predicates for introducing an unfamiliar referent}\label{3sec:332}

Though rare for first mention, a relativized deictic\is{relativized deictics} can also occur in this context, as shown in (\ref{3ex:22}).

\begin{exe}
\ex\label{3ex:22}
Relativized deictic\is{relativized deictics} for first mention of pragmatically specific referent. \newline
[\cite{ventur:76} 3:03, {\emph{A ayin etel aj Konejo}} `The Story of the Alligator and the Rabbit', E.\,S.]
\exi{}
\gll	pues 	{\textbf{jun}} 	{\textbf{tuul}} 		{\textbf{b'in}} 		{\textbf{a}} 	{\textbf{winik}} 	{\textbf{a}}	{\textbf{b'e'=e}}, \\
	so		one		{\textsc{clf.anim}}	{\textsc{hsy}}		{\textsc{art}}	man			{\textsc{art}}	{\textsc{d.3.nv=ev}} \\
\glt
\exi{}
\gll	tan-{∅} 				b'in 			u 		man-\"al. \\
	be.continuing-{\textsc{3b}}	{\textsc{hsy}}		{\textsc{3a}}	walk-{\textsc{intr.ipfv}} \\
\glt	`So {\textbf{this man}}, he was wandering along.'
\end{exe}

This is the first mention of the protagonist in a story.  The effect of combining the numeral + classifier + ART construction,\is{classifiers!numeral classifiers} commonly used for first mention of a referent that will continue to be topical (\sectref{3sec:334} below), with the relativized deictic\is{relativized deictics} {\emph{a b'e'}} `associated with neither speaker nor hearer and known through non-visual means', more often used for anaphoric reference, is similar to what \cite{prince:81} calls ``indefinite {\emph{this}}'' in English.\il{English}

\subsubsection{Emphatic pronoun\is{emphatic pronoun} for pragmatically specific reference}\label{3sec:333}
\largerpage[1.5]
The 3rd person emphatic pronoun\is{emphatic pronoun} {\emph{le'ek}} may also appear at first mention of a referent. Example (\ref{3ex:23}) occurs in the first line of the story, and the referent introduced with {\emph{le'ek}} is presented as syntactically equivalent to the one which is introduced with the numeral + classifier construction,\is{classifiers!numeral classifiers} very frequently used for first mentions.


\begin{exe}
\ex\label{3ex:23}
{\emph{Le'ek}} for first mention.  \newline
[\cite{ventur:76} 7:08, {\emph{A B’aalumoo’o}} `The Jaguars', A.\,T.]
\exi{}
\gll	jum	p'eel 		k'in 	b'in, \\
	one	{\textsc{clf.inan}}	day	{\textsc{hsy}}	\\
\glt
\exi{}
\gll	{\textbf{le'ek}} 	{\textbf{a}} 	{\textbf{b'aalum}} 	uy				et'ok \\
	3.{\textsc{emph}}	{\textsc{art}}		jaguar		{\textsc{3a}}(prevocalic)	companion \\
\glt
\exi{}
\gll	jun 	tuul 			aj 			leon=o \\
	one	{\textsc{clf.anim}}	{\textsc{gm.m}}	lion={\textsc{ev}} \\
\glt 
\exi{}
\gll	uy				ad'-aj-∅-oo'	 			b'in 	ti	 		u	 	b’ajil \\
	{\textsc{3a}}(prevocalic)	say-{\textsc{tr.pfv-3b-3.pl}}	{\textsc{hsy}}	{\textsc{prep}}		{\textsc{3a}}	self \\
\glt `One day {\textbf{that which is (a) jaguar}} together with a [{\textsc{num}} + {\textsc{clf}}] lion,\footnote{The word {\emph{leon}} `lion, jaguar' belongs to a subset of Mopan vocabulary that is lexically specified for gender.  For such nouns a gender marker is essentially obligatory and has no relationship to definiteness (see \citealt{contini:morava:danziger:18} for more on the Mopan gender markers).} they said to each other ...'
\end{exe}
\clearpage

In this case, both the jaguar introduced with {\emph{le'ek}} and the `lion' (also afterwards called {\emph{b'aalum}} `jaguar') are highly salient, and are mentioned multiple times in the discourse that follows. 

In light of (\ref{3ex:23}) and other examples of first mention, {\emph{le'ek}} `be it/be the one' must therefore be understood as an indicator of emphasis rather than primarily one of definiteness.


\subsubsection{Numeral classifier\is{classifiers!numeral classifiers} construction for pragmatically specific reference}\label{3sec:334}

The most common way in Mopan to introduce new referents that will remain topical in subsequent discourse is by means of the numeral {\emph{jun}} `one' (or other numeral where appropriate), followed by a numeral classifier\is{classifiers!numeral classifiers} and the nominal.  The nominal may or may not also be preceded by ART.  This is shown in (\ref{3ex:24}).


\begin{exe}
\ex\label{3ex:24}
{\emph{Jun}} `one' + classifier\is{classifiers!numeral classifiers} with and without ART for pragmatically specific new referent.
	\begin{xlista}
	\ex\label{3ex:24a}
	{\emph{Jun}} `one' + CLF with ART.  \newline
	[\cite{ventur:76} 5:09, {\emph{U Alka'b'eeb' Jesus}}, `The Chasing of Jesus', J.\,I.]
	\exi{}
	\gll	pues	k'och-ij				tub'a		yan-∅ \\
		so	arrive-{\textsc{3b.intr.pfv}}	where	exist-{\textsc{3b}} \\
	\glt
	\exi{}
	\gll	{\textbf{jun}}	{\textbf{teek}}		{\textbf{a}}	{\textbf{m\"ap=\"a}}. \\
		one		{\textsc{clf}}.plant	{\textsc{art}}		cocoyol\_palm={\textsc{ev}} \\
	\glt	`So he arrived at [a place] where there was {\textbf{a [{\textsc{num}} + {\textsc{clf}} + {\textsc{art}}] cocoyol palm}}.'
	%%
	\ex\label{3ex:24b}
	{\emph{Jun}} `one' + CLF without ART. \newline
	[\cite{ventur:76} 4:02, {\emph{U Kwentojil aj Konejo manyoso}}, `The Story of the Clever Rabbit', A.\,K'.]
	\exi{}
	\gll	entonses		b'in-ij				u		ka'		k\"ax-\"a'. \\
		then			go-{\textsc{3b.intr.pfv}}	{\textsc{3a}}	again	seek-{\textsc{3b.tr.irr}} \\
	\glt
	\exi{}	
	\gll	ke'en-{∅}			yalam	{\textbf{jun}}	{\textbf{teek}}		{\textbf{m\"ap}}.  \\
		be.located-{\textsc{3b}}	under	one		{\textsc{clf}}.plant	cocoyol\_palm \\
	\glt	`So he [puma] went off to look for him [rabbit] again. He [rabbit] was located under {\textbf{a [{\textsc{num}} + {\textsc{clf}}, no {\textsc{art}}] cocoyol palm}}.'
	\end{xlista}
\end{exe}

\newpage
Even though each example introduces a (single) cocoyol palm that is referred to again in subsequent discourse, one includes ART and the other does not.\footnote{{A referee asks whether {\emph{jun}} `one' in (\ref{3ex:24b}) is perhaps a type of indefinite determiner rather than the numeral `one'.  Although the numeral + classifier construction\is{classifiers!numeral classifiers} illustrated here is often translatable with an indefinite article in English,\il{English} this translation does not depend on presence vs. absence of ART.  The cocoyol palms in (\ref{3ex:24a}) and (\ref{3ex:24b}) are both new discourse referents, and in neither case is their singularity being contrasted with other possible numbers.  (Note also that other numerals can occur both with and without ART in a numeral + classifier construction\is{classifiers!numeral classifiers} in Mopan.)}} We propose a pragmatic explanation for presence/absence of ART in such cases.  When a lexeme, due to its meaning, is likely to be construed as an entity, the article can be omitted if the referent has lower discourse salience\is{discourse salience} than it would if it were marked by ART.\footnote{At the 2018 Workshop on Specificity, Definiteness and Article Systems across Languages (40th Annual Meeting of the Deutsche Gesellschaft f\"ur Sprachwissenschaft) we provided some quantitative evidence in support of differential discourse salience\is{discourse salience} of presence/absence of ART; see \cite{contini:morava:danziger:fc} for those data.}

We note, however, that lack of discourse salience\is{discourse salience} in this sense does not correspond precisely to Dryer's `pragmatic nonspecificity', since in both of the examples in (\ref{3ex:24}), the referent is mentioned again in later discourse. Difference in discourse salience\is{discourse salience} is, rather, a question of degree of importance of the referent as a protagonist in the discourse in question.  To illustrate, in (\ref{3ex:24a}) above, Jesus is fleeing from persecution and hides at the top of a cocoyol palm.  In the ensuing narrative when the pursuers ask the tree what it is hiding, it responds in a misleading way so as to protect Jesus. The tree is a salient protagonist in the story.  In (\ref{3ex:24b}), the cocoyol palm never speaks or takes on animacy,\is{animacy} and is eventually broken up and offered as food, losing its quality as an (individually identifiable) entity.  The word {\emph{m\"ap}} `cocoyol palm' in this second case is determinerless when first mentioned, and ---although it qualifies for Dryer's `pragmatic specificity'\is{specificity!pragmatic} because it is mentioned again in the same text--- it is not an important character in the story.  (Further indication of the difference in discourse salience\is{discourse salience} between the trees in these examples is the fact that the first one is introduced as the main argument of its clause whereas the second is introduced in a prepositional phrase.)

\subsubsection{Bare nominal\is{bare nominal} for pragmatically specific reference}\label{3sec:335}

It is also possible for a bare nominal\is{bare nominal} to introduce a new referent that will be mentioned again in the discourse, as shown in (\ref{3ex:25}).

\newpage
\begin{exe}
\ex\label{3ex:25}
Bare nominal\is{bare nominal} for pragmatically specific referent. \newline
[\cite{ulrich:ulrich:82}, `Trip to Belize', Jos\'e Mar\'ia Chowoj]
\exi{}
\gll	pwes 	ki'		keen-oo'			ti'i			i	jok-een		toj		ich	{\textbf{naj}}. \\
	well		good		{\textsc{1.quot-3.pl}}	{\textsc{3.obl}}	and	exit-{\textsc{1b}}	already	in	house \\
\glt	` ``Well, good!'' I said to them and went out of (the) {\textbf{house}}.'
\exi{}
\gll	b'in-o'on 		pach 	{\textbf{naj}}.  \\
	go-{\textsc{1b.pl}}	behind	house \\
\glt	`We went behind (the) {\textbf{house}}.'
\exi{}
\gll	pues te'=i					in 		jok-s-aj-oo'			u		foto \\
	well	{\textsc{d.loc.3.nv=scope}}	{\textsc{1a}} 	exit-{\textsc{caus-tr.pfv-3.pl}}	{\textsc{3a}}	photo \\
\glt
\exi{}
\gll	ti			k'och-ij				a	soldadoj=o. \\
	{\textsc{prep}}		arrive-{\textsc{3b.intr.pfv}}	{\textsc{art}}	soldier={\textsc{ev}} \\
\glt	`Well, I had taken their picture(s) when a policeman arrived.'
\exi{}
\gll	uch-ij 				u		cha'an	ich	{\textbf{naj}}.  \\
	happen-{\textsc{3b.intr.pfv}}	{\textsc{3a}}	gaze		in	house \\
\glt	`He looked around in (the) {\textbf{house}}.'
\end{exe}


In (\ref{3ex:25}), the narrator describes a visit to an acquaintance, whose house is an example of a referent whose specificity\is{specificity} and uniqueness\is{uniqueness} are given by the context.  The house is introduced with the bare nominal\is{bare nominal} {\emph{naj}}, and is mentioned two more times again with a bare nominal.\is{bare nominal}  Despite its specificity,\is{specificity} and despite the fact that it is mentioned more than once, the house is not an important participant in this narrative: it is mentioned merely in its capacity as location.\footnote{Note that the preposition {\emph{ich}} `inside' may also occur with an ART-marked nominal, i.\,e., it is not obligatorily followed by a bare nominal.\is{bare nominal}}


\subsection{Semantically specific but pragmatically nonspecific referents}\label{3sec:34}

Recall once again that for \cite{dryer:14}, a semantically specific referent involves an entailment of existence but that such referents can be either pragmatically specific (recurs in discourse ---Mopan examples of such cases were treated in the previous section), or pragmatically nonspecific. A semantically specific but pragmatically nonspecific referent in Dryer's terminology does not remain topical in the discourse: it is never mentioned again. In narrative at least, referents that receive only one mention are unlikely to be important protagonists in the discourse in which they occur. Dryer's pragmatic nonspecificity therefore coincides to a great extent with our `low discourse salience'\is{discourse salience} (although we have seen that the converse is not the case: Dryer's pragmatic specificity\is{specificity!pragmatic} can cover instances both of low and of high discourse salience,\is{discourse salience} see examples (\ref{3ex:24}a-b)).

\subsubsection{ART for semantically specific, pragmatically nonspecific referents}\label{3sec:341}

ART by itself is rarely used for referents which will not be mentioned again in the discourse (pragmatically nonspecific indefinites).  This is not surprising, given that such expressions are by definition not salient in the ensuing discourse and given that use of ART correlates with discourse salience.\is{discourse salience}  While not common, it is nevertheless possible for ART to appear with a semantically specific NP that is not mentioned again, as in example (\ref{3ex:26}).

\begin{exe}
\ex\label{3ex:26}
ART for semantically specific, pragmatically nonspecific referent. \newline
[\cite{ventur:76} 2:04, {\emph{U Kwentojil ix Pulya'aj}} `The Story of the Witch', A.\,K'.]
\exi{}
\gll	tan-{∅}				b'in		u		tz\"aj-ik-{∅}		a	ja'as=a. \\
	be.continuing-{\textsc{3b}}	{\textsc{hsy}}	{\textsc{3a}}	fry-{\textsc{tr.ipfv-3b}}	{\textsc{art}}	banana={\textsc{ev}} \\
\glt
\exi{}
\gll	tan-{∅}				b'in		u		tz\"aj-ik-{∅}		{\textbf{a}}	{\textbf{kamut=u}}. \\
	be.continuing-{\textsc{3b}}	{\textsc{hsy}}	{\textsc{3a}}	fry-{\textsc{tr.ipfv-3b}}	{\textsc{art}}		sweet.potato={\textsc{ev}} \\
\glt	`She was frying plantain(s).  She was frying {\textbf{sweet potato(es)}}.'
\end{exe}

{
Example (\ref{3ex:26}) is from a story in which two children, abandoned by their father in the forest, come upon a house where an old blind woman is cooking plantain and sweet potato (recall that plural specification is optional in Mopan).  They eventually steal the plantains but sweet potato is not mentioned again in the story.
}

\subsubsection{Relativized deictic\is{relativized deictics} for semantically specific, pragmatically nonspecific indefinites}\label{3sec:342}

Though rare, it is also possible for the relativized deictic\is{relativized deictics} {\emph{a b'e'}} to introduce a new discourse referent that is not mentioned again, as shown in example (\ref{3ex:27}).

\begin{exe}
\ex\label{3ex:27}
Relativized deictic\is{relativized deictics} for pragmatically nonspecific referent. \newline
[\cite{ventur:76} 1:03, {\emph{Jun tuul Winik etel Ma'ax}} `A Man and a Monkey', E.\,S.]
\exi{}
\gll 	entonses 	ti 			ka'		b'in 		jun 	tuul 			ilik 	b'in  \\
	then 		{\textsc{prep}} 	again	{\textsc{hsy}} 	one 	{\textsc{clf.anim}} 	just	{\textsc{hsy}} \\
\glt 
\exi{}
\gll	a 	koch		p'at-al				ti			uk'-ul \\		 
	{\textsc{art}}	last.one	abandon-{\textsc{intr.pfv}}	{\textsc{prep}}		drink-{\textsc{intr.ipfv}} \\
\glt 
\exi{}
\gll	ichil		{\textbf{a}}	{\textbf{tunich}}	{\textbf{a}}	{\textbf{b'e'=e}} \\
	inside	{\textsc{art}}	rock			{\textsc{art}}	{\textsc{d.3\_nv=ev}} \\
\glt 
\exi{}
\gll	entonses	pues		te'=ij=i,  \\
	then		so		{\textsc{loc.3.nv=scope=ev}} \\
\glt 
\exi{}
\gll	b'in-ij				b'in			u		chiit-t-ej \\
	go-{\textsc{3b.intr.pfv}}	{\textsc{hsy}}		{\textsc{3a}}	speak.to-{\textsc{t-3b.tr.irr}} \\
\glt 
\exi{}
\gll	a 	b'e'			a 	ma'ax=a. \\
	{\textsc{art}} 	{\textsc{d.3.nv}}	art	monkey={\textsc{ev}} \\
\glt	`Then when there was just one last (monkey) left behind, drinking from {\textbf{that rock}}, then at that point, he (hero) went and spoke to that monkey.'
\end{exe}

The hero of this story, a hunter who is thirsty, has spotted some monkeys drinking at a location that is not specified.  Fearing the monkeys, he hides until most of them depart, leaving one behind.  In (\ref{3ex:27}), the narrator mentions for the first time a rock that the monkey was drinking from.  Although the hero eventually befriends the monkey, the rock is not mentioned again.  This would not be an example of uniqueness\is{uniqueness} being given by the context, like the household fire in example (\ref{3ex:20}), because drinking at a stream in the forest does not presuppose drinking from a rock.  Possibly the demonstrative\is{demonstratives} in (\ref{3ex:27}) is meant to suggest that this monkey is in the same place where the others had been, even though that place was not explicitly described.


\subsubsection{Emphatic pronoun\is{emphatic pronoun} for semantically specific, pragmatically nonspecific indefinites}\label{3sec:343}

The independent pronoun {\emph{le'ek}} can be used for semantically specific but pragmatically nonspecific indefinites. In example (\ref{3ex:28}), a trickster rabbit convinces a puma that a large rock is in danger of falling over. But in fact the rock is firm ---a cloud passing overhead has created the illusion of instability.

\begin{exe}
\ex\label{3ex:28}
{\emph{Le'ek}} for semantically specific, pragmatically nonspecific indefinite. \newline
[\cite{ventur:76} 6:01, {\emph{Aj Koj etel aj konejo}} `The Puma and the Rabbit', M.\,X.]
\exi{}
\gll	pero			le'ek 			a	muyal \\
	{\textsc{disc}}		3.{\textsc{emph}}	{\textsc{art}}	cloud \\
\glt
\exi{}
\gll	a 	tan-{∅}				u		b’eel			ti			u		wich=i. \\
	{\textsc{art}}	be.continuing-{\textsc{3b}}	{\textsc{3a}}	go.{\textsc{ipfv}}	{\textsc{prep}}		{\textsc{3a}}	face={\textsc{ev}} \\
\glt	`But it was {\textbf{that which is a cloud}} that was passing over its face.'
\end{exe}

This sentence constitutes the first and only mention of the cloud. Here emphasis on its identity (as a cloud) contrasts with the appearance of instability of the rock. Although the cloud is not mentioned again, use of the emphatic pronoun\is{emphatic pronoun} highlights its role in the trick being played on the puma.

\subsubsection{Bare nominal\is{bare nominal} for pragmatically nonspecific referents}\label{3sec:344}

Pragmatically nonspecific referents are typically referred to with bare nominals\is{bare nominal} in Mopan, as shown in example (\ref{3ex:29}).

\begin{exe}
\ex\label{3ex:29}
{Bare nominal\is{bare nominal} for semantically specific but pragmatically nonspecific referent.} \newline
[\cite{ventur:76} 1:01, {\emph{Aj Jook'}} `The Fisherman', R.\,K'.]
\exi{}
\gll	pues		jak'-s-ab'-ij \\
	then		frighten-{\textsc{caus-pass-3b.intr.pfv}} \\
\glt
\exi{}
\gll 	b'in		uy(prevocalic)		ool 		u\_men	{\textbf{kan}}.  \\
	{\textsc{hsy}}	{\textsc{3a}} 			feeling	by 		snake \\
\glt	`Then he was startled by {\textbf{(a) snake}}.'
\end{exe}

The snake referred to here is never mentioned again in the story.  It does not contrast with any previously established expectation (as the cloud does in example (\ref{3ex:28})), nor does it play an important role in the plot.

We have already noted that a notion of discourse salience\is{discourse salience} ---importance of the referent as a participant in the surrounding narrative--- governs the distribution of bare nominals\is{bare nominal} and of ART in Mopan narratives, and that this is not necessarily coterminous with Dryer's contrast between pragmatic specificity\is{specificity!pragmatic} and pragmatic nonspecificity (whether a referent is or is not mentioned again in subsequent discourse). The examples in this and the previous section show once again that repeated mention, or lack of it, is at best an indirect marker of discourse salience:\is{discourse salience} a referent may be mentioned only once but play a significant role (example (\ref{3ex:28}), the cloud), and a referent may be mentioned more than once but play a peripheral role (example (\ref{3ex:25}), the house).

\subsection{Semantically nonspecific indefinites}\label{3sec:35}

Semantically nonspecific indefinites make no claim as to the actual existence of the referent.  

\subsubsection{ART for semantically nonspecific reference}\label{3sec:351}

Even though the category of semantically nonspecific is at the least definite end of Dryer's reference hierarchy,\is{reference hierarchy} it is possible for ART to occur in this context in Mopan, as shown in example (\ref{3ex:30}). 

\begin{exe}
\ex\label{3ex:30}
Use of ART for semantically nonspecific indefinite.  \newline
[\cite[][11]{verbeeck:99}, {\emph{U Kwentajil a Santo K'in}} `The Story of the Holy Sun', narrated by Alejandro Chiac]
\exi{}
\gll	sansamal	tatz'	tan-{∅}				b'in		u		b'el \\
	daily		far	be.continuing-{\textsc{3b}}	{\textsc{hsy}}	3{\textsc{a}}	go.{\textsc{ipfv}} \\
\glt
\exi{}
\gll	u		tz'on-o'			{\textbf{a}}	{\textbf{yuk=u}}. \\
	3{\textsc{a}}	shoot-{\textsc{3.btr.irr}}	{\textsc{art}}	antelope={\textsc{ev}} \\
\glt	`Every day he went far [into the woods] to shoot {\textbf{an [{\textsc{art}}] antelope}}.'
\end{exe}

In (\ref{3ex:30}), there is no entailment of existence of an antelope.  The fact that no particular antelope is being referred to (despite use of ART) can be inferred from the imperfective marking on the action of hunting, along with the time reference `every day', which make it highly unlikely that the same antelope would be involved on each occasion of hunting.

\subsubsection{Relativized deictic\is{relativized deictics} for semantically nonspecific reference}\label{3sec:352}

Given the strong association of {\emph{a b'e'}} `deictic stative 3rd person known through other than visual means' with anaphoric reference (\sectref{3sec:312} above), and its highlighting effect elsewhere, we would not expect it to be used for nonspecific referents, and indeed we did not find any examples of {\emph{a b'e'}} used for this purpose in our data.

\subsubsection{Independent pronoun {\emph{le'ek}} }\label{3sec:353}

We have found no examples of the independent pronoun {\emph{le'ek}} being used for semantically nonspecific referents. The semantics of this form (`that which is 3rd person') perhaps categorically preclude such usage.

\subsubsection{Numeral classifier\is{classifiers!numeral classifiers} construction}\label{3sec:354}

It is possible, though rare, for a numeral + classifier construction\is{classifiers!numeral classifiers} to be found with semantically nonspecific referents, as shown in (\ref{3ex:31}).

\begin{exe}
\ex\label{3ex:31}
Numeral + classifier construction\is{classifiers!numeral classifiers} for semantically nonspecific indefinite. \newline
[\cite{ventur:76} 1:05, {\emph{U kwentojil Santo K'in y Santo Uj}} `The Story of the Holy Sun and the Holy Moon', R.\,K.']
\exi{}
\gll	in 		tat=a, 			u 		k'ati 		{\textbf{jun}} 	{\textbf{tuul}} 		{\textbf{ix}} 		{\textbf{ch'up=u}}.  \\
	{\textsc{1a}}	father={\textsc{ev}},	{\textsc{3a}}	want		one		{\textsc{clf.anim}}	{\textsc{gm.f}}		woman={\textsc{ev}} \\
\glt 	`My father, he wants {\textbf{a woman/wife}}.'
\end{exe}

Example (\ref{3ex:31}) is uttered by a vulture to a woman whom he hopes to persuade to marry his father.  The father does not know this woman, so the woman referred to here is nonspecific.

\subsubsection{Bare nominal\is{bare nominal} for semantically nonspecific indefinite}\label{3sec:355}

In our discussion of examples (\ref{3ex:24}a-b) above, we mentioned that use of ART vs. a bare nominal\is{bare nominal} correlates with discourse salience\is{discourse salience} in the case of semantically specific referents.  This contrast also applies to semantically nonspecific referents, as shown in example (\ref{3ex:32}). In (\ref{3ex:32}), the speaker lists several hypothetical animals that he wants to hunt.  Some are marked by ART and some are bare nominals.\is{bare nominal}


\begin{exe}
\ex\label{3ex:32}
Semantically nonspecific indefinite reference. \newline
[Author's data, {\emph{Ix Che'il etel B\"ak'}} `Wild Woman', J.\,S.]
\exi{}
\gll	ix			kolool,	{\textbf{k'\"anb'ul}},	{\textbf{kox}} \\
	{\textsc{gm.f}}		partridge	pheasant		{cojolito (type of game bird)} \\
\glt
\exi{}
\gll	etel	{\textbf{a}}	{\textbf{kek'enche'}}	etel	{\textbf{a}}	{\textbf{yuk=u}} \\
	with	{\textsc{art}}	wild.pig			with	{\textsc{art}}	antelope={\textsc{ev}} \\
\glt
\exi{}
\gll	le'ek			kuchi			in		k'ati		tz'on-oo' 			pere		ma'		yan-{∅}		kut'an.\\
	{\textsc{3.emph}}	{\textsc{disc}}		{\textsc{1a}}	want		shoot-{\textsc{3.pl}} 	but		{\textsc{neg}}	exist-{\textsc{3b}}	{\textsc{3.quot}} \\
\glt	` ``[{\textsc{gm}}] Partridge\footnote{The word {\emph{kolool}} `partridge' is a feminine noun.  Recall (supra note 19) that for this subset of nouns a gender marker is essentially obligatory.}, [no {\textsc{art}}] {\textbf{pheasant}}, [no {\textsc{art}}] {\textbf{cojolito}} [type of game bird], and [{\textsc{art}}] {\textbf{wild pig}}, and [{\textsc{art}}] {\textbf{antelope}}, those are what I really want to hunt, but they aren't there!'' he said.'
\end{exe}

Even a hypothetical or non-existent referent can figure more or less centrally in discourse.  Recall that in example (\ref{3ex:30}) (\sectref{3sec:351} above), the protagonist repeatedly hunts for an antelope because he wants to impress a young woman with his prowess as a hunter.  Even though no specific antelope is being referred to in that example, a hypothetical antelope is important to the plot: the protagonist eventually tries to trick the woman by carrying a stuffed antelope skin past her house. In (\ref{3ex:32}), all the animal terms have equal status from the point of view of nonspecificity and all play the same syntactic role, but the referents differ in discourse salience:\is{discourse salience} ART is omitted before the names of the birds and retained before the names of the larger mammals that are more desirable as game.  The wild pig and antelope are also treated differently from the birds in that they are each introduced with the conjunction {\emph{etel}} `and/with'.


\section{Summary and conclusions}\label{3sec:4}

\tabref{3table:1} is a summary, according to Dryer's hierarchy, of the distribution of expressions that contribute to referential anchoring\is{referential anchoring} in Mopan narratives, as discussed in this chapter.  The table is not intended to make comprehensive quantitative claims. The double pluses mean that certain types of examples are easily found via investigation of multiple Mopan texts; the single pluses require more diligent searching.  Minus signs mean that we have not found any such examples despite diligent searching.


\begin{sidewaystable}
\begin{tabularx}{\textwidth}{Q@{}Z{1.8cm}Z{2.5cm}Z{3.1cm}Z{3.8cm}Z{2.2cm}@{}}
\lsptoprule
 & {{~\newline Anaphoric definites}} & {{~\newline Non-anaphoric\newline definites}} & {{Semantically\newline and pragmatically\newline specific indefinites}} & {{Semantically specific,\newline pragmatically\newline nonspecific indefinites}} & {{Semantically\newline nonspecific\newline indefinites}} \\
\midrule
{{ART}}						& ++ & ++ & +\footnote{Most likely to occur with high discourse salience.\is{discourse salience}} & + & + \\ %\hline
\tablevspace
{{Relativized deictic\is{relativized deictics} {\emph{b'e'}} }}		&++ & + & + & + & \minus \\ %\hline
\tablevspace
{{Emphatic pronoun\is{emphatic pronoun} {\emph{le'ek}} }}	&++ & + & + & + & \minus \\ %\hline
\tablevspace
{{NUM + CLF + ART}}				& \minus & \minus & + & \minus & \minus
\\
\tablevspace
{{NUM + CLF	}}				& \minus & \minus & + & + & + \\ %\hline
\tablevspace
{{Bare nominal}}					& + & + & + & +	 & + \\ %\hline
\lspbottomrule
\end{tabularx}
\caption{Summary of referential anchoring\is{referential anchoring} in Mopan in relation to Dryer's (\citeyear{dryer:14}) reference hierarchy\is{reference hierarchy} (with subdivisions for relative discourse prominence\is{discourse prominence} added)}\label{3table:1}
\end{sidewaystable}

{
The forms that have the most consistent connection to messages of definiteness are the relativized deictic\is{relativized deictics} predicate {\emph{b'e'}} and the emphatic pronoun\is{emphatic pronoun} {\emph{le'ek}}, found primarily at the most definite end of the hierarchy (and not found at the least definite end), and the {\emph{jun}} `one' + classifier construction,\is{classifiers!numeral classifiers} which is found primarily at the less definite end of the hierarchy, with a preference for contexts of specificity.\is{specificity}
}

Both ART and the bare nominal\is{bare nominal} option appear all across the hierarchy, from the maximally identifiable end (highly predictable anaphoric definites) to the least identifiable end (semantically and pragmatically nonspecific). Their distribution is not compatible with a semantics of (in)definiteness.\footnote{{For further discussion of Mopan in this connection, see \cite{contini:morava:danziger:fc}.}} Instead, ART is required in order to entitize lexical content that would not otherwise be construed as an entity/argument (see \sectref{3sec:21}, and \citealt{contini:morava:danziger:fc}).  With lexical content that lends itself to construal as an entity, ART is optional, and we have argued that its presence/absence is sensitive to the discourse salience\is{discourse salience} of the referent.  ART's tendency to occur most often at the definite end of Dryer's hierarchy follows from the fact that entities that are part of common ground between speaker and addressee tend also to be relatively salient in discourse.  But local contexts or nonlinguistic knowledge can lend salience\is{discourse salience} even to nonspecific indefinites. ART, in short, is not a form that is dedicated to signaling contrasts on the definiteness dimension.  Nevertheless, ART alone (without relativized deictic\is{relativized deictics} or numeral classifier\is{classifiers!numeral classifiers} construction) is one of the most common constructions with which arguments occur in Mopan.

Meanwhile, the fact that the bare nominal\is{bare nominal} construction is also allowable across all of the positions in the hierarchy makes clear that although  dedicated means for indicating definiteness or indefiniteness exist in Mopan, they are always optional. We conclude, then, that the status of a discourse referent with regard to relative familiarity,\is{familiarity} referentiality,\is{referentiality} specificity\is{specificity} and related notions normally considered as aspects of `definiteness' may be left unspecified in Mopan.\il{Mopan (Mayan)|)}  If it is found necessary to make such determination in a given case, this must frequently be accomplished through pragmatic inference, rather than via information explicitly signaled by particular grammatical forms.

\section*{Acknowledgments}
We thank the people of San Antonio village, Belize, and the National Institute for Culture and History, Belmopan, Belize. Many of the narratives from which examples are drawn were collected under a grant from the Wenner-Gren Foundation for Anthropological Research Grant (\#4850), and a Social Sciences and Humanities Research Grant of Canada Fellowship (\#452-87-1337); others with the support of the Cognitive Anthropology Research Group of the Max Planck Institute for Psycholinguistics, and of the University of Virginia, USA. 


\section*{Abbreviations and glosses}

\begin{tabularx}{.45\textwidth}{lQ}
{\textsc{a}} & set A case-role marker (Actor or Possessor) \\
{\textsc{anim}} & animate					  \\
{\textsc{b}} & set B case-role marker (Undergoer)		  \\
{\textsc{d}} & deictic 						  \\
{\textsc{disc}} & discourse particle 				  \\
{\textsc{emph}} & emphatic pronoun 				  \\
{\textsc{ev}} & prosodic echo vowel 				  \\
{\textsc{gm}} & gender marker 					  \\
\end{tabularx}
\begin{tabularx}{.45\textwidth}{lQ}
{\textsc{hsy}} & hearsay particle	                   \\
{\textsc{inan}} & inanimate\\
{\textsc{inch}} & inchoative \\
{\textsc{int}} & intensifier	\\
{\textsc{nv}} & non visible	\\
{\textsc{poss}} & Possessive	\\
{\textsc{prep}} & general preposition	\\
{\textsc{scope}} & scope clitic 	\\
{\textsc{t}} & transitivizer	\\
{\textsc{v}} & visible
\end{tabularx}


{\sloppy\printbibliography[heading=subbibliography,notkeyword=this]}
\end{document}
