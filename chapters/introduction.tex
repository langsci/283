\documentclass[output=paper]{langsci/langscibook}
\ChapterDOI{10.5281/zenodo.4049677}
\author{Kata Balogh\affiliation{Heinrich-Heine-Universität Düsseldorf}\and Anja Latrouite\affiliation{Heinrich-Heine-Universität Düsseldorf}\lastand  Robert D. Van Valin\quotesinglbase\ Jr.\affiliation{Heinrich-Heine-Universität Düsseldorf \& University at Buffalo}}
\title{Nominal anchoring: Introduction}  
\abstract{\noabstract}

\begin{document}
\maketitle

\section{Aims and motivations}\label{sec:intro:1}
It has been observed that a multitude of the world's languages can do without explicit formal marking of the concepts of definiteness and specificity\is{specificity} through articles (e.\,g., Russian, Tagalog, Japanese), while other languages (e.\,g., Lakhota) have very elaborate systems with more fine-grained distinctions in the domains of definiteness and specificity-marking.\is{specificity} The main questions that motivate this volume are: (1) How do languages with and without an article system go about helping the hearer to recognize whether a given noun phrase should be interpreted as definite, specific or non-specific? (2) Is there clear-cut semantic definiteness\is{definiteness!semantic} without articles or do we find systematic ambiguity regarding the interpretation of bare noun phrases? (3) If there is ambiguity, can we still posit one reading as the default? (4) What exactly do articles in languages encode that are not analyzed as straightforwardly coding (in)definiteness? (5) Do we find linguistic tools in these languages that are similar to those found in languages without articles?

The papers in this volume address these main questions from the point of view of typologically diverse languages. Indo-European is well represented by \ili{Russian}, \ili{Persian}, \ili{Danish} and \ili{Swedish}, with diachronic phenomena investigated in relation to the last two of these.  In terms of article systems,\is{article systems} they range from \ili{Russian}, which has no articles, the typical situation in most \ili{Slavic languages}, to \ili{Persian}, which has an indefinite article but no definite article, to the more complete systems found in Romance \il{Romance languages} and \ili{Germanic languages}. The three non-Indo-European languages investigated in this volume, namely Mopan (Mayan) \il{Mopan (Mayan)}, \ili{Vietnamese} and \ili{Siwi (Berber)}, are typologically quite diverse: Mopan \il{Mopan (Mayan)} is verb-initial and thoroughly head-marking, \ili{Vietnamese} is verb-medial and radically isolating, i.\,e., lacking inflectional and derivational morphology, and Siwi \il{Siwi (Berber)} is verb-initial with the signature Afro-Asiatic \il{Afro-Asiatic languages} trilateral roots which are the input to derivational and inflectional processes. What they have in common is the absence of articles signaling (in)definiteness. 

\section{Article systems\is{article systems} and related notions}

\citet[][p.4]{chesterman:91} points out that ``it is via the articles that definiteness is quintessentially realized, and it is in analyses of the articles that the descriptive problems are most clearly manifested. Moreover, it is largely on the basis of the evidence of articles in article-languages that definiteness has been proposed at all as a category in other languages.''

Here, we view definiteness as a denotational, discourse-cognitive category, roughly understood as identifiability\is{identifiability} of the referent to the speaker, instead of a grammatical (or grammaticalized) category, and therefore we can investigate the means that languages use for indicating definiteness or referential anchoring\is{referential anchoring} in general. Natural languages have various means to signal definiteness and/or specificity.\is{specificity} Languages differ in their article systems\is{article systems} as well as in the functions the set of articles they exhibit may serve. Simple article languages (e.\,g., \ili{English}, \ili{Hungarian}) generally distinguish definite and indefinite noun phrases by different articles, but they may also use their article inventory to code categories other than definiteness (e.\,g., Mopan Maya \il{Mopan (Mayan)}). Complex article languages like \ili{Lakhota}, which exhibits an elaborate and sophisticated system, always mark more than simple (in)definiteness. A great number of languages (e.\,g., \ili{Russian}, \ili{Tagalog}, \ili{Japanese}) have no or no clear-cut article systems\is{article systems} and rely on other means to encode definiteness distinctions. 

{
Most of the languages investigated in this volume belong to the last type. The means they use to help indicate how the referent of a noun phrase is anchored and intended to be interpreted include classifier systems\is{classifiers} (e.\,g., \ili{Vietnamese}, \ili{Chuj}), clitics (e.\,g., \ili{Romanian}), designated morphemes on nouns (e.\,g., \ili{Moksha}, \ili{Persian}) and syntactic position (e.\,g., \ili{Chinese}). In certain languages, alongside article systems\is{article systems} and morphosyntactic means, prosody plays a crucial role for the coding of (in)definiteness, for example  accent\is{accent} placement in Siwi \il{Siwi (Berber)} or tone in \ili{Bambara}.
}

\subsection{Basic notions: definiteness and specificity\is{specificity}}

{
In the cross-linguistic investigation and analysis of article systems\is{article systems} and noun phras-es, various different but related notions play a key role. In the analysis of various types of definite and indefinite noun phrases, the two most important notions are {\emph{definiteness}} and {\emph{specificity}},\is{specificity} together with further distinguishing notions of {\emph{uniqueness}},\is{uniqueness} {\emph{familiarity}},\is{familiarity} {\emph{discourse prominence}}\is{discourse prominence} and so on. In the following we give a brief introduction to these notions. 
Our aim is not to provide a detailed discussion of all notions and all theories, but to present an overview of the most important classical analyses relevant to the papers and their main issues in this volume. 
}

\subsubsection{Definiteness\is{definiteness|(}}

The notion of {\emph{definiteness}} itself is a matter of controversy, given the different uses of definite noun phrases for anaphoric linkage, relational dependencies, situational/deictic salience\is{discourse salience} or inherently uniquely referring nouns. The notion is used in a variety of ways by different authors. The classical analyses of definiteness distinguish two main lines of characterization: (1) the {\emph{uniqueness}}\is{uniqueness} analysis, following works by \citet{russell:05} and \citet{strawson:50}, and (2) the {\emph{familiarity}}\is{familiarity} account, after \citet{christophersen:39}, \citet{kamp:81} and \citet{heim:82}. 

{
In Russell's (\citeyear{russell:05}) analysis, indefinites have existential quantificational force, while definite\is{definite descriptions} descriptions\footnote{These mostly refer to noun phrases with a definite article, e.\,g., {\emph{the dog}}, but other expressions like possessive noun phrases and pronouns are also considered definite descriptions.\is{definite descriptions}} are considered referential. Definites assert {\emph{existence}} and {\emph{uniqueness}},\is{uniqueness} as illustrated in the logical translation of  sentences like (\ref{ex:russell}).
}

\begin{exe}
\ex\label{ex:russell}
The N is P. \\
$\exists x(N(x) \wedge \forall y(N(y) \rightarrow x=y) \wedge P(x))$ \\
a. there is an N (existence) \\
b. at most one thing is N (uniqueness)\is{uniqueness} \\
c. something that is N is P
\end{exe}


The meaning contribution of the definite article is to signal the existence of a unique referent (a-b), while the head noun provides sortal information of the referent (c). In the Russellian tradition, indefinites are distinguished from definites in terms of uniqueness,\is{uniqueness} as the predicate (sortal information) applies to exactly one referent. Russell's highly influential approach has inspired many theories on definiteness; similarly, various approaches point out critical issues in Russell's theory.  The most intriguing issues discussed in the literature are the problem of presuppositionality, the problem of incomplete descriptions and the problem of referentiality.\is{referentiality} To solve these crucial issues a great number of theories have been proposed over the decades. In Strawson's (\citeyear{strawson:50}) account, existence and uniqueness\is{uniqueness} are presupposed rather than asserted. He claims that if the presupposition fails, the sentence does not bear a truthvalue, i.\,e., it is neither true nor false. The incompleteness problem, where the definite description does not have a unique referent, inspired several authors \citep[e.\,g.,][]{strawson:50,mccawley:79,lewis:79,neale:90} to offer various solutions, like contextual restriction and the prominence/saliency approach. The latter was proposed by \citet{mccawley:79} and \citet{lewis:79}, who argues that definite descriptions\is{definite descriptions} refer to the most prominent or most salient referent of a given context. \citet{donnellan:66} argues that definite descriptions\is{definite descriptions} have two different uses: an attributive and a referential use. The former use can be characterized similarly to Russell's account, while the latter use requires a different analysis. Donnellan's famous example is (\ref{ex:donnellan}), which can be used in different ways in different situations. 

\begin{exe}
\ex\label{ex:donnellan}
Smith’s murderer is insane. \hfill\citep[][p.285]{donnellan:66}
\end{exe}

In a situation where the murderer is unknown (e.\,g., at the scene of the crime), the noun phrase `Smith's murderer' is understood attributively as meaning that whoever murdered Smith is insane. On the other hand, in a different situation where the murderer is known (e.\,g., at the trial), the noun phrase can be replaced by, for example, he, as it is used referentially, referring to the individual who is the murderer.

{
The other highly influential classical account of definites represents a different view. These theories follow the work by \citet{christophersen:39}, who accounts for the interpretation of definites in terms of {\emph{familiarity}}\is{familiarity} rather than uniqueness.\is{uniqueness} In his theory, definite descriptions\is{definite descriptions} must be discourse-old, already introduced in the given discourse context, and as such known to the hearer. Christophersen's familiarity\is{familiarity} account inspired famous theories in formal semantics: File Change Semantics [FCS]\is{File Change Semantics} of Heim (\citeyear{heim:82}) and the similar Discourse Representation Theory [DRT],\is{Discourse Representation Theory} which was developed in parallel and introduced by \citet{kamp:81} and \citet{drt:93}. One of the major contributions of these approaches is the solution for the so-called `donkey sentences' (\ref{ex:donkey.a}), and further issues of the interpretation of discourse anaphora\is{anaphora!discourse anaphora} (\ref{ex:donkey.b}). 
}

\begin{exe}
\ex
	\begin{xlista}
	\ex\label{ex:donkey.a}
	If a farmer$_i$ owns a donkey$_j$, he$_i$ beats it$_j$.
	\ex\label{ex:donkey.b}
	A student$_i$ came in. She$_i$ smiled.
	\end{xlista}
\end{exe}

In both sentences, the indefinite noun phrases can be referred to by an anaphoric expression in the subsequent sentence. Based on such examples, they propose a division of labour between indefinite and definite noun phrases. Indefinites like {\emph{a student}} introduce new discourse referents, while definite noun phrases like {\emph{the student}} pick up a referent that has already been introduced, similarly to anaphoric pronouns.

In his \citeyear{lobner:85} paper, L\"obner argues for a relational approach and against the uniqueness\is{uniqueness} approach, claiming that it is rather non-ambiguity that is essential for definiteness. \citet{lobner:85} distinguishes noun phrases by their type of use. The main distinction is into {\emph{sortal}} and {\emph{non-sortal}} nouns, where the latter is further divided into {\emph{relational}}\is{concept types!relational concept} and {\emph{functional}}\is{concept types!functional concept} nouns\is{noun types} and concepts.\is{concept types} Relational nouns\is{noun types!relational noun} include kinship terms\is{kinship terms} (e.\,g., {\emph{sister}}), social relations (e.\,g., {\emph{friend}}) and parts (e.\,g., {\emph{eye}}), while functional nouns\is{noun types!functional noun} are roles (e.\,g., {\emph{wife, president}}), unique parts (e.\,g., {\emph{head, roof}}), conceptual dimensions (e.\,g., {\emph{height, age}}) and singleton events (e.\,g., {\emph{birth, end}}). In the analysis of definite descriptions,\is{definite descriptions} \citet{lobner:85} distinguishes semantic and pragmatic definites.\is{definites!pragmatic definites} For semantic definites\is{definites!semantic definites} ``the referent of the definite is established independently of the immediate situation or context of the utterance'' \citep[][p.298]{lobner:85}, while pragmatic definites\is{definites!pragmatic definites} are ``dependent on special situations and context for the non-ambiguity of a referent'' \citep[][p.298]{lobner:85}. One- and two place functional concepts\is{concept type!functional concept} (\ref{ex:lobner.1}), as well as configurational uses (\ref{ex:lobner.2}), are considered semantic definites.\is{definites!semantic definites} Löbner claims that statements like (\ref{ex:lobner.2}) are impossible with sortal nouns.\is{noun types!sortal noun} 

\begin{exe}
\ex\label{ex:lobner.1}
1-place functional concepts:\is{concept types!functional concept} {\emph{the time, the last party, the other girl}}, etc. \\
2-place functional concepts:\is{concept types!functional concept} {\emph{my wife, the author, the president of France}}, etc.
\ex\label{ex:lobner.2}
He was the son of a poor farmer.	\hfill\citep[][ex.17]{lobner:85}
\end{exe}

As L\"obner argues, this distinction is significant in various ways; for example, functional nouns\is{noun types!functional noun} can only take the definite article (with the exception of existential contexts). Further examples he gives are of German cliticization (\ref{ex:loebner.exs}), where the cliticized article encodes a semantic definite\is{definites!semantic definites} as opposed to a non-cliticized one. In various languages, there are different articles, often referred to as weak and strong \citep{schwarz:19}, encoding semantic and pragmatic definites.\is{definites!pragmatic definites} This distinction can be found, for example, in the Fering (F\"ohr) dialect of \ili{Frisian} \citep[e.\,g.,][]{ebert:71} and in the Rheinland dialect of \ili{German} \citep[e.\,g.,][]{hartmann:82}.

\newpage
\begin{exe}
\ex\label{ex:loebner.exs}
{\textbf{German:}}
	\begin{xlista}
	\ex
	\gll Er {mu\ss} {ins/*in das} Krankenhaus.\\			
	he must {in=the/*in the} hospital\\
	\glt `He must go to hospital again.'\\\ 
	\hfill\citep[from][ex.52, our glosses]{lobner:85}
	
	\ex
	\gll Er {mu\ss} wieder {*ins/in das} Krankenhaus zurück, aus dem er schon entlassen war.\\
	he must again {in=the/in the} hospital back from the.{\sc{dat}} he already discharged was\\
	\glt `He has to go back to the hospital from which he had already been discharged.' \hfill\citep[from][ex.54, our glosses]{lobner:85}
	\end{xlista}
\end{exe}

\begin{exe}
\ex {\textbf{Fering (F\"ohr):}} \hfill\citep[][p.161]{ebert:71}
	\begin{xlista}
	\ex
	\gll Ik skal deel tu a/*di kuupman. \\
	I must down to the$_{\text{W}}$/the$_{\text{S}}$ grocer \\
	\glt `I have to go down to the grocer.'
	\ex
	\gll Oki hee an hingst keeft. *A/di hingst haaltet. \\
	Oki has a horse bought \ the$_{\text{W}}$/the$_{\text{S}}$ horse limps \\
	\glt `Oki has bought a horse. The horse limps.'	
	\end{xlista}
\end{exe}

As for the meaning contribution of the definite article, based on the different noun/concept types\is{concept types} and their uses, \citet{lobner:85} argues that the definite article indicates that the given noun must be taken as a functional concept. \is{definiteness|)}

\subsubsection{Specificity\is{specificity}}
The notion of {\emph{specificity}} \citep[see, e.\,g.,][]{vonheus:11} is also defined and characterized in different ways and in relation to a variety of factors. Specificity\is{specificity} is generally used to distinguish various readings of indefinites. A generally accepted view is that sentences like (\ref{ex:spec}) can be interpreted in two ways, depending on whether the speaker has a particular entity in mind, referred to by the indefinite noun phrase.

\begin{exe}
\ex\label{ex:spec}
{\emph{Mia kissed a student yesterday.}} \\
1. whoever Mia kissed is a student (non-specific) \\
2. there is a specific student whom Mia kissed (specific)
\end{exe}

As a linguistic notion, the opposition between the non-specific and the specific readings of indefinites is characterized in relation to a variety of factors. \citet{farkas:94} distinguishes referential, scopal and epistemic specificity.\is{specificity!epistemic} Specific indefinites refer to an individual, and hence can be anaphorically referred back to. With respect to the second reading of (\ref{ex:spec}), the sentence could be followed by {\emph{He is tall}}, while this is not possible after the first reading. In relation to other operators, specific indefinites take a wide scope. The epistemic opposition is very close to (if not the same as) the referential opposition, as it is characterized by the fact that, by using specific indefinites, the speaker has a referential intention,\is{referential intention} i.\,e., they have a certain individual in mind \citep{karttunen:68,farkas:94}. In addition to Farkas's  (\citeyear{farkas:94}) three-way distinction, \citet{vonheus:11,vonheus:19} proposes four more oppositions, namely partitivity,\is{partitivity} noteworthiness,\is{noteworthiness} topicality\is{topicality} and discourse prominence.\is{discourse prominence} As \citet{enc:91} argues, specific indefinites are discourse-linked and inferable: they refer to a part of a set previously introduced to the discourse. As a motivation, she shows that this distinction is overtly marked in \ili{Turkish}: accusative marked direct objects are interpreted specifically (\ref{ex:enc.a}), while unmarked objects are taken as non-specific (\ref{ex:enc.b}).

\begin{exe}
\ex
	\begin{xlista}
	\ex\label{ex:enc.a}
	\gll Ali bir piyano-yu kiralamak istiyor.\\
	Ali one piano-{\sc{acc}} to-rent wants\\
	\glt `Ali wants to rent a certain piano.' \hfill \citep[][ex.12]{enc:91}
	\ex\label{ex:enc.b}
	\gll Ali bir piyano kiralamak istiyor.\\
	Ali one piano to-rent wants\\
	\glt `Ali wants to rent a (non-specific) piano.' \hfill \citep[][ex.13]{enc:91}
	\end{xlista}
\end{exe}

{
The relevance of noteworthiness\is{noteworthiness} is often illustrated by the use of the marked indefinite {\emph{this N}} construction. Such examples can only be followed by newsworthy/interesting/particular information regarding the noun phrase. 
}

\begin{exe}
\ex He put this 31-cent stamp on the envelope, \hfill\citep[after][]{maclaran:82}
	\begin{xlista}
	\ex and only realized later that it was worth a fortune.
	\ex \#so he must want it to go airmail.
	\end{xlista}
\end{exe}	

Topicality\is{topicality} and discourse prominence\is{discourse prominence} are also closely related to specificity.\is{specificity} Indefinite noun phrases that are topical receive a specific interpretation. This can be shown by \ili{Hungarian} examples, where topicality\is{topicality} is syntactically marked by placement to a left-peripheral position within the clause (\ref{ex:hun}). 

\begin{exe}
\ex\label{ex:hun}
\gll Egy di\'ak be-kopogott az igazgató-hoz. \\
a student {\sc{vprt}}-knocked the director-{\sc{all}}\\
\glt `A (particular) student knocked at the director's office.'
\end{exe}

The left-peripheral topic\is{topic} position can only host referential and specific noun phrases \citep[e.\,g.,][]{ekiss:02}, and hence the indefinite noun phrase can only be in the topic\is{topic} position when it is interpreted specifically.


\section{Contributions}

The papers in this volume address to different degrees the general questions introduced in \sectref{sec:intro:1}. Most contributions report on research on different corpora and elicited data or present the outcome of various experimental studies. One paper presents a diachronic study of the emergence of article systems.\is{article systems} As mentioned before, the volume covers typologically diverse languages: \ili{Vietnamese}, \ili{Siwi (Berber)}, \ili{Russian}, \ili{Mopan (Mayan)}, \ili{Persian}, \ili{Danish} and \ili{Swedish}.

\subsection{Languages with articles}

If a language is analyzed as having an article, the standard expectation is that it will express either definiteness or indefiniteness. However, the number of papers introducing article-languages in which the determiners do not encode different degrees of identifiability\is{identifiability} and uniqueness\is{uniqueness} is on the rise \citep[e.\,g.,][]{lyon:15}. The crucial question is what features an element is required to exhibit to be counted as an article. If the answer is given in line with \citet{himmelmann:01} and others, then no functional element that does not convey some degree of specificity\is{specificity} is counted as an article. If Dryer's (\citeyear{dryer:14}) characterization of articles is adopted, then all functional elements that occur with high frequency in noun phrases, indicate argumenthood and vary for grammatical features are included in the category.

Eve Danziger and Ellen Contini-Morava 
adopt Dryer's (\citeyear{dryer:14}) view in their contribution {\emph{Referential anchoring\is{referential anchoring} without a definite article: The case of \ili{Mopan (Mayan)}}} and investigate all the means that Mopan utilizes in order to evoke relative identifiability\is{identifiability} and uniqueness.\is{uniqueness} While, based on formal and distributional criteria, the Yucatecan language \il{Yucatecan languages} Mopan exhibits a determiner of the type usually classified as an article, they find that this article does not encode any of the semantic notions of definiteness, specificity\is{specificity} and uniqueness.\is{uniqueness}  It merely serves to express that a given lexeme is used as an argument in the sentence. In their analysis and explanation, they build upon Dryer's (\citeyear{dryer:14}) definiteness hierarchy\is{definiteness hierarchy} and demonstrate that the article itself, as well as the bare nominal\is{bare nominal} form, can occur in any position in Dryer's definiteness hierarchy.\is{definiteness hierarchy} This observation leads to an investigation of exactly what the discourse-pragmatic function of the article is and how it can be calculated. The authors' conclusion is that the contribution of the article is best characterized by factors such as discourse salience,\is{discourse salience} which contexts or world knowledge may lend even to non-specific indefinites. 

In their paper, {\emph{The specificity marker}} -e {\emph{with indefinite noun phrases in Modern Colloquial Persian}},\il{Persian!Modern Colloquial} Klaus von Heusinger and Roya Sadeghpoor focus on the specificity\is{specificity} marker {\emph{-e}} and its compatibility with two indefinite markers and investigate the different kinds of indefinite readings that arise. In their experimental pilot studies, they test  and provide some support for the hypothesis that the difference in interpretation between the combinations lies in the anchoring of the referents, i.\,e., in whether the referent is construable as speaker-specific or non-speaker-specific. The studies thereby provide additional evidence for the need to assume a fine-grained approach in the investigation of specificity\is{specificity} and referential anchoring\is{referential anchoring} \citep{vonheus:02}. However, they also show that specificity-unrelated semantic properties like {\emph{animacy}}\is{animacy} need to be taken into account in the explanation of their results.

The contribution {\emph{Indirect anaphora\is{anaphora!indirect anaphora} from a diachronic perspective: The case of \ili{Danish} and \ili{Swedish}}} by Dominika Skrzypek is the only diachronic study in this volume. The author investigates different kinds of indirect anaphora\is{anaphora!indirect anaphora} ({\emph{associative anaphora,\is{anaphora!associative anaphora} bridging anaphora\is{anaphora!bridging anaphora}}}) as one of the steps in the grammaticalization\is{grammaticalization} process towards a definite article from the beginning of the 13th century until the middle of the 16th century. The paper is particularly concerned with the distribution and use of indirect anaphora\is{anaphora!indirect anaphora} and the features that the relationship between indirect anaphora\is{anaphora!indirect anaphora} and their anchor is based on. Looking at inalienable and other types of indirect anaphora,\is{anaphora!indirect anaphora} the author shows that indirect anaphora\is{anaphora!indirect anaphora} form a heterogeneous concept and are not easily positioned in the strong-weak definiteness dichotomy.\is{definiteness!strong}\is{definiteness!weak} The evidence points to the fact that the definite article did not spread uniformly through indirect anaphora\is{anaphora!indirect anaphora} in \ili{Danish} and \ili{Swedish}.

\subsection{Languages without articles}

{
In article-less languages, the encoding of definiteness is often a complex matter, where various linguistic factors play a role. Japanese and Chinese are both languages that are well known for lacking an article system. In Japanese, argument phrases are marked by case markers (nominative: {\emph{ga}}, accusative: {\emph{wo}}, dative: {\emph{ni}}) or non-case markers like the topic\is{topic} marker {\emph{wa}} or the additive marker {\emph{mo}} `also'. Consequently, definiteness is not straightforwardly grammaticalized, but rather considered an interpretational category \citep[e.\,g.,][]{tawa:93}, for which classifiers\is{classifiers} play a crucial role. The same holds for Chinese, which lacks case markers, but exhibits even more numeral classifiers\is{classifiers!numeral classifiers} than Japanese. These have been argued by \citet{cheng:sybesma:99} and others to play a crucial role for the definiteness reading of noun phrases, whenever numeral information is missing. However, \citet[][p.1129]{peng:04} notes that for indeterminate expressions ``there is a strong but seldom absolute correlation between the interpretation of identifiability\is{identifiability} or nonidentifiability and their occurrence in different positions in a sentence''. \citet{simpson:etal:11}, who study bare classifier\is{classifiers} definites in \ili{Vietnamese}, \ili{Hmong} and \ili{Bangla}, also find that classifiers\is{classifiers} are relevant to nominal anchoring.\is{nominal anchoring} However, the fact that bare noun phrases also seem to be able to receive definite interpretations weakens the claim that classifiers\is{classifiers} are the morphosyntactic key to definiteness interpretation, and rather points to the fact that a multilevel approach  proposed by \citet{heine:98} is better in explaining how definiteness or specificity\is{specificity} interpretations arise. 
}

Walter Bisang and Kim Ngoc Quang, in their study {\emph{(In)definiteness and \ili{Vietnamese} classifiers\is{classifiers}}}, %(p. xx-xx), 
contribute to our understanding of the classifier language \ili{Vietnamese}. They investigate which linguistic factors influence the interpretation of phrases with numeral classifiers\is{classifiers!numeral classifiers} [CL] in bare classifier\is{classifiers} constructions as either definite or indefinite and point out the licensing contexts for the different uses and readings of nominal classifiers.\is{classifiers!nominal classifiers} They find a striking clustering of definite interpretations with animacy\is{animacy} and subject status, whereby definiteness is understood as identifiability\is{identifiability} in discourse. Indefinite interpretations, on the other hand, are predominantly witnessed in certain sentence types (existential sentences\is{existential sentences} and thetic sentences)\is{thetic statements} and with certain types of verbs (verbs of appearance). A crucial finding is that noun class type,  following \citet{lobner:85,lobner:11}, and factors like {\emph{animacy}}\is{animacy} and {\emph{grammatical relation}} are less important than information structure\is{information structure} for the appearance of classifiers\is{classifiers} in definite and indefinite contexts. Classifiers\is{classifiers} are shown to be associated with pragmatic definiteness,\is{definiteness!pragmatic} rather than semantic definiteness,\is{definiteness!semantic} i.\,e., identifiability\is{identifiability} rather than uniqueness.\is{uniqueness} Furthermore, the authors provide evidence that contrastive topics,\is{topic!contrastive topic} contrastive focus\is{focus!contrastive focus} and focus particles\is{focus!focus particles} correlate with the use of classifier\is{classifiers} constructions. Similar to the constructions discussed for \ili{Persian} and \ili{Mopan (Mayan)} in this volume, the classifier\is{classifiers} construction in \ili{Vietnamese} can be once more viewed as a construction whose final interpretation depends, on the one hand, on discourse prominence\is{discourse prominence} and, on the other hand, on features of the morphosyntax-semantics interface that are well known for contributing to the overall saliency of a phrase.
													
In her contribution, {\emph{Accent on nouns and its reference coding in Siwi Berber}}, %(p. xx-xx), 
Valentina Schiattarella investigates definiteness marking in Siwi Berber,\il{Siwi (Berber)} an indigenous Berber language spoken in Egypt. In Siwi, a language without articles, it is claimed that the placement of accent\is{accent} on the last syllable versus the penultimate syllable encodes indefiniteness and definiteness respectively, i.\,e., the accent\is{accent} on the last syllable is generally assumed to encode indefiniteness and the accent\is{accent} on the penultimate syllable to encode definiteness. This default interpretation can be overridden, as Schiattarella shows in her paper. She analyzes various corpus data from spontaneous discourse and guided elicitations to further examine the role of various morphosyntactic means (e.\,g., possessive constructions, demonstratives,\is{demonstratives} prepositions and adpositional phrases) as well as pragmatic aspects (e.\,g., anaphoricity,\is{anaphoricity} familiarity,\is{familiarity} uniqueness,\is{uniqueness} reactivation and information structural considerations) in influencing the interpretation of noun phrases. The author, furthermore, finds that right- and left-detached constructions or the appearance of a demonstrative,\is{demonstratives} a possessive marker or relative clause in postnominal position influences the interpretation.

Olga Borik, Joan Borr\`as-Comes and Daria Seres, in {\emph{Preverbal (in)definites in \ili{Russian}: An experimental study}}, %(p. xx-xx), 
present an experimental study on \ili{Russian} bare nominal\is{bare nominal} subjects, and investigate the relationship between definiteness, linear order and discourse linking.\is{discourse linking (D-linking)} Given that \ili{Russian} lacks articles and has very flexible word order, it is widely assumed that (in)definiteness correlates with the position of a noun phrase in the clause, i.\,e., preverbal position is associated with a definite reading and postverbal with an indefinite interpretation. The authors experimentally verify that this correlation basically holds, but they also find that speakers accept a surprising number of cases in which a preverbal NP is interpreted as indefinite, which leads to the conclusion that \ili{Russian} bare nouns are basically indefinite. The unexpected correlations between position and interpretation lead to further investigations of the relevant factors involved and the suggestion that, regardless of topicality,\is{topicality} discourse linking\is{discourse linking (D-linking)} principles following \citet{pesetsky:87} and \citet{dyakonova:09} facilitate the use of indefinite nominals in the unexpected preverbal position.

\subsection{Summary}
{
The papers in this volume deal with pragmatic notions of definiteness and specificity.\is{specificity} The studies presented here provide the following findings regarding our initial motivating questions. On the issue of how languages with and without articles guide the hearer to the conclusion that a given noun phrase should be interpreted as definite, specific or non-specific, the studies in this paper argue for similar strategies. The languages investigated in this volume use constructions and linguistic tools that receive a final interpretation based on discourse prominence\is{discourse prominence} considerations and various aspects of the syntax-semantics interface. In case of ambiguity between these readings, the default interpretation is given by factors (e.\,g., familiarity,\is{familiarity} uniqueness)\is{uniqueness} that are known to contribute to the salience\is{discourse salience} of phrases, but may be overridden by discourse prominence.\is{discourse prominence} Articles that do not straightforwardly mark (in)definiteness encode different kinds of specificity.\is{specificity} In the languages studied in this volume, whether they have an article system or not, similar factors and linguistic tools are involved in the calculation process of interpretations.
}

\section*{Acknowledgments}
The volume contains revised selected papers from the workshop entitled {\emph{Specificity, definiteness and article systems across languages}} held at the 40th Annual Conference of the German Linguistic Society (DGfS), 7-9 March, 2018 at the University of Stuttgart. We very much appreciate the contributions of all participants in the workshop, who enriched the event with their presence, questions, presentations and comments. Special thanks to the DFG for all financial support and their funding of our project D04 within the SFB 991 in D\"usseldorf, which made the workshop and this editing work possible.
Many thanks to the series editors Philippa Cook, Anke Holler and Cathrine Fabricius-Hansen for carefully guiding us through from the submission to the completion of this volume.
%As for the reviewing process proper, we are indebted to the following reviewers:
%* Many thanks to the series editors Martin Haspelmath and Sebastian Nordhoff for carefully guiding us through from the submission to the completion of this volume. *


{\sloppy\printbibliography[heading=subbibliography,notkeyword=this]}
\end{document}
