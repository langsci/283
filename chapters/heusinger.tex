\documentclass[output=paper]{langsci/langscibook}
\ChapterDOI{10.5281/zenodo.4049685}

\author{Klaus von Heusinger\affiliation{Universität zu Köln}\lastand Roya Sadeghpoor\affiliation{Universität zu Köln}}

\title{The specificity marker -$e$ with indefinite noun phrases in Modern Colloquial Persian}
\shorttitlerunninghead{The specificity marker -$e$ in Persian} 

\abstract{Persian\il{Persian|(} has two indefinite markers, the prenominal {\emph{ye(k)}} and the suffixed {\emph{-i}}. Both forms express particular kinds of indefiniteness, as does their combination: for Modern Colloquial Persian,\il{Persian!Modern Colloquial|(} indefinites ending in {\emph{-i}} express a non-uniqueness or anti-definite implication and behave similarly to {\emph{any}} in English.\il{English} {\emph{Ye(k)}}, on the other hand, expresses an at-issue existence implication and behaves similarly to the English\il{English} {\emph{a(n)}} \citep{jasbi:16}. The combination of {\emph{ye(k)}} and {\emph{-i}} expresses an ignorance implication. Modern Colloquial Persian has the specificity\is{specificity|(} marker {\emph{-e}}, which can be combined with {\emph{ye(k) NP}}, as well as with the combined form of {\emph{ye(k) NP-i}}, but not with (solitary) {\emph{NP-i}} \citep{windfuhr:79,ghomeshi:03}. In this paper, we investigate the function of the indefinite form when combined with the specificity marker {\emph{-e}}, namely {\emph{ye(k) NP-e}} and {\emph{ye(k) NP-e-i}}. We present two pilot studies that tested our hypothesis, which is that the contrast between these two specific forms depends on whether the specificity is speaker-anchored, as for {\emph{ye(k) NP-e}}, or non-speaker anchored, as for {\emph{ye(k) NP-e-i}}. The results of the two studies provide weak support for this hypothesis, and provide additional evidence for the fine-grained structure of specificity as referential anchoring\is{referential anchoring} \citep{vonheus:02url}.}

\begin{document}
\maketitle

\section{Introduction}
Persian is a language with no definite marker and two indefinite markers. In Modern Colloquial Persian, the prenominal indefinite article {\emph{ye(k)}} `a(n)' marks an NP as indefinite and expresses an existential entailment `there is at least one N', as in (\ref{4ex:1}), similar to a noun phrase with the indefinite article in English.\il{English} In Modern Colloquial Persian, the suffixed (or enclitic) marker {\emph{-i}} is interpreted as a negative polarity item (NPI),\is{negative polarity item} as in (\ref{4ex:2}), similar to the English\il{English} {\emph{any}} \citep{jasbi:14,lyons:99,windfuhr:79}. Both indefinite markers can be combined into a complex indefinite, consisting of {\emph{ye(k) NP-i}}, which is interpreted as a free-choice item,\is{free-choice!free-choice item} as in (\ref{4ex:3}), or with a certain `flavor' of referential ignorance, as in (\ref{4ex:4}), similar to {\emph{some or other}} in English\il{English} \citep{jasbi:16}.\footnote{{Persian has a differential object marker\is{differential object marking} {\emph{-ra/-ro/-a/-o}} (generally glossed as {\emph{-rā}} or as {\sc{om, dom}} or {\sc{acc}}), which is obligatory with definite and specific direct objects, and optional with non-specific indefinite direct objects \citep{ghomeshi:03,karimi:03,karimi:18, lazard:57,lazard:92,windfuhr:79}.}}


\begin{exe}
\ex{\label{4ex:1}
\gll 	Emruz yek māšin tu xiābun didam. 	\\
  	today a car at street saw.{\sc{1sg}}	\\}\jambox*{\emph{ye(k) NP} (existential)}
\glt `Today I saw a car on the street.'

\ex{\label{4ex:2}
\gll	Māšin-i ro emruz tu xiābun didi?	 \\
	car-i rā today at street saw.{\sc{2sg}}	\\}\jambox*{\emph{NP-i} (negative polarity item\is{negative polarity item})}
\glt	`Did you see any cars on the street today?'

\ex\label{4ex:3}
Context: Ali wants to play the lottery. Reza is explaining to him how it is played.
\exi{}
\gll	Ye šomāreh-i  ro 	entexāb   kon        va  injā		alāmat 	bezan.			\hspace*{0.6cm} {\emph{ye(k) NP-i}}\hspace*{-4mm}\\
	ye number-i   rā 	choose    do.{\sc{2sg}}  and here  	mark    	do.{\sc{2sg}}	 ~ {(free choice)\hspace*{-4mm}}\\
\glt	`Choose a number and mark it here.'

\ex\label{4ex:4}
\gll	Yek bače-i		tu xiābun	gom	šode				bud.			\hspace*{-.1cm}{\emph{ye(k) NP-i} (ref. ignorance)}\\
	ye     child-i   	at street	lost	became.{\sc{3sg}} 	was.{\sc{3sg}}	  \\
\glt	`A/some child was lost in the street.'
\end{exe}

Modern Colloquial Persian has the optional suffix {\emph{-e}}, which we take to express specificity. The literature assumes different functions of this suffix, such as a demonstrative,\is{demonstratives} a definite, or a referential function (\citealt[][40]{windfuhr:79}; \citealt[][173-177]{hincha:61}; \citealt[][163]{lazard:57}; \citealt[][67]{ghomeshi:03}) or familiarity\is{familiarity} of the referent \citep{hedberg:etal:09} as in the anaphoric noun {\emph{pesar-e}} in (\ref{4ex:5}):

\begin{exe}
\ex\label{4ex:5}
\gll	Emruz	ye 	pesar	va	ye 	doxtar 	ro   did-am.    		Pesar-(e)	tās	bud. \\
	today	a	boy		and	a	girl      	rā   saw-{\sc{1sg}} 	boy-e	bald 	was.{\sc{3sg}} \\
\glt	`Today I saw a boy and a girl. The boy was bald.'
\end{exe}

{
The suffix {\emph{-e}} is typically used with demonstrative\is{demonstratives} and definite noun phrases, but it can also be combined with the indefinite constructions discussed above, which we take as evidence that it expresses specificity (or referential indefiniteness): (i) its combination with the indefinite marker {\emph{ye(k)}}, i.\,e., {\emph{ye(k) NP-e}}, as in (\ref{4ex:6}), yields a specific reading; (ii) it cannot be combined with suffixed indefinite {\emph{-i}}: *{\emph{NP-e-i}}, as in (\ref{4ex:7}), due to the incompatibility of the specific function of {\emph{-e}} and the free-choice function\is{free-choice!free-choice function} of {\emph{-i}}; (iii) the specific marker {\emph{-e}}, however, can be combined with the complex indefinite {\emph{ye(k) NP-e-i}}, yielding a specific reading in (\ref{4ex:8}), which is very similar to (\ref{4ex:6}).
}

\begin{exe}
\ex\label{4ex:6}
\gll	{\bf{Ye}}	{\bf{pesar-e}}	injā	kār	mikone. \\
	{\bf{a}}	{\bf{boy-e}}	here	work	do.{\sc{3sg}} \\
\glt	`A specific boy works here.'
\ex\label{4ex:7}
\gll	Diruz		{\bf{māšin-i}} jolo-e       dare      	xune  didi?       			\\
	yesterday	{\bf{car-i}}	    front.of-{\sc{e}} 	door.of 	home saw.{\sc{2sg}} 	\\
\glt
\exi{}
\gll	– *Na, man hič 	māšin-e-i  	nadidam. \\
	{} \ no I {\bf{any}} {\bf{car-e-i}}   not.saw.{\sc{1sg}} \\ 
\glt	`Did you see any cars in front of the house door yesterday?' \\
	Intended reading: `No, I didn't see any specific car.'
\ex\label{4ex:8}
\gll	Emruz	{\bf{ye}}	{\bf{māšin-e-i}}		az		pošt		behem 		zad. \\
 	today	{\bf{a}} 	{\bf{car-e-i}} 	 	from		behind	to.me		collided.{\sc{3sg}} \\
\glt	`Today a specific car collided into me from behind.'
\end{exe}

These data, then, raises the following questions. First, what are the differences in the meanings of the three forms expressing indefiniteness in (\ref{4ex:1}) through (\ref{4ex:4}) in Modern Colloquial Persian? Second, what is the contribution of the marker {\emph{-e}}? Does it express specificity or a different semantic pragmatic notion, such as referentiality,\is{referentiality} demonstrativeness, topicality,\is{topicality} or partitivity?\is{partitivity} Third, what is the function of the marker {\emph{-e}} with indefinite constructions, and, more specifically, what is the difference between the two (specific) indefinite constructions {\emph{ye(k) NP-e}} and {\emph{ye(k) NP-e-i}}? We assume the following functions of the three indefinite constructions \citep[cf.][]{jasbi:14,lyons:99,windfuhr:79}: (i) the indefinite marker {\emph{ye(k)}} signals a regular indefinite, i.\,e., it expresses an existential entailment, but does not encode specificity (like the English\il{English} {\emph{a(n)}}); (ii) the suffixed marker {\emph{-i}} is a negative polarity item\is{negative polarity item} (like the English\il{English} {\emph{any}}); (iii) the combination of the two markers, resulting in {\emph{ye(k) NP-i}}, shows an ignorance or free-choice implicature.\is{free-choice!free-choice implication}


Second, we assume that the marker {\emph{-e}} in Modern Colloquial Persian signals specificity in terms of ``referential anchoring'',\is{referential anchoring} in accordance with \cite{vonheus:02url}. An indefinite is referentially anchored if the speaker, or another prominent discourse referent, can readily identify the referent. This more fine-grained notion of specificity allows us to formulate our Hypothesis 1, about the semantic difference between the two indefinite constructions with the specificity marker, namely {\emph{ye(k) NP-e}} and {\emph{ye(k) NP-e-i}}; the specific indefinite construction {\emph{ye(k) NP-e}} only reflects the intention of the speaker (or speaker-oriented specificity), while the form {\emph{ye(k) NP-e-i}} only expresses the intention of another salient discourse participant (i.\,e., non-speaker-oriented specificity).

In \sectref{4sec:2}, we provide a brief overview of the variety of indefinites found in different languages, as well as the ranges of different functions that indefinites can take. In particular, we focus on the contrast between speaker-oriented specificity and non-speaker-oriented specificity. In \sectref{4sec:3}, we discuss the different functions of the indefinite markers in Modern Colloquial Persian and modify the approach of \cite{jasbi:16}. In \sectref{4sec:4}, we present some relevant data for the use of the marker {\emph{-e}} in Modern Colloquial Persian, and in \sectref{4sec:5}, we present the two pilot studies that addressed our hypotheses about the speaker-oriented specificity of these forms. Finally, \sectref{4sec:6} provides a discussion and a conclusion.


\section{Indefinites in the languages of the world}\label{4sec:2}
\largerpage
\subsection{Indefinite articles}\label{4sec:21}

Languages differ as to whether or not they mark indefinite noun phrases with special morphological means, such as indefinite articles. In Dryer's (\citeyear{dryer:05}) WALS sample, 57\% of the languages do not have indefinite articles.

\begin{table}[b]
\begin{tabularx}{\textwidth}{Xrr}
\lsptoprule
Type of article system & Instances & Percentages \\
\midrule
Indefinite word distinct from numeral for `one' & 91 & 19\% \\
Numeral for `one' is used as indefinite article & 90 & 19\% \\
Indefinite affix on noun & 23 & 5\% \\
No indefinite article but definite article & 81 & 17\% \\
Neither indefinite nor definite marker & 188 & 40\% \\
\midrule
Total & 473 & 100\% \\
\lspbottomrule
\end{tabularx}
\caption{Types of article systems\is{article systems} \citep{dryer:05} }\label{4table:1}
\end{table}


Among the 43\% of languages that do have an indefinite marker, we find some that have more than one indefinite marker or article, which often expresses the contrast between a specific reading, as in (\ref{4ex:9a}), and a non-specific reading, as in (\ref{4ex:9b}), from Lakhota,\il{Lakhota} North America \citep[][405]{latrouite:vanvalin:14}.\footnote{Abbreviations: {\sc{a}} `actor', {\sc{inan}} `inanimate'. }

\begin{exe}
\ex\label{4ex:9}
	\begin{xlista}
	\ex{\label{4ex:9a}
	\gll	Wówapi	waƞ			o$\langle$Ø-wá$\rangle$le.\\
		book		a[+specific]	look.for$\langle${\sc{inan-1sg.a}}$\rangle$ \\}\jambox*{[olé `look for']}
	\glt	`I'm looking for a [particular] book.'
	\ex\label{4ex:9b}
	\gll	Wówapi waƞží             o$\langle$Ø-wá$\rangle$le. \\
		book       a[\minus specific]   look.for$\langle${\sc{inan-1sg.a}}$\rangle$ \\
	\glt	`I'm looking for a book [any book will do].'
	\end{xlista}
\end{exe}

Moroccan Arabic\il{Arabic!Moroccan} provides a three-way system of indefinite marking: (i) bare nouns are not marked for specificity, as in (\ref{4ex:10a}); (ii) a specific indefinite article {\emph{wahed-l}}, composed of the numeral `one' and the definite article, as in (\ref{4ex:10b}); (iii) a non-specific indefinite article {\emph{shi}}, derived from the word for `thing', as in (\ref{4ex:10c}) (from \citealt{fassifehri:06}; see \citealt{brustad:00}: 26-31 for other Arabic dialects):

\begin{exe}
\ex\label{4ex:10}
	\begin{xlista}
	\ex\label{4ex:10a}
	\gll	Meryem  	bgha-t     		te-t-zewwej		b-{\bf{muhami}}		wa-layenni 	waldii-haa   \\
		Maryam 	wanted-{\sc{f}}  	to-{\sc{f}}-marry		with-{\bf{lawyer}}	but         		parents-her  \\
	\glt
	\exi{}
 	\gll	ma bghaw-eh-sh          /  	wa-layenni  	ma  	lqa-t-u-sh.  \\
      		not wanted-him-neg    /	but        		not 	met-her-him-{\sc{neg}} \\
	\glt	`Maryam wanted to marry {\bf{a lawyer}} but her parents don't like {\bf{him}}/but she has not met {\bf{one}} yet.'
	%%
	\ex\label{4ex:10b}
	\gll	Meryem  bgha-t       te-t-zewwej b-{\bf{wahed}} {\bf{r}}-{\bf{rajel}}    	wa-layenni  \\
      		Maryam wanted-{\sc{f}}   to-{\sc{f}}-marry   with-{\bf{one}} {\bf{the}}-{\bf{man}} 	but  \\
	\glt
	\exi{}
	\gll	ma  	lqa-t-u-sh. \\
		not 	met-her-him-{\sc{neg}} \\
	\glt	`Maryam wanted to marry a (specific) man but she hasn't found {\bf{him/(*one)}}.'
	%%
	\ex\label{4ex:10c}
	\gll	Meryem	bgha-t		te-t-zewwej		b-{\bf{shi}}           	{\bf{rajel}}   wa-layenni   \\
      		Maryam	wanted-{\sc{f}}	to-{\sc{f}}-marry  	with-{\bf{some}}  	{\bf{man}}    but  \\
	\glt
	\exi{}
	\gll	ma lqa-t-u-sh. \\
		not  met-her-him-{\sc{neg}} \\
	\glt	`Maryam wanted to marry a (non-spec.) man but she hasn't found {\bf{one/(*him)}}.'
	\end{xlista}
\end{exe}

\newpage
We will argue in this paper that Modern Colloquial Persian not only exhibits the specific vs. non-specific contrast, as in Lakhota\il{Lakhota} and Moroccan,\il{Arabic!Moroccan} but also allows us to morphologically mark a more fine-grained structure of specificity, namely whether the specific indefinite is oriented to the speaker or to some other prominent discourse referent within the context.

\subsection{Speaker- vs. non-speaker-oriented specificity}\label{4sec:22}

German,\il{German} like English\il{English} and other languages, has just one indefinite article (\ref{4ex:11a}). However, it has other means of marking the specificity or referentiality\is{referentiality} of an associated noun phrase. While the regular indefinite in (\ref{4ex:11a}) allows for both a wide- and a narrow-scope reading of the indefinite, the indefinite demonstrative\is{demonstratives} in (\ref{4ex:11b}) clearly signals a referential reading and forces a wide-scope reading:

\begin{exe}
\ex\label{4ex:11}
	\begin{xlista}
	\ex\label{4ex:11a}
	Jeder Student sagte ein Gedicht von Pindar auf. \\
	`Every student recited a poem by Pindar.'
	\ex\label{4ex:11b}
	Jeder Student sagte dieses Gedicht von Pindar auf. \\
	`Every student recited this$_{\text{indef}}$ poem by Pindar.'
	\end{xlista}
\end{exe}

Many languages also have special adjectives that can induce different degrees of specificity. \cite[][31]{ebert:etal:13} discuss the differences between the German\il{German} adjectives {\emph{ein bestimmter}} and {\emph{ein gewisser}}, both of which the authors translate as `a certain', even though the English\il{English} translation does not reflect the subtle differences in meaning of the German\il{German} adjectives. Their main observation is that both adjectives force the indefinite noun phrase to scope over the intentional verb {\emph{suchen}} `search' (\ref{4ex:12}a-b), while the regular indefinite also allows for the narrow-scope reading, as in (\ref{4ex:12c}):

{\small{
\begin{exe}
\ex\label{4ex:12}
	\begin{xlista}
	\ex\label{4ex:12a}
	Peter sucht eine bestimmte CD / zwei bestimmte CDs / bestimmte CDs. \\
	Peter searches a {\sc{bestimmt}} CD / two {\sc{bestimmt}} CDs / {\sc{bestimmt}} CDs. \\
 	`Peter is looking for a certain CD / two certain CDs / certain CDs.'	
	\exi{} \hfill$\exists$ $>$ {\sc{search}}	
	\ex\label{4ex:12b}
	Peter sucht eine gewisse CD / zwei gewisse CDs / gewisse CDs. \\
	Peter searches a {\sc{gewiss}} CD / two {\sc{gewiss}} CDs / {\sc{gewiss}} CDs. \\
 	`Peter is looking for a certain CD / two certain CDs / certain CDs.'
	\exi{} \hfill$\exists$ $>$ {\sc{search}}
	\ex\label{4ex:12c}
	Peter sucht eine CD / zwei CDs / CDs. \\
	Peter searches a CD / two CDs / CDs. \\
	`Peter is looking for a CD / two CDs / CDs.' %
	\hfill{\sc{search}} $>$ $\exists$, $\exists$ $>$ {\sc{search}}			
	\end{xlista}
\end{exe}
}}

The authors claim that the main difference between {\emph{ein bestimmter}} and {\emph{ein gewisser}} has to do with the bearer of the referential intention\is{referential intention} of that indefinite. For {\emph{ein gewisser}}, only the speaker of the sentence can have that referential intention.\is{referential intention} For {\emph{ein bestimmter}}, in contrast, the speaker or some other salient discourse agent, such as the subject of the sentence, can have this intention. This can be shown by the incompatibility of {\emph{ein gewisser}} with speaker ignorance in (\ref{4ex:13b}). The most natural reading of (\ref{4ex:13a}) is that Peter knows which CD, but the speaker does not. So, the speaker only reports the assertion that there is some source (e.\,g., the subject) that has this referential intention.\is{referential intention}

\begin{exe}
\ex\label{4ex:13}
	\begin{xlista}
	\ex\label{4ex:13a}
	Peter sucht eine bestimmte CD, aber ich weiß nicht, welche. \\
	`Peter is looking for a {\sc{bestimmt}} CD, but I do not know which one.'
	\ex\label{4ex:13b}
	Peter sucht eine gewisse CD, \#aber ich weiß nicht, welche. \\
	`Peter is looking for a {\sc{gewiss}} CD, \#but I do not know which one.'
	\end{xlista}
\end{exe}

We can rephrase Ebert et al.'s observation in terms of ``referential anchoring''\is{referential anchoring} in von Heusinger (\citeyear{vonheus:02url,vonheus:11url}; see also \citealt{onea:geist:11}). The idea is that specific indefinites are anchored to the discourse referent that holds the referential intention\is{referential intention} about the identity of the referent. In a default case, indefinites are anchored to the speaker of the utterance. However, they can also be anchored to some other salient discourse referent, such as the subject of the sentence or other (implicit) referents. (For more on the notion of salience\is{discourse salience} or prominence in discourse,\is{discourse prominence} see \citealt{vonheus:schum:19}.) We use this notion of speaker-oriented specificity vs. non-speaker-oriented specificity to account for the differences between the two specific indefinite constructions in Modern Colloquial Persian. That is, we will draw parallels between the two specific indefinite constructions in Modern Colloquial Persian and the contrast found for the German\il{German} specificity adjectives {\emph{ein gewisser}} vs. {\emph{ein bestimmter}}.


\section{Types of indefinites in Persian}\label{4sec:3}
\largerpage[2]
Persian is a language with two dominant registers, spoken and written Persian, both of which have informal and formal forms that are very distinct \citep{jasbi:14, lazard:57, lazard:92, modarresi:18, nikravan:14, windfuhr:79}. The language that we investigate in this paper is Standard Modern Colloquial Persian. The function of the indefinite marker varies with register; the specificity marker {\emph{-e}} is only used in Modern Colloquial Persian. In this section, we provide a brief overview of the way definiteness is expressed, the different indefinite forms in Standard Written Persian\il{Persian!Standard Written}, and the use and function of indefinite forms in Modern Colloquial Persian.

\subsection{Definiteness in Persian}\label{4sec:31}

Persian does not have a definite article, but it has two markers for indefiniteness (see the next section). To express definiteness, then, Persian typically uses bare noun phrases. This holds for different kinds of definite noun phrases. The definite in (\ref{4ex:14a}) is a familiar definite, (\ref{4ex:14b}) is a typical bridging\is{anaphora!bridging anaphora} definite, (\ref{4ex:14c}) shows a unique definite, and (\ref{4ex:14d}) is an example of generic use.\is{generic sentence}

\begin{exe}
\ex\label{4ex:14}
	\begin{xlista}
	\ex\label{4ex:14a}
	\gll	Anne  	yek xune    	xarid.      {\bf{Xune}}    labe  	marze  		kešvare. \\
		Anne  	a     house  	bought   {\bf{house}}   on   	edge.of		country.be.{\sc{3sg}} \\
	\glt	`Anne bought a house. The house is at the border of the country.'
	%%
	\ex\label{4ex:14b}
	\gll	Anne rafte         	 	bud           		ye marāseme arusi.		{\bf{Arus}}		xeyli \\
	    	Anne went.{\sc{3sg}} 	{\sc{aux.3sg}} 		a  ceremony  marriage 	{\bf{bride}} 	very  \\
	\glt
	\exi{}
	\gll	xošgel 		bud. \\
		beautiful	was.{\sc{3sg}} \\
 	\glt	`Anne went to a wedding. The bride was very beautiful.'
	%%
	\ex\label{4ex:14c}
	\gll	{\bf{Māh}}     	xeyli rošan   	mideraxše. \\
         	{\bf{moon}}  	very bright  	{\sc{prog}}.shine.{\sc{3sg}} \\
         \glt	`The moon shines very brightly.'
	%%
	\ex\label{4ex:14d}
	\gll	{\bf{Dianāsor}} 	60 milion  sāle qabl  	monqarez	šode. \\
		{\bf{dinosaur}}  	60 million year ago    	extinct		became.{\sc{3sg}} \\
   	\glt	`Dinosaurs became extinct 60 million years ago.'
	\end{xlista}
\end{exe}

{
There is controversy among scholars as to whether, in Persian, bare nouns are inherently definite \citep{krifka:modarresi:16}, or underspecified with respect to definiteness and genericity\is{genericity} \citep{ghomeshi:03}. Although it is not clear whether or not the non-specific indefinite nature of bare nouns can be detached from their generic (kind) reading,\is{generic sentence} \cite{dayal:17} argues, using Hindi\il{Hindi} as an example, against the view that bare nouns are ambiguous and can have either a definite or an indefinite reading. She concludes that bare singulars in articleless languages like Hindi\il{Hindi} are definite and not indefinite (specific/non-specific), and that their apparent indefiniteness is construction-specific or restricted to bare plurals. \cite{simik:burianova:18} claim that in Czech,\il{Czech} bare NPs,\is{bare nominal} where they are indefinite, cannot be specific. Rather, bare NPs are either definite or indefinite non-specific, which is in line with Dayal's argument. \cite{simik:burianova:18}, finally, annotate bare nouns for (in)definiteness, and their findings suggest that the definiteness of a bare noun is affected by its absolute position in the clause, and that indefinite bare NPs are unlikely to occur in clause-initial position (see also \citealtv{chapters/borik}). Note that this is also applicable to Persian: Persian bare nouns can express a non-definite reading, as in (\ref{4ex:15a}) with a kind-reading of `book', or a definite-reading of `book' as in (\ref{4ex:15b}). Note that a bare noun in the preverbal direct object position is typically interpreted as pseudo-incorporated\is{pseudo-incorporation} \citep[in the sense of][]{massam:01} as in (\ref{4ex:15c}), while a definite reading must be signaled by the object marker {\emph{-rā}} as in (\ref{4ex:15d}) (see \citealt{modarresi:14} for an analysis of bare direct objects in Persian):
}

\begin{exe}
\ex\label{4ex:15}
	\begin{xlista}
	\ex\label{4ex:15a}
	\gll	Roo  	miz  		ketābe. \\
 	 	on 		table 	book.be.{\sc{3sg}} \\
	\glt 	`There is a/some book on the table.'
	\ex\label{4ex:15b}
        \gll	Ketāb 	roo 	mize. \\
 	 	book 	on 	table.be.{\sc{3sg}} \\
 	\glt 	`The book is on the table.'
	\ex\label{4ex:15c}
	\gll 	Ali	ketāb	xarid. \\
		Ali	book		bought.{\sc{3sg}} \\
	\glt	`Ali bought book/books.'
	\ex\label{4ex:15d}
	\gll	Ali	ketāb-rā	xarid. \\
		Ali	book-rā	bought.{\sc{3sg}} \\
	\glt	`Ali bought the book.'
	\end{xlista}
\end{exe}


\subsection{Indefiniteness in Standard Written Persian}\label{4sec:32}

Standard Written Persian\il{Persian!Standard Written} has the suffixed\footnote{There is some controversy as to whether {\emph{-i}} is suffixed or enclitic. Herein we follow the works of \cite{ghomeshi:03, hincha:61, karimi:03, paul:08}. This does not affect our analysis in any way.} indefinite marker {\emph{-i}}, which has quite a large range of functions, and the independent lexeme {\emph{ye(k)}}, which derives from the numeral {\emph{yek}}, but behaves like a regular indefinite article.\footnote{\cite[][64-65]{ghomeshi:03} shows that the indefinite article {\emph{ye(k)}} is different from the numeral {\emph{yek}}. The former can appear without a classifier\is{classifiers!} (i), which is obligatory for numerals, as in (ii) (see also \citealtv{chapters/bisang} for Vietnamese),\il{Vietnamese} and the indefinite article can also appear with plurals, as in (iii).
\begin{exe}
\exi{(i)}
\gll	ye-(ta) ketab 		\hspace*{1cm}(ii)   	se-*(ta)        ketab         	\hspace*{1cm}(iii) 	ye ketab-ha-i \\
    	a-({\sc{cl}})    book 	{}				three-*({\sc{cl}})  book 	{}				a   book-{\sc{pl-ind}} \\
\glt  	`a book' 			\hspace*{2.05cm}`three books'           			\hspace*{2cm}`some books'
\end{exe}
} 
Both forms can be combined, yielding three different indefinite configurations: {\emph{ye(k) NP}}, {\emph{NP-i}}, and {\emph{ye(k) NP-i}}. For Standard Written Persian,\il{Persian!Standard Written} the suffixed {\emph{-i}} has indefinite readings, including readings that undergo negation and other operators. The use of {\emph{ye(k)}} is thought to express the typical ``cardinal'' reading of indefinites. There is no clear delimitation of the function of the combined form {\emph{ye(k) NP-i}}.

\cite{windfuhr:79} considers {\emph{NP-i}} to have three functions: (i) as {\emph{-i}} of `unit', the construction has similar functions as {\emph{a(an)}} in English;\il{English} (ii) as {\emph{-i}} of indefiniteness, the construction is very similar to what \cite{jasbi:16} describes as `antidefinite', similar to `any' or `some'; (iii) as demonstrative\is{demonstratives} {\emph{-i}}, the construction appears with relative clauses.\footnote{There is an ongoing discussion as to whether the use with relative clauses is a use of the suffixed article or a different morpheme (see discussion in \citealt[][65]{ghomeshi:03}).} Toosarvandani and Nasser (\citeyear{toosarvandani:nasser:17}) report that some traditional \citep{lambton:53} as well as contemporary linguists \citep{ghomeshi:03} assume that the indefinite determiner {\emph{yek+NP}} and the suffixed {\emph{NP-i}} can be equivalent in positive, assertive contexts, see example (\ref{4ex:16}) (mainly in non-contemporary or more literary usages); however, Toosarvandani and Nasser (\citeyear{toosarvandani:nasser:17}) provide examples that show a difference in distribution and meaning between the two constructions, mainly in negative, non-assertive contexts, see examples (\ref{4ex:17}) and (\ref{4ex:18}). In the following, the two indefinites' similarities and differences are discussed.

{
Since {\emph{-i}} is a suffix, it can occur with quantifiers. In fact, when universal quantifiers\is{universal quantifiers} such as {\emph{har}} (`every') and {\emph{hich}} (`no') are present, the suffixed {\emph{-i}} usually accompanies the NP. \cite[][90]{lyons:99} states that the ``suffix {\emph{-i}} semantically marks the noun phrase as non-specific or arbitrary in reference and is approximately equivalent to {\emph{any}} in nonassertive contexts and {\emph{some…or other}} in positive declarative contexts''. \cite[][64-65]{ghomeshi:03} argues that the two forms partly overlap, but that the suffixed {\emph{-i}} has a wider range of application. She does not discuss the combined form, however. \cite[][325]{paul:08} argues that {\emph{-i}} has the function of ``picking out and individuating entities''. He argues that this function should be kept separate from specificity and referentiality.\is{referentiality} \cite[][169-170]{hincha:61} assumes that {\emph{ye(k)}} expresses an individualized entity, while {\emph{-i}} signals an arbitrarily chosen element of a class. \cite[][16-19]{modarresi:14} focuses on the differences between bare nouns in an object position, and {\emph{ye(k) NP}} and {\emph{NP-i}} objects. The latter both introduce discourse referents and show scopal effects, while the bare noun does not. We cannot do justice to the whole discussion on indefinites in written Persian, but we try to summarize the main, and hopefully uncontroversial, observations in Table \ref{4table:2}.
}

\begin{table}
{{
\begin{tabularx}{\textwidth}{>{\raggedright}p{4cm}@{~}l>{\raggedright}Q@{~}p{1.9cm}}
\lsptoprule
Function & Form & Positive context & {Negative context}\\
\midrule 
definite & {\emph{NP}} & uniqueness and familiarity & wide scope \\
\tablevspace
indefinite -- expressing  `cardinality' or `existential entailment' 	& {\emph{ye(k) NP}} 		& cardinal reading,  existential `a/one N'			& variation between \mbox{wide and} \mbox{narrow scope} \\
\tablevspace
indefinite, existential 		& {\emph{NP-i}}		& existential/`one of a class'  `some or other'	& \mbox{NPI-narrow} \mbox{scope-`any'}\ \\
\tablevspace
indefinite -- expressing `cardinality' or  	`existential entailment' and speaker's indifference or ignorance 		& {\emph{ye(k) NP-i}}	& existential/arbitrary in reference/speaker’s	  indifference similar	 to German: `(irgend) jemand'	& wide scope \\
\lspbottomrule
\end{tabularx}
}}
\caption{{Definite and indefinite constructions in Standard Written Persian\il{Persian!Standard Written}}}\label{4table:2}
\end{table}

{
\largerpage
Semantically, {\emph{yek NP-i}} can express existence and signals that the referent is arbitrarily chosen \citep{lyons:99}. Pragmatically, it can show a speaker's indifference or ignorance, or a free-choice implication\is{free-choice!free-choice implication} \citep{jasbi:16}. In written form, the three indefinites behave similarly in positive declarative contexts, as shown in (\ref{4ex:16}a-c).
}

\newpage
\begin{exe}
\ex\label{4ex:16}
Context: There were three books. Ali bought one of them.
	\begin{xlista}
	\ex\label{4ex:16a}
	\gll	Ali  	ketāb-i  	xarid. \\
		Ali  	book-i  	bought.{\sc{3sg}} \\
	\glt	`He bought a/some book.'
	\ex\label{4ex:16b}
        	\gll	Ali  	yek ketāb   	xarid. \\
		Ali   	a     book    	bought.{\sc{3sg}} \\
	\glt	`He bought a book.'
	\ex\label{4ex:16c}
        	\gll	Ali  	yek ketāb-i 	xarid. \\
		Ali  	a     book-i   	bought.{\sc{3sg}} \\
	\glt	`Ali bought a/some book.'
	\end{xlista}
\end{exe}

Considering negation, {\emph{ye(k) NP-i}} takes a wide scope over negatives and questions,\is{questions} while {\emph{NP-i}} takes a narrow scope in the same context, and {\emph{ye(k) NP}} can take variable scope (\citealt[][8-9]{toosarvandani:nasser:17}; \citealt[][26-30]{modarresi:14}). The acceptability of a wide scope under negation with different indefinites is illustrated in (\ref{4ex:17}) and (\ref{4ex:18}). Context (\ref{4ex:17}) forces a narrow-scope reading for the indefinites, which is available for {\emph{NP-i}} in (\ref{4ex:17a}) and {\emph{ye(k) NP}} in (\ref{4ex:17b}), but not for {\emph{ye(k) NP-i}} in (\ref{4ex:17c}). The context in (\ref{4ex:18}) strongly suggests a wide-scope reading, which is not available for {\emph{NP-i}} in (\ref{4ex:18a}), but possible for {\emph{ye(k) NP}} in (\ref{4ex:18b}), and for {\emph{ye(k) NP-i}} in (\ref{4ex:18c}). (Note that the wide-scope reading goes hand in hand with the object marker {\emph{-rā}}.)

\begin{exe}
\ex\label{4ex:17}
{Context: There were three possible books I could buy. I didn't buy any of them.}
	\begin{xlista}
	\ex\label{4ex:17a}
	\gll	Man  {\bf{ketāb-i}}  	naxaridam. \\
                	I       	{\bf{book-i}}   	not.bought.{\sc{1sg}} \\
        \glt	`I didn't buy any books.'   \hfill $\neg > \exists$
	\ex\label{4ex:17b}
	\gll	Man 	hattā  	{\bf{ye}} {\bf{ketāb}} 	 ham  	naxaridam. \\
                	I        even  	{\bf{a}}   {\bf{book}}   also  	not.bought.{\sc{1sg}} \\
         \glt	`I didn't buy any book.'	\hfill $\neg > \exists$ 
	\ex\label{4ex:17c}
       	\gll	\#Man   {\bf{ye}} {\bf{ketāb-i}}   ro  	naxaridam. \\
                 \ \ I        {\bf{a}}    {\bf{book-i}}    rā   	not.bought.1{\sc{sg}} \\
       	\glt	Intended: `I didn't buy any book.'	\hfill *$\neg > \exists$
	\end{xlista}
%%
\ex\label{4ex:18}
{Context: There were three possible books that I could buy. I bought two of  them but not the third.}
	\begin{xlista}
	\ex\label{4ex:18a}
	\gll \#Man {\bf{ketāb-i}} naxaridam. \\
               \ \ I {\bf{book-i}} not.bought.{\sc{1sg}} \\
        \glt `There is a book I didn't buy.'  \hfill *$\exists > \neg$
	%
	\ex\label{4ex:18b}
	\gll Man {\bf{ye}} {\bf{ketāb}} ro naxaridam. \\
         I {\bf{a}} {\bf{book}} rā not.bought.{\sc{1sg}} \\
        \glt `There is a book I didn't buy.'  \hfill $\exists > \neg$
	%
	\ex\label{4ex:18c}
	\gll Man {\bf{ye}} {\bf{ketāb-i}} ro naxaridam. \\
          I a {\bf{book-i}} rā not.bought.{\sc{1sg}} \\
        \glt Intended: `There is a book I didn't buy.'	\footnote{(\ref{4ex:18c}) is felicitous in the written variety with {\sc{dom}} `rā'\is{differential object marking} whereas it is not felicitous in Modern Colloquial Persian.}  \hfill $\exists > \neg$
	\end{xlista}
\end{exe}

{
As shown in (\ref{4ex:18a}), {\emph{NP-i}} takes wide scope neither under negation nor under questions\is{questions} (similar to NPIs). However, in positive contexts (written form), it behaves similarly to simple indefinites and can have an existential or numerical implication. 
}

\subsection{Indefiniteness marking in Modern Colloquial Persian}\label{4sec:33}

One of the main distinctions between the system of indefinite forms in the written vs. spoken register is the semantic role of suffixed {\emph{-i}}. In the written register, {\emph{-i}} is a common way of marking an indefinite NP, whereas in colloquial Persian, {\emph{yek NP}} is common and {\emph{-i}} is very restricted as it is used as an NPI. \cite[][246]{jasbi:16} categorizes the indefinite markers in his native Tehrani colloquial Persian into three main categories: simple, complex, and antidefinite. He illustrates their difference in the following table:

\begin{table}

\begin{tabularx}{\textwidth}{Xllll}
\lsptoprule
	& Type		& Form		& Example	& Translation \\
\midrule
Definites & Bare & {\emph{NP}} & {\emph{māshin}} & the car \\[1mm]
\hdashline
			& Simple & {\emph{ye NP}}	 & {\emph{ye māshin}} & a car \\
Indefinites	 	& Antidefinite & {\emph{NP-i}} & {\emph{māshin-i}}	 & $\sim$ a/any car \\
			& Complex & {\emph{ye NP-i}} & {\emph{ye-māshin-i}} & $\sim$ some car or other \\
\lspbottomrule
\end{tabularx}
\caption{Definite and indefinite constructions in Modern Colloquial \mbox{Persian} \citep[][246]{jasbi:16}}\label{4table:3}
\end{table}

Jasbi calls {\emph{ye(k) NP}} a simple indefinite because it behaves similarly to {\emph{a(n)}} in English\il{English} and carries an existential inference (|$\lint$NP$\rint$|$\geq$1). On the other hand, {\emph{NP-i}} entails an antidefinite interpretation, meaning that it rejects any set that can have a unique inference (|$\lint$NP$\rint$|$\not=$1) and can have a non-existential implication (|$\lint$NP$\rint$|=0). Therefore, the respective set either is empty or contains more than one element. Now, the complex indefinite {\emph{ye(k) NP-i}} has an anti-singleton implication (|$\lint$NP$\rint$|$>$1), which is compositionally derived from the existential inference and the anti-uniqueness condition. The summary of the semantic differences proposed by \cite[][251]{jasbi:16} is provided in Table \ref{4table:4}.

\begin{table}

\begin{tabularx}{\textwidth}{XXXl}
\lsptoprule
 		& Type 	& Form 	& Cardinality \\
\midrule
Definite 	& Bare 	& {\emph{NP}} 	& |$\lint$NP$\rint$| = 1 \\[1mm]
\hdashline
		& Simple		& {\emph{ye NP}}	& |$\lint$NP$\rint$| $\geq$ 1 \\
Indefinite	& Antidefinite 	& {\emph{NP-i}}		& |$\lint$NP$\rint$| $\not=$ 1 \\
		& Complex		& {\emph{ye NP-i}}	& |$\lint$NP$\rint$| $>$ 1 \\
\lspbottomrule
\end{tabularx}
\caption{Cardinality implications for definites and indefinites in Modern Colloquial Persian \citep[][251]{jasbi:16}}\label{4table:4}
\end{table}

{
To summarize, the function of the different indefinite markers in Standard Written Persian\il{Persian!Standard Written} is controversial, and their function in Standard Colloquial Persian requires more investigation. Based on Jasbi's (\citeyear{jasbi:16}) semantic characterization (see Table \ref{4table:4}) and the examples discussed above as well as in the subsequent sections, we assume that the form {\emph{ye(k) NP}} corresponds to the unmarked indefinite, the form {\emph{NP-i}} only appears with negation, in conditionals,\is{conditionals} and in questions,\is{questions} and the combined form {\emph{ye(k) NP-i}} expresses a speaker's ignorance or indifference.
}


\section{The specificity marker {\emph{-e}} in Modern Colloquial Persian}\label{4sec:4}

Modern Colloquial Persian has the suffix {\emph{-e}}, which can optionally combine with bare, i.\,e., definite, noun phrases, demonstrative\is{demonstratives} noun phrases, and indefinite noun phrases. With bare noun phrases, {\emph{-e}} is assumed to express a demonstrative or definite function (\citealt[][40]{windfuhr:79}; \citealt[][163]{lazard:57}; \citealt[][67]{ghomeshi:03}; \citealt{toosarvandani:nasser:17}; \citealt{jasbi:20}). \cite[][173-177]{hincha:61} summarizes the distributional properties of {\emph{-e}}: it is always optional --- there are no conditions that makes its use obligatory. If used, it is always accented\is{accent} and attached directly to the stem. It stands in opposite distribution to the plural suffix {\emph{-hā}}, i.\,e., either {\emph{-hā}} or {\emph{-e}} can be used, but not both, which leads \cite[][175]{hincha:61} to assume that both suffixes share some features and express some contradictory features, such as number. \cite[][68]{ghomeshi:03} adds that {\emph{-e}} ``cannot attach to anything already of category D'', such as proper names, pronouns, and noun phrases containing possessors. It cannot combine with the suffixed marker {\emph{-i}}, but as we will discuss below, it can combine with the complex {\emph{ye(k) NP-i}}. With indefinite noun phrases, the suffix signals specificity. In the following, we first provide an overview of specific definite contexts that license the use of the marker, and then provide data on the possible combination of the marker with indefinite constructions. 


\subsection{Specificity marker {\emph{-e}} with definites}\label{4sec:41}

Modern Colloquial Persian can express (certain kinds of) definiteness by means of the marker {\emph{-e}}, which is absent in Standard Written Persian\il{Persian!Standard Written} (\citealt[][50]{windfuhr:79}; \citealt{ghomeshi:03}). The function of {\emph{-e}} is described as demonstrative,\is{demonstratives} definite, determinative, or referential. \cite[][176]{hincha:61} assumes that {\emph{-e}} signals that the NP refers to one particular or individualized entity (``Einzelgegenstand''). There is no comprehensive study of this marker.

\newpage
There is an interesting distribution of {\emph{-e}} with the unmarked bare noun. \cite{nikravan:14} argues that there is a functional difference between unmarked noun phrases, on the one hand, and noun phrases marked with {\emph{-e}} on the other. The former express weak definiteness\is{definiteness!weak} and the latter strong definiteness,\is{definiteness!strong} as is found in other languages with two definite articles \citep[see][]{schwarz:13}. Strong forms are used in anaphoric and situational contexts, while weak forms appear in encyclopedic, unique, and generic contexts. This contrast is illustrated in (\ref{4ex:19}).

\begin{exe}
\ex\label{4ex:19}
\gll	Emruz yek pesar va   yek doxtar ro	didam.      Pesar??(-e) ro       mišnāxtam. \\
	today a     boy    and a     girl      rā 	saw.{\sc{1sg}}  boy??(-e)   rā       knew.{\sc{1sg}} \\
\glt 	`Today I saw a boy and a girl. I knew the boy.'
\end{exe}

In (\ref{4ex:19}), {\emph{pesar}} `boy' is anaphoric and much more acceptable with the marker {\emph{-e}} than without it. Consequently, it is argued that in contexts where an explicit antecedent is present, the strong definite\is{definites!strong definites} is used. Other scholars propose that {\emph{-e}} marks familiarity\is{familiarity} of the associated referent\is{associated referent} \citep{hedberg:etal:09}.  The results of a questionnaire presented in \cite{nikravan:14} seem to indicate that there is a marginal effect of {\emph{-e}} towards a familiarity\is{familiarity} reading. However, it is unclear from her presentation whether the effect is statistically reliable or not. The results also show that the use of {\emph{-e}} is optional, as in (\ref{4ex:19}).

The use of {\emph{-e}} with different types of definite noun phrases (see (\ref{4ex:14}) above) provides further evidence that (i) the use of {\emph{-e}} is optional and (ii) {\emph{-e}} can only be used with referential definites, i.\,e., anaphorically used definites, as in (\ref{4ex:20a}), and definites in bridging\is{anaphora!bridging anaphora} contexts, as in (\ref{4ex:20b}). The use of {\emph{-e}} is ungrammatical for unique definites, as in (\ref{4ex:20c}), and generic uses,\is{generic sentence} as in (\ref{4ex:20d}).

\begin{exe}
\ex\label{4ex:20}
	\begin{xlista}
	\ex\label{4ex:20a}
	\gll	Anne yek	xune		xaride.    			{\bf{Xune(-he)}}		labe		marze	kešvare. \\
          	Anne a	house  	bought.{\sc{3sg}}	{\bf{house(-he)}}	on		edge.of	country.be.{\sc{3sg}} \\
         \glt	`Anne bought a house. The house is at the border of the country.'
	%%%
	\ex\label{4ex:20b}
      	\gll	Anne 	rafte			bud			ye	marāseme	arusi.    \\ 
         	Anne 	went.{\sc{3sg}}	{\sc{aux.3sg}} 	a  	ceremony  	marriage \\
	\glt	
	\exi{}
	\gll	{\bf{Arus(-e)}}   xeyli  xošgel     bud. \\
		{\bf{bride(-e)}}  very   beautiful was.{\sc{3sg}} \\
        \glt  	`Anne went to a wedding.  The bride was very beautiful.'
	%%%
	\ex\label{4ex:20c}
      	\gll	{\bf{Māh(*-e)}}    	xeyli rošan  	mideraxše. \\
          	{\bf{moon(*-e)}}  	very  bright  	{\sc{prog}}.shine.{\sc{3sg}} \\
       	\glt	`The moon shines very brightly.'
     	%%%
	\ex\label{4ex:20d}
	\gll	{\bf{Dianāsor(*-e)}} 	60 	milion	sāle  qabl 	monqarez		šode. \\
        		{\bf{dinosaur(*-e)}} 	60	million	year	ago	extinct		became.{\sc{3sg}} \\
        \glt	`Dinosaurs became extinct 60 million years ago.'
	\end{xlista}
\end{exe}

The referential function of {\emph{-e}} can also be shown in the contrast between a referential and an attributive reading of a definite NP \citep{donnellan:66, keenan:ebert:73}. Sentence (\ref{4ex:21}) strongly suggests an attributive or non-referential reading of the noun {\emph{barande-ye}} `the winner, whoever the winner will be'. In this reading, the use of {\emph{-e}} is ungrammatical, which confirms the assumption that \mbox{\emph{-e}} signals referentiality,\is{referentiality} in the sense that the hearer, as well as the speaker, can uniquely identify the referent of the noun phrase.

\begin{exe}
\ex\label{4ex:21}
\gll	{\bf{Barandeye(*-e)}}	in	mosābeqe  	yek safar  	be ālmān 	    migirad.	 \\
	{\bf{winner(*-e)}}.of	this	competition  	a     trip  to	Germany	get.{\sc{3sg}} \\
\glt	`The winner of this competition (whoever he/she is) will get a trip to Germany.'
\end{exe}

Therefore, we can conclude that the function of {\emph{-e}} is to mark referentially strong definites,\is{definites!strong definites} i.\,e., definites that refer to a discourse referent that was explicitly or implicitly introduced into the linguistic context.

\subsection{The suffix {\emph{-e}} with indefinites}\label{4sec:42}

The specificity marker {\emph{-e}} can combine with two of the three indefinite configurations, as in the examples (\ref{4ex:6})-(\ref{4ex:8}) above, repeated here as (\ref{4ex:22})-(\ref{4ex:24}).

\begin{exe}
\ex\label{4ex:22}
\gll	{\bf{Ye(k)}}	{\bf{pesar-e}}	injā	kār   	mikone. \\
	{\bf{a}}    	   	{\bf{boy-e}}    	here 	work do.{\sc{3sg}} \\
\glt	`A (specific) boy works here.'
%%
\ex\label{4ex:23}
\gll	Diruz        {\bf{māšin-i}}  jolo-e         dare     xune  didi? \\
	yesterday {\bf{car-i}}      front.of-{\sc{e}}   door.of home saw.{\sc{2sg}} \\
\glt
\exi{}
\gll	-- *Na, man hič  {\bf{māšin-e-i}}  	nadidam. \\
	{} \ no I      any  {\bf{car-e-i}}      		not.saw.{\sc{1sg}}\\
\glt	`Did you see any cars in front of the house door yesterday?' \\
	Intended: `No, I didn't see any specific car.'
%%
\ex\label{4ex:24}
\gll	Emruz 	{\bf{ye}} {\bf{māšin-e-i}}   	az     pošt 	   	behem 	zad. \\
 	Today 	{\bf{a}}   {\bf{car-e-i}} 	 from behind   	to.me	 smashed.{\sc{3sg}} \\
\glt	`Today some/other (specific) car smashed me from behind.'  
\end{exe}

The form {\emph{NP-i}} cannot combine with {\emph{-e}}. We speculate that this is due to a conflict of the referential meaning of {\emph{-e}} and the NPI-meaning of {\emph{NP-i}} in Modern Colloquial Persian.\footnote{The occurrence of {\emph{-e}} with {\emph{NP-i}} is not possible with restrictive relative clauses \citep{ghomeshi:03}.}

However, both forms with the indefinite article {\emph{ye(k)}} can combine with {\emph{-e}}, yielding {\emph{ye(k) NP-e}} and {\emph{ye(k) NP-e-i}}, respectively. With both indefinite constructions, the marker {\emph{-e}} signals referential and wide-scope readings of the indefinites. The regular indefinites {\emph{ye doxtar}} in (\ref{4ex:25a}) and {\emph{ye doxtar-i}} in (\ref{4ex:25c}) allow for (i) a wide-scope and (ii) a narrow-scope reading with respect to the universal quantifier.\is{universal quantifiers} However, the forms {\emph{ye doxtar-e}} in (\ref{4ex:25b}) and {\emph{ye doxtar-e-i}} in (\ref{4ex:25d}) only allow for a wide-scope, referential, or specific reading. We find the same contrast for indefinites in sentences with verbs of propositional attitudes,\is{propositional attitudes} as in (\ref{4ex:26}). The {\emph{-e}} marked indefinites can only take a wide scope with respect to the intensional operator {\emph{mixad}} `to want'.

\begin{exe}
\ex\label{4ex:25}
	\begin{xlista}
	\ex\label{4ex:25a}
	\gll	Hame	pesar-hā	bā 	ye doxtar	raqsidan. \\
		all	boy-{\sc{pl}}		with	a   girl		danced.{\sc{3pl}} \\
	\glt	(i) `There is a girl such that every boy danced with her.' \\
		(ii) `For every boy, there is a different girl, such that, that boy dances with her.'
	\ex\label{4ex:25b}
	\gll	Hame	pesar-hā	bā 	ye doxtar-e	raqsidan. \\
		all	boy-{\sc{pl}}	with	a   	girl-e		danced.{\sc{3pl}} \\
	\glt	(i) `There is a girl such that every boy danced with her.' \\
	\ex\label{4ex:25c}
	\gll	Hame	pesar-hā	bā 	ye doxtar-i		raqsidan. \\
		all	boy-{\sc{pl}}	with		a  girl-i		danced.{\sc{3pl}} \\
	\glt	(i) `There is a girl such that every boy danced with her.' \\
		(ii) `For every boy, there is a different girl, such that, that boy dances with her.' 
	\ex\label{4ex:25d}
	\gll	Hame	pesar-hā	bā 	ye doxtar-e-i	 raqsidan. \\
		all	boy-{\sc{pl}}	with		a  girl-e-i	 danced.{\sc{3pl}} \\
	\glt	(i) `There is a girl such that every boy danced with her.'
	\end{xlista}
%%%
\ex\label{4ex:26}
	\begin{xlista}
	\ex\label{4ex:26a}
	\gll	Ali	mixād		bā 	ye doxtar	dust 	še. \\
		Ali	want.{\sc{3sg}}	with 	a   girl		friend	become.{\sc{3sg}} \\
	\glt	(i) `Ali wants to make friends with a specific girl.' \\
		(ii) `Ali wants to make friends with a girl/whoever she may be.'
	\ex\label{4ex:26b}
	\gll	Ali	mixād		bā 	ye doxtar-e	dust 	še. \\
		Ali	want.{\sc{3sg}}	with 	a   girl		friend	become.{\sc{3sg}} \\
	\glt	(i) `Ali wants to make friends with a specific girl.'
%%% pagebreak forced	

	\ex\label{4ex:26c}
	\gll	Ali	mixād		bā 	ye doxtar-i	 dust 	še. \\
		Ali	want.{\sc{3sg}}	with 	a   girl-i	friend	become.{\sc{3sg}} \\
	\glt	(i) `Ali wants to make friends with a specific girl.' \\
		(ii) `Ali wants to make friends with a girl/whoever she may be.'
	\ex\label{4ex:26d}
	\gll	Ali	mixād		bā 	ye doxtar-e-i	dust 	še. \\
		Ali	want.{\sc{3sg}}	with 	a   girl-e-i	friend	become.{\sc{3sg}} \\
	\glt	(i) `Ali wants to make friends with a specific girl.'
	\end{xlista}
\end{exe}

We take the distribution of {\emph{-e}} discussed here as a good evidence that the marker encodes a specific or referential reading of the indefinite.\footnote{{Here we leave open what the exact semantics of the marker {\emph{-e}} is. \cite[][176]{hincha:61} describes it as ``punctualization''; \cite{jasbi:20b} assumes that the marker {\emph{-e}} creates a singleton set, thereby simulating wide-scope behavior. However, this approach would not explain why it can be used with certain definites and why it can be combined with the complex form {\emph{ye(k) NP-e-i}}, as it would include the combination of a singleton and an anti-uniqueness condition. An alternative approach is to assume that the marker is interpreted as an indexed choice function \citep{egli:vonheus:95,winter:97} that selects one element out of a set. This would explain the use with certain definites, and also the complementary distribution with the plural suffix {\emph{-hā}}. Such an account could provide an explanation for the definiteness effect on bare nouns. (The value for the index of the choice function is provided by the local situation or the local discourse, but not by encyclopedic knowledge.) In the form {\emph{ye(k) NP-e}}, the index is locally bound by the speaker, and for the form {\emph{ye(k) NP-e-i}}, the index can also be bound by other salient discourse referents.}}


\subsection{Specificity marker and referential anchoring\is{referential anchoring}}\label{4sec:43}

{
Epistemic specific indefinites express the ``referential intention''\is{referential intention} of the speaker. That is, the speaker signals with these expressions that he or she has already decided on the referent of the indefinite. Non-specific indefinites, on the other hand, assert the existence of an individual that falls under the descriptive content of the indefinite, but not a particular individual. The concept of epistemic specificity\is{specificity!epistemic} as speaker-oriented (or speaker-anchored) seems too narrow, however, as we also find (epistemic) specific indefinites where the speaker cannot identify the referent, but can recognize some other salient discourse participant. Therefore, \cite{vonheus:02url,vonheus:19} proposes the concept of ``referential anchoring'',\is{referential anchoring} modeling the dependency of the referent of the indefinite from some other salient discourse referent or participant (typically the speaker, the subject of the sentence, etc.). The discussion of the contrast between the specificity adjectives {\emph{ein gewisser}} and {\emph{ein bestimmter}} in \sectref{4sec:22} was explained along these lines: {\emph{ein gewisser}} is speaker-oriented, while {\emph{ein bestimmter}} is not obligatorily speaker-oriented, i.\,e., it can also be anchored to another salient agent in the discourse.
}

The two indefinite forms {\emph{ye(k) NP-e}} and {\emph{ye(k) NP-e-i}} are interpreted as specific or referential indefinites. We suggest that the difference between the two forms lies in the specificity orientation in epistemic contexts. It seems that the form {\emph{ye(k) NP-e-i}} is less acceptable in general; however, we still find examples such as the following on Twitter:\footnote{{The first anonymous reviewer pointed out that all the Twitter examples (\ref{4ex:27})-(\ref{4ex:30}) are speaker-oriented and would therefore contradict our hypothesis that the form {\emph{ye(k) NP-e-i}} is non-speaker-oriented. We think that it is difficult to judge this without more context. Moreover, we believe that, in most of the examples, the speaker signals that he or she is not able or willing to reveal the identity of the indefinite NP. The main point of the Twitter examples is to show that these forms are in current use, which contradicts some assumptions made in the literature.}}

\begin{exe}
\ex\label{4ex:27}
\gll	Tanhāi	vasate 	pārke  Mellat	nešastam,		ye xānum-e-i	dāre	 kenāram		Qurān 	mixune. \\
	alone	middle.of 	park.of Mellat	sitting.{\sc{1sg}}	a woman-e-i	{\sc{aux.3sg}} next.me	Quran 	reading.{\sc{3sg}} \\
\glt	`I am sitting alone in the middle of Mellat Park and some woman is reading the Quran next to me.'
%%
\ex\label{4ex:28}
\gll	Ye doxtar-e-i	tu bašgāham	hast,		ajab	heykali			dāre! \\
	a  girl-e-i		at gym.my		be.{\sc{3sg}} what	body-shape		have.{\sc{3sg}} \\
\glt	`There is some girl at the gym I go to, that has a perfect body-shape!'
%%
\ex\label{4ex:29}
\gll	Zošk		ke	berim, 	ye sag-e-i	ham hast 		unjā,		tāze			āšnā		šode			bāhām. \\
	Zoshk	that	go.{\sc{1pl}}	a dog-e-i 	also  be.{\sc{3sg}} 	there 	recently 		familiar 		got.{\sc{3sg}}		with.me \\
\glt	`If we go to Zoshk, there is some dog there that made friends with me last time.'
%%
\ex\label{4ex:30}
\gll	Ye dars		dāštam be esme ``Riāzi Pišrafte''.		Unjā			ye pesar-e-i	 bud	be	esme		Vahid	ya	Hamid. \\
	a   course	had.{\sc{1sg}} with name ``Math advanced''	there 	a boy-e-i	was.{\sc{3sg}}	with	name.of	Vahid 	or Hamid \\
\glt	`I had a course called ``Advanced Mathematics''. There was some boy named Vahid or Hamid.'
\end{exe}

We propose that the basic function of the suffixed indefinite article {\emph{-i}} in Modern Colloquial Persian is to signal speaker ignorance or indifference. Combining speaker ignorance with the epistemic specificity\is{specificity!epistemic} or referentiality\is{referentiality} might result in a semantic-pragmatic condition which we have termed non-speaker-oriented specificity (see discussion in \sectref{4sec:22} above). Therefore, we hypothesize that the difference between these two forms is the orientation or anchoring of the specificity relation. For {\emph{ye(k) NP-e}}, we assume that the indefinite is referentially anchored to the speaker, i.\,e., the indefinite is speaker-oriented specific. The form {\emph{ye(k) NP-e-i}}, in contrast, is referentially anchored to a discourse referent other than the speaker, i.\,e., it is non-speaker-oriented. We summarize this hypothesis in Table \ref{4table:5}.\footnote{Our second reviewer asks whether we assume a compositional semantics, which would provide an independent function for each marker, or whether we assume just one function for the whole construction. For a compositional approach, see \cite{jasbi:16} for the indefinite forms, and footnote 8 in this chapter, on the choice function approach to the specificity marker {\emph{-e}}. However, we have not yet developed full semantics for these configurations.}

\begin{table}

\fittable{\small{
\begin{tabular}{ccccc}
\lsptoprule
  			&  				& 	  	 	& Combination 			& 				\\
Indefinite 		& Cardinality		& Pragmatic 	& with specificity		& 				\\
form			& \citep{jasbi:16}	& difference	& marker {\emph{-e}}		& Assumed function	\\
\midrule
{\emph{ye(k)+NP}}		& |$\lint$NP$\rint$| $>$ 1		& normal indefinite 			& acceptable				& speaker-oriented 	\\
			& 						& marker like {\emph{a(n)}}	& 						& specificity		\\[1mm]
			
{\emph{NP-i}}			& |$\lint$NP$\rint$| $\not=$ 1		& with negations,			& ungrammatical			& --	\\
			& 						& conditionals, 			& 						& 	\\
			&						& and questions				& 						& 	\\[1mm]
			
{\emph{ye(k)+NP-i}}	& |$\lint$NP$\rint$| $>$ 1		& speaker's   				& less acceptable			& non-speaker 	\\
			&						& indifference or			&						& -oriented	\\
			&						& ignorance is				&						& specificity	\\
			&						& more likely				&						&			\\
\lspbottomrule
\end{tabular}
}}
\caption{Specificity marker {\emph{-e}} with different indefinite markers in Modern Colloquial Persian}\label{4table:5}
\end{table}

\largerpage
Our hypothesis makes clear predictions about the acceptability of sentences containing these forms in contexts that express a speaker orientation vs. a non-speaker orientation of the indefinite. We assume that the indefinite {\emph{ye ostād-e}} expresses a speaker orientation, which predicts that the continuation (\ref{4ex:31}i) is coherent, while the continuation (\ref{4ex:31}ii) is incoherent. For the indefinite {\emph{yek ostād-e-i}} in sentence (\ref{4ex:32}), we assume a non-speaker-orientation, which predicts that continuation (\ref{4ex:32}i) is not felicitous, while continuation (\ref{4ex:32}ii) is.

\begin{exe}
\ex\label{4ex:31}
\gll	Sārā  {\bf{ye}} {\bf{ostād-e}} 	{\bf{ro}}  	xeyli dust  dāre. \\
         Sara   {\bf{a}}   {\bf{professor-e}} {\bf{rā}} 	very  like   have	 \\
\glt	`Sara likes a specific professor very much.'
	\begin{xlista}
	\exi{(i)}
	\gll	Man midunam   kudum 	ostād. \\
		I       know.{\sc{1sg}} which   	professor \\
	\glt	`I know who he is.'
	\exi{(ii)}
	\gll	\#Vali nemidunam     kudum 	ostād. \\
		\ \ but  not.know.{\sc{1sg}} 	which  	professor \\
	\glt	`But I don't know which professor.'
	\end{xlista}
\ex\label{4ex:32}
\gll	Sārā  {\bf{ye}} {\bf{ostād-e-i}}      {\bf{ro}}   	xeyli dust dāre.  \\
         Sara   {\bf{a}}  {\bf{professor-e-i}} {\bf{rā}} 		very  like  have \\
\glt	`Sara likes some specific professor very much.'
	\begin{xlista}
	\exi{(i)}
	\gll	\#Man midunam   kudum ostād. \\
		\ \ I       know.{\sc{1sg}} 	which   professor \\
	\glt	\ \ `I know who he is.'
	\exi{(ii)}
	\gll	Vali nemidunam     kudum ostād. \\
		but not.know.{\sc{1sg}}  which  professor \\
	\glt	`But I don't know which professor.'
	\end{xlista}
\end{exe}

\largerpage[2]
We can summarize this prediction in Table 6 with the expected acceptability of the continuation.\footnote{The second reviewer also suggested that we test the examples (\ref{4ex:31})-(\ref{4ex:32}) without the specificity marker {\emph{-e}}, as in (31$'$) and (32$'$). The reviewer reported that his or her informants would accept the continuations (i) and (ii) for both sentences, but that the informants expressed a preference for (31$'$ii) and (32$'$i), which would be the opposite of the expectation expressed for (\ref{4ex:31})-(\ref{4ex:32}). We agree that both continuations are good for both sentences, but we do not share their preferences. We do not have any predictions with respect to (31$'$) and (32$'$). Note that both (\ref{4ex:31})/(\ref{4ex:32}), and (31$'$)/(32$'$), have the direct object marker {\emph{-rā}}, which is assumed to express specificity by itself. We cannot go into details about the difference between the function of {\emph{-e}} and {\emph{-rā}} here; however, our test items had examples with and without {\emph{-rā}}.

\begin{exe}
% \exi{(31$'$)}
\exp{4ex:31}
\gll	Sārā		{\bf{ye}}	{\bf{ostād}}	{\bf{ro}}	xeyli dust  dāre. 	\hspace*{1.25cm}(i) 	Man midunam   kudum ostād. \\
         Sara 	{\bf{a}}	{\bf{professor}}	{\bf{rā}}	very  like   have			  		{}	I     know.{\sc{1sg}} which professor \\
\glt	`Sara likes a specific professor very much.'					\hspace*{0.75cm}	`I know who he is.'
% \exi{(32$'$)}
\exp{4ex:32}
\gll	Sārā		{\bf{ye}}	{\bf{ostād-i}} 	{\bf{ro}}   	xeyli dust dāre. 		\hspace*{1cm}(ii)	Vali nemidunam     kudum ostād. \\
	Sara		{\bf{a}}	{\bf{professor-i}}	{\bf{rā}} very  like  have				{}	but  not.know.{\sc{1sg}} which  professor \\
\glt	`Sara likes some specific professor very much.'					\hspace*{0.35cm}	`But I don't know which professor.'
\end{exe}}\clearpage

\begin{table}
{\small{
\begin{tabularx}{\textwidth}{Ql@{}clQ}
\lsptoprule
Indefinite form 		& Epistemic type 	& Example 	& Context 	& Predited\newline acceptability \\
\midrule
{\emph{ye(k) NP-e}}		& speaker-specific		& (31)	& {(i) speaker knowledge}	& very good \\
{\emph{ye(k) NP-e}}		& speaker-specific		& (31)	& {(ii) speaker ignorance}	& \# \\
{{\emph{ye(k) NP-e-i}}}	& {non-speaker-specific}	& (32)	& {(i) speaker knowledge}	& \# \\
{{\emph{ye(k) NP-e-i}}}	& {non-speaker-specific	}& (32)	& {(ii) speaker ignorance}	& good \\
\lspbottomrule
\end{tabularx}
}}
\caption{{Prediction of type of epistemic specificity\is{specificity!epistemic} of {\emph{-e}} marked indefinites}}\label{4table:6}
\end{table}



\section{Empirical evidence for speaker orientation of specific noun phrases}\label{4sec:5}

{
In this section, we present two pilot acceptability studies that tested the predictions outlined above. In the first pilot, we used eight sentences, which we continued with either (i), a context that was only coherent with a speaker-oriented specific reading or (ii), a context that was only coherent with a non-speaker-oriented specific reading. The results show that simple indefinites with {\emph{ye(k) NP-e}}, regardless of their specificity orientation, are more acceptable than complex indefinites, but there were no clear effects of specificity orientation. We assume that our results might reflect a mix-up between different degrees of animacy\is{animacy} in the included indefinites. Therefore, we conducted a second pilot study with only human indefinites and a different design; as well as simple sentences and their speaker-oriented vs. non-speaker-oriented continuations, we also presented sentences that clearly signaled speaker ignorance in order to test whether informants can distinguish between different specificity orientations. The results of the second study not only confirm that speakers are capable of making this distinction, but also provide some support for our claim that the simple indefinite {\emph{ye(k) NP-e}} is speaker-oriented, and the complex indefinite {\emph{ye(k) NP-e-i}} is non-speaker-oriented.
}

\subsection{Experiment 1}\label{4sec:51}

{
Our hypothesis H1 is that in Modern Colloquial Persian {\emph{ye(k) NP-e}} always functions as speaker-specific (`gewiss NP'), while {\emph{ye(k) NP-e-i}} can only function as non-speaker-oriented. In order to test this hypothesis, we conducted a pilot questionnaire with speakers of Modern Colloquial Persian. We used a simple sentence, as seen in (\ref{4ex:33}), with simple indefinites with the marker {\emph{-e}} ({\emph{yek doktor-e}}), as well as complex indefinites with the marker {\emph{-e}} ({\emph{yek doktor-e-i}}). The first sentence with the critical item ({\emph{yek doktor-e}} or {\emph{yek doktor-e-i}}) is continued with either (i) an assertion that the speaker had knowledge of the referent, or (ii) a statement signaling the ignorance of the speaker. That is, continuation (i) strongly forces a speaker-specific reading and continuation, while (ii) forces a non-speaker-specific reading. Note that we did not test indefinites without the marker {\emph{-e}}, as we assume that there is ambiguity between a specific and non-specific interpretation.\footnote{In half of the examples the critical indefinite was the direct object, as in (\ref{4ex:31}), and a different argument in the other half, as in (\ref{4ex:33}). We found that this alternation had no significant effect, even though we added the differential case marker {\emph{-rā}} in the direct object instances. It is unclear what additional function this marker performs (see the discussion in the last footnote). We also balanced for animacy,\is{animacy} see the discussion below and Figure \ref{4fig:2}.}
}

\begin{exe}
\ex\label{4ex:33}
	\begin{xlista}
	\ex\label{4ex:33a}
	\gll	Mona bā  		{\bf{yek}} {\bf{doktor-e}}  	ezdevaj karde. \\
		Mona with  	{\bf{a}} {\bf{doctor-e}}  	marriage did.{\sc{3sg}} \\
	\glt	`Mona married a doctor.'
	\ex\label{4ex:33b}
	\gll	Mona bā  	{\bf{yek}} {\bf{doktor-e-i}}  	ezdevaj karde. \\
		Mona with  	{\bf{a}} {\bf{doctor-e-i}}  	marriage did.{\sc{3sg}} \\
	\glt	`Mona married a doctor.'
		\begin{xlista}
		\exi{(i)}
		\gll	Man midunam kudum 	doktor. \\
			I  	 know.{\sc{1sg}}	which	doctor \\
		\glt	`I know which doctor he is.'
		\exi{(ii)}
		\gll	Vali man nemidunam    	kudum 	doktor. \\
			but  I       not.know.{\sc{1sg}}    which    doctor \\
		\glt	`But I do not know which doctor he is.'
		\end{xlista}
	\end{xlista}
\end{exe}


\subsubsection{Participants and experimental technique}\label{4sec:511}

Twenty male and female participants participated in the study. Their native language was Persian and they had lived all or most of their lives in Iran. Their ages varied between 25 and 67. In terms of educational level, six participants had high school diplomas, ten had bachelor's degrees, and four had master's degrees. Participants read Persian written texts for at least one hour a day, and they spoke/heard Persian all or most of the day.

{
The study followed a 2x2 design with two different indefinite forms: (a) {\emph{ye(k) NP-e}} and (b) {\emph{ye(k) NP-e-i}} and two continuations: (i) ``I do know who/which'' for the speaker-oriented epistemic specificity\is{specificity!epistemic} and (ii) ``I do not know who/which'' for the non-speaker-oriented epistemic specificity.\is{specificity!epistemic} The assumption was that all forms are epistemically specific, as in Table \ref{4table:6} above. We had eight different sentences and created four lists using a Latin square design, so that each participant heard one sentence and two conditions each. Probable factors which might intervene with the evaluation, such as animacy,\is{animacy} position of NP in the sentence (direct object/indirect object), and direct/indirect speech, were equally present in all items. 
}

As we were testing Modern Colloquial Persian, i.\,e., spoken Persian, we read out the sentences to our participants at least once and asked them to evaluate the sentence on a scale from 1 for ``completely acceptable'' to 7 for ``completely unacceptable'' on the answer sheet, where they were also able to read the test sentence themselves.


\subsubsection{Results}\label{4sec:512}

We observed that participants complained (even verbally) about the appearance of {\emph{-e}} in {\emph{ye(k) NP-e-i}} in both speaker-specific and non-speaker-specific readings. This is also reflected in the acceptability scores. We summarize the pilot questionnaire with 20 participants in Table \ref{4table:7}, together with the expected acceptability.

\begin{table}

{\small{
\begin{tabularx}{\textwidth}{QQp{1cm}p{2cm}Q}
\lsptoprule
Indefinite\newline form		& Epistemic\newline type	&Mean value	& ~\newline Acceptability	&Predicted acceptability \\
\midrule
{\emph{ye(k) NP-e}} & speaker-specific & 2.725 & good & very good \\
\tablevspace
{\emph{ye(k) NP-e}} & non-speaker-specific & 3.025 & good & \# \\ 
\tablevspace
{\emph{ye(k) NP-e-i}} & speaker-specific & 4.675 & weak & \# \\
\tablevspace
{\emph{ye(k) NP-e-i}} & non-speaker-specific & 4.425 & weak & good \\
\lspbottomrule
\end{tabularx}
}}
\caption{Effect of {\emph{-e}} as specificity marker of indefinites on the kind of epistemicity (1\,=\,very good; 7\,=\,very bad)}\label{4table:7}
\end{table}

\begin{figure}[t]
% \includegraphics[width=\textwidth]{chapters/heusinger-fig1.png}
\pgfplotstableread{
1 2.7
2 3.0
3 4.7
4 4.4
}\dataset
\begin{tikzpicture}
\begin{axis}[ybar,
        width=1\textwidth,
        height=.3\textheight,
        ymin=0,
        %ymax=100,
        % ylabel={},
        xlabel={\small ye(k) NP-e \hspace{10em} ye(k) NP-e-i},
        % xlabel style = {yshift=-8mm},
        xtick=data,
        bar width=10mm,
        nodes near coords,
        xticklabels = {
            {speak-sec},
            {non-speak-sec},
            {speak-sec},
            {non-speak-sec},
        },
      axis y line*=right,
      axis x line*=bottom,
      legend style={at={(.8,1.05)}},
        x tick label style={align=center,text width=2cm},
        ticklabel style = {font=\footnotesize},
        ]
\addplot[draw=lsMidDarkBlue!80!black,fill=lsMidDarkBlue] table[x index=0,y index=1] \dataset;
% \addplot[draw=black,fill=lsMidGreen] table[x index=0,y index=2] \dataset;
% \legend{Definite,Indefinite}
\end{axis}
\end{tikzpicture}
\caption{{Acceptability (1\,=\,very good; 7\,=\,very bad) of simple and complex indefinites in non-speaker and speaker-oriented specificity contexts}}\label{4fig:1}
\end{figure}

\begin{figure}

% \includegraphics[width=\textwidth]{chapters/heusinger-fig2.png}

\pgfplotstableread{
1 1.9       2.4
2 4.3       3.9
3 3.6       3.7
4 5.1       5.0
}\dataset
\begin{tikzpicture}
\begin{axis}[ybar,
        width=1\textwidth,
        height=.3\textheight,
        ymin=0,
        %ymax=100,
        % ylabel={},
        xlabel={\small human \hspace{12em} non-human},
        % xlabel style = {yshift=-8mm},
        xtick=data,
        bar width=7mm,
        nodes near coords,
        xticklabels = {
            {ye(k) NP-e},
            {ye(k) NP-e-i},
            {ye(k) NP-e},
            {ye(k) NP-e-i},
        },
      axis y line*=right,
      axis x line*=bottom,
      legend style={at={(0,1)},anchor=north west},
        x tick label style={align=center,text width=2cm},
        ticklabel style = {font=\footnotesize},
        ]
\addplot[draw=lsMidDarkBlue!80!black,fill=lsMidDarkBlue] table[x index=0,y index=1] \dataset;
\addplot[draw=lsLightBlue!80!black,fill=lsLightBlue] table[x index=0,y index=2] \dataset;
\legend{speak-spec,non-speak-spec}
\end{axis}
\end{tikzpicture}
\caption{Acceptability (1 = very good; 7 = very bad) of simple and complex indefinites for human and non-human noun phrases in non-speaker and speaker-oriented specificity contexts}\label{4fig:2}
\end{figure}


{
\largerpage
Overall, we see that the form {\emph{ye(k) NP-e}} was more acceptable than the form {\emph{ye(k) NP-e-i}}, which confirms the intuition reported above. However, we also see that {\emph{ye(k)}} \mbox{\emph{NP-e}} performed well in both conditions (speaker- and non-speaker-specific), which went against our hypothesis. The judgment for the non-speaker-specificity condition is marginally weaker. The form {\emph{ye(k) NP-e-i}} was clearly weak-er; however, there is only a marginal difference between speaker-oriented specificity (slightly weaker) and non-speaker-oriented specificity. Interestingly, when distinguishing between human and non-human indefinites, as in Figure \ref{4fig:2}, we see that the non-human indefinites were less acceptable than the human indefinites. Furthermore, when looking at the human indefinites we can see that the simple indefinites ({\emph{ye(k) NP-e}}) were rated as slightly better in the speaker-specificity condition than in the non-speaker conditions (1.85 vs. 2.35). Complex indefinites ({\emph{ye(k) NP-e-i}}), on the other hand, were slightly better in the non-speaker-specificity condition than in the speaker-oriented specificity condition (3.9 vs. 4.3).
}




\subsubsection{Discussion}\label{4sec:513}

{
Our first pilot study shows that complex indefinites with the marker {\emph{-e}} are less acceptable than simple indefinites with the marker {\emph{-e}}. Animacy\is{animacy} is also an important factor: our study demonstrates that human indefinites were more acceptable than non-human indefinites. However, the predicted contrast between simple and complex indefinites in speaker- vs. non-speaker-oriented specificity contexts was not shown to be significant. We surmise that this contrast might be more pronounced with human indefinites, which led us to design a second pilot experiment.
}

\subsection{Experiment 2}\label{4sec:52}
In order to test the hypothesis that the two specific indefinites in Modern Colloquial Persian differ with respect to the referential anchoring\is{referential anchoring} of the indefinite, i.\,e., in the specificity orientation, in the second study, we focused on human indefinites. Additionally, we included some examples that provided contexts that signaled speaker ignorance in the first sentence. These examples were used to test whether participants were sensitive to the speaker- vs. non-speaker-orientation.

\subsubsection{Design}\label{3sec:521}

Experiment 2 was conducted to test for a feature which is only present in spoken colloquial Persian, namely the {\emph{-e}} marker with indefinite NPs. It followed the same 2x2 design with four lists as the first pilot study. There were 24 items consisting of 12 test and 12 filler items in each list. The experimental stimuli consisted of two sentences for each item. Since the feature under investigation was simple vs. complex indefinites with the marker {\emph{-e}}, the first sentence contained an indefinite noun phrase either with {\emph{yek NP-e}} or {\emph{yek NP-e-i}}. The second sentence forced either a speaker-specific reading of the indefinite in the first sentence, or a non-speaker-specific reading. 

In the speaker-specific continuation, we asserted the knowledge of the speaker about the identity of the referent of the indefinite. In the non-speaker-specific continuation, we asserted the ignorance of the speaker about the identity of the referent, thus forcing a non-speaker-specific reading.

\begin{exe}
\ex\label{4ex:34}
Critical items for Experiment 2 
	\begin{xlista}
	\ex\label{4ex:34a}
	Simple indefinite ({\bf\emph{yek NP-e}}) + speaker-specific continuation
	\exi{}
	\gll	Sara	emruz	az	{\bf{ye}} 	{\bf{vakil-e}}	 	vaqte 		mošāvere	gerefte. \\
		Sara	today	from	{\bf{a}} 	{\bf{lawyer-e}}	 	appointment	consulting 	took.{\sc{3sg}} \\
	\glt
	\exi{}
	\gll	Man ham ba	vakil-e 	čandinbar  	kar    kardam. \\
		I	also with lawyer-e	several.time work did.{\sc{1sg}} \\
	\glt
	\exi{}
	\gll	Kareš	xeyli 	xube. \\
		work.his 	very		good.be.{\sc{3sg}} \\
	\glt `Sara had a consultation appointment with a lawyer today. I have also consulted with the lawyer. His work is very good.'
	%%
	\ex\label{4ex:34b}
	Simple indefinite ({\bf\emph{yek NP-e}}) + non-speaker-specific continuation
	\exi{}
	\gll	Sara	emruz az		{\bf{ye}} {\bf{vakil-e}}	 vaqte 	        	mošāvere		gerefte.  \\
		Sara	today	 from		{\bf{a}} {\bf{lawyer-e}} 	appointment 	consulting    	took.{\sc{3sg}} \\
	\glt
	\exi{}
	\gll	Migan    vakil-e	 maroofe		  vali man čiz-i		  azaš	    nemidunam. \\
		say.{\sc{3pl}} lawyer-e known.be.{\sc{3sg}} but	I	   thing-{\sc{indef}} from.him not.know.{\sc{1sg}} \\
	\glt	`{Sara had a consultation appointment with a lawyer today. They say that this lawyer is well known, but I do not know anything about him.}'
	%%
%%% page break 

	\ex\label{4ex:34c}
	Complex indefinite ({\bf\emph{yek NP-e-i}}) + speaker-specific continuation
	\exi{}
	\gll	Sara emruz	az	    	{\bf{ye}} {\bf{vakil-e-i}}	vaqte 		mošāvere	gerefte. \\
		Sara today	from 		{\bf{a}} {\bf{lawyer-e-i}} 	appointment	consulting		took.{\sc{3sg}} \\
	\glt 
	\exi{}
	\gll	Man ham ba	vakil-e    čandinbār  kār    kardam.   \\
		I       also with	lawyer-e several.time work did.{\sc{1sg}}  \\
	\glt	
	\exi{}
	\gll	Kareš   xeyli	xube. \\
		work.his very	good.be.{\sc{3sg}} \\
	\glt	`Sara had a consultation appointment with a lawyer today. I have also consulted with the lawyer, several times. His work is very good.'
	%%
	\ex\label{4ex:34d}
	Complex indefinite ({\bf\emph{yek NP-e-i}}) + non-speaker-specific continuation
	\exi{}
	\gll	Sara emruz az     {\bf{ye}} {\bf{vakil-e-i}}     vaqte 	           mošāvere gerefte. \\
		Sara  today from {\bf{a}}   {\bf{lawyer-e-i}}  appointment consulting  took.{\sc{3sg}} \\
	\glt
	\exi{}
	\gll	Migan    vakil-e	 maroofe	       vali	man čiz-i		       azaš      nemidunam. \\
		Say.{\sc{3pl}} lawyer-e known.be.{\sc{3sg}} but	I	   thing-{\sc{indef}} from.him not.know.{\sc{1sg}} \\
	\glt	`{Sara had a consultation appointment with a lawyer today. They say that this lawyer is well known, but I do not know anything about him.}'
	\end{xlista}
\end{exe}

The test items also differed in their constructions: eight items had a third person subject (proper name), as in (\ref{4ex:34}), and four other items were constructions that showed a greater distance from the speaker, namely two items of the type ``They say...'', as in (\ref{4ex:35}), and two items of the form ``I heard...'', as in (\ref{4ex:36}). Note that we provide only the a-condition with {\emph{ye(k) NP-e}} and the specific continuation, as in (\ref{4ex:34a}).

\begin{exe}
\ex\label{4ex:35}
``They say...'' construction:
%%
\exi{}
\gll	Migan {\bf{ye}} {\bf{moalem-e}} tu madrese Tizhushān hast ke bečeha azaš xeyli mitarsan. Man ham bāhāsh 4ta dars daštam va oftadam. \\
	say.{\sc{3pl}} {\bf{a}} {\bf{teacher-e}} in school Tizhushan be.{\sc{3sg}} that	 student.{\sc{pl}} from.her very frighten.{\sc{3pl}} I also with.him {\sc{4cl}} course had.{\sc{3sg}} and failed \\
\glt	`They say there is a teacher in Tizhushan school that every student is afraid of. I also have had four courses with him and failed them all.'
%%%%%%%

\ex\label{4ex:36}
``I heard...'' construction:
\exi{}
\gll Šenidam {\bf{ye}} {\bf{pesar-e}} hast tu in mahale \\
heard.{\sc{1sg}} {\bf{a}} {\bf{boy-e}} be.{\sc{3sg}} in this neighborhood \\
\glt
\exi{}
\gll ke vase doxtara mozāhemat ijad mikone. \\
who for girl.{\sc{pl}} harassment make do.{\sc{3sg}} \\
\glt
\exi{}
\gll Man mišnasameš az vaghti bače bud. \\
I know.{\sc{1sg}} him.from when child was.{\sc{3sg}} \\
\glt `I have heard that there is a boy in this neighborhood who harasses girls. I have known him since he was a child.'


\end{exe}

\subsubsection{Results of Experiment 2}\label{4sec:522}
There was strong agreement in relation to the filler/control items, with marginal differences between participants ($<$ 0.8 points). The results of the test items can be summarized as follows. Firstly, in contrast to Experiment 1, Figure \ref{4fig:3} does not show a clear preference for simple indefinites. Rather, both types were rated very similarly. Secondly, we clearly see that the contexts which signaled speaker ignorance (``They say... '', ``I heard...'') preferred non-speaker-oriented specificity continuations. It shows that participants were aware of this contrast.

\begin{figure}

% \includegraphics[width=\textwidth]{chapters/heusinger-fig3.png}
\pgfplotstableread{
1 1.4       1.5
2 1.7       1.5
3 3.0       1.4
4 2.3       1.5
5 3.0       1.3
6 2.7       1.5
}\dataset
\begin{tikzpicture}
\begin{axis}[ybar,
        width=1\textwidth,
        height=.3\textheight,
        ymin=0,
        %ymax=100,
        % ylabel={},
        xlabel={`neutral' \hspace{5em} `I heard' \hspace{5em} `They say'},
        % xlabel style = {yshift=-8mm},
        xtick=data,
        bar width=7mm,
        nodes near coords,
        xticklabels = {
            {ye(k) NP-e},
            {ye(k) NP-e-i},
            {ye(k) NP-e},
            {ye(k) NP-e-i},
            {ye(k) NP-e},
            {ye(k) NP-e-i},
        },
      axis y line*=right,
      axis x line*=bottom,
      legend style={at={(0,.7)},anchor=south west},
        x tick label style={align=center,text width=2cm},
        ticklabel style = {font=\footnotesize},
        ]
\addplot[draw=lsMidDarkBlue!80!black,fill=lsMidDarkBlue] table[x index=0,y index=1] \dataset;
\addplot[draw=lsLightBlue!80!black,fill=lsLightBlue] table[x index=0,y index=2] \dataset;
\legend{speak-spec,non-speak-spec}
\end{axis}
\end{tikzpicture}

\caption{Acceptability (1 = very good; 5 = very bad) of simple and complex indefinites for human noun phrases in non-speaker and speaker-oriented specificity contexts, across types of constructions}\label{4fig:3}
\end{figure}

A more detailed inspection of the neutral contexts in Figure \ref{4fig:3} reveals a slight preference for the simple indefinite {\emph{ye(k) NP-e}} in speaker-oriented specificity contexts (1.39) vs. non-speaker-oriented specificity contexts (1.54), while the complex indefinite {\emph{ye(k) NP-e-i}} was rated slightly better in non-speaker-oriented specificity contexts (1.5) vs. speaker-oriented specificity contexts (1.67).

{
In summary, the direct comparison in neutral contexts between the simple and the complex indefinite with the marker {\emph{-e}} does not provide significant contrasts. It only suggests a preference of the simple indefinite for speaker-oriented specificity, while the complex indefinite prefers non-speaker-oriented specificity. However, constructions with ``They say...'' or ``I heard...,'', which clearly encode non-speaker-oriented specificity, show a preference for the complex indefinite. This supports our hypothesis for the difference between the two specific indefinites.
}

\section{Summary and open issues}\label{4sec:6}

{
Persian has two indefinite markers, prenominal {\emph{ye(k)}} and suffixed {\emph{-i}}. Both forms express particular kinds of indefiniteness, as does their combination. For Modern Colloquial Persian, indefinites with {\emph{-i}} express a non-uniqueness or anti-definite implication, and behave similarly to the English\il{English} {\emph{any}}. {\emph{Ye(k)}}, on the other hand, expresses an at-issue existence implication and behaves similarly to the English\il{English} {\emph{a(n)}}. Finally, the combination of {\emph{ye(k)}} and {\emph{NP-i}} expresses an ignorance implication \citep{jasbi:16}. The specificity marker {\emph{-e}} can be combined with {\emph{ye(k) NP}} and with the combined form {\emph{ye(k) NP-i}}, but not with (solitary) {\emph{NP-i}} \citep{windfuhr:79,ghomeshi:03}. Based on these semantic functions and on the comparison of the two specificity adjectives {\emph{ein gewisser}} and {\emph{ein bestimmter}} in German,\il{German} we hypothesized that the difference between the interpretation of the two indefinites lies in the anchoring of the indefinite either to the speaker or to some other salient discourse referent; the simple indefinite {\emph{ye(k) NP-e}} is interpreted as a speaker-oriented-specific referent. The complex indefinite {\emph{ye(k) NP-e-i}} is interpreted as a non-speaker-specific referent. 
}

{
In two pilot acceptability tasks, we tested these two indefinites in two contexts, one that suggested a speaker-specific interpretation of the indefinite, and a second that suggested a non-speaker-specific interpretation. The first study provided some support for our hypothesis, but we also found that type of indefinite and animacy\is{animacy} had a significant effect on interpretation. We therefore designed a second pilot study with only human indefinites. Additionally, we inserted constructions with ``I heard...'' and ``They say...'', which clearly suggest a non-speaker-oriented specificity. The results of the second study do not show a preference for the simple indefinite. However, they provide some evidence that, in neutral contexts, the simple indefinite is more acceptable with speaker orientation, and the complex with non-speaker orientation. Still, the evidence is very weak. Finally, in contexts that encode speaker ignorance (``They say...'', ``I heard...''), the complex indefinite was slightly more acceptable than the simple indefinite, which supports our original hypothesis.
}

{
In summary, we have seen that the complex system of indefinite marking in Modern Standard Persian\il{Persian!Modern Colloquial|)} provides a fruitful research environment for learning more about the formal marking of subtle semantic and pragmatic functions of noun phrases, such as specificity\is{specificity|)} and the referential anchoring\is{referential anchoring} of nominal expressions.\il{Persian|)}
}

\largerpage
\section*{Acknowledgments}
We would like to thank the audience of the workshop ``Specificity, Definiteness and Article Systems across Languages'' at the 40th Meeting of the German Linguistic Society, Stuttgart, March, 7-9, 2018 for their comments, two anonymous reviewers for their very helpful comments and suggestions, and the editors of this volume, Kata Balogh, Anja Latrouite, and Robert D. Van Valin, Jr. for all their work and continuous support. The research for this paper was funded by the Deutsche Forschungsgemeinschaft (DFG, German Research Foundation) -- Project ID: 281511265 -- SFB ``Prominence in Language'' in the project C04 ``Conceptual and referential activation in discourse'' at the University of Cologne, Department of German Language and Literature I, Linguistics.

{\sloppy\printbibliography[heading=subbibliography,notkeyword=this]}
\end{document}

