\documentclass[output=paper]{langsci/langscibook}
\ChapterDOI{10.5281/zenodo.4049687}

\author{Valentina Schiattarella\affiliation{University of Naples, L'Orientale}}

\title{Accent on nouns and its reference coding in Siwi Berber (Egypt)}  

\abstract{The aim of this article is to investigate the position of the accent\is{accent|(} on nouns in Siwi,\il{Siwi (Berber)|(} a Berber language spoken in the oasis of Siwa, Egypt, and to see how its alternation\is{accent!accent alternation} on the last or penultimate syllable functions in terms of reference coding. In Siwi, the role of the accent placed on nouns goes beyond the field of phonology: an analysis of original data from both spontaneous discourse and elicitations will show its functions in terms of attribution of (in)definiteness of nouns, in different environments. In order to proceed with the analysis, it is worth noting that Siwi, like all other Berber languages,\il{Berber languages} does not have definite or indefinite articles.}


\begin{document}
\maketitle


\section{Introductory remarks}\label{5sec:1}

Siwi is part of the Berber language\il{Berber languages} family (Afro-Asiatic phylum) spoken in Morocco, Algeria, Tunisia, Libya and Egypt, as well as in Mauritania, Mali, Niger and Burkina Faso. It is the easternmost of the Berber languages,\il{Berber languages} as it is the only one spoken in Egypt, in two oases: Siwa and El Gaṛa. These two oases are located in the Western Desert and are very close to the Libyan border. The main oasis, Siwa, is inhabited by over 25,000 people, including Siwi, Bedouins and Egyptians who have come from other parts of the country, and settled in the oasis mainly for work. Siwi people are almost entirely bilingual, as the vast majority of the population speaks Bedouin\il{Bedouin} and/or Egyptian Arabic\il{Arabic!Egyptian} and Siwi. 

Data for this article were collected over the course of several fieldwork trips between 2011 and 2018; collection methods mainly included recordings of spontaneous data (monologues and dialogues of variable length) and elicitation sessions with both male and female speakers, of different ages. All examples come from my corpus, which was transcribed and translated into English\il{English} with the help of my consultants. Accent will be marked with an acute accent (e.\,g., {\emph{á}}). When the position of the accent does not emerge clearly from the recording, I refer to the transcription sessions carried out with my consultants, as they reproduce exactly what the speakers say in the original recordings.

The aim of this article is to investigate the accent alternation\is{accent!accent alternation} on nouns as a marker of reference coding. The paper will be organized as follows: after an introduction on accent in general and in some Berber languages\il{Berber languages} in Sections \ref{5sec:11} and \ref{5sec:12} respectively, and an account of previous studies on Siwi in \sectref{5sec:13}, I will give an overview of the notions of definiteness and indefiniteness in \sectref{5sec:14}. In \sectref{5sec:2}, I will discuss the accent position when the noun is isolated (\sectref{5sec:21}) and when it is used in discourse (Sections \ref{5sec:22} and \ref{5sec:23}). I will then establish a hierarchy between accent position and other means that the language has to convey definiteness to the noun in \sectref{5sec:24}, and I will conclude, in \sectref{5sec:25}, by presenting a construction that shows clearly how speakers use the alternation of the position of the accent\is{accent!accent alternation} to mark a distinction between definite and indefinite reference.

\subsection{Some remarks on accent}\label{5sec:11}
\largerpage
%{
Scholars usually agree on the fact that accent has the function of establishing a contrast between accented and unaccented syllables \citep[][50]{garde:68}. Moreover, ``The term {\emph{stress}}\is{stress} is used here to refer to an abstract property of syllables within the domain of `words' (cf. Dixon and Aikhenvald 2002 for discussions of the notion {\emph{word}}). A stressed syllable is likely to be pronounced with more prominence\is{prominence} than unstressed syllables.'' \citep[][Section 1]{goedemans:vanderhulst:13}. The position of the accent is fixed in some languages (as in Czech\il{Czech} and French\il{French}), semifixed (Latin,\il{Latin} Polish\il{Polish}) or free (as in Russian,\il{Russian} Italian\il{Italian} and English\il{English}). Its position is sometimes predictable at the phonological level (for example, in light of syllable weight, the presence of non-accentuable morphemes, etc.) or at the morpho-syntactic level.
%}

Usually, the features of the accent are: ``greater loudness, higher pitch, greater duration and greater accuracy of articulation (most notably in vowels)'' \citep[][Section 1]{goedemans:vanderhulst:13}, but sometimes these features are not specific to accent and this makes its definition complicated. The F0 rise, for example, is also found in other prosodic phenomena, as well as vocalic length, which is linked not only to accentuation but also to the vowel's quantity.


\subsection{Some remarks on accent in Berber}\label{5sec:12}

Scholars unanimously agree that the accent has never been properly described in Berber.\il{Berber languages} This was in fact pointed out by A. Basset over sixty years ago \citep[][10]{basset:52}. The situation has not changed much since then, even though there has been a rise in interest in this issue over the last few years, especially by virtue of the tendency to study lesser-described Berber languages,\il{Berber languages} like those of the eastern part of the Berber area (Tunisia, Libya and Egypt), where the accent seems to have a more relevant role, when compared to the other Berber languages.\il{Berber languages}

An overview of the accent in the domain of Berber\il{Berber languages} studies was carried out by Vycichl and Chaker (\citeyear[][103-106]{vycichl:chaker:84}). In the first part of this study, 
Vycichl and Chaker (\citeyear[][103-105]{vycichl:chaker:84}) remark that in Guellala\il{Guellala} (Jerba, Tunisia), accent plays a peculiar role. The locative\is{locatives} is accented on the last syllable as it is in Tamezreṭ, Tunisia: {\emph{əlmáɣrəb}} `evening', {\emph{əlmaɣrə́b}} `in the evening'. Adjectives like {\emph{aməllal}} `white', distinguish a determined form {\emph{áməllal}} `the white', with the accent on the ancient definite article (the initial {\emph{a-}}, according to \citealt{vycichl:chaker:84}) while {\emph{amə́llal}} means `white' or `a white'. The authors add that in the past, accent probably played an important role, as in the Semitic languages.\il{Semitic languages}

%{
\cite{vycichl:chaker:84} also noticed that with the genitive preposition {\emph{n}}, the accent moves back one syllable: {\emph{agbə́n}} `house', {\emph{elbâb n ágbən}} `the door of the house'. This is found with other prepositions too: {\emph{amân}} `water', {\emph{y âman}} `into the water' \citep[][180-181]{vycichl:81}.\footnote{The English translation is mine. \cite[][180-181]{vycichl:81} uses the circumflex (ˆ) to mark accent on long vowels.} Other studies on the use of accent are those by \cite{louali:03} and \cite{louali:philippson:04} for Siwi, \cite{louali:philippson:05} for Siwi and Tuareg\il{Tuareg}, and \cite{lux:philippson:10} for a comparison of the accent in Tetserret\il{Tetserret} and Tamasheq\il{Tamasheq} (Niger).
%}

These studies show that the positioning of the accent in these Berber\il{Berber languages} varieties is different from that found in Siwi, which, in contrast, shares some characteristics with other varieties spoken in Tunisia and Libya. There is a dearth of studies and oral data concerning the latter, with the exception of those produced by \cite{brugnatelli:86,brugnatelli:05}, who compared the situation found in Nafusi\il{Nafusi} (Libya), where the location of the accent changes when the noun is preceded by a preposition \citep[][12]{beguinot:42}, with the one found in Jerba (Tunisia) or Siwa (according to \citealt{vycichl:chaker:84}):

\begin{exe}
\exi{} 
{\emph{uráġ}}: `fox' \\
{\emph{yefkû n úraġ}}: `he gave to the fox'.
\end{exe}

\largerpage
This ``movement'' of the accent on nouns was found by Brugnatelli in the Nafusi\il{Nafusi} texts after the prepositions {\emph{n, di, in, s, ded, af, denneg}} and with the exclamation particles {\emph{a/ai, ya}} \citep[][64-65]{brugnatelli:86}. The author remarks that in Beguinot's texts, the movement of the accent also takes place when the subject follows the verb \citep[][66]{brugnatelli:86}, where other Berber languages\il{Berber languages} have the annexed state. That is why the author concludes that there could be a relationship between accent and state distinction (free and annexed), which is no longer attested in Siwi and in other Berber\il{Berber languages} varieties spoken in Libya \citep[][68]{brugnatelli:86}. In Nafusi,\il{Nafusi} it seems that the position of the accent is important in distinguishing two different interpretations, in the case of kinship nouns such as:

\begin{exe}
\exi{}
{\emph{rûmmu}}: `my brother' \\
{\emph{rūmmû}}: `the brother, brother' \citep[][28-29]{beguinot:42}.
\end{exe}


\subsection{Previous studies on accent in Siwi}\label{5sec:13}

Accent on nouns in Siwi is insensitive to quantity as it can fall on the last or penultimate syllable of the same noun. Several authors agree that the accent on the last syllable codes indefiniteness and the accent on the penultimate syllable codes definiteness. While Louali does not recognize the function of marking locatives,\is{locatives} previously diagnosed by Vycichl, she confirms the possibility of coding the distinction between definite and indefinite forms through accent alternation\is{accent!accent alternation} \citep[][68-69]{louali:03}. According to \cite{louali:philippson:04,louali:philippson:05}, the function of the accent in Siwi is morpho-syntactic because it allows the distinction of the category of the verb (accent on the first syllable of the theme) and the category of the noun (when it is isolated, it has its accent on the last syllable).

{
Other factors that should be taken into consideration for the prediction of the position of the accent are the presence of prepositions ({\emph{i}} `to', {\emph{s}} `with, by means of', {\emph{n}} `of', {\emph{af}} `on', {\emph{d}} `with, comitative'), which has the consequence of moving the accent one or two syllables back \citep{louali:philippson:05}, and of possessive clitics, where the accent is always on the penultimate syllable. The authors also add that the position of the accent is linked to  pragmatic factors (\citeyear{louali:philippson:05}: 13). To conclude, Souag returns to the hypothesis formulated by Vycichl: ``In general, ultimate stress\is{stress} marks the indefinite, penultimate the definite'' \citep[][80]{souag:13}.
}

This overview leads us into the discussion of the corpus of data whose analysis illustrates the position that the accent can take on the noun, in different contexts and functions. The corpus is composed of a wordlist (highlighting the position of the accent when the noun is isolated) and spontaneous texts. Even though different factors can influence the elements that are at the base of accent formation (such as factors linked to the speaker, context, intonation\is{intonation} and position of the word in relation to the end of the prosodic unit), I decided to use this kind of sample in order to ascertain whether the accent has functions linked to morpho-syntax and pragmatics.

The position of the accent was determined through the use of PRAAT\footnote{Paul Boersma \& David Weenink. Praat: doing phonetics by computer [Computer program]. Version 6.1, retrieved 13 July 2019 from \url{http://www.praat.org}.} and the analysis confirms what has already been discovered by Louali and Philippson in \citeyear{louali:philippson:04}, namely that from a phonological point of view, in Siwi the accented syllable features higher pitch as its only consistent cue (\figref{5fig:1}). Higher pitch on the last syllable is not linked to the fact that the noun is at the end of an intonation\is{intonation} unit. The same goes for higher pitch on the penultimate syllable, which can be present even when a noun is at the end of an intonation\is{intonation} unit.

\begin{figure}
\centering
\includegraphics[width=12.5cm]{chapters/schiattarella-fig1.png}
\caption{In this picture from PRAAT, the first mention of {\emph{azidi}} `jackal' shows a higher pitch on the last syllable (azi{\textbf{dí}}), while the second mention shows a higher pitch on the penultimate syllable (a{\textbf{zí}}di). Intensity is indicated by the lower and lighter line.}\label{5fig:1}
\end{figure}


\subsection{Definiteness and indefiniteness}\label{5sec:14}

There are several ways definiteness and indefiniteness can be described: authors often use terms like uniqueness,\is{uniqueness} familiarity,\is{familiarity} specificity,\is{specificity} identifiability\is{identifiability} and referentiality\is{referentiality} to try to explain the properties of definite nouns. While indefinites are indeed associated with the fact that the referents are generic,\is{generics} non-specific and non-identifiable by both hearer and speaker,  definiteness is linked to concepts like familiarity,\is{familiarity} which refers to the possibility of a referent being recognized because it was mentioned previously (anaphora) or because it refers to the situation where it is immediately recognized by both the hearer and the speaker. \cite[][28]{christophersen:39} adds to the concept of familiarity\is{familiarity} the feature of being based on shared knowledge between the speaker and the hearer. Referents thus do not need to be mentioned before being considered as definite. Following this, anaphora is not be interpreted only in a strict sense (the referent is definite after it is first mentioned), as the definite noun can also solely be semantically linked to a previous referent, such as `the door' after talking about `a house'.

Another feature of definite nouns is uniqueness,\is{uniqueness} which is when there is only one possible referent the speaker could be referring to. Some nouns are more likely to be considered as definite because they are unique, such as individual nouns\is{noun types!individual noun}  ({\emph{sun, Pope}}, proper nouns), or nouns which are inherently relational, like some body parts and kinship terms\is{kinship terms} (like {\emph{brother, leg}}, etc.). Among inherently relational nouns,\is{noun types!relational noun} there are also the so-called `functional nouns'\is{noun types!functional noun} where, in addition, the referent is unique (like {\emph{nose, mother, father}}, etc.; \citealt[][307]{lobner:11}). That is why in many languages, in these specific cases, definiteness is not additionally marked by a definite marker, as this could be considered redundant.

There are also some syntactic constructions that help restrict the noun in order for it to be considered definite, such as relative clauses, adnominal possessive constructions and, to a lesser extent, adjective modification. Not all languages have definite or indefinite articles \citep{dryer:13a:5,dryer:13b:5}. Nevertheless, there is usually a way for the speaker to express whether the noun is definite or not. For example, some languages use demonstratives,\is{demonstratives} which act as definite articles. Possessives can also function as definite markers.

{
Definiteness can be strongly determined by pragmatics in many languages, especially those without articles. Indeed, information structure\is{information structure} and how information is conveyed is crucial and interacts with word order and with the possibility of determining whether a noun is a topic\is{topic} or a focus.\is{focus} Topic\is{topic} is related to what the information is about, as well as to the shared knowledge between the hearer and the speaker.  In contrast, focus\is{focus} is ``that portion of a proposition which cannot be taken for granted at the time of speech. It is the unpredictable or pragmatically non-recoverable element in an utterance. The focus\is{focus} is what makes an utterance into an assertion'' \citep[][207]{lambrecht:94}. That is why topic\is{topic} is usually associated with definiteness and focus\is{focus} with indefiniteness, even if this is not always the case.
}

A study on Polish\il{Polish} definiteness \citep{czardybon:17} has shown that topics\is{topic} are usually definite and in the preverbal position. When indefinites precede the verb, it is because they have to be considered to be the focus,\is{focus} in thetic constructions,\is{thetic statements} where the whole sentence is in focus\is{focus} \citep[][144]{lambrecht:94}. The possibility of interpreting nouns as definite or indefinite is then not linked to the position (pre- or post-verbal) but to their information structure\is{information structure} status. Information structure\is{information structure} also interacts with other morpho-syntactic means, which determine whether a noun is definite or not. 


\section{The function of the accent in Siwi}\label{5sec:2}

The aim of this second part is to show that the possibility of coding definiteness and indefiniteness can be conveyed by accent alternation,\is{accent!accent alternation} but also to highlight how other factors can interact in giving a definite/indefinite interpretation to the noun. Definiteness is indeed conveyed through a series of factors that interact with one another: position of the accent, semantics of the noun, information structure\is{information structure} and other morpho-syntactic elements.


\subsection{The position of the accent on nouns when isolated}\label{5sec:21}

Nouns, when isolated, carry the accent on the last syllable and, when preceded by a preposition, they carry it on the penultimate syllable. In order to illustrate this in this first section, I have used data from elicitations, because there are other factors to consider in discourse. My data confirm those presented by \cite{louali:philippson:05}, who state that the accent on the noun (except for kinship terms)\is{kinship terms} falls on the last syllable:

\begin{exe}
\ex\label{5ex:1}
\gll	aggʷ{\textbf{í}}d		\hspace*{1.4cm} aggʷid{\textbf{á}}n \\
	man.{\textsc{sg.m}}		{} man.{\textsc{pl.m}} \\
\glt
\ex\label{5ex:2}
\gll	talt{\textbf{í}}			\hspace*{1cm} təltaw{\textbf{é}}n \\
	woman.{\textsc{sg.f}}	{} woman.{\textsc{pl.f}} \\
\glt
\end{exe}

However, as already noted by Vycichl (\citeyear{vycichl:81}: 181; \citeyear{vycichl:05}: 207) and \cite[][12]{louali:philippson:05}, if the noun is preceded by a preposition, the accent is on the penultimate syllable:

\begin{exe}
\ex\label{5ex:3}
\gll	agb{\textbf{ə́}}n		\hspace*{0.95cm}	g 	{\textbf{á}}gbən \\
	house.{\textsc{sg.m}}	{} in 	house.{\textsc{sg.m}} \\
\glt
\ex\label{5ex:4}
\gll	am{\textbf{á}}n		\hspace*{1cm}	i	{\textbf{á}}man \\
	water.{\textsc{pl.m}}	{} to 	water.{\textsc{pl.m}} \\
\glt
\ex\label{5ex:5}
\gll	ul{\textbf{í}}			\hspace*{1cm}	sg 	{\textbf{ú}}li \\
	heart.{\textsc{sg.m}}	{} from	heart.{\textsc{sg.m}} \\
\glt
\end{exe}


\subsection{The position of the accent in discourse: accent on the last syllable}\label{5sec:22}

As mentioned in \sectref{5sec:13}, it is usually assumed that the accent on the last syllable codes indefiniteness and the accent on the penultimate syllable codes definiteness. In this section, I will start by analyzing nouns where the accent is placed on the last syllable. In existential predicative constructions,\is{existential sentences} the noun after {\emph{di}} `there is' has the accent on the last syllable, when this structure is used to introduce new referents into the discourse and they appear for the first time:

\begin{exe}
\ex\label{5ex:6}
\gll	di			{\textbf{aggʷíd}}. \\
	{\textsc{exist}}		man.{\textsc{sg.m}} \\
\glt	`There was a man.'
\ex\label{5ex:7}
\gll	di			{\textbf{taltí}}		/	d	{\textbf{aggʷíd}}. \\
	{\textsc{exist}}	woman.{\textsc{sg.f}}	/	and	man.{\textsc{sg.m}} \\
\glt	`There were a woman and a man.'
\end{exe}

This is also the case for the preposition {\emph{ɣuṛ}} `at' + pronoun, when it expresses possession and the referent is generic:\is{generics}

\begin{exe}
\ex\label{5ex:8}
\gll	šal n isíwan ɣúṛ-əs {\textbf{iǧəḅḅaṛə́n}} dabb. \\
	town.{\textsc{sg.m}} of siwa at-{\textsc{3sg}} palm\_tree.{\textsc{pl.m}} many \\
\glt	`Siwa has a lot of palm trees.'
\end{exe}

In the following example, both {\emph{di}} and {\emph{ɣuṛ}} are used to present all the main characters of the story:

\begin{exe}
\ex\label{5ex:9}
\gll	máṛṛa di {\textbf{aggʷíd}} / d::: ɣúṛ-əs {\textbf{tləččá}} / \\
	once	{\textsc{exist}} man.{\textsc{sg.m}} /	and at-{\textsc{3sg}} girl.{\textsc{sg.f}} / \\
\glt
\exi{}
\gll	d {\textbf{akəḅḅí}}. / abbá-nnəs n tlə́čča / \\
	and boy.{\textsc{sg.m}} / father.{\textsc{sg.m-poss.3sg}} of girl.{\textsc{sg.f}} / \\
\glt	`Once upon a time there was a man, he had a daughter and a son. The father of the girl...'
\end{exe}

At the end of the narrations, to recapitulate the topic,\is{topic} Siwi uses a non-verbal predication with a pronominal demonstrative\is{demonstratives} and a juxtaposed noun. This nominal predicate has the accent on the last syllable, because it is not taking up a specific referent, but is intended to recapitulate the subject of the narration:

\begin{exe}
\ex\label{5ex:10}
\gll	w-om			{\textbf{šahín}}. \\
	{\textsc{dem-2sg.f}}	tea.{\textsc{sg.m}} \\
\glt	`This is (about) the tea.'
\end{exe}

When something is non-specific and generic,\is{generics} and at the same time the speaker does not need to refer to a particular category in which the referent ought to be included, we see the use of the accent on the last syllable:

\begin{exe}
\ex\label{5ex:11}
\gll	əssn-ím-a	ánni	azídi	/ \\
	know.{\textsc{pfv-2pl-pragm}}	{\textsc{comp}}	jackal.{\textsc{sg.m}}	/ \\
\glt
\exi{}
\gll	l-í-ʕəṃṃaṛ	ɣúṛ-əs	{\textbf{ankán}}	/	{\textbf{agbə́n}}	/ \\
	{\textsc{neg-3sg.m}}-do.{\textsc{ipfv}}	at-{\textsc{3sg}}	place.{\textsc{sg.m}}	/	house.{\textsc{sg.m}}	/ \\
\glt
\exi{}
\gll	díma	i-ʕə́ṃṃaṛ	{\textbf{taṃɣáṛt}}	/ \\
	always	{\textsc{3sg.m}}-do.{\textsc{ipfv}}	cave.{\textsc{sg.f}}	/ \\
\glt
\exi{}
\gll	g	{\textbf{idrarə́n}}.	/ \\
	in	mountain.{\textsc{pl.m}}	/ \\
\glt	`Do you know that the jackal doesn't have a place, or a house, he is always in a cave, in [the] mountains.'
\end{exe}

{
Later in the narration, the protagonist (a hyena) finds a cave (accent on the last syllable), and when the same cave is mentioned again, the accent is placed on the penultimate syllable. This opposition will be analyzed in more detail in \ref{5sec:25}.
}

Sometimes, a referent appears after the first mention with the accent on the last syllable: this is the case when the referent does not need to be reactivated (through anaphora), but just to be mentioned again because, for example, the speaker needs to add new information, as in the following example where the {\emph{n}} + adjective construction highlights the beauty of the girl (whose birth was unexpected), even if she has just been mentioned in the previous intonation\is{intonation} unit. The fact that it is taken up again is not intended as a strategy to mark it as known, but to qualify it:

\begin{exe}
\ex\label{5ex:12}
\gll	əntátət	t-iráw	{\textbf{tləččá}}	/  {\textbf{tləččá}}	n	tkwáyəst \\
	{\textsc{idp.3sg.f}}	{\textsc{3sg.f}}-give\_birth.{\textsc{pfv}}	girl.{\textsc{sg.f}}	/  girl.{\textsc{sg.f}}	of	beautiful.{\textsc{sg.f}} \\
\glt	`(After she gave birth only to boys, we wanted a girl.) She gave birth to a girl'. A beautiful girl...'
\end{exe}

If the noun is indefinite, the accent falls on the last syllable, even if it is preceded by a preposition. It is therefore worth noting here that the presence of the preposition does not obligatorily trigger the presence of the accent on the penultimate syllable, contrary to the discussion in \sectref{5sec:21} with regard to nouns in isolation. In this case, the speaker is referring to a generic place:

\begin{exe}
\ex\label{5ex:13}
\gll	kan xsi-ṭ aẓəṛṛá šal s ə́nniǧ g-úni-ṭ i {\textbf{adrár}}. \\
	if want.{\textsc{pfv-2sg}} see.{\textsc{vn}} town.{\textsc{sg.m}} from up {\textsc{irr}}-climb.{\textsc{aor-2sg}} to mountain.{\textsc{sg.m}} \\
\glt	`If you want to see the town from the top, you will (have) to climb a mountain.'
%%%
\ex\label{5ex:14}
\gll	yə-mráq g {\textbf{ankán}} / y-ifá	aṭíl. \\
	{\textsc{3sg.m}}-reach.{\textsc{pfv}} in place.{\textsc{sg.m}} / {\textsc{3sg.m}}-find.{\textsc{pfv}} garden.{\textsc{sg.m}} \\
\glt	`He arrived in a place, he found a garden.'
\end{exe}

In the following example, the author is not referring to any specific palm tree, but he is asking the hearer to imagine a generic palm tree:

\begin{exe}
\ex\label{5ex:15}
\gll	ga-nə́-bdu		sg	{\textbf{aǧəḅḅáṛ}}. \\
	{\textsc{irr-1pl}}-start.{\textsc{aor}} from palm\_tree.{\textsc{sg.m}} \\
\glt	`Let's start from a palm tree.'
\end{exe}

If we look at the cases listed here, we can see that nouns where the accent is placed on the last syllable are not necessarily linked to first mention -- see for example nouns with the accent on the last syllable used when the speaker needs to recapitulate the topic\is{topic} of the narration, or when an already mentioned referent reappears, but is not crucial to the continuation of the narration. The accent on the last syllable is then linked to the fact that the speaker needs to present the referent, comment on it or recapitulate.


\subsection{The position of the accent in discourse: accent on the penultimate syllable}\label{5sec:23}

In this section, examples from spontaneous discourse are presented in order to show how placing the accent on the penultimate syllable sometimes indicates that a noun is definite. For each example, an explanation of the kind of definiteness expressed will be given.

The accent on the penultimate syllable can be used with nouns mentioned for the first time, but which are identifiable: the hearer knows who or what the referents that the speaker is talking about are, by virtue of information provided earlier by the speaker. In the following example, the speaker is talking about traditions in Siwa and the hearer understands immediately that he is talking about women from the oasis. The speaker is not referring to specific women, but rather to a category of people:

\begin{exe}
\ex\label{5ex:16}
\gll	{\textbf{təččíwen}} tə-ṛṭá-ya / \\
	girl.{\textsc{pl.f}} {\textsc{3sg.f}}-cover.{\textsc{pfv-pragm}} /  \\
\glt
\exi{}
\gll	{\textbf{təltáwen}} tə-ṛṭá-ya. / \\
	woman.{\textsc{pl.f}} {\textsc{3sg.f}}-cover.{\textsc{pfv-pragm}} / \\
\glt	`Girls are covered, women are covered.'
\end{exe}

{
The speaker is only referring to women in Siwa, and the fact of them being covered is considered as shared knowledge for both hearer and speaker. The same applies to the following example, where the sheikhs are mentioned for the first time:
}

\begin{exe}
\ex\label{5ex:17}
\gll	baʕdén	yə-ʕṃṛ-ín-a	/	albáb	/ \\
	then	3-do.{\textsc{pfv-pl-pragm}}	/	door.{\textsc{sg.m}}	/ \\
\glt
\exi{}
\gll	i-təṃṃá-n-as	albáb	n	šal.	/ \\
	3-say.{\textsc{ipfv-pl-3sg.dat}}	door.{\textsc{sg.m}}	of	town.{\textsc{sg.m}}	/ \\
\glt
\exi{}
\gll	{\textbf{ləmšáyəx}}	yə-ʕʕə́nʕən-ən	ə́gd-əs.	/	\\
	chief.{\textsc{pl.m}}	3-sit\_down.{\textsc{pfv-pl}}	in-{\textsc{3sg}}	/	\\
\glt 	{`Then they made a door, they call it ``door of the town''. The chiefs sat in it.'}
\end{exe}

Usually, when a noun appears again after a first mention it is considered to be anaphoric, but this anaphora can be also be associative: a noun is definite because it has a semantic relationship with what precedes it. In the following example, the oven is inferred from the fact that the speaker is talking about how to cook some dishes:

\begin{exe}
\ex\label{5ex:18}
\gll	kan	{\textbf{əṭṭáḅənt}}	tə-ḥmá-ya \\
	if	oven.{\textsc{sg.f}}	{\textsc{3sg.f}}-be\_hot.{\textsc{pfv-pragm}} \\
\glt	`when the oven was hot'
\end{exe}

In the next example, the window is mentioned for the first time, but the story is about a girl who has been kidnapped and is being held in a castle, so the presence of a window is retrievable from the situation:

\begin{exe}
\ex\label{5ex:19}
\gll	baʕdén tə-ẓṛ-á sg {\textbf{állon}}. \\
	then 3{\textsc{sg.f}}-see.{\textsc{pfv-3sg.m.do}} from window.{\textsc{sg.m}} \\
\glt	`Then (Jmila) saw him from the window.'
\end{exe}

In the following example, the well is mentioned for the first time, but it is coded as definite by virtue of it being clear that it is the only well that is present and perceivable by the characters in the castle of the sultan (visible situation use, \citealt{hawkins:78}: 110):

\begin{exe}
\ex\label{5ex:20}
\gll	t-uṭá	i	{\textbf{ánu}}	n	áman. \\
	3{\textsc{sg.f}}-fall.{\textsc{pfv}}	to	well.{\textsc{sg.m}}	of	water.{\textsc{pl.m}} \\
\glt	`(The ball) fell into the well of water.'
\end{exe}

Similarly, in the following example, the pot is mentioned for the first time, but the storyteller is asking the hearer to imagine that the woman is taking the only pot visible in the kitchen, in order to cook the chicken:

\begin{exe}
\ex\label{5ex:21}
\gll	t-ṛaḥ	tə-ṣṣáy	{\textbf{əṭṭánǧṛət}}	/ \\
	{\textsc{3sg.f}}-go.{\textsc{pfv}}	{\textsc{3sg.f}}-take.{\textsc{pfv}}	pot.{\textsc{sg.f}}	/ \\
\glt
\exi{}
\gll	tə-ɣṛə́ṣ	tyaẓə́ṭ.	/ \\
	{\textsc{3sg.f}}-slaughter.{\textsc{pfv}}	chicken.{\textsc{sg.f}}	/ \\
\glt	`She took the pot, she slaughtered a chicken.'
\end{exe}

Proper nouns, kinship terms\is{kinship terms} and toponyms,\is{toponym} which already have a high degree of referentiality,\is{referentiality} are usually accented on the penultimate syllable, as in the following examples:

\begin{exe}
\exi{}
{\emph{isíwan}}: Siwa or the people from Siwa \\
{\emph{šáli}}: the citadel in the oasis of Siwa \\
{\emph{ábba}}: father \\
{\emph{wə́ltma}}: sister. 
\end{exe}

{
Placement of the accent on the penultimate syllable is therefore linked to the need to present a referent as identifiable or unique. The uniqueness\is{uniqueness} of a referent can be linked both to its semantics and to its pragmatics (unique referent in the context of use). It also codes anaphora, as we will see in more detail in \sectref{5sec:25}.
}

\subsection{Interaction between accent position and other strategies to mark definiteness and indefiniteness}\label{5sec:24}

The examples discussed in \sectref{5sec:22} and \sectref{5sec:23} already seem to confirm the hypothesis regarding the variation in position of the accent as a means of coding definiteness or indefiniteness. Nevertheless, in this section, I will show that the position of the accent interacts with other devices, in order to convey a definite or indefinite interpretation to the noun. These elements sometimes override the accent alternation\is{accent!accent alternation} itself. If the noun is determined by a possessive clitic, the accent is always on the penultimate syllable:

\begin{exe}
\ex\label{5ex:22}
\gll	{\textbf{agbə́n}}-nək \\
	house.{\textsc{sg.m-poss.2sg.m}} \\
\glt	`your house'
\ex\label{5ex:23}
\gll	{\textbf{abbá}}-nnəs \\
	father.{\textsc{sg.m-poss.3sg}} \\
\glt	`his father'
\end{exe}

Adnominal possessive constructions (N + {\emph{n}} `of' + N) are definite most of the time, as the construction is a way to delimit the head noun. In the following example, the definite interpretation is conveyed by the entire construction, so the accent on the head noun can be on the last syllable:

\begin{exe}
\ex\label{5ex:24}
\gll	{\textbf{əddhán}}	n	{\textbf{isíwan}} \\
	oil.{\textsc{sg.m}}	of	Siwa \\
\glt	`the oil of Siwa'
\end{exe}

There are nevertheless cases where the interpretation of these constructions is indefinite, especially when they express part/whole relations, where the construction is used to refer to any generic part of the whole. In this case, both nouns have the accent on the last syllable:

\begin{exe}
\ex\label{5ex:25}
\gll	{\textbf{tḥəbbə́t}}	n	{\textbf{təṃẓén}} \\
	grain.{\textsc{sg.f}}	of	barley.{\textsc{pl.f}} \\
\glt	`a grain of barley'
\end{exe}

In general, when a noun is followed by a demonstrative,\is{demonstratives} and is therefore definite, it does not always have the accent on the penultimate syllable, as one would expect. The adnominal demonstrative\is{demonstratives} already codes definiteness, so most of the time, the noun has the accent on the last syllable:

\begin{exe}
\ex\label{5ex:26}
\gll	t-qad		{\textbf{tləččá}}	{\textbf{tat-ók}}	/ \\
	{\textsc{3sg.m}}-take.{\textsc{pfv}}	girl.{\textsc{sg.f}}	{\textsc{dem.f-2sg.m}}	/ \\
\glt 
\exi{}
\gll	tə-mráq	g	təṭ. \\
	{\textsc{3sg.f}}-reach.{\textsc{pfv}}	in	spring.{\textsc{sg.f}} \\
\glt	`She took this girl, she arrived at the spring.'
\end{exe}

Cases where nouns followed by demonstratives\is{demonstratives} have the accent on the penultimate syllable are nevertheless attested, especially (but not exclusively) when they appear in left-detached constructions:\footnote{This construction will be discussed in detail in \sectref{5sec:25}.}

\begin{exe}
\ex\label{5ex:27}
\gll	i-ʕəṃṃáṛ-ən	naknáf.	/ \\
	3-do.{\textsc{ipfv-pl}}	naknaf.{\textsc{sg.m}}	/ \\
\glt
\exi{}
\gll	{\textbf{náknaf}}	{\textbf{daw-érwən}}	/	smiyət-ə́nnəs	/ \\
	naknaf.{\textsc{sg.m}}	{\textsc{dem-2pl}}	/	name.{\textsc{sg.f-poss.3sg}}	/ \\
\glt	`They prepared the {\emph{naknaf}}. This {\emph{naknaf}} was called... ({\emph{tqaqish}}).'
\end{exe}

When there is a preposition + N + demonstrative,\is{demonstratives} the accent is on the penultimate syllable of the noun:

\begin{exe}
\ex\label{5ex:28}
\gll	sg	{\textbf{ə́lwoqt}}	tat-érwən \\
	from	time.{\textsc{sg.f}}	{\textsc{dem.f-2pl}} \\
\glt	`from that moment'
\ex\label{5ex:29}
\gll	ánni	gá-ḥḥ-ax	d	{\textbf{iṣəṛɣénən}}	daw-i-(y)óm. \\
	{\textsc{comp}}	{\textsc{irr}}-go.{\textsc{aor-1sg}}	and	bedouin.{\textsc{pl.m}}	{\textsc{dem-pl-2sg.f}} \\
\glt	`So I (can) go with these Bedouins.'
\end{exe}

We often find the noun in right-detached constructions, which have the function of reactivating a referent \citep[][280]{mettouchi:schiattarella:18}, with the accent on the last syllable. In this case, the fact that the noun is in a different intonation\is{intonation} unit is sufficient to indicate that the referent has already been mentioned, and it needs to be reactivated. It is, then, the construction itself, not the position of the accent, that codes this function:

\begin{exe}
\ex\label{5ex:30}
\gll	i-lə́hhu-n	/	{\textbf{təṛwawén}} \\
	3-be\_happy.{\textsc{ipfv-pl}}	/	child.{\textsc{pl.f}} \\
\glt	`They were happy, the children.'
\end{exe}

However, if a noun is inherently referential (proper nouns), the accent is on the penultimate syllable:

\begin{exe}
\ex\label{5ex:31}
\gll	əǧǧə́n	n	áddoṛ	/	y-uṭə́n	/	{\textbf{Ḥássnin}}.	/ \\
	one.{\textsc{m}}	of	time.{\textsc{sg.m}}	/	{\textsc{3sg.m}}-get\_sick.{\textsc{pfv}}	/	ḥassnin	/ \\
\glt
\exi{}
\gll	yə-ngə́r	yə-ṭṭís-a	g	ágbən.	// \\
	{\textsc{3sg.m}}-stay.{\textsc{pfv}}	{\textsc{3sg.m}}-rest.{\textsc{pfv-pragm}}	in	house.{\textsc{sg.m}}	// \\
\glt	`Once, he got sick, Hassnin. He stayed resting at home.'
\end{exe}

Most of the time, the referent in this right-detached construction is followed by a demonstrative,\is{demonstratives} and in this case, as mentioned in reference to examples (\ref{5ex:26}) and (\ref{5ex:27}), the position of the accent has no function in determining the definiteness of the noun.

\begin{exe}
\ex\label{5ex:32}
\gll	g-úɣi-x	ə́ǧǧət	gə́d-sən	/ \\
	{\textsc{irr}}-buy.{\textsc{aor-1sg}}	one.{\textsc{f}}	from-{\textsc{3pl}}	/ \\
\glt
\exi{}
\gll	{\textbf{amẓá}}	{\textbf{daw-óm}}. \\
	ogre.{\textsc{sg.m}}	{\textsc{dem-2sg.f}} \\
\glt `He thought: ``I will marry one of them'', this ogre.'
\end{exe}

Contrary to descriptive relative clauses where the relative marker is not obligatory, restrictive relative clauses are introduced by {\emph{(n) wən}} (`{\textsc{sg.m/pl}}' and sometimes `{\textsc{sg.f}}') or {\emph{tən}} (`{\textsc{sg.f}}') \citep{schiattarella:14}. Head nouns in these kinds of relative clauses must usually be considered definite and have the accent on the penultimate syllable:

\begin{exe}
\ex\label{5ex:33}
\gll	{\textbf{tálti}}	wәn	aggʷid-ə́nnәs	yә-ṃṃút \\
	woman.{\textsc{sg.f}}	{\textsc{rel}}	 man.{\textsc{sg.m-poss.3sg}} 	{\textsc{3sg.m}}-die.{\textsc{pfv}} \\
\glt	`the woman whose husband died'
\end{exe}

This does not mean that the definite interpretation is only given by the accent, because the restrictive relative clause is already a way to restrict a head noun, giving it a definite interpretation:

\begin{exe}
\ex\label{5ex:34}
\gll	yə-ṭlə́b	s	ɣúṛ-əs	{\textbf{tləččá}} \\
	{\textsc{3sg.m}}-ask.{\textsc{pfv}}	from	at-{\textsc{3sg}}	girl.{\textsc{sg.f}} \\
\glt
\exi{}
\gll	n	wən	yə-xs-ét. \\
	of	{\textsc{rel}}	{\textsc{3sg.m}}-want.{\textsc{pfv-3sg.f.do}} \\
\glt	`He asked for the girl he wanted.'
\end{exe}

Of course, not all head nouns of restrictive relative clauses should be considered as definite, such as in the following example where the head is an indefinite pronoun:

\begin{exe}
\ex\label{5ex:35}
\gll	kull	{\textbf{ə́ǧǧən}}	wən	ɣúṛ-əs	aṭíl \\
	every	one.{\textsc{m}}	{\textsc{rel}}	at-{\textsc{3sg}}	garden.{\textsc{sg.m}} \\
\glt	`everyone who has a garden'
\end{exe}

In general, when {\emph{ə́ǧǧən}} `one' is used alone, as an indefinite pronoun, and not as a numeral, the accent is always on the penultimate syllable. When it is a numeral, the accent can also be on the last syllable.

\begin{exe}
\ex\label{5ex:36}
\gll	mak	{\textbf{ə́ǧǧən}}	yə-xsá	anǧáf \\
	when	one.{\textsc{m}}	{\textsc{3sg.m}}-want.{\textsc{pfv}}	marry.{\textsc{vn}} \\
\glt	`when someone wants to get married'
\end{exe}


The accent falls on the last syllable when it expresses the locative.\is{locatives} In the introduction, I mentioned that according to \cite{louali:03} the locative\is{locatives} is not expressed by the position of the accent, but this form is in fact present in our corpus. This structure is only used when the place is referential and identifiable (so it is only possible with a toponym\is{toponym} or {\emph{ankán}} `place' + {\emph{n}} `of' + name of the place or when the name of the place is followed by a possessive). In this case, the referentiality\is{referentiality} of the noun is hierarchically more important than the fact that the accent is on the last syllable:


\begin{exe}
\ex\label{5ex:37}
\gll	gá-ḥḥ-aṭ	{\textbf{ankán}}	n	áʕṛus. \\
	{\textsc{irr}}-go.{\textsc{aor-2sg}}	place.{\textsc{sg.m}}	of	wedding.{\textsc{sg.m}} \\
\glt	`You will go to the wedding place.' 
\ex\label{5ex:38}
\gll	yə-ččá	{\textbf{əlgaṛá}}. \\
	{\textsc{3sg.m}}-eat.{\textsc{pfv}}	el\_gara \\
\glt	`He ate in El Gara.'
\ex\label{5ex:39}
\gll	ənnə́gr-ax		{\textbf{šalí}}. \\
	live.{\textsc{ipfv-1sg}}	shali \\
\glt	`I live in Shali.'
\end{exe}

Indeed, a generic noun\is{generics} indicating a place cannot mark a locative\is{locatives} solely by placing the accent on the last syllable (without the preposition):

\begin{exe}
\ex\label{5ex:40}
\gll	*i-nə́ddum		timədrást. \\
	{\textsc{3sg.m}}-sleep.{\textsc{ipfv}}	school.{\textsc{sg.f}} \\
\glt	Intended: `He sleeps at school.'
\end{exe}

Locatives\is{locatives} with the accent on the last syllable are also attested with nouns followed by a possessive. In this case, the accent is on the last syllable (which is unusual, because the accent of nouns with the possessive is always on the penultimate syllable, see examples (\ref{5ex:22}) and (\ref{5ex:23})):

\begin{exe}
\ex\label{5ex:41}
\gll	i-təčč	aksúm	{\textbf{timədrast-ənnə́s}}. \\
	{\textsc{3sg.m}}-eat.{\textsc{ipfv}}	meat.{\textsc{sg.m}}	school.{\textsc{sg.f-poss.3sg}} \\
\glt	`He eats meat in his school.'
\end{exe}


\subsection{Same referent in close intonation\is{intonation} units}\label{5sec:25}

In this section, I will analyze examples of structures where the opposition between the same noun with the accent on the last syllable and on the penultimate syllable is more visible. Indeed, in many cases in the corpus analyzed here, a noun is introduced for the first time, mostly with an existential predication ({\emph{di}} `there is'), and then taken up again with the accent on the penultimate syllable, in the following intonation\is{intonation} units:

\begin{exe}
\ex\label{5ex:43}
\gll	di	{\textbf{tləččá}}	tkwayə́st	t-ṛáwad-as \\
	{\textsc{exist}}	girl.{\textsc{sg.f}}	beautiful.{\textsc{sg.f}}	{\textsc{3sg.f}}-look.{\textsc{ipfv-3sg.dat}} \\
\glt
\exi{}
\gll	sg	állon.	 //	{\textbf{tlə́čča}}	t-nə́ddum. \\
	from	window.{\textsc{sg.m}}	//	girl.{\textsc{sg.f}}	{\textsc{3sg.f}}-sleep.{\textsc{ipfv}} \\
\glt	`There was a beautiful girl who was watching him from the window. (Afterwards) the girl was sleeping.'
%%
\ex\label{5ex:44}
\gll 	g-í-ftək	{\textbf{talís}} \\
	{\textsc{irr-3sg.m}}-open.{\textsc{aor}}	tank.{\textsc{sg.f}} \\
\glt
\exi{}
\gll	g-i-sə́-ssu	amán.	/ \\
	{\textsc{irr-3sg.m-caus}}-drink.{\textsc{aor}}	water.{\textsc{pl.m}}	/ \\
\glt
\exi{}
\gll	kan	lá-di	{\textbf{tális}} \\
	if	{\textsc{neg-exist}}	tank.{\textsc{sg.f}} \\
\glt	`He opens a tank and he drinks water. If there is no tank...'
%%
\ex\label{5ex:45}
\gll	máṛṛa	di	{\textbf{azidí}}	/	d	tɣátt	// \\
	time	{\textsc{exist}}  jackal.{\textsc{sg.m}}	/	and	goat.{\textsc{sg.f}}	// \\
\glt
\exi{}
\gll	{\textbf{azídi}}	kull	yom	/	g-í-f̣f̣aɣ.	/ \\
	jackal.{\textsc{sg.m}}	every	day	/	{\textsc{irr-3sg.m}}-go\_out.{\textsc{aor}}	/ \\
\glt	`Once upon a time there was a jackal and a goat. Every day, the jackal would go out.'
\end{exe}

Left-detached constructions, used to mark a subtopic shift to what is introduced in the preceding discourse \citep[][278]{mettouchi:schiattarella:18}, are also characterized by this alternation.\is{accent!accent alternation} First the referent is introduced with the accent on the last syllable; it is then taken up again in a subsequent intonation\is{intonation} unit with the accent on the penultimate syllable. In this case too, the function of the accent is anaphora:

\begin{exe}
\ex\label{5ex:46}
\gll	əlmanẓár	{\textbf{aməllál}}	/	{\textbf{amə́llal}}	dáw-om	/	w-om	{\textbf{tisə́nt}}.	/ \\
	view.{\textsc{sg.m}}		white.{\textsc{sg.m}}	/	white.{\textsc{sg.m}}	{\textsc{dem-2sg.f}}	/  {\textsc{dem-2sg.f}}	salt.{\textsc{sg.f}}	/\\
\glt
%%
\exi{}
\gll	{\textbf{tísənt}}	/	ənšní	n-xə́ddam-et. \\
	salt.{\textsc{sg.f}}	/	{\textsc{idp.1pl}}	{\textsc{1pl}}-work.{\textsc{ipfv-3sg.f.do}} \\
\glt	`A white view, this white, it is the salt. The salt, we work it.'
%%
\ex\label{5ex:47}
\gll	sad\_əlḥának	s	{\textbf{tiní}}	/	d	{\textbf{arə́n}}	/ \\
	sad\_əlḥanak	with	date.{\textsc{sg.f}}	/	and	flour.{\textsc{pl.m}}	/ \\
\glt
\exi{}
\gll	bass	{\textbf{tíni}}	d	{\textbf{árən}}	/ \\
	but	date.{\textsc{sg.f}}	and	flour.{\textsc{pl.m}}	/ \\
\glt
\exi{}
\gll	l-í-ḥaṭṭu-n-{\textbf{asən}}	amán.\\
	{\textsc{neg}}-3-put.{\textsc{ipfv-pl-3pl.dat}}	water.{\textsc{pl.m}} \\
\glt	`{\emph{Sad əlḥanak}} (is made) of dates and flour, but (to) the dates and flour, they don't add water.'
\end{exe}

It seems that in these constructions, the placement of the accent to mark first mention and anaphora is strictly linked to the spatial proximity of the same referent, probably because the alternation\is{accent!accent alternation} is more easily audible when the nouns are pronounced in a very short period of time, while it seems that other devices are needed to mark the anaphoric function of a noun that has already been mentioned, when the two instances of the noun being mentioned are far from each other. Moreover, in the constructions discussed in this paragraph, the noun, when taken up again, becomes the topic\is{topic} of the discourse, which is not always the case when referents that have already been mentioned reappear in a discourse.

\section{Discussion and conclusions}\label{5sec:5}

This paper has analyzed different morpho-syntactic, semantic and pragmatic factors which all contribute to the definiteness or indefiniteness of the noun, specifically when they interact with the position of the accent on the last or penultimate syllable. It appears that the assumption that the accent on the last syllable codes indefiniteness and the accent on the penultimate syllable codes definiteness is too simplistic: when other factors intervene, the situation can be different. After describing the pattern of the accent position when a noun is isolated, whether or not it is preceded by a preposition, I described the environments where it is more likely that the accent will be placed on the last and on the penultimate syllable.

{
The accent is on the last syllable when nouns are mentioned for the first time, especially through existential and possessive predications. Moreover, a noun that does not need to be reactivated has the accent on the last syllable, as the speaker is only mentioning it to allow the continuation of the narration, sometimes adding new information. The accent is on the penultimate syllable when a noun is anaphoric and the referent is taken up a few intonation\is{intonation} units after the first mention. The proximity here is crucial, as the anaphoric function could also be coded by demonstratives.\is{demonstratives} The same happens with left-detached constructions where the noun is first introduced into an intonation\is{intonation} unit, and is then taken up again in the following intonation\is{intonation} unit (with the accent on the penultimate syllable), and then reappears again in the form of a resumptive pronoun in the following intonation\is{intonation} unit. Anaphora can also be associative, with the referent only semantically linked to a previous noun. The accent is also placed on the penultimate syllable when the noun is identifiable and belongs to a recognizable category for both hearer and speaker, or when it refers to something which is clearly recognizable or perceivable as unique in the particular context of use.
}

Some syntactic constructions allow for the restriction, and consequently the definiteness, of the head noun, namely adnominal possessive constructions and restrictive relative clauses. In the first case, the second noun of the construction usually has the accent on the penultimate syllable. Finally, some nouns are semantically referential (proper nouns, toponyms,\is{toponym} kinship terms),\is{kinship terms} so they all have the accent on the penultimate syllable.

Nevertheless, we can observe that there are some factors that interact and override the function of the accent position, when conveying (in)definiteness, such as right- and left-detached constructions or when a noun is followed by a demonstrative\is{demonstratives} or a possessive, when it is a toponym,\is{toponym} for locatives\is{locatives} or when it is followed by a relative clause. Definiteness and indefiniteness in Siwi are thus coded in a complex way, and they are only achieved through the interaction of different elements, at different levels. The full range of aspects of interaction among all these means still needs to be studied in detail.

Further to what has been said so far, I will conclude by adding that Siwi allows all orders when only one argument (A, S or O) is present. When there are two arguments, only AVO is possible \citep[][288-289]{mettouchi:schiattarella:18}. Subject affixes on the verb are obligatory in Berber\il{Berber languages} and the presence of a co-referent lexical noun is rare. OV is quite a rare order, as is VA, and hence most of the time nouns before the verb are subjects and nouns after the verb are objects. As nouns can be both subject and object and can have the accent on the last or penultimate syllable in preverbal or post-verbal positions, there is no relationship, synchronically, between the coding of grammatical relation and the position of the accent.

In the corpus analyzed for this study, most of the nouns in preverbal position have the accent\is{accent|)} on the penultimate syllable, while most of the nouns in post-verbal position have the accent on the last syllable. One possible explanation is that nouns in preverbal position are topics,\is{topic} thus conveying known information, while post-verbal nouns are focus,\is{focus} thus conveying unpredictable or additional information. This hypothesis still needs to be fully analyzed.\il{Siwi (Berber)|)}

\section*{Acknowledgments}
I wish to thank here all the men and women consulted for this study, who accepted to collaborate with me during my stays in Siwa. I also wish to thank the editors of the volume, the reviewers, for their valuable comments and Amina Mettouchi, who gave me feedback on earlier versions of the paper.

\section*{Abbreviations}
\begin{tabularx}{.45\textwidth}{lQ}
{\textsc{a}}		& agent \\
{\textsc{aor}}		& aorist \\
{\textsc{caus}}	& causative \\
{\textsc{comp}}	& complementizer \\
{\textsc{dat}}		& dative/indirect object \\
{\textsc{dem}}		& demonstrative \\
{\textsc{do}}		& direct object \\
{\textsc{exist}}		& existential \\
{\textsc{f}}		& feminine \\
{\textsc{idp}}		& independent pronoun\\
{\textsc{ipfv}}		& imperfective \\
{\textsc{irr}}		& irrealis \\
{\textsc{m}}		& masculine \\
{\textsc{o}}		& object \\
{\textsc{pfv}}		& perfective \\
{\textsc{pl}}		& plural \\
{\textsc{poss}}	& possessive \\
\end{tabularx}
\begin{tabularx}{.45\textwidth}{lQ}
{\textsc{pragm}}	& pragmatic relevance marker \\
{\textsc{n}}		& noun \\
{\textsc{neg}}		& negative \\
{\textsc{rel}}		& relative marker \\
{\textsc{s}}		& subject \\
{\textsc{sg}}		& singular \\
{\textsc{v}}		& verb \\
{\textsc{vn}}		& verbal noun \\
1		& first person \\
2 		& second person \\
3		& third person \\
/		& end of a minor prosodic unit \\
//		& end of a major prosodic unit \\
\\
\end{tabularx}

%\noindent
%\begin{tabular}{ll}
%/		& end of a minor prosodic unit \\
%//		& end of a major prosodic unit \\
%\end{tabular}


{\sloppy\printbibliography[heading=subbibliography,notkeyword=this]}
\end{document}
