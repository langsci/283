\documentclass[output=paper]{langsci/langscibook}
\ChapterDOI{10.5281/zenodo.4049689}

\author{Dominika Skrzypek\affiliation{Adam Mickiewicz University, Poznań}}

\title{Indirect anaphora in a diachronic perspective: The case of Danish and Swedish}  

\abstract{In this paper, I offer a diachronic analysis of indirect anaphora\is{anaphora!indirect anaphora|(} (associative anaphora),\is{anaphora!associative anaphora} paying particular attention to the anchoring of the anaphor and the variation between definite and possessive NPs which appear in this type of bridging\is{anaphora!bridging anaphora} in Danish\il{Danish|(} and Swedish\il{Swedish|(} between 1220 and 1550. The study is based on a corpus of authentic texts evenly distributed across languages and genres. I argue that the expression of indirect anaphora is a crucial stage in the grammaticalization\is{grammaticalization} of the definite article, and that the study of the spread of the incipient definite article through this context can be described in terms of strong and weak definiteness.\is{definiteness!strong}\is{definiteness!weak}}

\shorttitlerunninghead{Indirect anaphora in a diachronic perspective}
\begin{document}
\maketitle

\section{Introductory remarks}\label{6sec:1}

Anaphora is one of the more widely studied discourse phenomena. The term itself is derived from Greek\il{Greek} (`carrying back', e.\,g., \citealt{huang:00}: 1) and is used to describe a relationship between two linguistic elements: an antecedent and an anaphor, as in the following example:

\begin{exe}
\ex\label{6ex:1}
I came into {\ul{a spacious room}}. {\emph{It}} was sparsely decorated and rather gloomy.
\end{exe}

The example given in (\ref{6ex:1}) includes what is often considered a typical antecedent (indefNP) and a typical anaphor (a pronoun). The simplicity of the example, however, is misleading, for anaphora is a complex linguistic and cognitive phenomenon, which has duly received a great deal of attention, both within linguistic paradigms and in other fields, such as (language) philosophy, psychology, cognitive science and artificial intelligence studies. Each is partly interested in different aspects of anaphora, and some studies subsume anaphora under a broader study of reference in discourse \citep[e.\,g.,][]{kibrik:11}. Anaphora is the central element of such theoretical proposals as Relevance Theory\is{Relevance Theory} \citep{sperber:wilson:12} and Centering Theory\is{Centering Theory} \citep{grosz:etal:95}. 

In historical linguistics, anaphora is singled out as the first stage of the grammaticalization\is{grammaticalization} of the definite article. What is originally a deictic element, usually a demonstrative\is{demonstratives} pronoun \citep[see][]{lyons:99}, begins to be used to point not only in a physical context, but also in text (anaphora). 

\begin{exe}
\ex\label{6ex:2}
I came into {\ul{a spacious room}}. (\ldots) \\
{\emph{The room}} was fully decorated but rather gloomy.
\end{exe}

The use of a demonstrative\is{demonstratives} to point within text involves a shift from situational to textual deixis\is{deixis} \citep{lyons:75}. As the grammaticalization\is{grammaticalization} progresses, new uses are found for the original pronoun, as it gradually transforms into a definite article \citep{mulder:carlier:11}. 

The first article-like use of the demonstrative\is{demonstratives} (i.\,e., a use in which, in article languages, the definite article would be used) is what could more precisely be termed direct anaphora.\is{anaphora!direct anaphora} In this type of reference the antecedent and the anaphor co-refer. A different type of anaphora is found in (\ref{6ex:3}).

\begin{exe}
\ex\label{6ex:3}
{\ul{My watch}} is dead. {\emph{The battery}} is flat. \citep[after][]{schwarz:00}
\end{exe}

Even though a co-referring antecedent for the battery is lacking, the NP is definite. Definite marking (such as a definite article) is normally a signal to the hearer that the referent of the definite NP (defNP) is known, identifiable or possible to locate, and here it seems to serve the same purpose. Moreover, it is clear that the two sentences in (\ref{6ex:3}) form a coherent text and the definite marking can be interpreted accordingly, in relation to another NP, namely {\emph{my watch}}. The element of the preceding discourse which makes the identification of the anaphor possible will be referred to as the anchor (after \citealt{fraurud:90}; see section \ref{6sec:2}). The relationship between {\emph{the battery}} and {\emph{my watch}} is anaphoric and the defNP {\emph{the battery}} is an anaphor, but since the two do not co-refer, I will use the term indirect anaphor to highlight the difference between this type of relation and the direct anaphora\is{anaphora!direct anaphora} described above. In the literature, this type of relation is also known as associative anaphora\is{anaphora!associative anaphora} or bridging.\is{anaphora!bridging anaphora} 

In this paper, I shall focus on this particular type of textual relation diachronically. In particular, I follow the typology of indirect anaphors in terms of their type of anchoring as presented by \cite{schwarz:00}, and address the question of the diachronic development from demonstrative\is{demonstratives} pronoun through an anaphoric marker to definite article and its relation to the proposed typology of direct and indirect anaphors. For the purpose of my study I have chosen two closely related languages, Danish and Swedish, representing the eastern branch of North Germanic.\il{Germanic languages!North Germanic} I base my study on a corpus of historical texts in each language spanning 330 years, from 1220 until 1550 (see section  \ref{6sec:3}). The corpus includes the oldest extant texts in each language in which there are only sporadic instances of the incipient definite article; by 1550 the article systems\is{article systems} of both languages have reached more or less the modern form \citep{strohwollin:16,skaftejensen:07}. I am particularly interested in how indirect anaphora is expressed throughout the time of the formation of the definite article. 

The aim of the paper is to fine-grain indirect anaphors and place them in a diachronic context of article grammaticalization.\is{grammaticalization} More specifically, I argue that not all indirect anaphors are marked as definites simultaneously, and that in this context the grammaticalizing definite article competes against two forms: bare nouns and possessives, in particular reflexive possessives. 

The paper is organized as follows: I begin by defining indirect anaphora in section \ref{6sec:2}, presenting this context in detail -- the aim of the section is to show how heterogeneous a context indirect anaphora is. In section \ref{6sec:3}, I present my sources and tagging principles, together with a brief overview of definiteness and its expressions in modern North Germanic languages.\il{Germanic languages!North Germanic} Section \ref{6sec:4} presents the results, with particular focus on the forms used as indirect anaphors and on the subtypes of these anaphors. In section \ref{6sec:5}, I discuss the possible relevance of the results for the grammaticalization\is{grammaticalization} of the definite article. I close with conclusions and ideas for further research in section \ref{6sec:6}. 


\section{Indirect anaphora}\label{6sec:2}

Indirect anaphora has been studied mainly synchronically and in the context of definiteness; it is therefore not surprising that it has been customary to focus on defNPs as indirect anaphors. The purpose of the studies has been to establish the link between the anaphor and its anchor, or to identify the anchor. This approach is not entirely fruitful in diachronic studies. In the context of article growth, there are few examples of definite articles in the oldest texts, while many NPs are used as indirect anaphors. Although it is interesting to see in what contexts the incipient definite article may be found, this does not give us a complete picture of its grammaticalization.\is{grammaticalization} 

For the purpose of a diachronic study it is more useful to consider the context itself, irrespective of the form of the indirect anaphor. Indirect anaphora is a type of bridging\is{anaphora!bridging anaphora} reference, which, following a long tradition, I take to be a relationship between two objects or events introduced in a text or by a text, a relationship that is not spelled out and yet constitutes an essential part of the content of the text, in the sense that without this information the lack of connection between the objects or events would make the text incoherent \citep{asher:lascarides:98}. This is illustrated by the following examples. 

\begin{exe}
\ex\label{6ex:4}
I met {\ul{two interesting people}} last night at a party. {\emph{The woman}} was a member of Clinton's Cabinet. 
\ex\label{6ex:5}
In the groups there was {\ul{one person}} missing. {\emph{It was Mary who left}}. 
\ex\label{6ex:6}
John {\ul{partied}} all night yesterday. He's going to get drunk {\emph{again}} today.
\ex\label{6ex:7}
Jack was going to commit {\ul{suicide}}. He got {\emph{a rope}}. 
\ex\label{6ex:8}
Jack locked himself out again. He had left {\emph{his keys}} on the kitchen table. 
\exi{}
\hspace*{-0.8cm}(examples (\ref{6ex:4})--(\ref{6ex:7}) after \citealt[][83]{asher:lascarides:98})
\end{exe}

{
It may be noted that there is a variety of expressions treated as bridging\is{anaphora!bridging anaphora} here, including, but not limited to, defNPs. In (\ref{6ex:8}), it would be possible to use a defNP instead of the possessive, and most likely it would also be possible to replace the indefNP in (\ref{6ex:7}) with a defNP `the rope'. The variation in form of indirect anaphors has not been given due attention in studies thus far, while it is of fundamental importance in a diachronic study. I wish to argue for a widening of the scope of study to include other expressions, first and foremost possessive NPs (possNPs). 
}

{
For indirect anaphors, although there is no antecedent, we are (mostly) able to identify some connected entity, event/activity or scenario/frame in the preceding discourse as serving a similar function (`my watch' for `the battery'). If nominal, the `antecedent' has been termed a {\emph{trigger}} \citep{hawkins:78} or an {\emph{anchor}} \citep{fraurud:90} for the anaphor. The two notions differ in terms of how they paint the process of referent identification. {\emph{Trigger}} implies that with its articulation a number of stereotypically connected entities are activated in the hearer's mind, from among which he/she is then free to choose when the anaphor appears. Thus:
}

\begin{exe}
\ex\label{6ex:9}
{We chose {\ul{a quiet restaurant}}. {\emph{The menus}} were modest, yet {\emph{the food}} was great.}
\end{exe}

The utterance of the indefNP `a quiet restaurant' triggers a series of connected entities, such as menus, waiters, food, other guests, cloakrooms etc. In other words, it opens up a new reference frame or reference domain (Referenz-domäne, \citealt{schwarz:00}) within which these can be found. On hearing defNPs such as `the waiter' or `the table' the hearer will automatically interpret them as belonging to the restaurant mentioned earlier (though the restaurant itself may not be a familiar one, since it is presented with an indefNP). Were the speaker to choose a referent from outside this frame and mark it as definite, the hearer would probably have more trouble interpreting it correctly:

\begin{exe}
\ex\label{6ex:10}
We chose {\ul{a quiet restaurant}}. {\emph{The hairdresser}} was rather heavy-handed and he pulled my hair with unnecessary force.
\end{exe}

And yet, it seems unlikely that on hearing the phrase `a quiet restaurant' the hearer automatically sees in his/her mind's eye a series of entities connected with it. In fact, were he/she to do so, it would be a very uneconomical procedure, since only some of the potential indirect anaphors will be used in the following discourse. For the most part, only some of the potential triggers become actual triggers, and when they do, only some of the wide range of possible indirect anaphors are used. Consider the following examples:

\begin{exe}
\ex\label{6ex:11}
	\begin{xlista}
	\ex\label{6ex:11a}
	Hanna hat Hans {\ul{erschossen}}. {\emph{Der Knall}} war bis nach Galdbach zu hören. \\
	`Hanna has shot Hans dead. The bang could be heard all the way to Galdbach.'
	\ex\label{6ex:11b}
	Hanna hat Hans {\ul{erschossen}}. {\emph{Die Wunde}} blutet furchtbar. \\
	`Hanna has shot Hans dead. The wound is bleeding awfully.'
	\ex\label{6ex:11c}
	Hanna hat Hans {\ul{erschossen}}. {\emph{Das Motiv}} war Eifersucht. \\
	`Hanna has shot Hans dead. The motive was jealousy.'
	\ex\label{6ex:11c}
	Hanna hat Hans {\ul{erschossen}}. Die Polizei fand {\emph{die Waffe}} im Küchenschrank. \\
	`Hanna has shot Hans dead. The police found the weapon in the kitchen cabinet.'
	\end{xlista}
\exi{} (\citealt[][38]{schwarz:00}; she calls the collection of entities/processes activated with the use of a trigger `konzeptueller Skopus')
\end{exe}

Another term for the antecedent-like entity in preceding discourse is {\emph{anchor}}, to my knowledge first introduced by \cite{fraurud:90}. In contrast to the term {\emph{trigger}}, it takes into account the actual anaphor and the process of accessing the referent by searching for an `anchor' in the previous discourse. This term also has the value of being equally applicable to indirect and direct anaphors (the most obvious anchor would be the co-referring entity).

The examples quoted above show how heterogeneous indirect anaphora is. There are a number of relations between the anchor and the anaphor. Authors differ in their typologies of indirect anaphors; however, all of them distinguish between at least two major types. Following \cite{schwarz:00} I will refer to the first type as semantic (based on lexical knowledge) and the second as conceptual (based on knowledge of the world). The former can be further subdivided into meronymic (part-whole relations) and lexical/thematic (other semantic roles), and the latter into scheme-based and inference-based. The types are illustrated with examples below.\\[-2mm]

\noindent
{\emph{Semantic types}} \\
a. meronymic relations

	\begin{exe}
	\ex\label{6ex:12}
	{\ul{A new book}} by Galbraith is in bookstores now. On {\emph{the cover}} there is a picture of {\emph{the author}}.
	\end{exe}

\noindent
b. lexical/thematic relations
	\begin{exe}
	\ex\label{6ex:13}
	{\ul{A new book}} on climate change is in bookstores now. {\emph{The author}} claims that mankind has only twenty years in which to make changes.
	\end{exe}

\noindent
{\emph{Conceptual types}} \\
a. scheme-based
	\begin{exe}
	\ex\label{6ex:14}
	{\ul{A charge}} of negligent homicide against Daw Bauk Ja could be withdrawn at the request of the {\emph{plaintiff}}.
	\end{exe}

\noindent
b. inference-based 
	\begin{exe}
	\ex\label{6ex:15}
	Wussten Sie […] dass der Schrei in Hitchcocks „Psycho`` deshalb so echt wirkt, weil der Regisseur genau in dem Moment der Aufnahme eiskaltes Wasser durch {\emph{die Leitung}} pumpen ließ? \\
	Did you know (...) that the scream in Hitchcock's {\emph{Psycho}} seems so real because at the moment of filming the director let cold water to be pumped through the pipe?\\[1mm]
	(\citealt[][102]{consten:04}; own translation)
	\end{exe}

To successfully interpret an anaphor of the conceptual type, a degree of knowledge of the world is necessary. The interpretation of the defNP {\emph{die Leitung}} `the pipe' relies on familiarity\is{familiarity} with the Hitchcock film and the fact that the famous scene with the scream takes place in a shower. 

There are a number of other typologies of indirect anaphors (notably \citealt{irmer:11}; see also \citealt{zhao:14} for an overview of studies of indirect anaphora), though most make similar divisions. I follow M. Schwarz's (\citeyear{schwarz:00}) typology, since unlike the majority of other studies it is grounded in authentic texts and not constructed examples, and therefore seems best suited for a study of authentic examples, which is the subject of this paper. It should be noted, however, as Schwarz herself frequently does, that when studying authentic texts one is often forced to classify examples that may fit more than one category, depending on what seems to be the anchor or what type of relation between the anchor and the anaphor is identified. It is also possible that in authentic texts the anaphor is accessible through more than one anchor.

Finally, a note on the form of the indirect anaphor is necessary here. Traditionally, the point of departure for all classifications has been defNPs without a co-referring antecedent. The aim of studies has been to explain their definiteness in the absence of an antecedent. However, in recent years, when the concept of bridging\is{anaphora!bridging anaphora} has become more established, more and more authors have appreciated that bridging\is{anaphora!bridging anaphora} can also occur in the absence of definites \citep[][107]{asher:lascarides:98}. In his discussion of totality (exhaustivity,\is{exhaustivity} completeness), \cite{hawkins:78} shows that the definite can only occur in bridging\is{anaphora!bridging anaphora} when it refers uniquely, e.\,g., {\emph{car -- the engine}} but {\emph{car -- a tyre}}, yet the underlying relationship between engine and car seems to be the same as that between tyre and car. It has also been demonstrated that possessives may introduce new, anchored referents \citep{willemse:etal:09}. Those authors found that in a considerable number of cases PM (= possessum) referents of possessive NPs are first mentions with inferential relations to the context \citep[][24]{willemse:etal:09}. In the following, I will concentrate on the context itself and study the variety of forms found in it in historical Danish and Swedish texts. 


\section{Sources and tagging}\label{6sec:3}

The corpus used in this study consists of 29 texts in Danish and Swedish, written between 1220 and 1550, in three genres representative of the period studied: legal, religious and profane prose. From each text I chose passages with ca. 150 NPs in each (if the text was long enough), preferably high narrativity passages. The texts were divided into three periods: Period I (1220--1350), Period II (1350--1450) and Period III (1450--1550). The proposed periodization has been used in previous studies of article grammaticalization\is{grammaticalization} and other diachronic studies of Swedish \citep{delsing:12}. A total of 5822 NPs (nominal NPs only) were tagged and analyzed. The tool used for tagging and generating statistics is called DiaDef (see Figure \ref{6fig:1} below), and was tailor-made for the project. It enables us to tag each NP for all data we assume to be in some way relevant for the choice of article, such as function in sentence (subject, object, etc.), referential status\is{referential status} (new, unique, generic,\is{generics} anaphoric, etc.) and other information (case, number, gender, animacy,\is{animacy} countability, etc.). 

The languages considered are both North Germanic languages\il{Germanic languages!North Germanic} of the eastern variety. The extant texts consist of Runic inscriptions from ca. 200 AD onwards; the oldest extant Danish and Swedish texts written in the Latin alphabet are legal texts from ca. 1220. For this project I look at texts from 1220 to 1550, which is a time of radical change in the grammars of both languages, including loss of case and the emergence of (in)definiteness. 

\begin{table}[H]
\centering
\begin{tabular}{llll}
\lsptoprule
  &   & number of & number of  \\
{\bf{language}} & {\bf{period}} & texts {\bf{tagged}} & nominal NPs {\bf{extracted}} \\
\midrule
Danish 	& Period I (1200–1350)	& 7	& 1097 \\ 
		& Period II (1350–1450)	& 5	& 1016 \\ 
		& Period III (1450–1550)	& 4	& 787 \\[2mm] 
Swedish	& Period I (1200–1350)	& 5	& 1194 \\ 
		& Period II (1350–1450)	& 5	& 1093 \\ 
		& Period III (1450–1550)	& 3	& 635 \\[2mm] 
\midrule
		& 				& 29	& 5822 \\
\lspbottomrule
\end{tabular}
\caption{An overview of the sources}\label{6table:1}
\end{table}

A detailed list of quoted source texts can be found in the Sources. When quoting examples from the corpus I note the language (DA for Danish and SW for Swedish), the source text (e.\,g., SVT for {\emph{Sju vise mästare}}; the abbreviations are also given in the Sources) and the date of its composition. 

A note on the definite article in North Germanic\il{Germanic languages!North Germanic} is necessary here. The definite article is a suffix that is always attached to the noun (in the Insular Scandinavian languages\il{Scandinavian languages} Icelandic\il{Icelandic} and Faroese,\il{Faroese} to the case-inflected form of the noun). Its origins are to be found in the distal demonstrative\is{demonstratives} {\emph{hinn}} `yon' \citep[e.\,g.,][]{perridon:89}. Apart from the suffixed article, there are other exponents of definiteness, i.\,e., the weak form of the adjective (in the continental languages Danish, Swedish and Norwegian\il{Norwegian} and in Faroese\il{Faroese} merely an agreement phenomenon, in Icelandic\il{Icelandic} possibly retaining an original meaning of definiteness; see \citealt{naert:69}) and a preposed determiner, originally a demonstrative\is{demonstratives} {\emph{sá}} (in younger texts {\emph{den}}) `this'. Both the suffixed article and the preposed determiner can be combined within one NP in Swedish, Norwegian\il{Norwegian} and Faroese\il{Faroese} (so-called double definiteness) but are exclusive in Danish and Icelandic.\il{Icelandic} The variety of NPs is illustrated below using the example of the noun `house' (neuter in all languages) in the singular. 

\begin{exe}
\ex\label{6ex:16}
\begin{tabbing}
hus-{\bf{et}}\hspace*{5.2cm} \= (Danish, Norwegian,\il{Norwegian} Swedish) \\
hús-ið \> (Faroese,\il{Faroese} Icelandic)\il{Icelandic} \\
house-{\sc{def}} 
\end{tabbing}

\ex\label{6ex:17}
\begin{tabbing}
{\bf{det}}\hspace*{1.3cm} \= 	stor{\bf{e}}\hspace*{1.3cm} \= huset\hspace*{1.3cm} \= (Norwegian) \\
{\bf{det}}  \> stor{\bf{a}} \> huset \> (Swedish) \\
{\bf{det}} \> stor{\bf{e}} \> hus \> (Danish) \\
--- \> stóra \> hús-ið \> (Icelandic) \\
hið/tað \> stóra \> hús-ið \> (Faroese) \\
{\sc{def}} \> large-{\sc{def}} \> house-{\sc{def}} 
\end{tabbing}

\ex\label{6ex:18}
\begin{tabbing}
et\hspace*{1.52cm} \= hus\hspace*{3.8cm} \= (Danish, Norwegian) \\
ett \> hus \> (Swedish) \\
eitt \> hús \> (Faroese) \\
--- \> hús \> (Icelandic) \\
{\sc{indef}} \> house
\end{tabbing}
\end{exe}

For excerption, I define bridging\is{anaphora!bridging anaphora} as widely as possible. Direct anaphora\is{anaphora!direct anaphora} (co-reference) is tagged as DIR-A, uniques as U, generics\is{generics} as G, new discourse referents as NEW (when there is no connection to previous discourse whatsoever), and non-referential uses as NON-REF. For all other types of reference I use the tag INDIR-A.

\begin{figure}[H]
\centering
\includegraphics[width=12.6cm]{chapters/skrzypek-fig1.png}
\caption{DiaDef print screen}\label{6fig:1}
\end{figure}

The DiaDef program allows us to excerpt all NPs tagged as INDIR-A and sort them, according to the form of the NP, into: BN (bare noun), -IN (incipient definite article), POSS (possessive), DEN (demonstrative\is{demonstratives} {\emph{den}} `this'), DEM (other demonstrative\is{demonstratives} elements) and EN (incipient indefinite article). For the purpose of the present study the possessives are further subdivided into POSS-GEN (genitive, e.\,g., {\emph{Jans}} `Jan-{\sc{gen}}'), POSS-PRO (possessive pronoun, e.\,g., {\emph{hans}} `his') and POSS-REFL (reflexive possessive pronoun, e.\,g., {\emph{sin}} `his-{\sc{refl}}').

{
I did not expect to find large discrepancies between texts in different languages and from different periods with respect to the number of indirect anaphors in each. NPs tagged as indirect anaphors constitute ca. 25\% of all NPs in the material (Table \ref{6table:2}), with only slight variation between languages and periods. This confirms an intuitive expectation that this type of textual relation does not depend on the period. It may depend on the genre chosen; I have therefore concentrated on choosing passages of high narrativity\footnote{Old Danish and Old Swedish texts include a number of passages that can best be termed case studies, leading to the establishment of a precedent. These usually tell a short story with a number of discourse referents. I chose passages of this type over mere formulations of legal rules whenever possible.} from each genre, including legal prose. 
}

\begin{table}[H]
\begin{tabular}{m{70pt}m{70pt}m{70pt}}
\lsptoprule
{\textbf{period}} & {\textbf{Danish}} & {\textbf{Swedish}}\\ 
\midrule
1200--1350	& 24.52\%		& 25.80\% \\ 
1350--1450	& 23.72\%		& 19.67\% \\ 
1450--1550	& 29.61\%		& 23.62\% \\ 
average		& 25.95\%		& 23.03\% \\ 
\lspbottomrule
\end{tabular}
\caption{Percentage of indirect anaphors in the corpus}\label{6table:2}
\end{table}

\filbreak
\section{Results}\label{6sec:4}

I sorted all indirect anaphors according to the form of the NP. Table \ref{6table:3} below presents an overview of the results for each language and period. 

\begin{table}[H]
\centering
\begin{tabular}{m{55pt}m{55pt}m{55pt}m{55pt}m{55pt}}
\lsptoprule
  & {\textbf{NP form}} & {\textbf{Period I}} & {\textbf{Period II}} & {\textbf{Period III}} \\
  &   & 1200--1350 & 1350--1450 & 1450--1550 \\
\midrule
{\textbf{Danish}} & BN		& 21.85\%		& 5.39\%		& 4.72\%	\\ 
 & POSS-refl	& 10.74\%		& 12.45\%		& 12.88\% \\ 
 & POSS-pro	& 26.30\%		& 36.93\%		& 24.46\% \\ 
 & POSS-gen	& 9.26\%		& 12.03\%		& 21.89\% \\ 
 & -IN		& 7.04\%		& 10.79\%		& 18.88\% \\ 
 & DEM		& 2.22\%		& 1.66\%		& 1.72\% \\ 
 & DEN		& 5.93\%		& 6.64\%		& 4.72\% \\ 
 & EN		& 0.74\%		& 4.15\%		& 0.86\% \\ 
 \cmidrule{3-5}
 & 			& 84.08\% 	& 90.04\%		& 90.13\% \\ 
\midrule
{\textbf{Swedish}} & BN & 36.04\%	& 9.30\%	& 8.00\% \\ 
 & POSS-refl	& 4.22\%		& 20.93\%		& 15.33\% \\ 
 & POSS-pro	& 10.06\%		& 22.79\%		& 22.67\% \\ 
 & POSS-gen	& 20.78\%		& 10.23\%		& 20.67\% \\ 
 & -IN		& 8.77\%		& 17.67\%		& 11.33\% \\ 
 & DEM		& 3.90\%		& 5.58\%		& 6.67\% \\ 
 & DEN		& 4.22\%		& 4.65\%		& 6.67\% \\ 
 & EN		& 2.27\%		& 4.65\%		& 0.67\% \\ 
  \cmidrule{3-5}
 & 			& 90.26\%	 	& 86.50\% 	& 92.01\% \\
\lspbottomrule
\end{tabular}
\caption{Indirect anaphors in Old Danish and Old Swedish according to form}\label{6table:3}
\end{table}

First, a comment on the presentation of the results is necessary. I give percentages for each NP form used in an indirect anaphoric context; e.\,g., of all NPs tagged as INDIR-A in Swedish Period I, 36.04\% were BNs. As can be seen from the totals (shown in italics), the forms I chose for the study cover the majority of indirect anaphors, but not all. There are other types of NPs that can be found in the material, including nouns with adjectival modifiers (adjectives in the weak or strong form) but without any other determiners. However, their frequencies were low enough for them not to be reported. 

The general results show the expected patterns -- a decreasing frequency of BNs in bridging\is{anaphora!bridging anaphora} reference together with a rising frequency of -IN, the incipient definite article. The high frequencies of BNs in Period I are to be expected, since in both languages the process of article grammaticalization\is{grammaticalization} most likely began some time before the oldest texts were written (see \citealt{skrzypek:12}: 74 for an overview of proposed dating by different authors). The period 1220--1550 is the time when the definite article grammaticalizes in both languages. In many contexts, indirect anaphora being one of them, it comes to be used instead of BNs. We can further see that other NP types are on the rise in both languages, most notably possNPs (with reflexive possessive in Swedish and pronominal, non-reflexive possessive in Danish), not only the incipient definite article. PossNPs are the strongest competitor to defNPs in the material studied. 

The results reported in Table \ref{6table:3} above show indirect anaphora without subdividing the context into semantic and conceptual anaphors (see section \ref{6sec:2}). They show that the context is by no means exclusively expressed by defNPs, and that possNPs in particular show high frequencies.

They also show that the major change taking place between Period I and Period II is the reduction of zero determination. In the material chosen, no BNs were found in anaphoric uses of NPs (they were still found with uniques and generics;\is{generics} see also \citealt{skrzypek:12}), but since the definite article is not yet fully grammaticalized it is not the default option for determination. Speakers therefore make use of other elements, most notably different types of possessives. 

In the following part of the paper I will focus on the variation between defNPs and possNPs in indirect anaphora.


\subsection{Semantic indirect anaphora -- mereological relations\is{mereological relations}}\label{6sec:41}

Although it may seem that I have already fine-grained the concept of indirect anaphora, the first subtype, mereological relations,\is{mereological relations} is by no means homogeneous. Within it we find such different relations between anchor and anaphor as object -- material ({\emph{bicycle -- the steel}}), object -- component ({\emph{joke -- the punchline}}), collective -- member ({\emph{deck -- the card}}), mass -- portion ({\emph{pie -- the slice}}), etc. There are a number of examples of mereological relations\is{mereological relations} found in the material. With limited material at my disposal, I was not able to find examples of each type of mereological relation\is{mereological relations} in the Danish and Swedish texts to enable a systematic study of all sub-types for all periods in both languages. Very well represented are examples of inalienable possession, i.\,e., body parts, items of clothing or weaponry. 

\filbreak
The NPs found in semantic indirect anaphora include BNs, possNPs and defNPs, although in Period I inalienables seem to be found only as BNs or possNPs and not as defNPs. 

\begin{exe}
\begin{samepage}
\ex\label{6ex:19}
(DA\_VL 1300)
\exi{}
\gll Æn of swa {worthær} at man {mistær} allæ sinæ {tændær} af {\emph{sin}} {\emph{høs}}.\\
%
and if so be {that} man loses all his teeth {from} {his.{\sc{refl}}} head \\
%
\glt `If it should happen that a man loses all his teeth.'
\end{samepage}
%%%%%%%%
\ex\label{6ex:20}
(DA\_Mar 1325)
\exi{}
\gll iak kom þa fuul sørhilika til miin kæra sun ok þahar iak sa hanum slaa-s mæþ næua (...) ok spytta-s i {\emph{anlæt}} ok krona-s mæþ þorna. \\
%
I came then fully sorrowful to my dear son and when I saw him beat-{\sc{pass}} with fists (...) and spit-{\sc{pass}} in face and crown-{\sc{pass}} with thorns \\
%
\glt `I came full of sorrow to my dear son and as I saw he was beaten with fists and spat in the face and crowned with thorns.' 
%%%%%%%%
\ex\label{6ex:21}
(SW\_Bur 1330)
\exi{}
\gll at hon varþ hauande mæþ guz son ii {\emph{sino}} {\emph{liue}}\\
that she became pregnant with God.{\sc{gen}} son in her.{\sc{refl}} womb\\
\glt `that she carried God's son in her womb'
%%%%%%%%
\ex\label{6ex:22}
(SW\_AVL 1225)
\exi{}
\gll Uærþær maþer dræpin (...) þa skal {\emph{uighi}} a þingi lysæ. \\
be man killed (...) then shall murder on ting declare \\
\glt `If a man is killed then the murder shall be made public on a ting.' \\
\end{exe}

{
In Period II, inalienables no longer appear as BNs, but either with a (reflexive) possessive pronoun or the incipient definite article. It should be noted here that North Germanic languages\il{Germanic languages!North Germanic} have retained two possessive pronouns: the regular possessive, corresponding to the English {\emph{his/her/its}}, and the reflexive possessive, {\emph{sin/sitt}}, which is used when the possessor is the subject of the clause. The default marking of inalienables in Period II seems to be the possessive, and the incipient definite article is at first only found with inalienables in direct anaphora\is{anaphora!direct anaphora} (i.\,e., such body parts or items of clothing that are not only connected with an owner known from previous discourse, but have also been mentioned themselves).
}

\begin{exe}
\begin{samepage}
\ex\label{6ex:23}
(SW\_Jart 1385)
\exi{}
\gll {\ul{Kwinna-n}} gik bort ok faldadhe han j {\emph{sinom}} {\emph{hwiff}} som {\ul{hon}} hafdhe a {\emph{sino}} {\emph{hofdhe}}. \\
woman-{\sc{def}} went away and folded him in her scarf which she had on her head \\
\glt `The woman went away and folded him in her scarf which she had on her head.'
\end{samepage}
%%%%%%%%
\begin{samepage}
\ex\label{6ex:24}
(SW\_Jart 1385)
\exi{}
\gll Tha syntis quinno-n-na hwifwir allir blodhoghir ok water aff blodh swa at {\emph{blodh-in}} flöt nidhir vm quinno-n-na kindir. Hulkit herra-n saa, ropadhe ok sagdhe hwar slo thik j thit änlite älla sarghadhe. Ok quinnna-n lypte vp {\emph{sina}} {\emph{hand}} ok strök sik vm {\emph{änliti-t}} ok tha hon tok nidhir {\emph{hand-in-a}} tha war hon al blodhogh. \\
%
then was-seen woman-{\sc{gen-def}} scarf all bloody and wet of blood so that blood-{\sc{def}} flew down about woman-{\sc{gen-def}} cheeks which master-{\sc{def}} saw screamed and said who hit you in your face or hurt and woman-{\sc{def}} lifted up her hand and stroked herself about face-{\sc{def}} and when she took down hand-{\sc{acc-def}} then was {she (=\,the hand)} all bloody \\
%s
\glt 
{`Then the woman's scarf seemed all bloodied and wet with blood so that the blood flew down the woman's cheeks. Which the master saw, screamed and said ``Who hit you in your face or hurt (you)?''. And the woman lifted her hand and stroked her face and when she took the hand away it was all bloodied.'}
\end{samepage}
\end{exe}

Example (\ref{6ex:24}) illustrates well the division of labour between the (reflexive) possessive and the incipient definite article. The possessive is used if the inalienable is mentioned for the first time (indirect anaphora). The definite article is used only in further mentions, i.\,e., in direct anaphora\is{anaphora!direct anaphora} (thus {\emph{your face -- the face, her hand -- the hand}}). Naturally, we could simply treat such examples as direct anaphors. However, it is clear that they are both co-referring with an antecedent and accessible via their anchors. It seems that this double identity, as direct and indirect anaphors, constitutes a bridging\is{anaphora!bridging anaphora} context (in the sense of \citealt{heine:02}) for defNPs to spread to indirect anaphora with meronyms.\is{meronyms} By the end of Period II and the beginning of Period III the definite article starts being used also in indirect anaphora (first mention of an inalienable possessum connected with a known discourse referent), as shown in (\ref{6ex:25}) and (\ref{6ex:26}).

\begin{exe}
\ex\label{6ex:25}
(SW\_ST 1420)
\exi{}
\gll Tha bar {\ul{keysari-n}} vp {\emph{hand-ena}} oc slogh hona widh {\emph{kinben-it}} at hon størte til iordh-inna. \\
Then bore emperor-{\sc{def}} up hand-{\sc{def}} and hit her at cheekbone-{\sc{def}} that she fell to earth-{\sc{def}} \\
\glt `Then the emperor lifted his hand and hit her on the cheekbone so that she fell down.'
\ex\label{6ex:26}
(DA\_Jer 1480) 
\exi{}
\gll Tha begynthe løffwe-n som hwn war wan gladeligh at løpe i clostereth (...) eller rørdhe {\emph{stiærth-en}}. \\
then began lion-{\sc{def}} as she was accustomed gladly to run in monastery-{\sc{def}} (…) or wagged tail-{\sc{def}}  \\
\glt `Then the lion began, as she was accustomed to, to gladly run in the monastery (…) or wagged her tail.'
\end{exe}

It should be noted that BNs are found in indirect anaphora even in Period III; however, as illustrated in examples (\ref{6ex:27}) and (\ref{6ex:28}), these occurrences may be lexicalizations rather than indirect anaphors. 

\begin{exe}
\ex\label{6ex:27}
(DA\_KM 1480)
\exi{}
\gll Jamwnd-z {\ul{hoffui-t}} bløde bodhe giømmen {\emph{mwn}} ok {\emph{øren}}.  \\
Jamund-{\sc{gen}} head-{\sc{def}} bled both through mouth and ears \\
\glt `Jamund's head bled through both mouth and ears.' 
\ex\label{6ex:28}
(DA\_Kat 1480)
\exi{}
\gll badh meth {\emph{mwndh}} {\emph{oc}} {\emph{hiærthe}}. \\
prayed with mouth and heart \\
\glt `(She) prayed with mouth and heart.' 
\end{exe}

\subsection{Semantic, lexical/thematic}\label{6sec:42}

The lexical/thematic type is based on our lexical knowledge of certain elements forming more or less stereotypical events or processes, e.\,g., a court case involves a judge, one or more hearings, a charge, a plaintiff and so on. In Period I we find mostly BNs in this type of indirect anaphora (example (\ref{6ex:29})), but a few instances of the incipient definite article have been found as well (example (\ref{6ex:30})). 

\filbreak
\begin{exe}
\ex\label{6ex:29}
(SW\_AVL 1225)
\exi{}
\gll Sitær konæ i bo dör {\emph{bonde}}. \\
sits wife in house dies husband \\
\glt `If a wife is alive and the husband dies.' 
\ex\label{6ex:30}
(SW\_OgL 1280)
\exi{}
\gll Nu dræpær maþ-ær man koma til arua man-zs-in-s ok fa {\emph{drapar-a-n}} ok hugga þær niþær a fötær þæs döþ-a. \\
now kills man-{\sc{nom}} man.{\sc{acc}} come to heir man-{\sc{gen-def-gen}} and get killer-{\sc{acc-def}} and cut there down on feet this.{\sc{gen}} dead-{\sc{gen}} \\
\glt `If a man kills another, comes to the man's heir and gets the killer and cuts (him) down at the feet of the deceased.'
\end{exe}

This context allows defNPs as early as Period I. I have not found possNPs in this type of indirect anaphora. In Period II the lexical type is regularly found with defNPs, in pairs such as {\emph{tjuven}} `the thief' -- {\emph{stölden}} `the larceny', {\emph{wighia}} `ordain' -- {\emph{vixlenne}} `the ordination', {\emph{henger}} `hangs' -- {\emph{galghan}} `the gallows', {\emph{rida}} `ride' -- {\emph{hästen}} `the horse', {\emph{fördes död}} `a dead (man) was carried' -- {\emph{baren}} `the stretcher'. Typical for this type of indirect anaphora is that the anchor need not be nominal and the anaphor may be accessible through a VP.


\subsection{Conceptual scheme-based anaphors}\label{6sec:43}

The conceptual types of indirect anaphora are resolved not (only) through lexical knowledge but rather through familiarity\is{familiarity} with stereotypical relations between objects or events and objects. The NPs found in this type are either BNs (in Period I) or defNPs. PossNPs, on the other hand, are seldom found in this type at all, irrespective of the period. I have located some examples of possNPs that may be considered indirect anaphors; it should be noted that they, such as example (31), sound natural with a reflexive possessive in Modern Swedish as well and the choice between defNP and possNP may be a question of stylistics rather than grammatical correctness.

\begin{samepage}
\begin{exe}
\ex\label{6ex:31}
(SW\_HML 1385)
\exi{}
\gll Diäfwl-en saa hans dirue oc reede hanom snaru. (...) Oc baþ munk-in sik inläta i {\emph{sin}} {\emph{cella}}. \\
devil-{\sc{def}} saw his courage and prepared him trap (…) and asked monk-{\sc{def}} himself allow in his.{\sc{refl}} cell \\
\glt `And he (the devil) asked the monk to let him in his (= the monk's) cell.'
\ex\label{6ex:32}
(DA\_Kat 1488)
\exi{}
\gll Ther sancta katherina thette fornam tha luckthe hwn sik hardeligh i {\emph{syn}} {\emph{cellæ}} och badh jnderligh till gudh. \\
when saint Catherine this understood then locked she herself firmly in her.{\sc{refl}} cell and prayed passionately to God \\
\glt `When Saint Catherine understood this, she locked herself away in her cell and prayed passionately to God.'
\end{exe}
\end{samepage}

However, the most commonly found NP forms in this type of indirect anaphora are either BNs (in Period I) or defNPs (sporadically in Period I, regularly in Period II and Period III), such as {\emph{tjuvnad}} `larceny'\,--\,{\emph{malseghanden}} `the plaintiff' (larceny is prosecuted, somebody sues, this person is called a plaintiff), {\emph{skuld krava}} `debt demand'\,--\,{\emph{guldit}} `the gold' (the debt is to be paid, it is possible to pay it in gold).

\subsection{Conceptual inference-based}\label{6sec:44}

This type of indirect anaphora is the least accessible. To correctly identify the referent, the hearer must not only consider the textual information or stereotypical knowledge of the world, but also make inferences allowing him/her to resolve the anaphor. It should be noted that some authors do not consider this type anaphoric at all, e.\,g., \cite{irmer:11}. 

In the corpus, this type is expressed either by BNs or by defNPs. No possNPs were found here. An interesting fact, however, is that defNPs may be found as early as Period I. 

\begin{exe}
\ex\label{6ex:33}
(SW\_AVL 1225)
\exi{}
\gll Maþær far sær aþalkono gætær uiþ barn dör sv fær aþra gætær viþ barn far hina þriðiu þör bonde þa konæ er livændi þa skal af takæ hemfylgh sinæ alt þet ær vnöt ær hun ællær hænær börn þa skal hin ælsti koldær boskipti kræfiæ takær af þriþiung af {\emph{bo-n-o}}. \\
man gets himself wife begets by child dies this gets another begets by child gets that third dies peasant than woman is alive than shall of take dowry her all that which unused is she or her children than shall that oldest brood division demand take of third-part of estate-{\sc{dat-def}} \\
\glt `If a man marries a woman and has a child with her, after her death marries again and fathers a child and marries for the third time and dies, leaving the widow, she or her children should retrieve her dowry --all of it that is unspoilt-- then the children of the first marriage demand a part in the estate and should be awarded a third of it.'
\ex\label{6ex:34}
(SW\_Jart 1385)
\exi{}
\gll Nu j the stund-in-ne for ther fram vm en prästir mz gud-z likama til en siukan man ok klokka-n ringde for gud-z likama. \\
now in this hour-{\sc{dat-def}} travelled there forward about a priest with God-{\sc{gen}} body to a sick man and bell-{\sc{def}} rang for God-{\sc{gen}} body \\
\glt `At this hour a priest was travelling to a sick man, carrying the wafer and the bell rang to announce him.'
\end{exe}

I have not found a single example of indirect anaphora that could be classified as conceptual inference-based which would be expressed by a possNP. In this type of anaphora defNPs occur early -- they are found, though only sporadically, at the beginning of Period I (while the meronymic type is not expressed with defNPs until the end of Period II). To begin with, however, BNs are prevalent. Gradually, they are suppressed by defNPs, without going through the possNP phase which the meronymic types seem to have done. This type of indirect anaphora may be seen as the one reserved for the definite article, since no other element, possessive or demonstrative,\is{demonstratives} can appear here. 

\begin{table}[H]
\centering
\begin{tabular}{lcccc}
\lsptoprule
					& {\bf{BN}}	& {\bf{POSS-REFL}}	& {\bf{POSS-PRO}}	& {\bf{-IN}} \\ 
\midrule
Period I (1220–1350)	& +			& +				& +				& \minus \\ 
Period II (1350–1450)	& \minus / (+)	& +				& +				& (+) \\ 
Period III (1450–1550)	& \minus / (+)	& +				& +				& + \\ 
\lspbottomrule
\end{tabular}
\caption{NP forms of indirect anaphora in Old Danish and Old Swedish}\label{6table:4}
\end{table}


\section{Discussion: indirect anaphora and grammaticalization\is{grammaticalization} of the definite article}\label{6sec:5}

The grammaticalization\is{grammaticalization} of the definite article is a relatively well-studied development, yet a number of questions remain unresolved. The first models proposed in the literature show the path from (distal) demonstrative\is{demonstratives} to definite article in one step \citep{greenberg:78} or focus on the first stage of development, i.\,e., textual deixis\is{deixis} and direct anaphora\is{anaphora!direct anaphora} \citep[J.][]{lyons:75}. \cite{diessel:99} sees definite articles as derived from adnominal anaphoric demonstratives,\is{demonstratives} while C. \cite{lyons:99} argues that the origins of the definite are to be found in exophoric use (when the referent is present and accessible in the physical context) and in anaphoric use (when the referent is also easily accessible, though through discourse rather than the physical situation). Common to J. \cite{lyons:75}, \cite{diessel:99} and C. \cite{lyons:99} is the focus on the initial stages of grammaticalization\is{grammaticalization} as the shift from demonstrative\is{demonstratives} to definite article. However, none of these proposals account for the fact that what truly distinguishes a definite article from a demonstrative\is{demonstratives} is the possibility of being used in indirect rather than direct anaphora,\is{anaphora!direct anaphora} a context where the use of demonstratives\is{demonstratives} is allowed only marginally, if at all (see \citealt{charolles:99} for a discussion of demonstrative\is{demonstratives} use in indirect anaphora). Demonstratives\is{demonstratives} may, on the other hand, be used in direct anaphora\is{anaphora!direct anaphora} without exhibiting any other properties of or grammaticalizing into definite articles. It seems therefore that the critical shift from a demonstrative\is{demonstratives} to a definite article takes place where the demonstrative/incipient article appears in indirect anaphora \citep[see also][]{mulder:carlier:11,skrzypek:12}. 

\begin{exe}
\ex
{demonstrative\is{demonstratives} $\rightarrow$ direct anaphora\is{anaphora!direct anaphora} $\rightarrow$ {\bf{indirect anaphora}} $\rightarrow$ unique  ($\rightarrow$ generic)\is{generics}}
\end{exe}


What remains unclear is both the course of the development from direct to indirect anaphora and the course {\emph{through}} indirect anaphora (which is not a homogeneous context, as demonstrated above). Also, the variation between definite article and other elements such as possessive pronouns and incipient indefinite article has not been given enough attention.

Recently, \cite{carlier:simonenko:16} have proposed that the development of the definite article in French\il{French} proceeds from strong to weak definiteness,\is{definiteness!strong}\is{definiteness!weak} with the strong-weak dichotomy, as proposed by \cite{schwarz:09}, basically corresponding to the long-debated origins of definite meaning in either familiarity\is{familiarity} (strong definiteness)\is{definiteness!strong} or uniqueness\is{uniqueness} (weak definiteness).\is{definiteness!weak} Based on diachronic data from Latin\il{Latin} and French,\il{French} Carlier and Simonenko suggest that the developments may be partly independent and that the weak and strong patterns unite in a single definite article with time. They note that in Classical Latin\il{Latin} direct anaphoric relations are increasingly marked by demonstratives,\is{demonstratives} among them the incipient definite article {\emph{ille}}, yet the indirect anaphoric relations remain unmarked in both Classical and Late Latin\il{Latin} and are marked with the l-article first in Old French.\il{French} As Carlier and Simonenko claim, the original semantics of the l-articles involved an identity relation with a context-given antecedent (strong definiteness).\is{definiteness!strong} With time, an alternative definite semantics emerged, involving a presupposition of uniqueness\is{uniqueness} rather than an identity relation (weak definiteness).\is{definiteness!weak} 

These two types of definiteness may be expressed by different definite articles, as has been noted for some German dialects (Austro-Bavarian German)\il{German!Austro-Bavarian} and North Frisian\il{Frisian} \citep{ebert:71}, or they may correspond to different behaviours of the one definite article, as in Standard German\il{German} \citep{schwarz:09}. 

In a diachronic context, the division into strong and weak definiteness\is{definiteness!weak} leaves indirect anaphora neither here nor there. Its resolution depends on textual anchoring (familiarity);\is{familiarity} however, it also depends on the uniqueness\is{uniqueness} presupposition. Consider examples (\ref{6ex:35}) and (\ref{6ex:36}).

\begin{exe}
\ex\label{6ex:35}
I took a taxi to the airport. The driver was a friendly elderly man.
\ex\label{6ex:36}
He drove to the meeting but arrived late due to a problem with a tyre.
\end{exe}

The use of the defNP {\emph{the driver}} is based on both familiarity\is{familiarity} (with the vehicle mentioned earlier) and uniqueness\is{uniqueness} (there only being one driver per car). The use of the indefNP {\emph{a tyre}} is motivated by there being more than one in the given context, the anchor being the verb {\emph{drove}} suggesting a vehicle, of which a tyre (the faulty tyre in this case) is a part (making the driver late). There is familiarity\is{familiarity} (we assume the existence of a vehicle) but no uniqueness.\is{uniqueness} 
It is therefore not easy to place indirect anaphora in the strong-weak definiteness\is{definiteness!strong}\is{definiteness!weak} dichotomy. It may be that some types of indirect anaphora show more similarities with strong definites\is{definites!strong definites} while others have a closer affinity with weak definites.

This would explain the relative discrepancy between inalienables and other types of indirect anaphora. The inalienable relationship between the anchor and the anaphor is based on familiarity\is{familiarity} (the anaphor being a part of the anchor) but not necessarily uniqueness.\is{uniqueness} In this textual relation it is possible (and in most contexts most natural) to use the defNP {\emph{benet}} `the leg' referring to either of the two legs, just as it is to say {\emph{fickan}} `the pocket' irrespective of how many pockets there are in the outfit worn. 


\section{Conclusions}\label{6sec:6}

The model of the grammaticalization\is{grammaticalization} of definiteness is imperfect, as is our understanding of the category itself. It is a recurring problem in many linguistic descriptions that definites are defined mainly as text-deictic (this also applies to grammars of article-languages), whereas corpus studies show that this is not the (whole) case. While an extended deixis\is{deixis} in the form of direct anaphora\is{anaphora!direct anaphora} is understandable, it is by no means certain that it is the original function of the article. Also, it is present in many languages that cannot be claimed to have definite articles, like the Slavic languages,\il{Slavic languages} and has not led (yet?) to the formation of a definite article. Perhaps the origins of the article are to be sought among the bridging\is{anaphora!bridging anaphora} uses, including in their widest sense (conceptual inferential). 

The results of my study show that indirect anaphora is a heterogeneous context and that the incipient definite article does not spread through it uniformly in Danish\il{Danish|)} and Swedish.\il{Swedish|)} It appears relatively early in semantic lexical types ({\emph{a book -- the author}}) and in conceptual types; in these contexts its main competitor is the original BNs. However, it is late in appearing in semantic meronymic types, in particular those involving inalienable possession. In this context there is strong competition from the reflexive possessive pronouns. 

As indirect anaphora\is{anaphora!indirect anaphora|)} constitutes a crucial element of the grammaticalization\is{grammaticalization} of the definite article, it should be addressed in any account of the development of that article.

\section*{Acknowledgments}
The research presented in this paper was financed by a research grant from the Polish National Science Centre (NCN) `Diachrony of definiteness in Scandinavian languages' number 2015/19/B/HS2/00143. The author gratefully acknowledges this support.

\section*{Sources}
Danish \\
DA\_VL = Valdemars lov, ca. 1300. Source: {\small\texttt{http://middelaldertekster.dk}} \\ 
DA\_Jer = Af Jeronimi levned, ca. 1488, Source: Gammeldansk læsebog, 341-345 \\
{DA\_Kat = Af Katherine legende, ca. 1488. Source: Gammeldansk læsebog, 346-347} \\
DA\_KM = Karl Magnus Krønike, ca. 1480. Source: Poul Lindegård Hjorth, Udg. for Universitets-Jubilæets Danske Samfund, J. H. Schultz \\
DA\_Mar = Mariaklagen efter et runeskrevet Haandskrift-Fragment i Stockholms Kgl. Bibliotek. J. Brøndum-Nielsen \& A. Rohmann (editors). 1929. Copenhagen. Manuscript Cod. Holm. A120.\\[2mm]
%%
Swedish -- all texts downloaded from Fornsvenska textbanken.  \\
SW\_AVL = Äldre Västgötalagen, ca. 1225 \\
SW\_Bur = Codex Bureanus, ca. 1330 \\
SW\_Jart = Järteckensbok, ca. 1385 \\
SW\_ST = Själens Tröst, ca. 1420 \\
SW\_OgL = Östgötalagen, ca. 1290 \\
SW\_HML = Helga manna leverne, ca. 1385



{\sloppy\printbibliography[heading=subbibliography,notkeyword=this]}
\end{document}
